\documentclass[a4paper]{article}
\usepackage[utf8]{inputenc}
\usepackage[T1]{fontenc}
% \usepackage{fourier}
\usepackage{textcomp}
\usepackage{hyperref}
\usepackage[english]{babel}
\usepackage{url}
% \usepackage{hyperref}
% \hypersetup{
%     colorlinks,
%     linkcolor={black},
%     citecolor={black},
%     urlcolor={blue!80!black}
% }
\usepackage{graphicx} \usepackage{float}
\usepackage{booktabs}
\usepackage{enumitem}
% \usepackage{parskip}
% \usepackage{parskip}
\usepackage{emptypage}
\usepackage{subcaption}
\usepackage{multicol}
\usepackage[usenames,dvipsnames]{xcolor}
\usepackage{ocgx}
% \usepackage{cmbright}


\usepackage[margin=1in]{geometry}
\usepackage{amsmath, amsfonts, mathtools, amsthm, amssymb}
\usepackage{thmtools}
\usepackage{mathrsfs}
\usepackage{cancel}
\usepackage{bm}
\newcommand\N{\ensuremath{\mathbb{N}}}
\newcommand\R{\ensuremath{\mathbb{R}}}
\newcommand\Z{\ensuremath{\mathbb{Z}}}
\renewcommand\O{\ensuremath{\emptyset}}
\newcommand\Q{\ensuremath{\mathbb{Q}}}
\newcommand\C{\ensuremath{\mathbb{C}}}
\newcommand\F{\ensuremath{\mathbb{F}}}
% \newcommand\P{\ensuremath{\mathbb{P}}}
\DeclareMathOperator{\sgn}{sgn}
\DeclareMathOperator{\diam}{diam}
\DeclareMathOperator{\LO}{LO}
\DeclareMathOperator{\UP}{UP}
\DeclareMathOperator{\card}{card}
\DeclareMathOperator{\Arg}{Arg}
\DeclareMathOperator{\Dom}{Dom}
\DeclareMathOperator{\Log}{Log}
\DeclareMathOperator{\dist}{dist}
% \DeclareMathOperator{\span}{span}
\usepackage{systeme}
\let\svlim\lim\def\lim{\svlim\limits}
\renewcommand\implies\Longrightarrow
\let\impliedby\Longleftarrow
\let\iff\Longleftrightarrow
\let\epsilon\varepsilon
\usepackage{stmaryrd} % for \lightning
\newcommand\contra{\scalebox{1.1}{$\lightning$}}
% \let\phi\varphi
\renewcommand\qedsymbol{$\blacksquare$}

% correct
\definecolor{correct}{HTML}{009900}
\newcommand\correct[2]{\ensuremath{\:}{\color{red}{#1}}\ensuremath{\to }{\color{correct}{#2}}\ensuremath{\:}}
\newcommand\green[1]{{\color{correct}{#1}}}

% horizontal rule
\newcommand\hr{
    \noindent\rule[0.5ex]{\linewidth}{0.5pt}
}

% hide parts
\newcommand\hide[1]{}

% si unitx
\usepackage{siunitx}
\sisetup{locale = FR}
% \renewcommand\vec[1]{\mathbf{#1}}
\newcommand\mat[1]{\mathbf{#1}}

% tikz
\usepackage{tikz}
\usepackage{tikz-cd}
\usetikzlibrary{intersections, angles, quotes, calc, positioning}
\usetikzlibrary{arrows.meta}
\usepackage{pgfplots}
\pgfplotsset{compat=1.13}

\tikzset{
    force/.style={thick, {Circle[length=2pt]}-stealth, shorten <=-1pt}
}

% theorems
\makeatother
\usepackage{thmtools}
\usepackage[framemethod=TikZ]{mdframed}
\mdfsetup{skipabove=1em,skipbelow=1em}

\theoremstyle{definition}

\declaretheoremstyle[
    headfont=\bfseries\sffamily\color{ForestGreen!70!black}, bodyfont=\normalfont,
    mdframed={
        linewidth=1pt,
        rightline=false, topline=false, bottomline=false,
        linecolor=ForestGreen, backgroundcolor=ForestGreen!5,
    }
]{thmgreenbox}

\declaretheoremstyle[
    headfont=\bfseries\sffamily\color{NavyBlue!70!black}, bodyfont=\normalfont,
    mdframed={
        linewidth=1pt,
        rightline=false, topline=false, bottomline=false,
        linecolor=NavyBlue, backgroundcolor=NavyBlue!5,
    }
]{thmbluebox}

\declaretheoremstyle[
    headfont=\bfseries\sffamily\color{NavyBlue!70!black}, bodyfont=\normalfont,
    mdframed={
        linewidth=1pt,
        rightline=false, topline=false, bottomline=false,
        linecolor=NavyBlue
    }
]{thmblueline}

\declaretheoremstyle[
    headfont=\bfseries\sffamily, bodyfont=\normalfont,
    numbered = no,
    mdframed={
        rightline=true, topline=true, bottomline=true,
    }
]{thmbox}

\declaretheoremstyle[
    headfont=\bfseries\sffamily, bodyfont=\normalfont,
    numbered=no,
    % mdframed={
    %     rightline=true, topline=false, bottomline=true,
    % },
    qed=\qedsymbol
]{thmproofbox}

\declaretheoremstyle[
    headfont=\bfseries\sffamily\color{NavyBlue!70!black}, bodyfont=\normalfont,
    numbered=no,
    mdframed={
        rightline=false, topline=false, bottomline=false,
        linecolor=NavyBlue, backgroundcolor=NavyBlue!1,
    },
]{thmexplanationbox}

\declaretheorem[
    style=thmbox, 
    % numberwithin = section,
    numbered = no,
    name=Definition
    ]{definition}

\declaretheorem[
    style=thmbox, 
    name=Example,
    ]{eg}

\declaretheorem[
    style=thmbox, 
    % numberwithin = section,
    name=Proposition]{prop}

\declaretheorem[
    style = thmbox,
    numbered=yes,
    name =Problem
    ]{problem}

\declaretheorem[style=thmbox, name=Theorem]{theorem}
\declaretheorem[style=thmbox, name=Lemma]{lemma}
\declaretheorem[style=thmbox, name=Corollary]{corollary}

\declaretheorem[style=thmproofbox, name=Proof]{replacementproof}

\declaretheorem[style=thmproofbox, 
                name = Solution
                ]{replacementsolution}

\renewenvironment{proof}[1][\proofname]{\vspace{-1pt}\begin{replacementproof}}{\end{replacementproof}}

\newenvironment{solution}
    {
        \vspace{-1pt}\begin{replacementsolution}
    }
    { 
            \end{replacementsolution}
    }

\declaretheorem[style=thmexplanationbox, name=Proof]{tmpexplanation}
\newenvironment{explanation}[1][]{\vspace{-10pt}\begin{tmpexplanation}}{\end{tmpexplanation}}

\declaretheorem[style=thmbox, numbered=no, name=Remark]{remark}
\declaretheorem[style=thmbox, numbered=no, name=Note]{note}

\newtheorem*{uovt}{UOVT}
\newtheorem*{notation}{Notation}
\newtheorem*{previouslyseen}{As previously seen}
% \newtheorem*{problem}{Problem}
\newtheorem*{observe}{Observe}
\newtheorem*{property}{Property}
\newtheorem*{intuition}{Intuition}

\usepackage{etoolbox}
\AtEndEnvironment{vb}{\null\hfill$\diamond$}%
\AtEndEnvironment{intermezzo}{\null\hfill$\diamond$}%
% \AtEndEnvironment{opmerking}{\null\hfill$\diamond$}%

% http://tex.stackexchange.com/questions/22119/how-can-i-change-the-spacing-before-theorems-with-amsthm
\makeatletter
% \def\thm@space@setup{%
%   \thm@preskip=\parskip \thm@postskip=0pt
% }
\newcommand{\oefening}[1]{%
    \def\@oefening{#1}%
    \subsection*{Oefening #1}
}

\newcommand{\suboefening}[1]{%
    \subsubsection*{Oefening \@oefening.#1}
}

\newcommand{\exercise}[1]{%
    \def\@exercise{#1}%
    \subsection*{Exercise #1}
}

\newcommand{\subexercise}[1]{%
    \subsubsection*{Exercise \@exercise.#1}
}


\usepackage{xifthen}

\def\testdateparts#1{\dateparts#1\relax}
\def\dateparts#1 #2 #3 #4 #5\relax{
    \marginpar{\small\textsf{\mbox{#1 #2 #3 #5}}}
}

\def\@lesson{}%
\newcommand{\lesson}[3]{
    \ifthenelse{\isempty{#3}}{%
        \def\@lesson{Lecture #1}%
    }{%
        \def\@lesson{Lecture #1: #3}%
    }%
    \subsection*{\@lesson}
    \testdateparts{#2}
}

% \renewcommand\date[1]{\marginpar{#1}}


% fancy headers
\usepackage{fancyhdr}
\pagestyle{fancy}

\makeatother

% notes
\usepackage{todonotes}
\usepackage{tcolorbox}

\tcbuselibrary{breakable}
\newenvironment{verbetering}{\begin{tcolorbox}[
    arc=0mm,
    colback=white,
    colframe=green!60!black,
    title=Opmerking,
    fonttitle=\sffamily,
    breakable
]}{\end{tcolorbox}}

\newenvironment{noot}[1]{\begin{tcolorbox}[
    arc=0mm,
    colback=white,
    colframe=white!60!black,
    title=#1,
    fonttitle=\sffamily,
    breakable
]}{\end{tcolorbox}}

% figure support
\usepackage{import}
\usepackage{xifthen}
\pdfminorversion=7
\usepackage{pdfpages}
\usepackage{transparent}
\newcommand{\incfig}[1]{%
    \def\svgwidth{\columnwidth}
    \import{./figures/}{#1.pdf_tex}
}

% %http://tex.stackexchange.com/questions/76273/multiple-pdfs-with-page-group-included-in-a-single-page-warning
\pdfsuppresswarningpagegroup=1


\title{Homework 1}

\author{Lance Remigio}
\begin{document}
\maketitle

\begin{problem}
   \begin{enumerate}
       \item[(a)] Define \( f: [0,\infty  ] \to \R  \) by \( f(t) = \frac{ t  }{  1 + t  }  \) show that \( f  \) is an increasing function.
       \item[(b)] Let \( (X,d) \) be a metric space. Define \( \tilde{d}  : X \times X \to \R  \) by
           \[  \tilde{d}(x,y) = \frac{ d(x,y) }{ 1 + d(x,y) }. \]
           Prove that \( \tilde{d} \) is a metric on \( X  \).
   \end{enumerate} 
\end{problem}
\begin{proof}
\begin{enumerate}
    \item[(a)] We will show that \( f: [0,\infty ] \to \R  \) defined by \( f(t) = \frac{ t  }{  1 + t }  \) is an increasing function; that is, for any \( a,b \in [0,\infty ] \) with \( a \leq b  \), we have
        \[  \frac{  a  }{ 1 + a  }  \leq \frac{ b  }{ 1 + b }. \]
        To this end, let \( a,b \in [0,\infty ] \) such that \( a \leq b  \). Observe that
            \begin{align*}
                a \leq  b  &\Longleftrightarrow ab + a \leq ab +  b  \\
                        &\Longleftrightarrow a (b + 1) \leq b (a + 1) \\
                        &\Longleftrightarrow \frac{ a  }{ 1 + a  }  \leq \frac{ b }{  1 + b } .
            \end{align*}
            Hence, we see that \( f(t) = \frac{ t }{ 1+ t }  \) must be an increasing function.
        \item[(b)] 
            \begin{enumerate}
        \item[(i)] Let \( x,y \in X  \). Since \( (X,d) \) is a metric space, we know that \( d(x,y) = 0  \) if and only if \( x = y  \). Thus, we see that  
            \begin{align*}
                \tilde{d}(x,y) = 0 &\Longleftrightarrow \frac{ d(x,y) }{ 1 + d(x,y)  } = 0 \\
                           &\Longleftrightarrow d(x,y) = 0 \\
                           &\Longleftrightarrow x = y.
            \end{align*}
            Hence, property (ii) is satisfied.
        \item[(ii)] Let \( x,y \in X  \). Since \( d(x,y) = d(y,x) \) for all \( x,y \in X  \) (because \( (X,d) \) is a metric space), we see that  
            \[ \tilde{d}(x,y) = \frac{ d(x,y) }{ 1 + d(x,y) } = \frac{ d(y,x) }{ 1 + d(y,x) }  = \tilde{d}(y,x).  \]
            Thus, property (iii) is satisfied.
        \item[(iii)] Now, we will show that \( \tilde{d} \) satisfies the triangle inequality. Let \( x,y,z \in X  \).
We need to consider a few cases when proving the triangle inequality:
            \begin{enumerate}
                \item[(I)] \( d(x,y) \leq d(x,z) \).
                \item[(II)] \( d(x,y) \leq d(z,y) \)
                \item[(III)] \( d(x,y) > d(x,z) \) and \( d(x,z) > d(z,y) \).
            \end{enumerate}
            We proceed with each case as follows:
            \end{enumerate}
            \begin{enumerate}
                \item[(I)] If \( d(x,y) \leq d(x,z) \), then by using the monotonicity of \( \tilde{d}(x,y) \), we can see that 
                    \[  \tilde{d}(x,y) = \frac{ d(x,y) }{ 1 + d(x,y) } \leq \frac{ d(x,z) }{ 1 + d(x,z) } = \tilde{d}(x,z) \leq \tilde{d}(x,z) + \tilde{d}(z,y). \]
                \item[(II)] Similarly, if \( d(x,y) \leq d(z,y) \), we have by using the monotonicity of \( \tilde{d}(x,y) \) that
                    \[  \tilde{d}(x,y) = \frac{ d(x,y) }{  1 + d(x,y) } \leq \frac{ d(z,y) }{ 1 + d(z,y) } = \tilde{d}(z,y) \leq \tilde{d}(x,z) + \tilde{d}(z,y). \]
                \item[(III)] If both \( d(x,y) > d(x,z) \) and \( d(x,z) > d(z,y) \), then by the triangle inequality property of \( (X,d) \), we see that                   
                    \begin{align*}
                        \tilde{d}(x,y) &= \frac{ d(x,y) }{ 1 + d(x,y) }  \\
                           &\leq \frac{ d(x,z) + d(z,y)  }{ 1 + d(x,y) } \\
                           &= \frac{ d(x,z) }{ 1 + d(x,y)  }  + \frac{ d(z,y) }{ 1 + d(z,y) } \\
                           &\leq \frac{ d(x,z) }{ 1 + d(x,z) }  + \frac{ d(z,y) }{ 1 + d(z,y) } \\
                           &= \tilde{d}(x,z) + \tilde{d}(z,y).
                \end{align*} 
            \end{enumerate}
            Thus, we can see that \( \tilde{d}(x,y) \) satisfies the triangle inequality.
\end{enumerate}
Since properties (I) through (IV) are satisfied, we can conclude that \( \tilde{d} \) is a metric on \( X  \).
\end{proof}

\begin{problem}
    Let \( X = \R^{n} \). For \( \vec{ x }  = \begin{pmatrix} {x}_{1} \\ \vdots \\ {x}_{n} \end{pmatrix} , \vec{ y }  = \begin{pmatrix} {y}_{1} \\ \vdots \\ {y}_{n} \end{pmatrix}  \). Define \( {d}_{1}(\vec{ x } , \vec{ y } ) = \displaystyle  \sum_{ i=1  }^{ n } | {x}_{i} - {y}_{i} |  \) and \( {d}_{\infty }(\vec{ x } , \vec{ y } ) = \max \{ | {x}_{i} - {y}_{i} | : 1 \leq i \leq n   \}    \).
    Show that \( {d}_{1}  \) and \( {d}_{\infty } \) are metrics on \( \R^{n} \).
\end{problem}
\begin{proof}
We will show that the following functions
\begin{enumerate}
    \item[(1)]  \( {d}_{1}(\vec{ x } , \vec{ y } ) = \displaystyle \sum_{ i=1  }^{ n } | {x}_{i} - {y}_{i} |  \) 
    \item[(2)] \( {d}_{\infty }(\vec{ x } , \vec{ y } ) = \max \{ | {x}_{i} - {y}_{i} | : 1 \leq i \leq n  \} = \max_{1 \leq i \leq n} | {x}_{i} - {y}_{i} |     \).
\end{enumerate}

First, we will show that (1) is a metric on \( \R^{n} \). 
\begin{enumerate}
    \item[(i)] Suppose \( \vec{ x }  = \vec{ y }  \). Then observe that 
        \begin{align*}
            \vec{ x }  = \vec{ y } &\iff {x}_{i} = {y}_{i} \ \forall 1 \leq i \leq n \\
                                   &\iff {x}_{i} - {y}_{i} = 0 \ \forall 1 \leq i \leq n \\
                                   &\iff | {x}_{i} - {y}_{i} |  = 0 \ \forall 1 \leq i \leq n \\
                                   &\iff \sum_{ i=1  }^{ n } | {x}_{i} - {y}_{i} | = 0 \  \forall 1 \leq i \leq n \\
                                   &\iff {d}_{1}(\vec{ x } , \vec{ y } ) = 0. 
        \end{align*}
        Thus, property (i) is satisfied.
    \item[(ii)] Let \( \vec{ x } , \vec{ y }  \in \R^{n} \). Then 
        \[  {d}_{1}(\vec{ x } , \vec{ y } ) = \sum_{ i=1  }^{ n } | {x}_{i} - {y}_{i} |  = \sum_{ i=1  }^{ n } | {y}_{i} - {x}_{i} | = {d}_{1}(\vec{ y } , \vec{ x } ). \]
        Thus, property (ii) is satisfied.
    \item[(iii)] Let \( \vec{ x } , \vec{ y }, \vec{ z }  \in \R^{n} \). By the triangle inequality of the standard metric \( | \cdot |  \) on \( \R \), we have 
        \begin{align*}
            {d}_{1}(\vec{ x } , \vec{ y } ) = \sum_{ i=1  }^{ n } | {x}_{i} - {y}_{i} | &\leq \sum_{ i=1  }^{ n } [ | {x}_{i} - {z}_{i} |  + | {z}_{i} - {y}_{i} | ] \\
                                                                                        &= \sum_{ i=1  }^{ n } | {x}_{i} - {z}_{i} |  + \sum_{ i=1  }^{ n } | {z}_{i} - {y}_{i} |  \\
                                                                                        &= {d}_{1}(\vec{ x } , \vec{ z } ) + {d}_{1}(\vec{ z } , \vec{ y } ).
        \end{align*}
\end{enumerate}
Thus, properties (i) through (ii) of a metric space are satisfied. Hence, \( {d}_{1} \) is a metric on \( \R^{n} \).

Now, we will show that (2) is a metric on \( \R^{n} \).
\begin{enumerate}
    \item[(i)] Observe that 
        \begin{align*}
            {d}_{\infty }(\vec{ x } , \vec{ y } ) = 0 &\implies \max_{1 \leq i \leq n } | {x}_{i} - {y}_{i} | = 0  \\
                                                      &\implies 0 \leq | {x}_{i} - {y}_{i} |  \leq 0 \ \  \forall 1 \leq i \leq n \\
                                                      &\implies | {x}_{i} - {y}_{i} |  = 0 \ \ \forall 1 \leq i \leq n \\
                                                      &\implies {x}_{i} - {y}_{i} = 0 \ \ \forall 1 \leq i \leq n \\
                                                      &\implies {x}_{i} = {y}_{i} \ \ \forall 1 \leq i \leq n \\
                                                      &\implies \vec{ x }  = \vec{ y }. 
        \end{align*}
        Conversely, we have  
        \begin{align*}
            \vec{ x }  = \vec{ y } &\implies {x}_{i} = {y}_{i} \ \ \forall 1 \leq i \leq n  \\
                                   &\implies {x}_{i} - {y}_{i} = 0 \ \ \forall 1 \leq i \leq n \\ 
                                   &\implies | {x}_{i} - {y}_{i} | = 0 \ \ \forall 1 \leq i \leq n \\
                                   &\implies \max_{1 \leq i \leq n} | {x}_{i} - {y}_{i} |  = 0 \\
                                   &\implies {d}_{\infty }(\vec{ x }, \vec{ y } ) = 0. 
        \end{align*}
    \item[(ii)] Let \( \vec{ x } , \vec{ y }  \in \R^{n} \). Then we have 
        \[  {d}_{\infty }(\vec{ x } ,\vec{ y } ) = \max_{1 \leq i \leq n} | {x}_{i} - {y}_{i} |  = \max_{1 \leq i \leq n} | {y}_{i} - {x}_{i} |  = {d}_{\infty }(\vec{ y } , \vec{ x } ). \]
    \item[(iii)] Let \( \vec{ x } ,\vec{ y } , \vec{ z }  \in \R^{n} \). Our goal is to show that 
        \[  {d}_{\infty }(\vec{ x } , \vec{ y } ) \leq {d}_{\infty }(\vec{ x } , \vec{ z } ) + {d}_{\infty }(\vec{ z } ,  \vec{ y } ). \]
        Note that 
        \[  | {x}_{i} - {z}_{i} |  \leq \max_{1 \leq i \leq n} | {x}_{i} - {z}_{i} | \ \ \text{and} \ \ | {z}_{i} - {y}_{i} |  \leq \max_{1 \leq i \leq n} | {z}_{i} - {y}_{i} |  \tag{\(\forall 1 \leq i \leq n\)}  \]
        Adding the two inequalities above gives us 
        \[  | {x}_{i} - {z}_{i} |  + | {z}_{i} - {y}_{i} |  \leq \max_{1 \leq   i \leq n} | {x}_{i} - {z}_{i} |  + \max_{1 \leq i \leq n} | {z}_{i} - {y}_{i} |.  \tag{\( 1 \leq i \leq n \)} \]
        Using the triangle inequality property of the standard metric \( | \cdot |  \) on \( \R  \), we can see that 
        \[  | {x}_{i} - {y}_{i} |  \leq | {x}_{i} - {z}_{i} |  + | {z}_{i} - {y}_{i} |. \]
        Hence, we have 
        \[  | {x}_{i} - {y}_{i} |  \leq  \max_{1 \leq   i \leq n} | {x}_{i} - {z}_{i} |  + \max_{1 \leq i \leq n} | {z}_{i} - {y}_{i} |. \]
        Notice that the right-hand side of the above inequality is an upper bound of the set \( \{ | {x}_{i} - {y}_{i} | : {x}_{i}, {y}_{i} \in \R , 1 \leq i \leq n \}  \). Thus, we see that 
        \[  \max_{1 \leq i \leq n } | {x}_{i} - {y}_{i} | \leq \max_{1 \leq   i \leq n} | {x}_{i} - {z}_{i} |  + \max_{1 \leq i \leq n} | {z}_{i} - {y}_{i} |. \]
        By definition, we can conclude that
        \[ {d}_{\infty }(\vec{ x } , \vec{ y } ) \leq {d}_{\infty }(\vec{ x } , \vec{ z } ) + {d}_{\infty }(\vec{ z } , \vec{ y } ).  \]
\end{enumerate}
Hence, we can conclude that \( {d}_{\infty } \) forms a metric on \( \R^{n} \).
\end{proof}

\begin{problem}
    Let \( X = \R^{n} \) and \( {d}_{1} \) and \( {d}_{\infty } \) be as defined as in problem 2. 
    \begin{enumerate}
        \item[(a)] 
            \begin{enumerate}
                \item[(i)] Show that \( {d}_{\infty }(\vec{ x } , \vec{ y } ) \leq {d}_{1} (\vec{ x } , \vec{ y } ) \) for all \( \vec{ x } , \vec{ y } \in \R^{n} \).  
                \item[(ii)] Let \( \vec{ {x}_{0} } \in \R^{n} \) and \( r > 0  \). Let 
                    \begin{align*}
                        {B}_{1} &= \{ \vec{ x  }  \in \R^{n} : {d}_{1}(\vec{ x } , \vec{ {x}_{0} } ) < r \}  \\
                        {B}_{2} &= \{ \vec{ x } \in \R^{n} : {d}_{\infty }(\vec{ x } , \vec{ {x}_{0} } ) < r \}. 
                    \end{align*}
                    Which one of the following holds and why?
                    \[  {B}_{1} \subseteq {B}_{2} \ \text{or} \ {B}_{2} \subseteq {B}_{1}. \]
            \end{enumerate}
        \item[(b)] Show that \( {d}_{1}(\vec{ x } , \vec{ y } ) \leq n \cdot {d}_{\infty }(\vec{ x } , \vec{ y } ). \)
        \item[(c)] Fix \( \vec{ {x}_{0} }  \in \R^{n} \) and \( r > 0  \). Prove that 
            \[  \{ \vec{ x }  \in \R^{n} : {d}_{\infty }(\vec{ x } , \vec{ {x}_{0} } ) < \frac{ r  }{ n  }  \}  \subseteq  \{ \vec{ x }  \in \R^{n} : {d}_{1}(\vec{ x } , \vec{ {x}_{0} } ) < r  \}. \]
        \item[(d)] Prove that \( M \subseteq  \R^{n} \) is open with respect to the metric \( {d}_{1} \) if and only if \( M  \) is open with respect to the metric \( {d}_{\infty } \).
    \end{enumerate}
\end{problem}
\begin{proof}
\begin{enumerate}
    \item[(a)] 
        \begin{enumerate}
            \item[(i)] Note that for all \( \vec{ x } , \vec{ y } \in \R^{n} \), we have 
        \[  | {x}_{i} - {y}_{i} |  \leq \sum_{ i=1  }^{ n } | {x}_{i} - {y}_{i} |. \tag{\( \forall 1 \leq i \leq n \)}  \]
        Furthermore, the right-hand side of the above inequality is an upper bound for the set 
        \[  \{ | {x}_{i} - {y}_{i}  |  : 1 \leq i \leq n , \ \ \vec{ x } , \vec{ y  }  \in \R^{n} \}. \]
        Hence, we have 
        \[  \max_{1 \leq i \leq n} | {x}_{i} - {y}_{i} |  \leq \sum_{ i=1  }^{ n } | {x}_{i} - {y}_{i} | \implies {d}_{\infty }(\vec{ x } , \vec{ y } ) \leq {d}_{1}(\vec{ x } , \vec{ y } ). \]
        \item[(ii)] We claim that \( {B}_{2} \subseteq  {B}_{1} \) holds. Let \( \vec{ y }  \in {B}_{2} \). By part (i), we can see that \( {d}_{\infty }(\vec{ y } , \vec{ {x}_{0} } ) \leq {d}_{1}(\vec{ y } , \vec{ {x}_{0} } ) < r  \). Hence, we have \( \vec{ y }  \in {B}_{1} \) and so \( {B}_{2} \subseteq {B}_{1} \).
        \end{enumerate}
    \item[(b)] Our goal is to show that \( {d}_{1} (\vec{ x } ,\vec{ y } ) \leq n \cdot {d}_{\infty }(\vec{ x } , \vec{ y } ) \) for any \( \vec{ x } , \vec{ y }  \in \R^{n} \). Let \( \vec{ x } , \vec{ y }  \in \R^{n} \). Observe that 
        \begin{align*}
            {d}_{1}(\vec{ x } , \vec{ y } ) = \sum_{ i=1  }^{ n } | {x}_{i} - {y}_{i} | &\leq \sum_{ i=1  }^{ n } \max_{1 \leq i \leq n} | {x}_{i} - {y}_{i} |   \\
                                                                                        &= n \cdot \max_{1 \leq i \leq n} | {x}_{i} - {y}_{i} | \\ 
                                                                                        &= n \cdot {d}_{\infty }(\vec{ x } , \vec{ y }).
        \end{align*}
        Thus, we have \( {d}_{1}(\vec{ x } ,\vec{ y } ) \leq n \cdot {d}_{\infty }(\vec{ x } , \vec{ y } ) \).
    \item[(c)] Fix \( \vec{ {x}_{0} }  \in \R^{n} \) and \( r > 0  \). Set
        \begin{align*}
            {D}_{1} &= \{ \vec{ x } \in \R^{n} : {d}_{\infty }(\vec{ x } , \vec{ {x}_{0} } ) < \frac{ r }{ n }  \},  \\
            {D}_{2}&= \{ \vec{ x }  \in \R^{n} : {d}_{1}(\vec{ x } , \vec{ {x}_{0} } ) < r \}.
        \end{align*}
        We will show that \( {D}_{1} \subseteq {D}_{2} \). Let \( \vec{ y }  \in {D}_{1} \). Then  
        \[  {d}_{\infty }(\vec{ y } , \vec{ {x}_{0} } ) < \frac{ r }{ n }  \iff n \cdot {d}_{\infty }(\vec{ y } , \vec{ {x}_{0} } ) < r.  \]
        Using part (b), we can write
        \[  {d}_{1}(\vec{ y } , \vec{ {x}_{0} } ) < r. \]
        Hence, \( \vec{ y }  \in {D}_{2} \). So, \( {D}_{1} \subseteq {D}_{2} \).
    \item[(d)] \( (\Longrightarrow) \) Suppose that \( M  \) is an open set with respect to the metric \( {d}_{1} \). We will show that \( M  \) is open with respect to the metric \( {d}_{\infty } \); that is, we want to show that for all \( \vec{ x  }  \in M  \), there exists a \( \delta > 0  \) such that \( B_{{d}_{\infty }}(\vec{ x } ; \delta) \subseteq  M  \). To this end, let \( \vec{  x  }  \in M \). By assumption, we can see that there exists \( \hat{\delta} > 0  \) such that \( {B}_{{d}_{1}}(\vec{ x } , \hat{\delta}) \subseteq M  \).

        We claim that \( \hat{\delta} \) can be used as the same \( \delta \) we were looking for. Indeed, we can see by part (ii) of (a) that \( {B}_{{d}_{\infty}}(\vec{ x } ; \delta) \subseteq {B}_{{d}_{1}}(\vec{ x }, \delta ) \subseteq M    \). Hence, we have \( M  \) must be open with respect to the metric \( {d}_{\infty }  \).

        \( (\Longleftarrow) \) Suppose that \( M  \) is an open set with respect to the metric \( {d}_{\infty } \). Our goal is to show that \( M \) is open with respect to the metric \( {d}_{1} \); that is, for all \( \vec{ y }  \in M \), we need to find a \( \delta > 0  \) such that \( B_{{d}_{1}}(\vec{ x } ; \delta ) \subseteq  M  \). Let \( \vec{ y }  \in M  \). By assumption, we can find a \( \hat{\delta} > 0  \) such that \( {B}_{{d}_{\infty }}(\vec{ y } ; \hat{\delta}) \subseteq  M \). If \( \vec{ x } \in {B}_{{d}_{\infty }}(\vec{ y } ; \hat{\delta})  \), then \( {d}_{\infty }(\vec{ x } , \vec{ y } ) < \hat{\delta} \). Set \( \delta = \hat{\delta} - {d}_{\infty }(\vec{ x } , \vec{ y } ) > 0  \). It suffices to show that \( {B}_{{d}_{1}}(\vec{ y } ; \delta) \subseteq {B}_{{d}_{\infty }} (\vec{ y } ; \hat{\delta})  \). Let \(  \vec{ z }  \in {B}_{{d}_{1}}(\vec{ y } ; \delta )  \). Then
        \begin{align*}
            {d}_{1}(\vec{ z } , \vec{ y } ) < \delta &\implies {d}_{1}(\vec{ z } , \vec{ y } ) < \hat{\delta} - {d}_{\infty }(\vec{ x } , \vec{ y } ).
        \end{align*}
        Since \( {d}_{1}(\vec{ z } , \vec{ y } ) < {d}_{1}(\vec{ z } ,\vec{ y } ) + {d}_{\infty }(\vec{ x } , \vec{ y } ) \), we can conclude that \[  {d}_{1}(\vec{ z } , \vec{ y } ) < \hat{\delta} \implies \vec{ z }  \in {B}_{{d}_{\infty }}(\vec{ y }; \hat{\delta} ) \implies {B}_{{d}_{1}}(\vec{ y  }  ; \delta) \subseteq {B}_{{d}_{\infty }}(\vec{ y } ; \hat{\delta}). \] 
        Hence, \( M  \) is open with respect to \( {d}_{1} \).
\end{enumerate}
\end{proof}

\begin{problem}
    Let \( I  \) be an indexing set and \( \{ {X}_{i} \}_{i \in I} \) be a collection of subsets of \( X  \). We define
    \begin{align*}
        \bigcup_{ i \in I  }^{  }  {X}_{i} &= \{ x \in X : x \in {X}_{i} \ \ \text{for some} \ \ i \in I  \}  \\ 
        \bigcap_{ i \in I  }^{  }  {X}_{i} &= \{ x \in X : x \in {X}_{i} \ \ \text{for all} \ \ i \in I  \}.
    \end{align*}
    Let \( (X,d) \) be a metric space.
    \begin{enumerate}
        \item[(i)] Let \( \{ {M}_{i} \}_{i \in I} \) be a collection of open sets in \( X  \). Show that \( \bigcup_{ i \in I  }^{  }  {M}_{i} \) is also open.
        \item[(ii)] Let \( {M}_{1}  \) and \( {M}_{2} \) be two open sets in \( X  \). Prove that \( {M}_{1} \cap {M}_{2} \) is open.
        \item[(iii)] Let \( {x}_{0} \in X  \) and \( r > 0  \). Show that \( B({x}_{0}; r)  \) is open.
        \item[(iv)] Let \( \{ {K}_{i} \}_{i \in I} \) be a collection of closed sets in \( X  \). Prove that \( \bigcap_{ i \in I  }^{  }  {K}_{i} \) is closed.
        \item[(v)] Let \( {K}_{1} \) and \( {K}_{2} \) be closed in \( X  \). Prove that \( {K}_{1} \cup {K}_{2} \) is closed.
        \item[(vi)] Let \( {x}_{0} \in X  \) and \( r > 0  \). Prove that \( \overline{B}({x}_{0};r) \) is closed. 
        \item[(vii)] Let \( M \subseteq X  \). Prove that \( \overline{M} \) is closed in \( X  \).
        \item[(viii)] Let \( {x}_{0} \in X  \) and \( r > 0  \). Prove that 
            \[  \overline{B({x}_{0};r)} \subseteq \overline{B}({x}_{0}; r). \]
            Is it always true that 
            \[  \overline{B({x}_{0};r)} = \overline{B}({x}_{0};r)? \]
            Justify your answer.
    \end{enumerate}
\end{problem}
\begin{proof}
\begin{enumerate}
    \item[(i)] Let \( x \in \bigcup_{ i \in I  }^{  } {M}_{i} \). Our goal is to find a \( \delta > 0  \) such that \( B(x,\delta) \subseteq \bigcup_{ i \in I  }^{  }  {M}_{i} \). Now, notice that 
        \[  x \in \bigcup_{ i \in I  }^{  } {M}_{i} \implies \exists k \in I \ \text{such that} \ x \in {M}_{k}.   \]
        Since \( {M}_{k} \) is an open set, there exists \( \hat{\delta} > 0  \) such that \( B(x, \hat{\delta}) \subseteq {M}_{k} \). But this means that 
        \[  B(x,\hat{\delta}) \subseteq {M}_{k} \]
        since \( {M}_{k} \subseteq \bigcup_{ i\in I  }^{  }  {M}_{i} \).
    \item[(ii)] Let \( x \in {M}_{1} \cap {M}_{2} \). Our goal is to show that there exists \( \delta > 0   \) such that \( B(x,\delta) \subseteq  {M}_{1} \cap {M}_{2} \). If \( x \in {M}_{1} \cap {M}_{2} \), then \( x \in {M}_{1} \) and \( x \in {M}_{2} \). Then \( x \in {M}_{1} \) and \( {M}_{1} \) is open implies there exists \( {\delta}_{1} > 0  \) such that \( B(x,{\delta}_{1}) \subseteq {M}_{1} \). Likewise, \( x \in {M}_{2} \) an \( {M}_{2}  \) is open implies that there exists \( {\delta}_{2} > 0  \) such that \( B(x,{\delta}_{2}) \subseteq {M}_{2} \). Choose \( \delta = \frac{ 1 }{ 2 }  \min \{ {\delta}_{1}, {\delta}_{2} \}  \). Our goal is to show that \( B(x,\delta) \subseteq {M}_{1} \cap {M}_{2} \). By the way \( \delta  \) was constructed, observe that  
        \begin{center}
            \( B(x,\delta) \subseteq B(x,{\delta}_{1}) \subseteq {M}_{1} \) and \( B(x,\delta) \subseteq B(x, {\delta}_{2}) \subseteq  {M}_{2} \).
        \end{center}
        If \( y \in B(x,\delta) \), then \( y \in {M}_{1} \) and \( y \in {M}_{2} \) by the above. Hence, we have 
        \[  B(x,\delta) \subseteq {M}_{1} \cap {M}_{2}. \]
    \item[(iii)] Our goal is to show that \( B({x}_{0}, r)  \) is open; that is, we want to show that for every \( x \in B({x}_{0}, r) \), there exists \( \delta > 0  \) such that \( B(x,\delta) \subseteq  B({x}_{0},r) \). Let \( x \in B({x}_{0} ,r) \). Then 
        \[  d(x,{x}_{0}) < r \implies r - d(x,{x}_{0}) > 0.  \]
        Choose \( \delta = r - d(x,{x}_{0}) \). Now, our goal is to show that 
        \[  B(x,\delta) \subseteq B({x}_{0}, r). \]
        Let \( y \in B(x,\delta) \). Then
        \[  d(y,x) < \delta = r - d(x,{x}_{0}) \implies d(y,x) + d(x, {x}_{0}) < r. \]
        By the triangle inequality, we can see that 
        \[  d(y,{x}_{0}) \leq d(y,x) + d(x,{x}_{0}).  \]
        This implies that 
        \[  d(y,{x}_{0}) < r \implies y \in B({x}_{0},r). \]
        Hence, \( B(x,\delta) \subseteq  B({x}_{0}, r). \)
    \item[(iv)] Let \( \{ {K}_{i} \}_{i \in I} \) be a collection of closed sets in \( X  \). Our goal is to show that \( \bigcap_{  i \in I  }^{  }  {K}_{i} \) is closed. It suffices to show that 
        \[  \Big(  \bigcap_{ i \in I  }^{  }  {K}_{i} \Big)^{c} \ \text{is open}. \]
        The above can be rewritten in the following way
        \[  \Big(  \bigcap_{ i \in I }^{  }  {K}_{i} \Big)^{c} = \bigcup_{ i \in I  }^{  }  {K}_{i}^{c}. \]
        Since each \( {K}_{i}  \) is closed, we can see that \( {K}_{i}^{c} \) is open. Using part (i), we can conclude that \( \Big(  \bigcap_{ i \in I  }^{  }  {K}_{i} \Big)^{c} \) is open and so 
        \[  \bigcap_{ i \in I  }^{  }  {K}_{i} \] must be closed.
    \item[(v)] Apply part (iv) to \( i \in I \) on \( {K}_{1}, {K}_{2}, \emptyset, \emptyset, \dots  \). 
    \item[(vi)] Our goal is to show that \( \overline{B}({x}_{0}, r)\) is a closed set. To this end, we will show that \( [\overline{B}({x}_{0},r)]^{c} \) is an open set. If this holds, then we can conclude that \( \overline{B}({x}_{0},r)\) is a closed set. Let \( x \in [\overline{B}({x}_{0},r)]^{c} \). Then we have \( d(x,{x}_{0}) > r \). Our goal is to find \( \epsilon > 0  \) such that \( B(x,\epsilon) \subseteq [\overline{B}({x}_{0}, r)]^{c} \). Then \( d(x,p) > r \). Choose \( \epsilon = d(x,p) - r > 0  \). 
        Using this chosen radius \( \epsilon  \), let \( y \in B(x,\epsilon) \). In order for \( y  \) to be contained within \( [\overline{B}({x}_{0},r)]^{c} \), we have to show that \( d(p,y) > r \). Using the triangle inequality, we have 
        \begin{align*}
            d(x,{x}_{0}) \leq d(x,y) + d(y,{x}_{0}) \Longrightarrow d(y,{x}_{0}) &\geq d(x,{x}_{0}) - d(x,{x}_{0}) \\   
                                                               &> d(x,{x}_{0}) - \epsilon \tag{\( y \in B(x, \epsilon) \)} \\ 
                                                               &= r.
        \end{align*}
        Indeed, we can now see that \( B(x,\epsilon) \subseteq [\overline{B}({x}_{0},r)]^{c} \). This tells us that \( [\overline{B}({x}_{0},r)]^{c} \) is open and so \( \overline{B}({x}_{0},r) \) is closed.
    \item[(vii)] Our goal is to show that \( (\overline{E})^{c} \) is open. We need to show that every point of \( (\overline{E})^{c} \) is an interior point of \( (\overline{E})^{c} \). Let \( p \in (\overline{E})^{c} \). We have 
\begin{align*}
    p \in (\overline{E})^{c} &\Longrightarrow p \notin \overline{E} \\ 
                             &\Longrightarrow p \notin (E \cup E') \\
                             &\Longrightarrow p \notin E \ \wedge \ p \notin E'.
\end{align*}
Note that 
\begin{align*}
    p \notin E' &\Longrightarrow \exists \epsilon > 0 \  B(p,\epsilon) \cap (E \setminus \{ p \} ) = \emptyset \\
                &\Longrightarrow \exists \epsilon > 0 \  B(p,\epsilon) \cap E = \emptyset. \tag{1}
\end{align*}
In what follows, we will show that \(  B(p,\epsilon) \cap E' = \emptyset \). So, we have
\begin{align*}
    &B(p,\epsilon) \cap (E \cup E') = \emptyset \\
                                       &\Longrightarrow  B(p,\epsilon)\cap \overline{E} = \emptyset \\
                                       &\Longrightarrow  B(p,\epsilon)\subseteq (\overline{E})^{c}.
\end{align*}
Thus, we have that \( p  \) is an interior point of \( (\overline{E})^{c}\). It remains to show that \( B(p,\epsilon) \cap E'  = \emptyset\). Assume for sake of contradiction that \(  B(p,\epsilon) \cap E' \neq \emptyset \). Let \( q \in B(p,\epsilon) \cap E' \). Then we have \( q \in B(p,\epsilon)  \) and \( q \in E' \). Because \( B(p,\epsilon) \) is an open set, there exists \( \delta > 0  \) such that \( B(q,\delta) \subseteq B(p,\epsilon) \) and that \(  B(q,\delta) \cap (E \setminus  \{ q \} ) \neq \emptyset \), respectively. But note that since \(  B(q,\delta) \subseteq B(p,\epsilon)  \) and \( E \setminus  \{ q  \} \subseteq E   \) implies that 
\[  B(p,\epsilon) \cap E \neq \emptyset. \]
which contradicts (1).
    \item[(viii)] We can see immediately that \( B({x}_{0}, r) \subseteq \overline{B}({x}_{0} ,r)    \). By part (vi), we can see that \( \overline{B}({x}_{0},r) \) is a closed set. As a consequence, we have that 
        \[  \overline{B({x}_{0},r)} \subseteq  \overline{B}({x}_{0},r). \]

        In general, it is not true that 
        \[  \overline{B({x}_{0},r)} = \overline{B}({x}_{0},r). \]
        Consider the interval \( [0,1] \) in \( \R  \) with the discrete metric. Clearly, we see that \( 1/2 \in [0,1] \). If we let \( \epsilon = 1   \), then 
            \[  B(\frac{ 1 }{ 2 } , 1) = \{ x \in \R : d(x,1/2) < 1 \} = \{ 1/2 \}  \]
            since the only case when the inequality is satisfied is when \( x =  1/2  \). If we consider the closure of this neighborhood, we just get 
            \[  \overline{B(\frac{ 1 }{ 2 } ,1)} = \{ 1/2 \}. \]
            Now, consider the closed ball 
            \[   B(\frac{ 1 }{ 2 } ,1) = \{ x \in \R : d(x,1/2) \leq 1  \}. \]
            Observe that for any \( x \in \R  \), either \( x = 1/2 \) or \( x \neq 1/2 \) in \( [0,1] \), the inequality of the set above we always be satisfied; that is, the set will just be all elements contained in \( [0,1]  \). Thus, we see that \( B(\frac{ 1 }{ 2 } ,1) = [0,1] \) and, in this case, that \( \overline{B}(\frac{ 1 }{ 2 } ,1) \neq \overline{B(\frac{ 1 }{ 2 } ,1)} \).  
\end{enumerate} 
\end{proof}

\begin{problem}
    In this problem, we will establish some key inequalities that will be useful later in the class. 
    \begin{enumerate}
        \item[(i)] \textbf{Young's Inequality:} Let \( a,b \in \R  \), \( a \geq 0  \), \( b \geq 0  \), and \( p > 1  \). Let \( q = \frac{  p  }{  p - 1  }  \). Then
            \[ ab \leq \frac{ a^{q} }{ q  }  + \frac{ b^{p} }{ p }. \]
           Prove Young's Inequality. 
       \item[(ii)] \textbf{Holder's Inequality:} Let \( p > 1  \). For \( \vec{ x }  \in \R^{n} \), \( \vec{ x  }  = \begin{pmatrix} {x}_{1} \\ \vdots \\ {x}_{n}   \end{pmatrix}  \), define \( \|\vec{ x } \|_p = \displaystyle \Big[ \sum_{ i=1  }^{ n } | {x}_{i} |^{p} \Big]^{1/p} \). Let \( q = \frac{ p  }{  p - 1  } \). For \( \vec{ x } , \vec{ y } \in \R^{n} \), prove that 
           \[  \sum_{ i=1  }^{ n } | {x}_{i} {y}_{i} | \leq \|\vec{ x } \|_{p} \|\vec{ y } \|_{q} \]
           where \( \vec{ x }  = \begin{pmatrix} {x}_{1} \\ \vdots \\ {x}_{n} \end{pmatrix} \), \( \vec{ y  } = \begin{pmatrix} {y}_{1} \\ \vdots \\ {y}_{n} \end{pmatrix} \)
        \item[(iii)] \textbf{Minkowski's Inequality} Let \( p , \vec{ x } ,  \) and \( \vec{ y }  \) be as in (ii). Prove that 
            \[  \|\vec{ x }  + \vec{ y } \|_p \leq \|\vec{ x } \|_{p} + \|\vec{ y } \|_{p}. \]
        \item[(iv)] Let \( X = \R^{n} \), \( p > 1  \). For \( \vec{ x  } , \vec{ y }  \in \R^{n} \), define \( {d}_{p}(\vec{ x } , \vec{ y } ) = \|\vec{ x }  - \vec{ y } \|_{p} \). Prove that \( {d}_{p} \) is a metric on \( \R^{n} \).
    \end{enumerate}
\end{problem}

\begin{proof}
\begin{enumerate}
    \item[(i)] Let \( p  \) and \( q  \) be positive real numbers with \( \frac{ 1 }{ p }  + \frac{ 1 }{ q }  = 1  \). Let \( a,b \in \R  \) be nonnegative. We have the following cases:  
            \begin{enumerate}
                \item[(1)] \( a = 0  \) and \( b = 0  \)
                \item[(2)] \( a = 0  \) and \( b > 0  \)
                \item[(3)] \( a > 0  \) and \( b = 0  \)
                \item[(4)] \( a > 0  \) and \( b > 0  \).
            \end{enumerate}
            We proceed with the proof of the result with the following cases.
            \begin{enumerate}
                \item[(1)] If \( a = 0  \) and \( b = 0  \), then the result is immediate. 
                \item[(2)] If \( a = 0  \) and \(  b > 0  \), then we immediately have  
                    \[  ab = 0 \leq \frac{ a^{p} }{ p  } + \frac{ b^{q} }{ q }  = \frac{ b^{q} }{ q }. \]
                \item[(3)] If \( b = 0  \) and \( a > 0  \), then we similarly have
                    \[  ab = 0 \leq \frac{ a^{p} }{ p  } + \frac{ b^{q} }{ q }  = \frac{ a^{p} }{ p }. \]
                \item[(4)] Suppose \(  a > 0  \) and \( b > 0  \). By the property of logarithms, we see that
                    \[  ab = e^{\ln a } e^{\ln b} = e^{\ln a + \ln b}. \]
                Also, we see that 
                \[  \ln(a^{p}) = p \ln a \ \text{and} \ \ln(b^{q})=  q \ln b.  \]
                Now, observe that 
                \[  e^{\ln a + \ln b} = e^{\frac{ p }{ p } \ln a  + \frac{ q }{ q }  \ln b } = e^{\frac{ 1 }{ p }  \ln (a^{p}) + \frac{ 1 }{ q }  \ln(b^{q})}.    \]
                Notice that \( e^{t} \), when differentiated twice, is a strictly positive function. Thus, \( e^{t} \) is convex for all \( t \in \R  \) our knowledge of calculus. Thus, we can use Jensen's inequality to conclude that
                \[  e^{\frac{ 1 }{ p }  \ln (a^{p}) + \frac{ 1 }{ q }  \ln (b^{q})} \leq   \frac{ 1 }{ p } e^{  \ln (a^{p})} + \frac{ 1 }{ q }  e^{\ln (b^{q})}  = \frac{ 1 }{ p }  a^{p} + \frac{ 1 }{ q }  b^{q}. \]
            \item[(ii)] Let \( a = \Big(  \sum_{ i=1  }^{ n } | {x}_{i} |^{p} \Big)^{\frac{ 1 }{ p }} \) and \( b = \Big(  \sum_{ i=1  }^{ n } | {y}_{i} |^{q} \Big)^{\frac{ 1 }{ q } } \). Note that if \( a = 0  \) or \( b = 0  \), then both sides of the above inequality will be zero. Hence, it suffices to show the result when \( a \neq 0  \) and \( b \neq 0  \). For each \( i \in \{ 1, \dots, n \}  \) let \( {u}_{i} = \frac{ | {x}_{i} |  }{ a }  \) and \( {v}_{i} = \frac{ | {y}_{i} |  }{ b }  \). Using part (a), we can see that 
                \begin{align*}
                    \sum_{ i=1  }^{ n } \Big|  \Big(  \frac{ {x}_{i} }{ a }  \Big) \Big(  \frac{ {y}_{i} }{ b }  \Big) \Big| = \sum_{ i=1  }^{ n } {u}_{i} {v}_{i} &\leq \sum_{ i=1  }^{ n } \Big(  \frac{ {u}_{i}^{p} }{ p }  + \frac{ {v}_{i}^{q} }{ q }  \Big) \\
                                                                                                                                                                   &= \frac{ 1 }{ p } \sum_{ i=1  }^{ n } \frac{ | {x}_{i} |^{p} }{  a^{p} }  + \frac{ 1 }{ q }  \sum_{ i=1  }^{ n } \frac{ | {y}_{i} |^{q} }{ b^{q} } \\
                                                                                                                                                                   &= \frac{ 1 }{ p a^{p} }  \sum_{ i=1  }^{ n } | {x}_{i} |^{p} + \frac{ 1 }{ q b^{q} }  \sum_{ i=1  }^{ n } | {y}_{i} |^{q} \\
                                                                                                                                                                   &= \Big(  \frac{ 1 }{ p a^{p} }   \Big) a^{p} + \Big(  \frac{ 1  }{ q b^{q} }  \Big) b^{q} \\
                                                                                                                                                                   &= \frac{ 1 }{ p }  + \frac{ 1 }{ q }  = 1.
                \end{align*}
            \end{enumerate}
            Now, we have 
    \begin{align*}  \sum_{ i=1  }^{ n } \Big| \Big(  \frac{ {x}_{i} }{ a }  \Big) \Big(  \frac{ {y}_{i} }{ b }  \Big) \Big| \leq 1 &\Longrightarrow \sum_{ i=1  }^{ n }  \Big|  \frac{ {x}_{i} }{ a }  \Big| \Big|  \frac{ {y}_{i} }{ b }  \Big| \leq 1  \\
        &\Longrightarrow \frac{ 1 }{ ab } \sum_{ i=1  }^{ n } | {x}_{i} |  | {y}_{i} | \\
        &\Longrightarrow \sum_{ i=1  }^{ n } | {x}_{i} {y}_{i}  | \leq ab = \Big(  \sum_{ i=1  }^{ n } | {x}_{i} |^{p} \Big)^{\frac{ 1 }{ p } } \Big(  \sum_{ i=1  }^{ n } | {y}_{i} |^{q} \Big)^{\frac{ 1 }{ q }}.
    \end{align*}
    Thus, we have 
    \[  \sum_{ i=1  }^{ n } | {x}_{i} {y}_{i} | \leq \|\vec{ x } \|_{p} \|\vec{ y } \|_{q}. \]
    \item[(iii)] Suppose \( \vec{ x } , \vec{ y }  \in \R^{n} \). Notice that if \( \sum_{ i=1  }^{ n } | {x}_{i} + {y}_{i} |^{p} = 0  \), then Minkowski's inequality immediately follows. Hence, it suffices to show the result when \( \sum_{ i=1  }^{ n } | {x}_{i} + {y}_{i} |^{p} \neq 0  \). Note that if \( p = 1  \), then Minkowski's inequality immediately follows via applying the triangle inequality (on the standard metric \( | \cdot |  \) on \( \R  \)) and distributing the summation. Thus, suppose further that \( p > 1  \). Let \( q  \) be such that \( \frac{ 1 }{ p }  + \frac{ 1 }{ q }  = 1  \) (where \( p > 1  \) and \( q  \) is a positive real number). Hence, we have that 
        \begin{align*}
            \sum_{ i=1  }^{ n } | {x}_{i} + {y}_{i} |^{p} &= \sum_{ i=1  }^{ n } | {x}_{i} + {y}_{i} |  | {x}_{i} + {y}_{i} |^{p-1} \\
                                                          &\leq \sum_{ i=1  }^{ n } (| {x}_{i}  |  + | {y}_{i} | )| {x}_{i} + {y}_{i} |^{p-1} \\
                                                          &= \sum_{ i=1  }^{ n } | {x}_{i} | {x}_{i} + {y}_{i}  |^{p-1} + \sum_{ i=1  }^{ n } | {y}_{i} |  | {x}_{i} + {y}_{i} |^{p-1} \\
                                                          &\leq \Big(  \sum_{ i=1  }^{ n } | {x}_{i} |^{p} \Big)^{\frac{ 1 }{ p }} \Big(  \sum_{ i=1  }^{ n } | {x}_{i} + {y}_{i} |^{(p-1)q} \Big)^{\frac{ 1 }{ q } }  + \Big(  \sum_{ i=1  }^{ n } | {y}_{i} |^{p} \Big)^{\frac{ 1 }{ p } } \Big(  \sum_{ i=1  }^{ n } | {x}_{i} + {y}_{i} |^{(p-1)q} \Big)^{\frac{ 1 }{ q } } \tag{Holder's Inequality} \\
                                                          &= \Bigg( \Big( \sum_{ i=1  }^{ n } | {x}_{i} |^{p} \Big)^{\frac{ 1 }{ p }} + \Big(  \sum_{ i=1  }^{ n } | {y}_{i} |^{p} \Big)^{\frac{ 1 }{ p } }   \Bigg) \Big(  \sum_{ i=1  }^{ n } | {x}_{i} + {y}_{i} |^{p} \Big)^{\frac{ 1 }{ q } }.
        \end{align*}
\end{enumerate}
    Dividing \( \Big(  \sum_{ i=1  }^{ n } | {x}_{i} + {y}_{i} |^{p} \Big)^{\frac{ 1 }{ q } } \) by both sides, we see that 
    \[  \frac{ \sum_{ i=1  }^{ n } | {x}_{i} + {y}_{i} |^{p} }{ \Big(  \sum_{ i=1  }^{ n } | {x}_{i} + {y}_{i} |^{p} \Big)^{\frac{ 1 }{ q } } } \leq  \Bigg( \Big(  \sum_{ i=1  }^{ n } | {x}_{i} |^{p} \Big)^{\frac{ 1 }{ p }} + \Big(  \sum_{ i=1  }^{ n } | {y}_{i} |^{p} \Big)^{\frac{ 1 }{ p }}     \Bigg) \]
    which can be re-written to 
    \[  \Big(  \sum_{ i=1  }^{ n } | {x}_{i} + {y}_{i} |^{p} \Big)^{1 - \frac{ 1 }{ q } } \leq  \Big(  \sum_{ i=1  }^{ n } | {x}_{i} |^{p} \Big)^{\frac{ 1 }{ p }} + \Big(  \sum_{ i=1  }^{ n } | {y}_{i} |^{p} \Big)^{\frac{ 1 }{ p }}.   \]
    Now, observe that 
    \[  \frac{ 1 }{ p }  + \frac{ 1 }{ q }  = 1 \Longrightarrow \frac{ 1 }{ q }  = 1 - \frac{ 1 }{ p }. \]
    If we set 
    \[  A =  \sum_{ i=1  }^{ n } | {x}_{i} + {y}_{i} |^{p},   \]
    then we see that 
    \[  A^{1 - \frac{ 1 }{ q } } = A^{1 - \Big(  1 - \frac{ 1 }{ p }  \Big)} = A^{\frac{ 1 }{ p }}. \]
    Thus, we see that 
            \[  \Big( \sum_{ i=1  }^{ n } | {x}_{i} +  {y}_{i} | \Big)^{\frac{ 1 }{ p } }  \leq \Big(  \sum_{ i=1  }^{ n } | {x}_{i} |^{p} \Big)^{\frac{ 1 }{ p } } +  \Big(  \sum_{ i=1  }^{ n } | {y}_{i} |^{p} \Big)^{\frac{ 1 }{ p } }  \]
            and so we conclude that
            \[  \|\vec{ x }  + \vec{ y } \|_{p} \leq \|\vec{ x } \|_{p} + \|\vec{ y } \|_{p}. \]
    \item[(iv)] Let \( X = \R^{n} \) and \( p > 1  \). We will show that 
        \[  {d}_{p}(\vec{ x } , \vec{ y } ) = \|\vec{ x }  - \vec{ y } \|_{p} \]
        is a metric on \( \R^{n} \).
        \item[(I)] Let \( \vec{ x } ,\vec{ y }  \in \R^{n}  \). Suppose \( {d}_{p}(\vec{ x }, \vec{ y } ) = 0  \). By definition of \( d(\vec{ x }, \vec{ y } )  \) and by property (2) of norms, we have 
            \begin{align*}
               {d}_{p}(\vec{ x } ,\vec{ y } ) = 0  &\Longrightarrow \|\vec{ x }  - \vec{ y } \| = 0  \\
                           &\Longrightarrow \vec{ x }  - \vec{ y }  = 0 \\
                           &\Longrightarrow \vec{ x }  = \vec{ y } .
            \end{align*}
            This shows property (ii).
        \item[(II)] Let \( \vec{ x } ,\vec{ y }  \in \R^{n}  \). Then by property (3) of norms, we see that 
            \[  {d}_{p}(\vec{ x } ,\vec{ y } ) = \|\vec{ x }  - \vec{ y }  \| = \| -(\vec{ y } -\vec{ x } )\| = | -1 | \| \vec{ y }  - \vec{ x }  \| = {d}_{p}(\vec{ y } ,\vec{ x } ).  \]
            Thus, property (iii) is satisfied.
        \item[(III)] Let \( \vec{ x } ,\vec{ y } ,\vec{ z }  \in V  \). Then by the triangle inequality property of norms, we see that 
            \begin{align*}
                {d}_{p}(\vec{ x } ,\vec{ y } ) = \|\vec{ x }  - \vec{ y } \|  &= \|(\vec{ x }  - \vec{ z } ) + (\vec{ z }  - \vec{ y } )\|  \\
                                    &\leq \|\vec{ x }  - \vec{ z }  \| + \| \vec{ z }  - \vec{ y } \| \\
                                    &= {d}_{p}(\vec{ x } ,\vec{ z } ) + {d}_{p}(\vec{ z } ,\vec{ y } ).
            \end{align*}
            Hence, we can conclude that \( {d}_{p} \) is a metric on \( \R^{n} \).
\end{proof}


\end{document}

