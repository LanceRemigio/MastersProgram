\subsection{Plan}


\subsection{Learning Objectives}

\begin{itemize}
    \item Basics of Metric Spaces
    \item Discuss topological aspects of metric spaces
\end{itemize}

\subsection{Metric Spaces}

On \( \R  \), we have the usual notion of distance between two real numbers \( x  \) and \( y  \) defined by the function \( d(x,y) = | x - y  |  \). We learned that this function \( d: \R \times \R \to \R  \) enjoys certain properties:
\begin{enumerate}
    \item[(i)] \( d(x,y) = 0  \) if and only if \( x = y  \); (Nondegeneracy)
    \item[(ii)] \( d(x,y) = d(y,x) \) for all \( x,y \in \R  \); (Symmetricity)
    \item[(iii)] \( d(x,y) \leq d(x,z) + d(z,y) \) for all \( x,y,z \in \R  \). (Triangle Inequality)
\end{enumerate}

We would like to extend this idea and define a notion of distance in a general situation by using these properties.

\begin{definition}[Metric]
    Let \( X  \) be a non-empty set. A metric \( d  \) on \( X  \) is a function \( d: X \times X \to \R  \) such that 
    \begin{enumerate}
        \item[(i)] \( d(x,y) = 0  \) if and only if \( x \) and \( y  \) are equal;
        \item[(ii)] \( d(x,y) = d(y,x) \) for all \( x,y \in X  \);
        \item[(iii)] \( d(x,z) \leq d(x,y) + d(y,z) \) for all \( x,y, z \in X  \).
    \end{enumerate}
\end{definition}

\begin{eg}
   Assume that \( d  \) is a metric on \( X  \). Show that \( d(x,y) \geq 0  \) for all \( x,y \in X  \). 
\end{eg}

\begin{eg}
   Let \( X  = \{ a,b \}   \). Is it possible to define a metric on \( X  \)? \textbf{Yes} with the discrete metric.
\end{eg}

\begin{definition}[Metric Spaces]
    A \textbf{metric space} is a pair \( (X,d)  \) where \( X  \) is a non-empty set and \( d  \) is a metric on \( X  \).
\end{definition}

\begin{eg}
    \begin{enumerate}
        \item[(i)] Let \( X  \) be a non-empty set. Define \( d: X \times X \to \R  \) by 
            \[  d(x,y) = 
            \begin{cases}
                1 &\text{if} \ x \neq y  \\
                0 &\text{if} \ x = y 
            \end{cases}. \]
            Then \( d  \) is a metric on \( X  \) (called the \textbf{discrete metric}) and \( (X,d) \) is a metric space.
        \item[(ii)] 
    \end{enumerate}
\end{eg}




