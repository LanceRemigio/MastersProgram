\documentclass[a4paper]{article}
\usepackage[utf8]{inputenc}
\usepackage[T1]{fontenc}
% \usepackage{fourier}
\usepackage{textcomp}
\usepackage{hyperref}
\usepackage[english]{babel}
\usepackage{url}
% \usepackage{hyperref}
% \hypersetup{
%     colorlinks,
%     linkcolor={black},
%     citecolor={black},
%     urlcolor={blue!80!black}
% }
\usepackage{graphicx} \usepackage{float}
\usepackage{booktabs}
\usepackage{enumitem}
% \usepackage{parskip}
% \usepackage{parskip}
\usepackage{emptypage}
\usepackage{subcaption}
\usepackage{multicol}
\usepackage[usenames,dvipsnames]{xcolor}
\usepackage{ocgx}
% \usepackage{cmbright}


\usepackage[margin=1in]{geometry}
\usepackage{amsmath, amsfonts, mathtools, amsthm, amssymb}
\usepackage{thmtools}
\usepackage{mathrsfs}
\usepackage{cancel}
\usepackage{bm}
\newcommand\N{\ensuremath{\mathbb{N}}}
\newcommand\R{\ensuremath{\mathbb{R}}}
\newcommand\Z{\ensuremath{\mathbb{Z}}}
\renewcommand\O{\ensuremath{\emptyset}}
\newcommand\Q{\ensuremath{\mathbb{Q}}}
\newcommand\C{\ensuremath{\mathbb{C}}}
\newcommand\F{\ensuremath{\mathbb{F}}}
% \newcommand\P{\ensuremath{\mathbb{P}}}
\DeclareMathOperator{\sgn}{sgn}
\DeclareMathOperator{\diam}{diam}
\DeclareMathOperator{\LO}{LO}
\DeclareMathOperator{\UP}{UP}
\DeclareMathOperator{\card}{card}
\DeclareMathOperator{\Arg}{Arg}
\DeclareMathOperator{\Dom}{Dom}
\DeclareMathOperator{\Log}{Log}
\DeclareMathOperator{\dist}{dist}
% \DeclareMathOperator{\span}{span}
\usepackage{systeme}
\let\svlim\lim\def\lim{\svlim\limits}
\renewcommand\implies\Longrightarrow
\let\impliedby\Longleftarrow
\let\iff\Longleftrightarrow
\let\epsilon\varepsilon
\usepackage{stmaryrd} % for \lightning
\newcommand\contra{\scalebox{1.1}{$\lightning$}}
% \let\phi\varphi
\renewcommand\qedsymbol{$\blacksquare$}

% correct
\definecolor{correct}{HTML}{009900}
\newcommand\correct[2]{\ensuremath{\:}{\color{red}{#1}}\ensuremath{\to }{\color{correct}{#2}}\ensuremath{\:}}
\newcommand\green[1]{{\color{correct}{#1}}}

% horizontal rule
\newcommand\hr{
    \noindent\rule[0.5ex]{\linewidth}{0.5pt}
}

% hide parts
\newcommand\hide[1]{}

% si unitx
\usepackage{siunitx}
\sisetup{locale = FR}
% \renewcommand\vec[1]{\mathbf{#1}}
\newcommand\mat[1]{\mathbf{#1}}

% tikz
\usepackage{tikz}
\usepackage{tikz-cd}
\usetikzlibrary{intersections, angles, quotes, calc, positioning}
\usetikzlibrary{arrows.meta}
\usepackage{pgfplots}
\pgfplotsset{compat=1.13}

\tikzset{
    force/.style={thick, {Circle[length=2pt]}-stealth, shorten <=-1pt}
}

% theorems
\makeatother
\usepackage{thmtools}
\usepackage[framemethod=TikZ]{mdframed}
\mdfsetup{skipabove=1em,skipbelow=1em}

\theoremstyle{definition}

\declaretheoremstyle[
    headfont=\bfseries\sffamily\color{ForestGreen!70!black}, bodyfont=\normalfont,
    mdframed={
        linewidth=1pt,
        rightline=false, topline=false, bottomline=false,
        linecolor=ForestGreen, backgroundcolor=ForestGreen!5,
    }
]{thmgreenbox}

\declaretheoremstyle[
    headfont=\bfseries\sffamily\color{NavyBlue!70!black}, bodyfont=\normalfont,
    mdframed={
        linewidth=1pt,
        rightline=false, topline=false, bottomline=false,
        linecolor=NavyBlue, backgroundcolor=NavyBlue!5,
    }
]{thmbluebox}

\declaretheoremstyle[
    headfont=\bfseries\sffamily\color{NavyBlue!70!black}, bodyfont=\normalfont,
    mdframed={
        linewidth=1pt,
        rightline=false, topline=false, bottomline=false,
        linecolor=NavyBlue
    }
]{thmblueline}

\declaretheoremstyle[
    headfont=\bfseries\sffamily, bodyfont=\normalfont,
    numbered = no,
    mdframed={
        rightline=true, topline=true, bottomline=true,
    }
]{thmbox}

\declaretheoremstyle[
    headfont=\bfseries\sffamily, bodyfont=\normalfont,
    numbered=no,
    % mdframed={
    %     rightline=true, topline=false, bottomline=true,
    % },
    qed=\qedsymbol
]{thmproofbox}

\declaretheoremstyle[
    headfont=\bfseries\sffamily\color{NavyBlue!70!black}, bodyfont=\normalfont,
    numbered=no,
    mdframed={
        rightline=false, topline=false, bottomline=false,
        linecolor=NavyBlue, backgroundcolor=NavyBlue!1,
    },
]{thmexplanationbox}

\declaretheorem[
    style=thmbox, 
    % numberwithin = section,
    numbered = no,
    name=Definition
    ]{definition}

\declaretheorem[
    style=thmbox, 
    name=Example,
    ]{eg}

\declaretheorem[
    style=thmbox, 
    % numberwithin = section,
    name=Proposition]{prop}

\declaretheorem[
    style = thmbox,
    numbered=yes,
    name =Problem
    ]{problem}

\declaretheorem[style=thmbox, name=Theorem]{theorem}
\declaretheorem[style=thmbox, name=Lemma]{lemma}
\declaretheorem[style=thmbox, name=Corollary]{corollary}

\declaretheorem[style=thmproofbox, name=Proof]{replacementproof}

\declaretheorem[style=thmproofbox, 
                name = Solution
                ]{replacementsolution}

\renewenvironment{proof}[1][\proofname]{\vspace{-1pt}\begin{replacementproof}}{\end{replacementproof}}

\newenvironment{solution}
    {
        \vspace{-1pt}\begin{replacementsolution}
    }
    { 
            \end{replacementsolution}
    }

\declaretheorem[style=thmexplanationbox, name=Proof]{tmpexplanation}
\newenvironment{explanation}[1][]{\vspace{-10pt}\begin{tmpexplanation}}{\end{tmpexplanation}}

\declaretheorem[style=thmbox, numbered=no, name=Remark]{remark}
\declaretheorem[style=thmbox, numbered=no, name=Note]{note}

\newtheorem*{uovt}{UOVT}
\newtheorem*{notation}{Notation}
\newtheorem*{previouslyseen}{As previously seen}
% \newtheorem*{problem}{Problem}
\newtheorem*{observe}{Observe}
\newtheorem*{property}{Property}
\newtheorem*{intuition}{Intuition}

\usepackage{etoolbox}
\AtEndEnvironment{vb}{\null\hfill$\diamond$}%
\AtEndEnvironment{intermezzo}{\null\hfill$\diamond$}%
% \AtEndEnvironment{opmerking}{\null\hfill$\diamond$}%

% http://tex.stackexchange.com/questions/22119/how-can-i-change-the-spacing-before-theorems-with-amsthm
\makeatletter
% \def\thm@space@setup{%
%   \thm@preskip=\parskip \thm@postskip=0pt
% }
\newcommand{\oefening}[1]{%
    \def\@oefening{#1}%
    \subsection*{Oefening #1}
}

\newcommand{\suboefening}[1]{%
    \subsubsection*{Oefening \@oefening.#1}
}

\newcommand{\exercise}[1]{%
    \def\@exercise{#1}%
    \subsection*{Exercise #1}
}

\newcommand{\subexercise}[1]{%
    \subsubsection*{Exercise \@exercise.#1}
}


\usepackage{xifthen}

\def\testdateparts#1{\dateparts#1\relax}
\def\dateparts#1 #2 #3 #4 #5\relax{
    \marginpar{\small\textsf{\mbox{#1 #2 #3 #5}}}
}

\def\@lesson{}%
\newcommand{\lesson}[3]{
    \ifthenelse{\isempty{#3}}{%
        \def\@lesson{Lecture #1}%
    }{%
        \def\@lesson{Lecture #1: #3}%
    }%
    \subsection*{\@lesson}
    \testdateparts{#2}
}

% \renewcommand\date[1]{\marginpar{#1}}


% fancy headers
\usepackage{fancyhdr}
\pagestyle{fancy}

\makeatother

% notes
\usepackage{todonotes}
\usepackage{tcolorbox}

\tcbuselibrary{breakable}
\newenvironment{verbetering}{\begin{tcolorbox}[
    arc=0mm,
    colback=white,
    colframe=green!60!black,
    title=Opmerking,
    fonttitle=\sffamily,
    breakable
]}{\end{tcolorbox}}

\newenvironment{noot}[1]{\begin{tcolorbox}[
    arc=0mm,
    colback=white,
    colframe=white!60!black,
    title=#1,
    fonttitle=\sffamily,
    breakable
]}{\end{tcolorbox}}

% figure support
\usepackage{import}
\usepackage{xifthen}
\pdfminorversion=7
\usepackage{pdfpages}
\usepackage{transparent}
\newcommand{\incfig}[1]{%
    \def\svgwidth{\columnwidth}
    \import{./figures/}{#1.pdf_tex}
}

% %http://tex.stackexchange.com/questions/76273/multiple-pdfs-with-page-group-included-in-a-single-page-warning
\pdfsuppresswarningpagegroup=1


\title{Math 230B: Homework 1}
\author{Lance Remigio}

\begin{document}
\maketitle

\begin{problem}
    Let \( m,n \in \N \). Consider the function \( f: \R \to \R  \) defined by
    \[  f(x) = 
    \begin{cases}
        x^{m} \sin \frac{ 1 }{ x^{n} } &\text{if} \ x \neq 0 \\
        0 &\text{if} \ x = 0 
    \end{cases}. \]
    \begin{enumerate}
        \item[(i)] Prove that \( f  \) is differentiable at all \( x \neq 0  \).
        \item[(ii)] Prove that \( m > 1   \), then \( f  \) is differentiable at \( 0  \).
        \item[(iii)] Prove that if \( m > 1 + n  \), then \( f'  \) is continuous at \( 0  \).
        \item[(iv)] Prove that if \(  m > 2 + n  \), then \( f'  \) is differentiable on \( \R  \).
        \item[(v)] Prove that if \( m > 2 + 2n  \), then \( f''  \) is continuous at \( 0  \).
        \item[(vi)] Prove that if \( 2 + n < m \leq 2 + 2n \), then \( f''  \) is not continuous at \( 0  \).
    \end{enumerate}
\end{problem}
\begin{proof}
\begin{enumerate}
    \item[(i)] Suppose \( x \neq 0  \). Notice that
        \begin{enumerate}
            \item[(1)] \( x^{m} \) is a polynomial that is differentiable for any \( x \neq 0  \).
            \item[(2)] \( \frac{ 1 }{ x^{n} }   \) is a rational function which is differentiable for any \( x \neq 0  \).
            \item[(3)] \( \sin x  \) is a trigonometric function which is differentiable for any \( x \neq 0  \).
        \end{enumerate}
        By (1), (2), and (3), we conclude via a combination of the Chain Rule and Algebraic Differentiability Theorem that \( f  \) is differentiable for all \( x \neq 0  \).
    \item[(ii)] Suppose \( m > 1  \). Our goal is to show that \( f  \) is differentiable at \( 0  \); that is, we will show that 
        \[  \lim_{ x \to 0 }  \frac{ f(x) - f(0) }{ x - 0 }  = \lim_{ x \to 0 }  \frac{ x^{m} \sin \frac{ 1 }{ x^{n} }   }{ x  } = \lim_{ x \to 0 }  x^{m-1} \sin \frac{ 1 }{ x^{n} }. \tag{*}  \]
        Since \( | \sin \frac{ 1 }{ x^{n} }  | \leq 1  \), we can see that 
        \[  0 \leq | x^{m-1} \sin \frac{ 1 }{ x^{n} }  | =  | x^{m-1} |  \Big| \sin \frac{ 1 }{ x^{n} }  \Big|   \leq | x^{m-1}  |. \tag{**}   \]
        Since \( m - 1 > 0  \) and \( x^{m-1} \) is a polynomial that is continuous everywhere on \(\R  \), we have \( \lim_{ x \to 0 }  x^{m-1} = 0  \). As a consequence, we also have \( \lim_{ x \to 0 } | x^{m-1} | = 0  \). By applying the Squeeze Theorem for functions to the inequality in (**), we conclude that 
        \[  \lim_{ x \to 0  }  | x^{m-1} \sin \frac{ 1 }{ x^{n} } |  = 0 \iff \lim_{ x \to 0 }  x^{m-1} \sin \frac{ 1 }{ x^{n} } = 0.  \]
        But this implies that the limit in (*) exists and so \( f  \) is differentiable at \(  0  \).
    \item[(iii)] Computing \( f' \), we have 
        \[  f'(x) = 
        \begin{cases}
            m x^{m-1} \sin \frac{ 1 }{ x^{n} }  - x^{m-n-1} \cos \frac{ 1 }{ x^{n} } &\text{if} \ x \neq 0 \\
            0 &\text{if} \ x = 0 
        \end{cases}. \]
        We wills show that both 
        \[  m x^{m-1} \sin \frac{ 1 }{ x^{n} } \ \text{and} \ x^{m-n-1} \cos \frac{ 1 }{ x^{n} }  \]
        approach \( 0 \) as \( x \to 0  \). Since \( | \sin x  | \leq 1  \) for all \( x \in \R  \), we have  
        \[   0 \leq | m x^{m-1} \sin \frac{ 1 }{ x^{n} }  | \leq | m x^{m-1} |. \tag{I}  \]
        Since \( m - 1 > n > 0  \) and \( x^{m-1} \) is a polynomial which is continuous at \( 0  \), the Algebraic Continuity Theorem implies that 
        \[  \lim_{ x \to 0 }  m x^{m-1} = 0 \iff \lim_{ x \to 0 }  | m x^{m-1} |  = 0.  \]
        Using the Squeeze Theorem on (I), we conclude that 
        \[  \lim_{ x \to 0 }  | m x^{m-1} \sin \frac{ 1 }{ x^{n} }  | = 0 \iff \lim_{ x \to 0 }  m x^{m-1} \sin \frac{ 1 }{ x^{n} } = 0.  \]
        Using a similar argument, we can prove that 
        \[  x^{m-n-1} \cos \frac{ 1 }{ x^{n} } \]
        is continuous at \( 0  \).
        Indeed, we have 
        \[  0 \leq | x^{m-n-1} \cos \frac{ 1 }{ x^{n} }  | \leq | x^{m-n-1} |. \tag{\( | \cos x  | \leq 1  \ \forall x \in \R  \)}  \]
        Notice that \( m - n - 1 > 0  \) and that \( x^{m-n-1}  \) is a polynomial which is continuous everywhere on \( \R  \), we have \( \lim_{ x \to 0 }  x^{m-n-1}  = 0 \). Hence, 
        \[  \lim_{ x \to 0 }  | x^{m-n-1}  |  = 0.  \]
        Applying the Squeeze Theorem, we have 
        \[  \lim_{ x \to 0 } | x^{m-n-1} \cos \frac{ 1 }{ x^{n} }  | = 0 \iff \lim_{ x \to 0 } x^{m-n-1} \cos \frac{ 1 }{ x^{n} } = 0.  \]
        Using the Algebraic Limit theorem for functions, we can conclude that as \( x \to 0  \) 
        \[  f'(x) = m x^{m-1} \sin \frac{ 1 }{ x^{n} }  - x^{m-n-1} \cos \frac{ 1 }{ x^{n} } \to 0  = f'(0) \]
        and so \( f'(x) \) is at continuous at \( 0 \).
    \item[(iv)] Our goal is to show that \( f' \) is differentiable on \( \R  \). Let \( c \in \R  \). Suppose \( c = 0  \). We will show that the limit 
        \begin{align*}
            \lim_{ x \to 0 } \frac{ f'(x) - f'(0) }{  x - 0  }  &= \lim_{ x \to 0 }  \frac{ m x^{m-1} \sin \frac{ 1 }{ x^{n} }  - x^{m-n-1} \cos \frac{ 1 }{ x^{n} }  }{ x  }  \\ 
                                                                &= \lim_{ x \to 0 } \Big[ m x^{m-2} \sin \frac{ 1 }{ x^{n} }  - x^{m-n-2} \cos \frac{ 1 }{ x^{n} } \Big]     \\
                                                                &= 0.
    \end{align*}
    To prove this, we will use the same approach we used in part (iii); that is, we will show that each corresponding function of the above limit exists and equals \( 0  \). For te first function, notice that 
        \[   0 \leq | m x^{m-2} \sin \frac{ 1 }{ x^{n} }  | \leq | m x^{m-2} |. \tag{I}  \]
        Since \( x^{m-2} \) (note that \( m - 2 > 0  \) by assumption) is a polynomial that continuous everywhere, we have \( \lim_{ x \to 0 }  x^{m-2} = 0  \). This holds if and only if \( \lim_{ x \to 0 }  | x^{m-2} |  = 0  \). By applying the Squeeze Theorem to the inequality in (I), we have 
        \[  \lim_{ x \to 0 } | m x^{m-2} \sin \frac{ 1 }{ x^{n} }  | = 0  \iff \lim_{ x \to 0 }  m x^{m-2} \sin \frac{ 1 }{ x^{n} } = 0.  \]
        Now, we will show that 
        \[  \lim_{ x \to 0 }  x^{m-n-2} \cos \frac{ 1 }{ x^{n} } = 0. \]
        Again, with a similar argument used in part (iii), we have 
        \[  0 \leq | x^{m-n-2} \cos \frac{ 1 }{ x^{n} }  | \leq | x^{m-n-2} |. \tag{\( | \cos x  | \leq 1  \ \forall x \in \R  \)}  \]
        Since \( x^{m-n-2} \) is a polynomial which is continuous everywhere on \( \R  \), we have \( \lim_{ x \to 0 }  x^{m-n-2} = 0  \). This holds if and only if
        \[  \lim_{ x \to 0 }  | x^{m-n-2} |  = 0.  \]
        Using the Squeeze Theorem on the inequality above, we have 
        \[  \lim_{ x \to 0 }  | x^{m-n-2}  \cos \frac{ 1 }{ x^{n} }  |  = 0 \iff \lim_{ x \to 0 }  x^{m-n-2} \cos \frac{ 1 }{ x^{n} }  = 0.   \]
        Using the Algebraic limit theorem for functions, we can conclude that \( f'  \) is differentiable for \( c = 0  \). Suppose \( c \neq 0  \). From part (iii), we have         
        \[  f'(x) = m x^{m-1} \sin \frac{  1 }{ x^{n} } - x^{m-n-1} \cos \frac{ 1 }{ x^{n} }. \]
        Note the following:
        \begin{enumerate}
            \item[(1)] \( x^{m-1} \) and \( x^{m-n-1} \) are polynomials which are differentiable everywhere on \( \R  \).
            \item[(2)] \( \sin x  \) and \( \cos x  \) are trigonometric functions which are differentiable everywhere on \( \R  \).
            \item[(3)] \( \frac{ 1 }{ x^{n} }  \) is a rational function that is differentiable for every \( x \neq 0  \) in \( \R  \).
            \item[(4)] \( \sin (\frac{ 1 }{ x^{n} } ) \) is differentiable for every \( x \neq 0  \) in \( \R  \) by the chain rule.
        \end{enumerate}
        From (1), (2), (3), and (4) we can use the Algebraic differentiability theorem to conclude that \( f'(x) \) is indeed differentiable.
    \item[(v)] Using a combination of the product rule and chain rule, we have    
        \[  f''(x) = 
        \begin{cases}
            [m(m-1)x^{m-2} - n x^{m-2n-2}] \sin \frac{ 1 }{ x^{n} } \\
            -[mn x^{m-n-2} + n (m - n -1) x^{m-n-2} \cos \frac{ 1 }{ x^{n} } &\text{if} \ x \neq 0 ] \\
            0 &\text{if} \ x = 0 
        \end{cases} \]
        By assumption, we have that 
        \[  m > 2 + 2n \implies m - 2n - 2 > 0 \]
        and similarly
        \[  m - n - 2 > n > 0 \ \text{and} \ m - 2 > 0.  \]
        Hence, we can see that \( x^{m-2} \) and \( x^{m-2n-2}  \) are polynomials which are continuous everywhere on \( \R  \). In particular, these two polynomials are continuous at \( 0  \). Since \( | \sin x  | \leq 1  \) for all \( x \in \R  \), we have
        \[  0 \leq \Big| [m(m-1) x^{m-2} - n x^{m-2n-2}] \sin \frac{ 1 }{ x^{n} }  \Big| \leq | m(m-1) x^{m-2} - n x^{m-2n-2} |. \tag{*}    \]
        Since \( x^{m-2} \) and \( x^{m-2n-2}  \) are both continuous at \( 0  \), the Algebraic Continuity Theorem implies that 
        \[  \lim_{ x \to 0 }  m(m-1) x^{m-2} - n x^{m-2n-2} = 0  \]
        which holds if and only if 
        \[ \lim_{ x \to 0 } | m(m-1) x^{m-2} - n x^{m-2n-2} | = 0.    \]
        Then using the Squeeze Theorem on (*), we can conclude that 
        \[  \lim_{ x \to 0 } \Big| [m(m-1)x^{m-2} - n x^{m-2n-2}] \sin \frac{ 1 }{ x^{n} }  \Big| = 0   \]
        which holds if and only if
        \[ \lim_{ x \to 0 } [m(m-1) x^{m-2} - n x^{m-2n-2}] \sin \frac{ 1 }{ x^{n} } = 0. \tag{1}  \]
        Analogously, we can prove that  
        \[  \lim_{ x \to 0 }  [mn x^{m-n-2} + n (m-n-1) x^{m-n-2}] \cos \frac{ 1 }{ x^{n} }  = 0. \tag{2}  \]
        Using the Algebraic Limit Theorem for functions, we can conclude that 
        \[  \lim_{ x \to 0 }  f''(x) =  f''(0).  \] 
    \item[(vi)] Computing \( f'' \), we have 
        \[  f''(x) = 
        \begin{cases}
            m(m-1)x^{m-2} \sin \Big(  \frac{ 1 }{ x^{n} }  \Big)  + n^{2} x^{m-2n-2} \sin \Big(  \frac{ 1 }{ x^{n} } \Big) \\ - n (2m - n - 2) x^{m-n-2} \cos \Big(  \frac{ 1 }{ x^{n} }  \Big)  &\text{if} \ x \neq 0  \\
            0 &\text{if} \ x = 0.
        \end{cases} \]
        Define \( {a}_{k} = \frac{ 1 }{ \sqrt[n]{ 2 \pi k  + \frac{ \pi }{ 2 }  }  }  \). We can see immediately that \( {a}_{k } \to 0  \) and that  
        \begin{align*}
            \sin \frac{ 1 }{ ({a}_{k})^{n} } &= \sin \Big(2 \pi k  + \frac{ \pi }{ 2 }  \Big) = 1,  \\ 
            \cos \frac{ 1 }{ ({a}_{k})^{n} }  &= \cos \Big(  2 \pi k  + \frac{ \pi }{ 2 }  \Big) = 0 
        \end{align*}
        Now, for all \( k  \geq 1  \), we can see that 
        \begin{align*}
            \lim_{ k  \to  \infty  }  f''({a}_{k}) &= \lim_{ k  \to  \infty   }  \frac{ 1  }{ ({a}_{k})^{(2 + 2n) - m} }.
    \end{align*}
    Note that if \( m = 2 + 2n \), we just have \( \lim_{ k  \to + \infty  } f''({a}_{k}) = - n^{2}  \). Otherwise, \( \lim_{ k  \to  + \infty  } \frac{ 1 }{ ({a}_{k})^{(2+2n) - m} } = \infty  \). That is,
    \[  \lim_{ k  \to \infty  }  f''({a}_{k}) \neq 0 = f''(0). \]
    So, we conclude that \( f'' \) is not continuous at \( 0  \).
\end{enumerate}
\end{proof}
\begin{problem}
    Let \( f: \R \to \R  \) be defined as follows:
    \[  f(t) = 
    \begin{cases}
        e^{-\frac{ 1 }{ t } } &\text{if} \ t < 0 \\
        0 &\text{if} \ t \leq 0 
    \end{cases}. \]
    Prove that \( f  \) is infinitely differentiable at \( 0  \) with \( f^{(n)}(0) = 0  \) for all \( n \in \N \).
\end{problem}
\begin{proof}
Our goal is to show that \( f  \) is infinitely differentiable at \( 0  \). We will consider two cases for this. Note that for \( t \leq 0  \), one can immediately see that via induction that \( f^{(n)}(0) = 0  \) for all \( n \in \N \). Thus, it remains to be shown that if \( t > 0  \), then \( f^{(n)}(0) = 0  \) for all \( n \in \N \). First, we will show that for all \( n \in \N \)
\[  f^{(n)}(t) = e^{-1/t} \frac{ {P}_{n}(t) }{ t^{2n} }.   \]
Using a combination of the chain rule and product rule, we have that 
\[ f'(t) = \frac{ e^{-1/t} }{ t^{2} }. \]
Suppose that the result in (*) holds for \( n = k  \). We will show that it also holds for \( n = k +1 \) case. Indeed, we see that 
\begin{align*}
    f^{(k+1)}(t) = \frac{ d }{ dt } [f^{(k)}(t)] &= \frac{ d }{ dt } \Big[ {P}_{k}(t) \cdot \frac{ e^{-1/t} }{ t^{2k+2} } \Big]  \\
    \implies f^{(k+1)}(t) &= \frac{ d }{ dt }  [{P}_{k}(t)] \frac{ e^{-1/t} }{ t^{2k} }  +  t^{2} {P}_{k}(t) \frac{ e^{-1/t} }{ t^{2k} } - 2k {P}_{k}(t) \frac{ e^{-1/t} }{ t^{2k+1} }  \\
                          &= \underbrace{\Big[ \frac{ 1 }{ t } \frac{ d }{ dt } [{P}_{k}(t)] + t {P}_{k}(t) - 2k {P}_{k}(t)\Big]}_{\text{polynomial of degree at most}n+1} \frac{ e^{-1/t} }{ t^{2k+1} }  \\
                          &= {P}_{k+1}(t) \frac{ e^{-1/t} }{ t^{2k+1} }.
\end{align*} 
This implies that 
\[  f^{(k+1)}(t) = {P}_{k+1}(t) \frac{ e^{-1/t} }{ t^{2k+1} }. \]
Our goal is to show that \( f  \) is infinitely differentiable at \( 0  \) with \( f^{(n)} (0) = 0  \) for all \( n \in \N \). We proceed via induction on \( n \in \N \). We will start with proving that \( f  \) is differentiable once. Indeed,  
\begin{align*}
    \lim_{ t \to 0 }  \frac{ f(t) - f(0) }{ t  } &= \lim_{ t \to 0 } \frac{ e^{\frac{ -1 }{ t } } - 0  }{ t  }   \\
                                                 &= \lim_{ t \to 0 }  \frac{ 1 }{ t } e^{- \frac{ 1 }{ t } } \\
                                                 &= \lim_{ t \to 0  }  \frac{ 1 }{ t e^{1/t} } \\
                                                 &= 0.
\end{align*}
Hence, \( f'(0)  \) exists and so \( f'(0) = 0  \) by the above. Suppose for our induction hypothesis that \( f  \) is differentiable at \( 0  \) \( k = n  \) times. Our goal is to show that \( f  \) is differentiable \( n = k + 1  \) times. Note that 
where \( {P}_{n}(t) \) is a polynomial of at most degree \( n \).
Since \( f^{(k+1)}(t) \) exists, we have 
\begin{align*}
    f^{(k+1)}(0) &= \lim_{ t  \to 0 }  \frac{ f^{(k)}(t) - f^{(k)}(0)  }{t - 0  }  \\
            &= \lim_{ t \to 0 }  \frac{ 1 }{ t }  \cdot \frac{ e^{-1/t} {P}_{k}(t)  }{ t^{2k} } \\
            &= \lim_{ t \to 0  } \Big[ {P}_{k}(t) \cdot \frac{ e^{-1/t} }{ t^{2k+1} } \Big] \\
            &= \lim_{ t \to 0  }  {P}_{k}(t) \cdot \lim_{ t \to 0 }  \frac{ e^{-1/t} }{ t^{2k+1} }  \\
\end{align*}
Note that \( \lim_{ t \to 0  }  \frac{ e^{-1/t} }{ t^{2k+1} }  = 0  \) by L'Hopital's rule. Indeed, let's induct on \( n \in \N  \) to show that this is true. Let our base case be \( n = 1  \).
\[  \lim_{ t \to 0^{+} } \frac{ t^{-1} }{ e^{1/t} } \underbrace{=}_{\frac{ \infty  }{ \infty  }} \lim_{ t \to 0^{+} } \frac{ - t^{-2} }{ - t^{-2} e^{1/t} }  = \lim_{ t \to 0^{+} } \frac{ 1 }{ e^{1/t} }  = 0.    \]
Suppose the result holds for \( n = k   \); that is,  
\[  \lim_{ t \to 0^{+} }  \frac{ t^{-k} }{ e^{1/t} }  = 0.  \]
Our goal is to show that the result holds \( n = k + 1  \) (where \( k \geq 1  \)). Indeed, 
using L'Hopital's Rule, we have
\[  \lim_{ t \to 0^{+} } \frac{ t^{-(k+1)} }{ e^{1/t} } \underbrace{=}_{ \frac{ \infty  }{ \infty  }    } -(k+1)  \lim_{ t \to 0^{+} }  \frac{  t^{-k} }{ -e^{1/t}  } = 0. \] Hence, we conclude that 
\[  f^{(k+1)}(0) = 0. \]
\end{proof}

\begin{problem}
   Let \( f: I \to \R  \) where \( I \subseteq  \R   \) is an interval. Let \( c \in I  \). Recall that in class we proved that \( f  \) is differentiable at \( c  \) if and only if \( \displaystyle \lim_{ h \to 0 } \frac{ f(c+h)  - f(c) }{ h }  \) exists. Use this result to prove that \( f  \) is differentiable at \( c  \) if and only if  
   \[  \exists L \in \R \ \text{such that} \ \lim_{ h \to 0 } \frac{ f(c+h) - f(c) - Lh }{ h } = 0. \]
\end{problem}
\begin{proof}
Suppose that \( f  \) is differentiable at \( c  \). Then
\[  \lim_{ h \to 0 }  \frac{ f(c+h) - f(c) }{ h  } \ \text{exists}    \]
and so 
\[  \lim_{ h \to 0 }  \frac{ f(c+h) - f(c) }{ h  } = L  \]
for some \( L \in \R  \). Now, the right-hand side can be written in the following way:
\[  L = \lim_{ h \to 0 }  L = \lim_{ h \to 0 }   \frac{ L h }{ h }.  \]
Note that the quantity \( \frac{ h }{ h }  \) holds because of the \( \epsilon-\delta \) definition of the derivative. Now, we have 
\begin{align*}
    \lim_{ h \to 0 }  \frac{ f(c+h) - f(c) }{ h   } = L  &\implies \lim_{ h \to 0 }  \frac{ f(c+h) - f(c) }{ h  } = \lim_{ h \to 0 } \frac{L h }{ h }  \\
                                                         &\implies \lim_{ h \to 0 }  \frac{ f(c+h) - f(c) }{ h }  - \lim_{ h \to 0 } \frac{ Lh  }{ h } = 0.
\end{align*}
By the Algebraic Limit Theorem for functions, we conclude that 
\[  \lim_{ h \to 0 } \frac{ f(c+h) - f(c) - Lh }{ h } = 0. \]

(\( \Longrightarrow \)) Suppose that there exists an \( L \in \R  \) such that 
\[  \lim_{ h \to 0 } \frac{ f(c+h) - f(c) - Lh }{ h  } = 0.  \]
Our goal is to show that \( f  \) is differentiable at \( c  \); that is,  
\[  \lim_{ h \to 0 } \frac{ f(c+h) - f(c) }{ h }  \ \text{exists}.  \]
Then we have 
\begin{align*}
    \lim_{ h \to 0 }  \frac{ f(c+h) - f(c) }{ h } &= \lim_{ h \to 0} \Big[  \frac{ f(c+h) - f(c)  }{ h  }  - L + L \Big] \\
                                                &= \lim_{ h \to 0 } \Big[ \frac{ f(c+h) - f(c) - Lh }{ h }  +  L \Big] \\
                                                &= \lim_{ h \to 0 }  \frac{ f(c+h) - f(c) - Lh }{ h }  + \lim_{ h \to 0 }  L \tag{ALT for Functions} \\
                                                &= 0 + L \\
                                                &= L. 
\end{align*}
Hence, we can see that 
\[  \lim_{ h \to 0 }  \frac{ f(c+h)-  f(c) }{ h  } \ \text{exists} \]
and so we conclude that \( f  \) is differentiable at \( c  \).
\end{proof}

\begin{problem}
    Let \( g: A \to \R  \) where \( A  \) is a nonempty subset of \( \R  \). Suppose \( 0  \) is an interior point of \( A  \). Use the \( \epsilon-\delta \) definition of limit to prove that \( \lim_{ h \to 0 }  g(h) = L  \), then \( \lim_{ h \to 0 }  g(-h) = L  \). 
\end{problem}
\begin{proof}
Our goal is to show that \( \lim_{ h \to 0 }  g(-h) = L \); that is, for all \( \epsilon > 0  \), there exists \( \delta > 0  \) such that whenever \( 0 < | h  |  < \delta \), we have 
\[  | g(-h) - L  |  < \epsilon. \]
Let \( \epsilon > 0  \) be given. Since \( \lim_{ h \to 0 } g(h) = L  \), we can find a \( \hat{\delta} > 0  \) such that whenever \( 0 < | y  |  < \hat{\delta} \), 
\[  | g(y) - L  | < \epsilon.   \]
Since \( 0  \) is an interior point of \( A  \), there exists an \( \tilde{\delta} > 0  \) such that \( {N}_{\tilde{\delta}}(0) \subseteq  A  \); that is, we have
\[  (-\tilde{\delta}, \tilde{\delta}) \subseteq A. \]
Set \( \delta = \min \{ \hat{\delta}, \tilde{\delta} \} > 0   \). We claim that this \( \delta  \) is the same \( \delta \) we were looking for. Observe that \( | h  |  = | -(-h) | = | -h  |  \). Thus, if \( h \in A  \), then \( 0 < |  h  |  < \delta \) implies that  \( 0 < | -h  |  < \delta \). By setting \( y = -h  \), we can write
\[  | g(-h) - L  | < \epsilon \]
which is our desired result.
\end{proof}

\begin{problem}
    Let \( f: I \to \R  \) where \( I \subseteq \R   \) is an interval. Let \( c  \) be an interior point of \( I  \). Assume \( f  \) is differentiable at \( c  \). 
    \begin{enumerate}
        \item[(a)] Recall that \( f'(c) = \displaystyle \lim_{ h \to 0 }  \frac{ f(c+h) - f(c) }{ h }   \). Use this and the result of Exercise 4 to show that
            \[  f'(c) = \lim_{ h \to 0 }  \frac{ f(c) - f(c-h) }{ h }. \]
        \item[(b)] Use the result of (a) to prove that 
            \[  f'(c) = \lim_{ h \to 0 }  \frac{ f(c+h) - f(c-h) }{ 2h }. \]
    \end{enumerate}
\end{problem}
\begin{proof}
\begin{enumerate}
    \item[(a)] Define \( g: I \to \R  \) by
        \[  g(h)  = \frac{ f(c+h) - f(c) }{ h }.  \]
        Notice that 
        \[  g(-h) = \frac{ f(c -h) - f(c) }{ -h } = \frac{ f(c) - f(c-h) }{ h }. \]
        By exercise 4, we can see that  
        \[  f'(c) = \lim_{ h \to 0 }  g(h) = \lim_{ h \to 0 } g(-h)  \]
        Hence, we have 
        \[  f'(c) = \lim_{ h \to 0 }  \frac{ f(c) - f(c-h) }{ h }. \]
    \item[(b)] For \( h  \) sufficiently small, we have  
        \begin{align*}
            \frac{ f(c+h) - f(c-h)  }{ 2h } &= \frac{ f(c+h) - f(c) }{ 2h }  + \frac{ f(c) - f(c-h) }{ 2h }  \\
                                            &= \frac{ 1 }{ 2 }  \cdot \frac{ f(c+h) - f(c) }{ h }  + \frac{ 1 }{ 2 } \cdot \frac{ f(c) - f(c-h) }{ h }  \\ 
        \end{align*}
        Now, taking the limit as \( h \to 0  \), we have by part (a) (and using the Algebraic Limit Theorem for functions) that
        \[ \lim_{ h \to 0 }  \frac{ f(c+h) - f(c-h) }{  2h } = \frac{ 1 }{ 2 } f'(c) + \frac{ 1 }{ 2 }  f'(c) = f'(c).  \]
\end{enumerate}
\end{proof}

\begin{problem}
    Recall that in one of the homework assignments of Math 230A we proved that \( \sin x  \) and \( \cos x  \) are continuous functions on \( \R  \). We also proved that \( \displaystyle \lim_{ x \to 0 }  \frac{ \sin x  }{  x  }  = 1  \)
   \begin{enumerate}
       \item[(i)] Use this result to show that 
           \[  \lim_{ h \to 0 }  \frac{ \cos h - 1  }{ h  }  = 0.  \]
        \item[(ii)] Use (i) to show that \( f: \R \to \R  \) defined by \( f(x) = \sin x  \) is differentiable at all points \( c \in \R  \) and \( f'(c) = \cos c  \) for all \( c \in \R  \).
   \end{enumerate} 
\end{problem}
\begin{proof}
\begin{enumerate}
    \item[(i)] Suppose \( \lim_{ x \to 0 }  \displaystyle \frac{ \sin x  }{  x  }  = 1  \). Then for for a sufficiently small neighborhood of zero, we may write
        \begin{align*}
            \frac{ \cos h - 1  }{  h  }  &= \Big(  \frac{ \cos h - 1  }{ h  }  \Big) \Big( \frac{ \cos h + 1  }{  \cos + 1  }  \Big) \\
                                         &= \frac{ \cos^{2} h - 1  }{  h (\cos h +  1 ) } \\
                                         &= \frac{ - \sin^{2} h  }{  h (\cos h + 1 ) } \\ 
                                         &= \frac{ \sin h  }{  h  }  \cdot \frac{ - \sin h  }{  \cos h + 1  }.
        \end{align*}
        Note that the first term of the product in the last equality above exists by assumption and the second term exists because 
        \[  \lim_{ h \to 0 }  \frac{ - \sin h  }{  \cos h + 1  } = 0.  \]
        Indeed, \( \sin h  \) and \( \cos h  \) are both continuous functions, and so \( \lim_{ h \to 0 } (-\sin h) =  - \sin 0 = 0   \) and \( \lim_{ h \to 0 } (  \cos h + 1 ) = 2 \) along with the Algebraic Continuity Theorem implies that the above limit holds. 
        Now, using the Algebraic Limit Theorem for functions, we can write that 
        \begin{align*}
            \lim_{ h \to 0 }  \frac{ \cos h - 1  }{  h  } &= \lim_{ h \to 0 } \Big(  \frac{ \sin h  }{  h  }  \cdot \frac{ - \sin h  }{ \cos h + 1  } \Big)  \\
                                                          &= \lim_{ h \to 0 }  \frac{ \sin h  }{ h  }  \cdot \lim_{ h \to 0 } \frac{ - \sin h  }{  \cos h + 1  } \\
                                                          &= 1 \cdot 0 \\ 
                                                          &= 0
        \end{align*}
        which is our desired result.
        \item[(ii)] By the summation trigonometric identity  
            \begin{align*}
                \frac{ \sin (c + h) - \sin h  }{  h } &=  \frac{ [\sin c \cos h + \cos c \sin h ] - \sin c  }{ h }   \\
                                                      &= \frac{ \sin c (1 - \cos h) + \cos c \sin h }{ h } \\
                                                      &=  \sin c \cdot \frac{ 1 - \cos h  }{  h  }  + \cos c \cdot \frac{ \sin h  }{ h }. 
            \end{align*}
            Using part (i) along with the fact that \( \lim_{ x \to 0 } \displaystyle  \frac{ \sin x  }{  x  }  = 1  \), we have 
            \begin{align*}
                \lim_{ h \to 0 }  \frac{ \sin (c+h) - \sin h  }{ h  } &= \lim_{ h \to 0 }  \Big(  \sin c \cdot \frac{ 1 - \cos h  }{  h  }   \Big) + \lim_{ h \to 0 }  \Big(  \cos c \cdot \frac{ \sin h  }{ h }  \Big) \\
                                                                      &= \sin c \cdot \lim_{ h \to 0 }  \frac{ 1 - \cos h  }{ h  }  + \cos c \cdot \lim_{ h \to 0 }  \frac{ \sin h  }{ h } \\ 
                                                                      &= \sin c \cdot 0 + \cos c \cdot 1 \\
                                                                      &= \cos c.
            \end{align*}
            Clearly, we can see that the limit above does exist. Now, we can conclude that 
            \[  f'(c) = \cos c. \]
    \end{enumerate}
\end{proof}

\begin{problem}
    Prove the following theorem.
\end{problem}
\begin{theorem}[Generalized Mean Value Theorem]
    If \( f \) and \( g  \) are continuous on the closed interval \( [a,b] \) and differentiable on the open interval \( (a,b) \), then there exists a point \( c \in (a,b) \) where 
    \[ [f(b) - f(a)]g'(c) = [g(b)- g(a)]f'(c). \]
\end{theorem}
\begin{proof}
    Suppose that \( f: [a,b] \to \R   \) and \( g: [a,b] \to \R   \) are continuous on the closed interval \( [a,b] \) and differentiable on the open interval \( (a,b) \). Our goal is to show that there exists a point \( c \in (a,b) \) where 
    \[  [f(b) - f(a)]g'(c) = [g(b) - g(a) f'(c).] \]
    To this end, define the function \( h : [a,b] \to \R  \) by \[ h(x) = [f(b)-f(a)]g(x) - [g(b)-g(a)]f(x) .\] Our goal is to show that \( h  \) is continuous on \( [a,b] \) and differentiable on the open interval \( (a,b) \). Indeed, knowing that \( f  \) and \( g  \) are continuous on \( [a,b] \) implies, by the Algebraic Continuity Theorem, that \( h(x) \) is continuous. Furthermore, \( f  \) and \( g  \) are differentiable on \( (a,b) \), and so \( h(x) \) must also be differentiable by the Algebraic differentiability Theorem. Also, we have 
    \begin{align*}
        h(b) - h(a) &= [f(b) - f(a)]g(b) - [g(b) - g(a)]f(b) \\ 
                    &- \Big(  [f(b) - f(a)] g(a) - [g(b)-g(a)]f(b) \Big)  \\
                    &= f(b)g(b) - f(a)g(b) - g(b)f(b) + g(a)f(b)  \\
                    &- f(b)g(a) + f(a)g(a) + g(b)f(b) - g(a)f(b) \\
                    &= 0 
    \end{align*}
    Thus, we have \( h(b) = h(a) \) and so, the Rolle's Theorem implies that there exists a \( c \in (a,b)  \) such that \( h'(c) =  0 \). Hence, we have 
    \[ h'(c) = [f(b) - f(a)]g'(c) - [g(b) - g(a)]f'(c) = 0   \]
    and so 
    \[  [f(b) - f(a)]g'(c) = [g(b) - g(a)]f'(c) \]
    which is our desired result.
\end{proof}

\begin{problem}
    Prove the following theorem.
\end{problem}
\begin{theorem}[ ]
   Let \( I \subseteq  \R   \) be an interval and \( f: I \to \R  \) be a differentiable function. Prove that 
   \[  \forall x \in I, f'(x) > 0 \implies f \ \text{is strictly increasing on} \ I.  \]
\end{theorem}
\begin{proof}
    Suppose that for all \( x \in I  \), we have \( f'(x) > 0  \). Our goal is to show that \( f  \) is strictly increasing on \( I  \); that is, for all \( {x}_{1}, {x}_{2} \in I \) with \( {x}_{1} < {x}_{2} \), we have that \( f({x}_{1}) < f({x}_{2}) \). Let \( {x}_{1}, {x}_{2} \in I  \) with \( {x}_{1} < {x}_{2} \). Since \( f  \) is differentiable on \( I  \), we must also have that \( f  \) is continuous on \( I  \). Consider the open interval \( ({x}_{1}, {x}_{2}) \) in \( I  \). Then \( f  \) must be differentiable on \( ({x}_{1}, {x}_{2}) \) and continuous on \( [{x}_{1}, {x}_{2}] \). By the Mean Value Theorem, there exists a \( c \in ({x}_{1}, {x}_{2}) \) such that 
    \[  f'(c) = \frac{ f({x}_{2}) - f({x}_{1}) }{ {x}_{2} - {x}_{1} }. \]
    By assumption, we can see that \( f'(c) > 0  \). Since \( {x}_{2} - {x}_{1} > 0  \), we can see that
    \[  f({x}_{2}) - f({x}_{1}) > 0 \iff f({x}_{2}) > f({x}_{1}) \ \forall {x}_{1}, {x}_{2} \in I.  \]
    
\end{proof}

\begin{problem}
    Let \( f: \R \to \R  \) be a differentiable function and \( C > 0  \).
    \begin{enumerate} 
        \item[(i)] Suppose \( | f(u) - f(v) | \leq C | u - v  |  \) for all \( u,v \in \R  \). Prove that \( | f'(x) | \leq C  \) for all \( x \in \R  \).
        \item[(ii)] Suppose \( | f'(x) | \leq C  \) for all \( x \in \R  \). Prove that \( | f(u) - f(v) |  \leq C | u -v  |  \) for all \( u,v \in \R  \).
    \end{enumerate}
\end{problem}
\begin{proof}
    Let \( f: \R \to \R  \) be a differentiable function and \( C > 0 \).
    \begin{enumerate}
        \item[(i)] Our goal is to show that \( | f'(x) | \leq C  \) for all \( x \in \R  \). To this end, let \( x \in \R  \). To show the result, we must show that  
            \[  -C \leq \lim_{ \hat{y} \to x  }  \frac{ f(\hat{y}) - f(x) }{  \hat{y} - x } \leq C. \tag{*}    \]
            By assumption, we can see that 
            \begin{align*}
            | f(\hat{y}) - f(x) | \leq C | \hat{y} - x  |  &\iff \Big| \frac{ f(\hat{y}) - f(x) }{  \hat{y} - x  }  \Big|  \leq C \\  
                                                           &\iff - C \leq \frac{ f(\hat{y}) - f(x) }{  \hat{y} - x  }  \leq C.
        \end{align*}
        Since \( f  \) is differentiable on \( \R  \), we can see that 
        \[  \lim_{ \hat{y} \to x  }  \frac{ f(\hat{y}) - f(x) }{ \hat{y} - x  } \ \text{exists}.  \]
        Applying the Order Limit Theorem for functions on the above inequality implies that 
        \[ - C \leq \lim_{ \hat{y} \to x  }  \frac{ f(\hat{y}) - f(x) }{ \hat{y} - x   } \leq C    \]
        which tells us further that
        \[  | f'(x)  |  \leq C.  \]
        \item[(ii)] Suppose \( | f'(x) | \leq C  \) for all \( x \in \R  \). Our goal is to show that          
            \[  | f(u) - f(v) |  \leq C | u - v  | \ \forall u,v \in \R.   \] 
            Let \( u,v \in \R  \). Consider the closed interval \( [u,v] \subseteq \R  \). Since \( f  \) is continuous on \( \R  \), it follows immediately that \( f  \) must also be continuous on \( [u,v] \) (since \( f  \) is differentiable on \( \R  \)). Furthermore, \( f  \) is differentiable on the open interval \( (u,v) \) since \( f  \) is differentiable on \( \R  \). By the Mean Value Theorem, there exists a \( \xi \in (a,b) \) such that 
            \[  f'(\xi) = \frac{ f(u) - f(v) }{ u - v  }. \]
            By assumption, we can see that \( | f'(\xi) | \leq C   \) and so 
            \[   \Big| \frac{ f(u) - f(v) }{ u - v  }    \Big| = | f'(\xi) | \leq C.  \]
            Thus, we have 
            \[  | f(u) - f(v) | \leq C | u - v |  \]
            which is our desired result.
    \end{enumerate}
\end{proof}


\begin{problem}
    Let \( f: \R \to \R  \) be given \( f(x) = x^{5} + x^{3} - x^{2} + 5x + 3  \).
    \begin{enumerate}
        \item[(i)] Prove that there exists a solution to the equation \( f(x) = 0  \).
        \item[(ii)] Prove that there cannot be more than one solution to the equation \( f(x) = 0  \).
    \end{enumerate}
\end{problem}
\begin{proof}
\begin{enumerate}
    \item[(i)] We proceed by using the Intermediate Value Theorem to show the result. Since \( f  \) is continuous everywhere (because \( f  \) is a polynomial), we can just consider a closed interval \( [-1,1] \). We will show that \( f(-1) < 0  \) and \( f(1) > 0  \). Indeed, we have 
        \begin{align*}
            f(-1) &= (-1)^{5} + (-1)^{3} - (-1)^{2} + 5(-1) + 3 \\
                  &= -5 < 0 
        \end{align*}
        and
        \begin{align*}
            f(1) &= (1)^{5} + (1)^{3} - (1)^{2} + 5(1) + 3  \\
                 &> 9 > 0. 
        \end{align*}
        Thus, the intermediate value theorem implies that there exists \( \hat{c} \in [-1,1] \) such that \( f(\hat{c}) = 0  \).
    \item[(ii)] Suppose for sake of contradiction that there exists more than one solution \( {c}_{1}, {c}_{2} \in [-1,1] \) such that \( f({c}_{1}) = 0  \) and \( f({c}_{2}) = 0  \). Thus, there exists \( \tilde{c} \in (-1,1) \) such that  
        \[  f'(\tilde{c}) = \frac{ f({c}_{2}) - f({c}_{1}) }{  {c}_{2} - {c}_{1} }  = 0.  \]
        Since \( f  \) is also differentiable everywhere on \( \R  \), we have 
\end{enumerate}
\[  f'(x) = 5 x^{4} + 3 x^{2} - 2x  + 5. \]
But note that \( 3x^{2} -2x + 5  \) is a positive quadratic for all \( x \in \R  \). Hence, \( f'(x) > 0  \) for all \( x \in \R  \) which contradicts the fact that \( f'(\tilde{c}) = 0   \).
\end{proof}

\begin{problem}
    In class,we gave a proof of L'Hopital's Rule. If we add the following three assumptions to the hypotheses of the corresponding theorem, then we can give a shorter proof of H'opital's Rule:
    \begin{enumerate}
        \item[(i)] \( f'(a) \) and \( g'(a) \) exist. 
        \item[(ii)] \( g'(a) \neq 0 \).
        \item[(iii)] \( f' \) and \( g' \) are continuous at \( a  \).
    \end{enumerate}
    Here is the shorter proof:
    \[  L = \lim_{ x \to a }  \frac{ f'(x) }{ g'(x) }  = \frac{ f'(a) }{ g'(a) } = \lim_{ x \to a }  \frac{ \frac{ f(x) - f(a) }{ x - a  }  }{  \frac{ g(x) - g(a) }{ x - a  }  } = \lim_{ x \to a }  \frac{ f(x) - f(a) }{  g(x) - g(a) }  = \lim_{ x \to a }  \frac{ f(x)  }{  g(x) }.  \]
\end{problem}
\begin{solution}
    The first equality
    \[  \lim_{ x \to a }  \frac{ f'(x) }{ g'(x) }  = \frac{ f'(a) }{  g'(a) } \]
    holds because of (iii) and (ii). The third equality holds because of (i) and by definition of the derivative. Since are the referring to limits of functions, we can justify multiplying and dividing by \( x - a  \). The last equality holds because \( f(a) = 0  \) and \( g(a) = 0  \) from our original set of assumptions.
\end{solution}

\begin{problem}
    Let \( n \in \N  \) and suppose that \( f : \R \to \R  \) is a differentiable function for which the equation \( f'(x) = 0  \) has at most \( n - 1  \) solutions. Prove that the equation \( f(x) = 0  \) has at most \( n  \) solutions.
\end{problem}
\begin{proof}
Suppose for sake of contradiction that \( f(x) = 0  \) has at least \( n \) solutions. Denote the roots by
\[  {x}_{1} < {x}_{2} < \cdots < {x}_{n} < {x}_{n+1}. \]
Now, notice that 
\[  f({x}_{1}) = f({x}_{2}) = \cdots = f({x}_{n}) = f({x}_{n+1}) = 0.  \]
Since \( f  \) is differentiable on \( \R  \), we can find an \( {c}_{i} \in ({x}_{i}, {x}_{i} ) \) for \( 1 \leq i \leq n  \) such that \( f'({c}_{i}) = 0  \) by the Mean Value Theorem. This implies that \( f'(x) \) has \( n  \) solutions which contradicts our assumption that \( f'(x)  \) has \( n - 1  \) solutions.
\end{proof}

\begin{problem}
    Let \( f: \R \to \R  \) be defined by
    \[  f(x) = 
    \begin{cases}
        \frac{ x }{ 2 }  + x^{2} \sin \frac{ 1 }{ x }  &\text{if} \ x \neq 0 \\
        0 &\text{if} \ x = 0. 
    \end{cases} \]
    \begin{enumerate}
        \item[(i)] Prove that \( f  \) is differentiable at all \( x \neq 0  \).
        \item[(ii)] Prove that \( f'(0) = \frac{ 1 }{ 2 }  \).
        \item[(iii)] Prove that \( f  \) is NOT increasing on any open interval containing \( 0  \). 
    \end{enumerate}
\end{problem}
\begin{proof}
\begin{enumerate}
    \item[(i)] Note that \( x  \) and \( x^{2} \) are polynomials which is differentiable for all \( x \neq 0  \) in \( \R  \), \( \sin x  \) is differentiable for all \( x \neq 0  \) in \( \R  \), and \( \frac{ 1 }{ x }  \) is a rational function which is also differentiable for all \( x \neq 0  \) in \( \R  \). By the algebraic differentiability theorem, we have that \( f(x) \) is a differentiable function for all \( x \neq 0  \).
    \item[(ii)] Observe that
        \begin{align*}
            f'(0) &= \lim_{ x \to 0 }  \frac{ f(x) - f(0) }{ x - 0  }  \\
                  &= \lim_{ x \to 0 }  \Big[ \frac{ 1 }{ 2 }  + x \sin \frac{ 1 }{ x } \Big] \\
                  &= \lim_{ x \to 0 }  \Big[ \frac{ 1 }{ 2 }  + \frac{ \sin \frac{ 1 }{ x }  }{ \frac{ 1 }{ x }  } \Big]  \\
                  &= \frac{ 1 }{ 2 } + 0 \tag{Algebraic Limit Theorem}  \\ 
                  &= \frac{ 1 }{ 2 }
        \end{align*}
        where 
        \[  \lim_{ x \to 0 }  \frac{ \sin \frac{ 1 }{ x }  }{ \frac{ 1 }{ x }  } = 0. \]
        Indeed, we have 
        \[  0 \leq | x \sin \frac{ 1 }{ x }  | \leq | x  |. \]
        Applying the squeeze theorem for functions to the inequality above, we have that 
        \[  \lim_{ x \to 0 }  \Big| x \sin \frac{ 1 }{ x }  \Big|  = 0  \]
        which further implies that 
        \[  \lim_{ x \to 0 }  x \sin \frac{ 1 }{ x }  = 0.  \]

    \item[(iii)] Define \( a_{n} = \frac{ 1 }{ 2 \pi n  }  \to 0  \). Our goal is to show that \( f'({a}_{n}) < 0  \). Computing \( f'(x ) \) for \( x \neq 0  \), we have 
        \begin{align*}
           f'(x) &= \frac{ 1 }{ 2 }  + \Big[ 2x \sin \frac{ 1 }{ x }  - \cos \frac{ 1 }{ x } \Big]. 
        \end{align*}
        Then we have 
        \begin{align*}
            f'({a}_{n}) &= \frac{ 1 }{ 2 }  + \Big[ \frac{ 1 }{ \pi  } n \cdot \sin 2 \pi n - \cos 2 \pi n \Big] \\
                        &= \frac{ 1 }{ 2 }  + [0 -  1] \\
                        &= \frac{ 1 }{ 2 }  - 1 \\
                        &= \frac{ -1 }{ 2 } < 0.  
        \end{align*}
        Hence, \( f'({a}_{n})  < 0 \) and so we conclude that \( f  \) is NOT increasing on any open interval containing zero. 
\end{enumerate}
\end{proof}


\begin{problem}
    Let \( I \subseteq  \R   \) be an interval. Let \( f: I \to \R  \) be a differentiable function. 
    \begin{enumerate}
        \item[(a)] Show that if there exists some \( L \geq 0  \) such that \( | f'(x) | \leq L  \) for all \( x \in I  \), then \( f  \) is uniformly continuous.
        \item[(b)] Is the converse true? Prove it or give a counterexample. 
    \end{enumerate}
\end{problem}

\begin{proof}
\begin{enumerate}
    \item[(a)] Suppose that there exists some \( L \geq 0  \) such that \( | f'(x) |  \leq L  \). Our goal is to show that \( f  \) is uniformly continuous; that is, we need to show that for any \( \epsilon > 0  \), there exists \( \delta > 0  \) such that whenever \( | x - y  |  < \delta  \), we have
        \[  | f(x) - f(y) |  < \epsilon. \]
\end{enumerate}
Let \( x,y \in \R  \) and let \( \epsilon > 0  \). Suppose, without loss of generality, that \( x < y  \). Since \( f \) is differentiable on \( I  \) and \( (x,y) \subseteq  I  \), we can see that \( f  \) is also differentiable on \( (x,y) \). Furthermore, \( f  \) being differentiable on \( I \) implies that \( f  \) is continuous on \( I  \) and so \( f  \) is continuous on \( [x,y] \). By the Mean Value Theorem, we can find an \( \ell \in (x,y) \) such that   
\[  f'(\ell) = \frac{ f(x) - f(y) }{ x - y  }. \]
By assumption, we can see that for \( L > 0 \) we have 
\begin{align*}
| f'(\ell) | \leq L &\implies \Big|  \frac{ f(x) - f(y) }{ x - y  }  \Big| \leq L \\   
                    &\implies | f(x) - f(y) | \leq L | x - y |.
\end{align*} 
Now, choose \( \delta = \frac{ \epsilon  }{  L  }  \). Then whenever \( | x - y  |  < \delta \), we can see that 
\[  | f(x) - f(y)  |  \leq L | x - y  |  < L \cdot \frac{ \epsilon  }{  L  }   =  \epsilon. \]
Hence, we conclude that \( f  \) must be uniformly continuous on \( \R  \).
    \item[(b)] Consider the function \( f: (0,\infty ) \to \R  \) defined by \( f(x) = \sqrt{ x }  \). We claim that this function is uniformly continuous on \( \R  \) but its derivative \( f'(x) = \frac{ 1 }{ 2 }  x^{-1/2} \) is not bounded above for some \( L \geq 0  \). Indeed, \( f  \) is differentiable on \( (0,\infty ) \) since the following limit  
        \begin{align*}
            \lim_{ x \to c } \frac{ f(x) - f(c) }{ x - c  } &= \lim_{ x \to c  }  \frac{ \sqrt{ x  }  - \sqrt{ c  }   }{  x - c  }  \\
                                                            &= \lim_{ x \to c }  \frac{ 1  }{  \sqrt{ x  }  + \sqrt{ c  }  }  \\
                                                            &= \frac{ 1 }{ 2 \sqrt{  c  }  }
        \end{align*}
        exists.

        To show that \( f  \) is uniformly continuous, let \( x,y \in (0,\infty )  \) and let \( \epsilon > 0  \). Choose \( \delta = \sqrt{ \epsilon }  \). Since \( x,y \in (0,\infty ) \), we have 
        \[  | \sqrt{ x }  - \sqrt{ y }  | \leq | \sqrt{ x }  + \sqrt{ y }  |.  \]
        Observe that if \( | x - y  |  < \delta \), then we have
        \begin{align*}
            | \sqrt{ x }  - \sqrt{ y }  |^{2} &\leq | \sqrt{ x } - \sqrt{ y }  |  | \sqrt{ x  }  + \sqrt{ y }  |   
                                              = | x - y  |
                                              < \sqrt{ \epsilon }
        \end{align*}
        which further implies that 
        \[  | \sqrt{ x } - \sqrt{ y }  |  < \epsilon. \]
        Thus, \( f(x) = \sqrt{ x }   \) is uniformly continuous on \( (0,\infty)  \).

        Now, we want to show that the derivative of \( f  \) is NOT bounded above. Note that 
        \[  f'(x) = \frac{ 1 }{ 2 }  x^{-1/2}. \]
        
        We need to show that for all \( M > 0  \) such that there exists \( \hat{x} \in (0,\infty ) \) such that \( | f'(x) |  > M  \). To this end, let \( \epsilon > 0  \). Choose \( \hat{x} = \frac{ 1 }{ 64 M^{2} } > 0   \). Then we have 
        \begin{align*}
            | f'(\hat{x}) | = \frac{ 1 }{ 2 \sqrt{ \hat{x} }  } &= \frac{ 1 }{ 2 }  \cdot \frac{ 1  }{  \sqrt{  1/64 M^{2} }  }  \\
                                                                &= \frac{ 1 }{ 2 } \cdot \frac{ 1 }{ 1/8M } \\ 
                                                                &= 4M \\
                                                                &> M. 
        \end{align*}
        Hence, \( f  \) is not bounded above.
\end{proof}



\begin{problem}
    Let \( f: (a,b) \to \R  \) be differentiable on \( (a,b) \) with \( | f'(x)  |  \leq M  \) for \( x \in (a,b) \) and \( M \geq 0  \). Prove that 
    \[  \lim_{ x \to b^{-} }  f(x) \] 
    exists.
\end{problem}
\begin{proof}
    Suppose \( f: (a,b) \to \R  \) is a differentiable function on \( (a,b) \) with \( | f'(x) | \leq M  \) for all \( x \in (a,b) \). By Exercise 14, \( f  \) must be uniformly continuous on \( (a,b) \). By Exercise 16 of Homework 10 from Math 230A, we can find a continuous function \( F: [a,b] \to \R  \) such that \( F |_{(a,b)} = f \). As a consequence, we have \( \lim_{ x \to b } F(x) = F(b) \). Since we are only referring to the limit 
    \[  \lim_{ x \to b^{-} }  f(x) \]
    and \( F |_{(a,b)} = f  \), it follows that 
    \[  \lim_{ x \to b } f(x) = F(b) \]
    exists. Clearly, if this holds, then 
    \[  \lim_{ x \to b^{-} } f(x) = F(b) \]
    must also hold.
\end{proof}

\begin{problem}
    Let \( f: (0,1] \to \R  \) be differentiable with \( 0 < f'(x) < 1  \) for all \( x \in (0,1] \). Define a sequence \( ({a}_{n}) \):
    \[  {a}_{n} = f \Big(  \frac{ 1 }{ n }  \Big) \]
    Prove that \( \lim_{ n \to \infty  } {a}_{n}  \) exists.
\end{problem}

\begin{proof}
    Our goal is to show that \( \lim_{ n \to \infty  } {a}_{n} \) exists in \( \R  \). Since \( \R  \) is a complete metric space, it suffices to show that \( {a}_{n} = f \Big(  \frac{ 1 }{ n }  \Big) \) is a Cauchy sequence; that is, for all \( \epsilon > 0  \) there exists an \( N \in \N  \) such that for any \( n,m > N \)
    \[  | {a}_{n} - {a}_{m} | < \epsilon.   \]
    To this end, let \( \epsilon > 0  \) be given. Since \( 0 < f'(x) < 1 \) for all \( x \in (0,1] \), we have    
    \[  | {a}_{n} - {a}_{m} |  = \Big| f \Big(  \frac{ 1 }{ n }  \Big) - f \Big(  \frac{ 1 }{ m }  \Big) \Big| \leq \Big| \frac{ 1 }{ n }  - \frac{ 1 }{ m }  \Big| \]
    by exercise 9.
            Choose \( \hat{N} = \frac{ 2 }{ \epsilon }  \). Then for any \( n,m > N  \), we have  
            \[  \Big| \frac{ 1 }{ n }  - \frac{ 1 }{ m }  \Big| \leq \Big| \frac{ 1 }{ n }  \Big|  + \Big| \frac{ 1 }{ m }  \Big|  < \frac{ \epsilon }{ 2 } + \frac{ \epsilon }{ 2 } = \epsilon.  \]
            Hence, we see that \( {a}_{n} = f \Big(  \frac{ 1 }{ n }  \Big) \) is a Cauchy sequence.
\end{proof}

\begin{problem}
    Let \( f: (0,\infty ) \to \R \) be a differentiable function. Prove that, if \( \lim_{ x \to  + \infty   }  f(x) = M  \in \R  \), then there exists a sequence \( ({x}_{n}) \) such that \( | f'({x}_{n}) | \to 0 \).
\end{problem}
\begin{proof}
    Suppose \( \lim_{ x \to \infty  }  f(x) = M  \). Our goal is to show that \( | f'({x}_{n}) | \to 0  \). By assumption, let \( \epsilon = \frac{ 1 }{ n }  \) for all \( n \in \N  \). Then there exists \( {\zeta}_{n} > 0  \) such that for any \( x > {\zeta}_{n} \), we have 
    \[  | f(x) - M  |  < \frac{ 1 }{ 2n }. \]
    In particular, we have \( {\zeta}_{n} + 2 > {\zeta}_{n} + 1 > {\zeta}_{n} \). Since \( f \) is differentiable on \( (0,\infty ) \), \( f  \) must also be continuous on \( (0,\infty ) \). Consider the open interval \( ({\zeta}_{n} + 2, {\zeta}_{n}+1) \) for all \( n \in \N \). Since \( f \) is continuous on \( (0,\infty ) \), \( f  \) is also continuous on \( [{\zeta}_{n} + 1, {\zeta}_{n} + 2 ] \).  By the Mean Value Theorem, we can find an \( {x}_{n} \in ({\zeta}_{n} + 1, {\zeta}_{n} + 2) \) such that 
    \[  f'({x}_{n}) = \frac{ f({\zeta}_{n}+2) - f({\zeta}_{n} + 1) }{ {\zeta}_{n} + 2 - ({\zeta}_{n} + 1) }  = f({\zeta}_{n} + 2) - f({\zeta}_{n} + 1). \]
    By the triangle inequality, we have
    \begin{align*}
         0 < | f'({x}_{n}) | &\leq | f({\zeta}_{n} + 2) - M  |  + | M - f({\zeta}_{n} + 1) |  
                        < \frac{ 1 }{ 2n }  + \frac{ 1 }{ 2n }  
                        = \frac{ 1 }{ n }.
    \end{align*}
    By using the Squeeze Theorem for sequential limits on the inequality above, we conclude that \( | f'({x}_{n}) | \to 0  \).

\end{proof}

\begin{problem}
    Let \( f: [0,1] \to [0,1] \) be continuous on \( [0,1] \) and differentiable on \( (0,1) \). Show that if \( f'(x) \neq 1  \) for all \( x \in (0,1) \), then there exists a unique \( {x}_{0} \in [0,1] \) such that \( f({x}_{0}) = {x}_{0} \).
\end{problem}
\begin{proof}
    By exercise 16 of Homework 10 of 230A, we have that \( f: [0,1] \to [0,1] \) being continuous on \( [0,1] \) implies that there exists a \( c \in [0,1] \) such that \( f(c) = c  \). Now, we want to show that this element \( c \in [0,1] \) is unique. Suppose for sake of contradiction that there exists \( {c}_{1}, {c}_{2} \in [0,1] \) such that \( f({c}_{1}) = {c}_{1} \) and \( f({c}_{2}) = {c}_{2} \). Now, \( f  \) must be differentiable on \( (0,1)  \) implies that \( f  \) is differentiable on \( ({c}_{1}, {c}_{2}) \). By the Mean Value Theorem, there exists \( \hat{c} \in (0,1) \) such that 
    \[  f'(\hat{c}) = \frac{ f({c}_{2}) - f({c}_{1}) }{ {c}_{2} - {c}_{1} }  = \frac{ {c}_{2} - {c}_{1}  }{  {c}_{2} - {c}_{1} }  =  1. \]
    But this contradicts the fact that \( f(x) \neq 1  \) for all \( x \in [0,1] \). Hence, the element \( c \in [0,1] \) must be unique.
\end{proof}


\begin{problem}
    Let \( f: [0,\infty ) \to \R   \) be continuous on \( [0,\infty ) \) with \( f(0) = 0 \). Assume that \( f  \) is differentiable on \( (0,\infty  ) \) with \( f'  \) is increasing on \( (0,\infty ) \). Let \( g : (0,\infty  ) \to \R  \) be defined as \( g(x) = \frac{ f(x) }{ x  }  \). Prove that \( g  \) is increasing on \( (0,\infty ) \).
\end{problem}
\begin{proof}
Our goal is to show that \( g  \) is increasing on \( (0,\infty ) \); that is, for all \( x,y 
in (0,\infty )\) with \( x < y \), we have 
\[  g(x) < g(y). \]
Let \( x,y \in (0,\infty ) \). Since \( f  \) and \( g  \) are differentiable on \( (0,\infty ) \) and continuous on \( [0,\infty) \), there exists \( \hat{x} \in (0,x) \) and \( \hat{y} \in (0,y) \) such that  
\[  f'(\hat{x}) = \frac{ f(x) - f(0) }{  x - 0  } = \frac{ f(x)  }{ x  }   \]
and 
\[  f'(\hat{y}) = \frac{ f(y) - f(0) }{ y - 0  } = \frac{ f(y) }{ y }.  \]
Since \( f'  \) is increasing on \( (0,\infty ) \) and \( \hat{x} < \hat{y} \), we have 
\[  f'(\hat{x}) < f'(\hat{y}) \iff \frac{ f(x) }{ x  }  < \frac{ f(y) }{ y } \iff g(x) < g(y). \]
Hence, we can conclude that \( g  \) is increasing.
\end{proof}


\begin{problem}
    Let  \( f : [0,+\infty) \to \R  \) be continuous function, which is differentiable on \( (0,+ \infty ) \).
    \begin{enumerate}
        \item[(a)] Prove that if \( \lim_{ x \to  + \infty  } f'(x) = 0  \), then \( f  \) is uniformly continuous on \( [0,\infty ) \).
        \item[(b)] Give an example of such a function with unbounded derivative.
    \end{enumerate}
\end{problem}
\begin{proof}
\begin{enumerate}
    \item[(a)] Suppose \( \lim_{ x \to + \infty  }  f'(x) = 0  \). Our goal is to show that \( f  \) is uniformly continuous on \( [0,+\infty) \). Let \( x,y \in [0,+\infty) \). Since \( \lim_{ x \to \infty  } f'(x) = 0   \), we know that for any \( \epsilon > 0  \), there exists \( R > 0  \) such that for any \( x \geq R  \), we have  
        \[  | f'(x) | < 1. \]
        Now, for any \( x \in [R, +\infty) \), we have \( | f'(x) | < 1  \). By exercise 14, \( f  \) is uniformly continuous on \( [R,\infty) \). Next, consider the closed and bounded interval \( [0,R] \subseteq \R  \). By Heine-Borel, \( [0,R] \) is a compact set. Since \( f  \) is continuous on \( [0,+\infty) \), \( f  \) is also continuous \( [0,R] \); that is, \( f  \) is uniformly continuous on \( [0,R] \). Our goal is to, for any given \( x,y [0,+\infty) \), find a \( \delta > 0  \) such that whenever \( | x - y | < \delta \), we have
        \[  | f(x) - f(y) | < \epsilon. \]
        Since \( f  \) is uniformly continuous on \( [0,R] \), we can find a \( {\delta}_{1} > 0 \) such that whenever \( | x - y  | < {\delta}_{1} \), we have
        \[  | f(x) - f(y) | < \epsilon. \tag{1} \]
        Similarly, since \( f  \) is uniformly continuous on \( [R,+\infty) \), we can find a \( {\delta}_{2} > 0   \) such that whenever \( |  x - y  |  < {\delta}_{2} \), we have
        \[  | f(x) - f(y) | < \epsilon. \tag{2} \]
        Lastly, \( f  \) being uniformly continuous on \( [R-1,R+1] \) implies that there exists \( {\delta}_{3} > 0  \) such that whenever \( | x - y  |  < {\delta}_{3} \), we have 
        \[  | f(x) - f(y) | < \epsilon. \tag{3} \]
        Now, choose \( \delta = \min \{ {\delta}_{1} , {\delta}_{2}, {\delta}_{3}, 1  \}  \) and suppose \( x,y \in [0,\infty) \). We will consider three cases. 
        \begin{enumerate}
            \item[(I)] Let \( x,y \in [0,R] \). Then suppose \( |  x- y  |  < \delta < {\delta}_{1} \). From (1), we can conclude that \( f \) is uniformly continuous.
            \item[(II)] Let \( x,y \in [R,+\infty) \). Then suppose \( | x- y  |  < \delta < {\delta}_{2} \). From (3), we can conclude that \( f  \) is uniformly continuous.
            \item[(III)] Let \( x \in [0,R] \) and \( y \in [R,+\infty) \). Then whenever \( |  x - y  | < \delta < 1  \), we can conclude that \( x,y \in [R-1, R+1] \). Then \( |  x- y  |  < \delta < {\delta}_{3} \) implies that 
                \[  | f(x) - f(y) |  < \epsilon. \]
                Hence, \( f  \) is uniformly continuous.
        \end{enumerate}
    \item[(b)] Refer to the example in exercise 14 (\( f(x) = \sqrt{ x }  \)). We see that \( \lim_{ x \to +\infty  } f'(x) = 0  \) and \( f  \) is uniformly continuous on \( [0,\infty) \), but \( f  \) is unbounded near zero. 
\end{enumerate}
\end{proof}

\begin{problem}
    Let \( f  \) be differentiable on \( (a,b) \) and let \( c \in (a,b) \). Suppose \( f'(c) > 0  \). Prove that there exists some \( \delta > 0  \) such that \( f(x) < f(c) \) for \( x \in (c - \delta, c) \) and \( f(x) > f(c) \) for \( x \in (c, c + \delta) \).
\end{problem}
\begin{proof}
Suppose that \( f  \) is differentiable on \( (a,b) \). By definition, we have 
\[  \lim_{ x \to c }  \frac{ f(x) - f(c) }{  x - c  } = f'(c)  \ \text{exists}. \]
Hence, for all \( \epsilon > 0  \), there exists a \( \delta > 0  \) such that whenever \( 0 < |  x - c  |  < \delta \), we have 
\[  \Big| \frac{ f(x) - f(c) }{ x - c  }  - f'(c) \Big| < \epsilon. \]
In particular, let \( \epsilon = \frac{ f'(c) }{ 2 }   \). Then  
\[  \Big| \frac{ f(x) - f(c) }{  x - c  }  - f'(c) \Big| < 1 \tag{*}  \]
which further implies that 
\[  \frac{ f'(c)   }{ 2 }  + f'(c) < \frac{ f(x) - f(c) }{ x - c  }  < \frac{ f'(c) }{ 2 }  + f'(c) \implies \frac{ f'(c)   }{ 2 } (x-c) < f(x) - f(c) < \frac{ 3 f'(c) }{ 2 }  (x - c).  \]
Suppose \( x \in (c-\delta, c) \). It immediately follows that \( x < c  \) and so \( x - c < 0  \). Then we have 
\begin{align*}
    f(x) - f(c) < \underbrace{\frac{ 3 f'(c) }{ 2  }}_{>0}  \underbrace{(x - c)}_{<0} 
                < 0.
\end{align*}
Hence, we have \( f(x) < f(c) \). Now, suppose \( x \in (c , c + \delta) \). Similarly, we have \( x > c  \) which implies that \( x - c > 0  \). Then we have 
\begin{align*}
    f(x) - f(c )&> \underbrace{\frac{ f'(c) }{ 2  }}_{>0}  \underbrace{(x - c)}_{x > 0} 
                > 0.
\end{align*}
Hence, we have \( f(x) > f(c) \).
\end{proof} 


Consider the following problem: Let \( f  \) be continuous on \( [a,b] \) such that  
\[  \int_{ a }^{ x } f(t) \ dt = \int_{ x }^{ b }  f(t)  \ dt \ \forall x \in [a,b].  \]
Prove that \( f(x) = 0  \) on \( [a,b] \).


\end{document}

