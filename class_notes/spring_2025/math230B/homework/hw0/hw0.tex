\documentclass[a4paper]{article}
\usepackage[utf8]{inputenc}
\usepackage[T1]{fontenc}
% \usepackage{fourier}
\usepackage{textcomp}
\usepackage{hyperref}
\usepackage[english]{babel}
\usepackage{url}
% \usepackage{hyperref}
% \hypersetup{
%     colorlinks,
%     linkcolor={black},
%     citecolor={black},
%     urlcolor={blue!80!black}
% }
\usepackage{graphicx} \usepackage{float}
\usepackage{booktabs}
\usepackage{enumitem}
% \usepackage{parskip}
% \usepackage{parskip}
\usepackage{emptypage}
\usepackage{subcaption}
\usepackage{multicol}
\usepackage[usenames,dvipsnames]{xcolor}
\usepackage{ocgx}
% \usepackage{cmbright}


\usepackage[margin=1in]{geometry}
\usepackage{amsmath, amsfonts, mathtools, amsthm, amssymb}
\usepackage{thmtools}
\usepackage{mathrsfs}
\usepackage{cancel}
\usepackage{bm}
\newcommand\N{\ensuremath{\mathbb{N}}}
\newcommand\R{\ensuremath{\mathbb{R}}}
\newcommand\Z{\ensuremath{\mathbb{Z}}}
\renewcommand\O{\ensuremath{\emptyset}}
\newcommand\Q{\ensuremath{\mathbb{Q}}}
\newcommand\C{\ensuremath{\mathbb{C}}}
\newcommand\F{\ensuremath{\mathbb{F}}}
% \newcommand\P{\ensuremath{\mathbb{P}}}
\DeclareMathOperator{\sgn}{sgn}
\DeclareMathOperator{\diam}{diam}
\DeclareMathOperator{\LO}{LO}
\DeclareMathOperator{\UP}{UP}
\DeclareMathOperator{\card}{card}
\DeclareMathOperator{\Arg}{Arg}
\DeclareMathOperator{\Dom}{Dom}
\DeclareMathOperator{\Log}{Log}
\DeclareMathOperator{\dist}{dist}
% \DeclareMathOperator{\span}{span}
\usepackage{systeme}
\let\svlim\lim\def\lim{\svlim\limits}
\renewcommand\implies\Longrightarrow
\let\impliedby\Longleftarrow
\let\iff\Longleftrightarrow
\let\epsilon\varepsilon
\usepackage{stmaryrd} % for \lightning
\newcommand\contra{\scalebox{1.1}{$\lightning$}}
% \let\phi\varphi
\renewcommand\qedsymbol{$\blacksquare$}

% correct
\definecolor{correct}{HTML}{009900}
\newcommand\correct[2]{\ensuremath{\:}{\color{red}{#1}}\ensuremath{\to }{\color{correct}{#2}}\ensuremath{\:}}
\newcommand\green[1]{{\color{correct}{#1}}}

% horizontal rule
\newcommand\hr{
    \noindent\rule[0.5ex]{\linewidth}{0.5pt}
}

% hide parts
\newcommand\hide[1]{}

% si unitx
\usepackage{siunitx}
\sisetup{locale = FR}
% \renewcommand\vec[1]{\mathbf{#1}}
\newcommand\mat[1]{\mathbf{#1}}

% tikz
\usepackage{tikz}
\usepackage{tikz-cd}
\usetikzlibrary{intersections, angles, quotes, calc, positioning}
\usetikzlibrary{arrows.meta}
\usepackage{pgfplots}
\pgfplotsset{compat=1.13}

\tikzset{
    force/.style={thick, {Circle[length=2pt]}-stealth, shorten <=-1pt}
}

% theorems
\makeatother
\usepackage{thmtools}
\usepackage[framemethod=TikZ]{mdframed}
\mdfsetup{skipabove=1em,skipbelow=1em}

\theoremstyle{definition}

\declaretheoremstyle[
    headfont=\bfseries\sffamily\color{ForestGreen!70!black}, bodyfont=\normalfont,
    mdframed={
        linewidth=1pt,
        rightline=false, topline=false, bottomline=false,
        linecolor=ForestGreen, backgroundcolor=ForestGreen!5,
    }
]{thmgreenbox}

\declaretheoremstyle[
    headfont=\bfseries\sffamily\color{NavyBlue!70!black}, bodyfont=\normalfont,
    mdframed={
        linewidth=1pt,
        rightline=false, topline=false, bottomline=false,
        linecolor=NavyBlue, backgroundcolor=NavyBlue!5,
    }
]{thmbluebox}

\declaretheoremstyle[
    headfont=\bfseries\sffamily\color{NavyBlue!70!black}, bodyfont=\normalfont,
    mdframed={
        linewidth=1pt,
        rightline=false, topline=false, bottomline=false,
        linecolor=NavyBlue
    }
]{thmblueline}

\declaretheoremstyle[
    headfont=\bfseries\sffamily, bodyfont=\normalfont,
    numbered = no,
    mdframed={
        rightline=true, topline=true, bottomline=true,
    }
]{thmbox}

\declaretheoremstyle[
    headfont=\bfseries\sffamily, bodyfont=\normalfont,
    numbered=no,
    % mdframed={
    %     rightline=true, topline=false, bottomline=true,
    % },
    qed=\qedsymbol
]{thmproofbox}

\declaretheoremstyle[
    headfont=\bfseries\sffamily\color{NavyBlue!70!black}, bodyfont=\normalfont,
    numbered=no,
    mdframed={
        rightline=false, topline=false, bottomline=false,
        linecolor=NavyBlue, backgroundcolor=NavyBlue!1,
    },
]{thmexplanationbox}

\declaretheorem[
    style=thmbox, 
    % numberwithin = section,
    numbered = no,
    name=Definition
    ]{definition}

\declaretheorem[
    style=thmbox, 
    name=Example,
    ]{eg}

\declaretheorem[
    style=thmbox, 
    % numberwithin = section,
    name=Proposition]{prop}

\declaretheorem[
    style = thmbox,
    numbered=yes,
    name =Problem
    ]{problem}

\declaretheorem[style=thmbox, name=Theorem]{theorem}
\declaretheorem[style=thmbox, name=Lemma]{lemma}
\declaretheorem[style=thmbox, name=Corollary]{corollary}

\declaretheorem[style=thmproofbox, name=Proof]{replacementproof}

\declaretheorem[style=thmproofbox, 
                name = Solution
                ]{replacementsolution}

\renewenvironment{proof}[1][\proofname]{\vspace{-1pt}\begin{replacementproof}}{\end{replacementproof}}

\newenvironment{solution}
    {
        \vspace{-1pt}\begin{replacementsolution}
    }
    { 
            \end{replacementsolution}
    }

\declaretheorem[style=thmexplanationbox, name=Proof]{tmpexplanation}
\newenvironment{explanation}[1][]{\vspace{-10pt}\begin{tmpexplanation}}{\end{tmpexplanation}}

\declaretheorem[style=thmbox, numbered=no, name=Remark]{remark}
\declaretheorem[style=thmbox, numbered=no, name=Note]{note}

\newtheorem*{uovt}{UOVT}
\newtheorem*{notation}{Notation}
\newtheorem*{previouslyseen}{As previously seen}
% \newtheorem*{problem}{Problem}
\newtheorem*{observe}{Observe}
\newtheorem*{property}{Property}
\newtheorem*{intuition}{Intuition}

\usepackage{etoolbox}
\AtEndEnvironment{vb}{\null\hfill$\diamond$}%
\AtEndEnvironment{intermezzo}{\null\hfill$\diamond$}%
% \AtEndEnvironment{opmerking}{\null\hfill$\diamond$}%

% http://tex.stackexchange.com/questions/22119/how-can-i-change-the-spacing-before-theorems-with-amsthm
\makeatletter
% \def\thm@space@setup{%
%   \thm@preskip=\parskip \thm@postskip=0pt
% }
\newcommand{\oefening}[1]{%
    \def\@oefening{#1}%
    \subsection*{Oefening #1}
}

\newcommand{\suboefening}[1]{%
    \subsubsection*{Oefening \@oefening.#1}
}

\newcommand{\exercise}[1]{%
    \def\@exercise{#1}%
    \subsection*{Exercise #1}
}

\newcommand{\subexercise}[1]{%
    \subsubsection*{Exercise \@exercise.#1}
}


\usepackage{xifthen}

\def\testdateparts#1{\dateparts#1\relax}
\def\dateparts#1 #2 #3 #4 #5\relax{
    \marginpar{\small\textsf{\mbox{#1 #2 #3 #5}}}
}

\def\@lesson{}%
\newcommand{\lesson}[3]{
    \ifthenelse{\isempty{#3}}{%
        \def\@lesson{Lecture #1}%
    }{%
        \def\@lesson{Lecture #1: #3}%
    }%
    \subsection*{\@lesson}
    \testdateparts{#2}
}

% \renewcommand\date[1]{\marginpar{#1}}


% fancy headers
\usepackage{fancyhdr}
\pagestyle{fancy}

\makeatother

% notes
\usepackage{todonotes}
\usepackage{tcolorbox}

\tcbuselibrary{breakable}
\newenvironment{verbetering}{\begin{tcolorbox}[
    arc=0mm,
    colback=white,
    colframe=green!60!black,
    title=Opmerking,
    fonttitle=\sffamily,
    breakable
]}{\end{tcolorbox}}

\newenvironment{noot}[1]{\begin{tcolorbox}[
    arc=0mm,
    colback=white,
    colframe=white!60!black,
    title=#1,
    fonttitle=\sffamily,
    breakable
]}{\end{tcolorbox}}

% figure support
\usepackage{import}
\usepackage{xifthen}
\pdfminorversion=7
\usepackage{pdfpages}
\usepackage{transparent}
\newcommand{\incfig}[1]{%
    \def\svgwidth{\columnwidth}
    \import{./figures/}{#1.pdf_tex}
}

% %http://tex.stackexchange.com/questions/76273/multiple-pdfs-with-page-group-included-in-a-single-page-warning
\pdfsuppresswarningpagegroup=1


\title{Homework 4}
\author{Lance Remigio}
\begin{document}
\maketitle

\begin{problem}
   If \( E  \) is nonempty subset of a metric space \( (X,d) \), define the distance from \( x \in X  \) to \( E  \) by 
   \[  \dist(x,E) = \inf_{z \in E } d(x,z). \]
   \begin{enumerate}
       \item[(a)] Prove that \( \dist (x,E) = 0  \) if and only if \( x \in \overline{E} \).
      \item[(b)] Prove that if \( E  \) is compact, then the infimum in the definition above is attained, that is, if \( x \in X  \) and \( E  \) is compact, then there exists \( a \in E  \) such that \( \dist (x,E ) = d(x,a) \).
      \item[(c)] Prove that if \( x = \R^{n} \) and if \( E  \) is closed, then the in the definition above is attained, that is, if \( x \in \R^{n} \) and \( E  \) is closed, then there exists \( a \in E  \) such that \( \dist (x,E) = d(x,a) \).
        \item[(d)] Prove that \( \dist(x,E) = \dist (x, \overline{E}) \).
        \item[(e)] Prove that \( {d}_{E} : X \to \R  \) defined by \( {d}_{E} (x) = \dist (x,E) \) is uniformly continuous function on \( X  \), by showing that 
            \[  | {d}_{E}(x) - {d}_{E}(y) |  \leq d(x,y) \ \ \ \forall x \in X , y \in X. \]
   \end{enumerate} 
\end{problem}

\begin{proof}
\begin{enumerate}
    \item[(1-a)]  \( (\Longrightarrow)  \) Suppose \( \dist(x,E) = 0  \). Our goal is to show that \( x \in \overline{E} \); that is, we want to show that for all \( \epsilon > 0  \), 
        \[  {N}_{\epsilon}(x) \cap E \neq \emptyset. \]
        Let \( \epsilon > 0  \) be given. Since \( \dist(x,E) = \inf_{z \in E } d(x,z) = 0  \), there exists \( {z}_{1} \in E  \) such that 
        \[  d(x,{z}_{1}) < \dist(x,E) + \epsilon = 0 + \epsilon = \epsilon.  \]
        Thus, \( {z}_{1} \in {N}_{\epsilon}(x) \). Since we also have \( {z}_{1} \in E  \), it follows that \( {N}_{\epsilon}(x) \cap E \neq \emptyset \) as desired.

        \( (\Longleftarrow) \) Suppose \( x \in \overline{E} \). Our goal is to show that \( \dist(x,E) = 0  \); that is, we need to show that \( \inf_{z \in E } d(x,z) = 0  \). To this end, it suffices to prove that 
        \[  \forall z \in E \ \ d(x,z) \geq 0 \tag{i}  \]
        and
        \[ \forall \epsilon > 0 \ \ \exists z \in E \ \ \text{such that} \ \ d(x,z) < 0 + \epsilon \tag{ii} \]
\end{enumerate}
We see that (i) follows immediately because \( d  \) defines a metric on \( X  \). To show (ii), let \( \epsilon > 0  \) be given. Since \( x \in \overline{E} \), \( {N}_{\epsilon}(x) \cap E \neq \emptyset \). So, there exists \( {z}_{1} \) such that \( {z}_{1} \in E  \) and \( {z}_{1} \in {N}_{\epsilon}(x) \). Hence, \( {z}_{1} \in E  \) such that \( d(x,{z}_{1}) < \epsilon \). Note that \( {z}_{1} \) is the same \( z  \) we were looking for. This conclude the proof for the backwards direction. 

\item[(1-b)] We know that if \( A \subseteq  \R   \) is a nonempty set that is bounded below, then \( \inf A \in \overline{A} \) and so there exists a sequence \( ({a}_{n})  \) in \( A  \) such that \( {a}_{n} \to \inf A  \). We have \( \dist(x,E) = \inf_{z \in E } d(x,z) \). So, there exists a sequence \( ({z}_{n}) \) in \( E  \) such that \( d(x,{z}_{n}) \to \dist(x,E) \). Now, since \( E  \) is compact, \( ({z}_{n}) \) contains a subsequence \( ({z}_{{n}_{k }}) \) that converges to a point \( a \in E  \). Thus, we have 
    \[  {z}_{{n}_{k }} \to a \implies d(x, {z}_{{n}_{k}}) \to d(x,a) \]
    and 
    \[  d(x,{z}_{n}) \to \dist(x,E) \implies d(x,{z}_{{n}_{k}}) \to \dist(x,E) \]
    imply that
    \[  \dist(x,E) = d(x,a) \]
    by the uniqueness of limits.
\item[(1-c)] Recall that in \( \R^{n} \) every closed and bounded set is compact. Pick any point \( p \in E  \). Let \( r = d(x,p) \). Let \( S = \overline{{N}_{r}(x)} \cap E  \) (clearly, \( p \in S  \) and since \( S \subseteq \overline{{N}_{r}(x)}  \) and \( \dist(x,S) \leq r \)).

    In what follows, we will show that \( \dist(x,S) = \dist(x,E) \).
    \begin{remark}
        Note that since \( S  \) is the intersection of closed sets, it is closed. Also, 
        \[  S \subseteq  \overline{{N}_{r}(p)} = \{ z \in X : d(x,z) \leq r  \}  \subseteq  {N}_{2r}(p). \]
        So, \( S  \) is bounded. Since \( S  \) is closed and bounded, it is compact. Thus, by (1-b), there exists \( z \in S  \) such that \( d(x,z) = \dist(x,S) \). Since \( \dist(x,S) = \dist(x,E)  \), the claim in proved.  
    \end{remark}
    First, note that 
    \[  \dist(x,S) = \inf_{z \in S } d(x,z) \underbrace{\geq}_{S \subseteq E} \inf_{z \in E } d(x,z) = \dist(x,E).  \]
    Hence, \( \dist(x,S) \geq \dist(x,E) \). From here, we just need to prove that \( \dist(x,E) \geq \dist(x,S)  \). Our goal is to show that 
    \[  \forall z \in E \ \ d(x,z) \geq \dist(x,S). \]
    Let \( z \in E \) be given. If \( z \in S  \), then \( d(x,z) \geq \inf_{w \in S} d(x,w) = \dist(x,S) \). If \( z \notin S = \overline{{N}_{r}(x)} \cap E  \), then since \( z \in E  \), we can conclude that \( z \notin \overline{{N}_{r}(x)} \) and so \( d(x,z) \geq r \geq \dist(x,S) \) as desired.
        \item[(1-d)] First note that \( E \subseteq  \overline{E} \) (in genreal, if \( A \subseteq  B  \), then \( \inf A \geq \inf B  \)). So, we have 
            \[  \dist(x,E) = \inf_{z \in E } d(x,z) \geq \inf_{z \in E } d(x,z) = \dist(x,\overline{E}). \]
            It suffices to show that \( \dist(x,\overline{E}) \geq \dist(x,E)  \), that is,
            \[  \inf_{z \in \overline{E}} d(x,z) \geq \dist(x,E). \]
            That is, our goal is to show that 
            \[  \forall z \in \overline{E} \ \ d(x,z) \geq \dist(x,E). \]
            Let \( z \in \overline{E}  \) be given. By definition, we have 
            \[  \forall \epsilon > 0 \ \ {N}_{\epsilon}(z) \cap E \neq \emptyset.  \]
            Hence, there exists \( {p}_{\epsilon} \in {N}_{\epsilon}(z) \cap E  \) and so 
            \[  \dist(x,E) \leq d(x, {p}_{\epsilon}) \leq d(x,z) + d(z,{p}_{\epsilon}) < d(x,z) + \epsilon. \]
            That is, 
            \[  \forall \epsilon  > 0 \ \ d(x,z) + \epsilon > \dist(x,E). \]
            Thus, 
            \[  d(x,z) \geq \dist(x,E). \]
        \item[(1-e)] Recall that \( {d}_{E} : X \to \R  \) is uniformly continuous if and only if for all \( \epsilon > 0  \), there exists \( \delta > 0  \) such that for all \( x ,y \in X  \) if \( d(x,y) < \delta \), then
            \[  | {d}_{E}(x) - {d}_{E}(y) |  < \epsilon. \tag{*} \]
            If we prove that 
            \[  \forall x, y \in X \ \ | {d}_{E}(x) - {d}_{E}  | \leq d(x,y), \tag{**}  \]
            then (*) will hold by setting \( \delta = \epsilon  \) (or any positive nymber less than \( \epsilon \)). So, it suffices to show that (**) holds. Let \( x,y \in X  \) be given. We have  
            \[  {d}_{E}(x) = \inf_{z \in E} d(x,z) \implies \forall z \in E \ \ {d}_{E}(x) \leq d(x,z).   \]
            Then we have 
            \[ \forall z \in E \ \ {d}_{E}(x) \leq d(x,y) + d(y,z) \]
            which can be further rewritten into
            \[ \forall z \in E  \ \   {d}_{E}(x) - d(x,y) \leq d(y,z). \]
            This tells us that \( {d}_{E}(x) - d(x,y) \) is a lower bound for the set
            \[  \{ d(y,z) : z \in E  \}.  \]
            Hence, we have that 
            \[  {d}_{E}(x) - d(x,y) \leq \inf_{z \in E} d(y,z) = {d}_{E}(y) \]
            and so 
            \[  {d}_{E}(x) - {d}_{E}(y) \leq d(x,y). \tag{1} \]
            Switching the roles of \( x  \) and \( y  \) in the argument above, we can derive a similar result; that is,
            \[ - ({d}_{E}(x) - {d}_{E}(y)) =   {d}_{E}(y) - {d}_{E}(x) \leq d(y,x) = d(x,y). \tag{2} \]
            Thus, (1) and (2) imply that 
            \[  | {d}_{E}(x) - {d}_{E}(y) |  \leq d(x,y) \]
            which proves that \( {d}_{E} \) is a uniformly continuous function on \( X  \) as desired.
\end{proof}

\begin{problem}
    Let \( A  \) and \( B  \) be nonempty subsets of a metric space \( (X,d) \). The distance between \( A  \) and \( B  \) is defined as follows:
    \[  \dist(A,B)  = \inf \{  d(x,y) : x \in A , y \in B \}. \]
    (Note that in case \( A = \{ x  \}  \), \( \dist(\{ x \}, B ) = \dist(x,B) \) which was introduced in the previous exercise.) Prove that 
    \[  \dist(A,B) = \inf_{x \in A} \dist(x,B) = \inf_{y \in B} \dist(y,A). \]
\end{problem}
\begin{proof}
Here we will prove a more general claim: 
Let \( A  \) and \( B  \) be any two nonempty sets (not necessarily in a metric space) and let \( F: A \times B \to \R  \) be a function that is bounded below; that is, the set \( \{ F(x,y) : (x,y) \in A \times B  \}  \) is bounded below. Let 
\begin{align*}
    G: A \to \R &, G(x) = \inf_{y \in B} F(x,y) \\
    H: B \to \R &, H(y) = \inf_{x \in A } F(x,y).
\end{align*}
Then
\begin{enumerate}
    \item[(1)] \( \inf_{(x,y) \in A \times B } F(x,y) = \inf_{x \in A } G(x) \);
    \item[(2)] \( \inf_{(x,y) \in A \times B } F(x,y) = \inf_{y \in B} H(y) \).
\end{enumerate}
Here we will prove (1). The proof of (2) is analogous.
Let \( L = \inf_{(x,y) \in A \times B} F(x,y) \). Our goal is to show that \( L = \inf_{x \in A } G(x) \). To this end, it suffices to show that 
\begin{enumerate}
    \item[(i)] \( L \leq G(x)  \) for all \( x \in A  \)
    \item[(ii)] \( \forall \epsilon > 0  \), \( \exists x \in A  \) such that \( G(x) < L + \epsilon  \).
\end{enumerate}
Indeed, let \( x \in A  \). Then we have 
\begin{align*}
    \forall y \in B \ \ (x,y) \in A \times B &\implies \forall y \in B \ \ L \leq F(x,y) \\
                                             &\implies  L \  \text{is a lower bounded of} \ \{ F(x,y) : y \in B \} \\
                                             &\implies  L \leq \inf_{y \in B} F(x,y) = G(x).
\end{align*}
This proves (i). Now, we will show (ii). Let \( \epsilon > 0  \) be given. Then
\[  L = \inf_{(x,y) \in A \times B } F(x,y) \implies \exists ({x}_{0}, {y}_{0}) \in A \times B \ \text{such that} \ F({x}_{0}, {y}_{0}) < L + \epsilon. \]
Thus, we have 
\[  G({x}_{0}) = \inf_{y \in B} F({x}_{0}, y) \leq F({x}_{0}, {y}_{0}) < L + \epsilon. \]
From this, we can see that \( {x}_{0} \) is the same \( x  \) we were looking for.
\end{proof}

\begin{problem}
    Let \( (X,d) \) be a metric space. Prove that if \( A  \) and \( B  \) are two nonempty disjoint sets in \( X  \) such that \( A  \) is \textbf{compact} and \( B  \) is \textbf{closed}, then \( \dist(A,B) > 0  \).
\end{problem}
\begin{proof}
Assume for contradiction that \( \dist(A,B) = 0  \). We have 
\[  0 = \dist(A,B) = \inf_{x \in A} {d}_{B}(x). \tag{See Exercise 2} \]
In exercise 1, we proved that \( {d}_{B}: X \to \R  \) is uniformly continuous. As a consequence, \( {d}_{B} : A \to \R  \) is continuous. Since \( A  \) is compact, it follows from the Extreme Value Theorem that 
\[  \exists a \in A \ \text{such that} \ \inf_{x \in A } {d}_{B}(x) = {d}_{B}(a). \]
Since \( \inf_{x \in A } {d}_{B}(x) = \dist(A,B) = 0  \), we can conclude that 
\[  {d}_{B}(a) = 0.  \] 
It follows from part (a) of exercise 1 that \( a \in \overline{B} \). Since \( B  \) is closed, we have \( \overline{B} = B  \) and so \( a \in B  \). Thus, \( A \cap B \neq \emptyset  \) since \( a \in A  \) and \( a \in B  \) which is a contradiction!
\end{proof}

\begin{problem}
    Let \( E  \) be a nonempty subset of \( \R^{n} \). Let \( t > 0  \) be a fixed positive number. Let \( A = \{  x \in \R^{n} :  \dist(x,E) \geq t  \}  \). Prove that 
    \[  \circ{A} = \{ x \in \R^{n} : \dist(x,E) > t  \}. \]
\end{problem}
\begin{proof}
It suffices to show the following two s
\end{proof}


\end{document}


