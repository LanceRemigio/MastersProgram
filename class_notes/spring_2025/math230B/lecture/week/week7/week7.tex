\section{Lectures 17-18}

\subsection{Plan}

\begin{itemize}
    \item Series of functions
    \item Cauchy Criterion for Uniform Convergence of Series
    \item Weiertstrass M-Test
\end{itemize}

\begin{theorem}[Term-by-Term Continuity Theorem]
Let \( A \subseteq  (X,d)  \) be nonempty. Suppose for all \( n \in \N \) \( {f}_{n} : A \to \R  \) is a sequence of continuous functions, and \( \sum_{ n=1  }^{ \infty  } {f}_{n} \) converges uniformly to \( f: A \to \R  \). Then \( f: A \to \R  \) is continuous.    
\end{theorem}
\begin{proof}
Applying the corresponding theorem for sequences of functions to the sequence of partial sums \( {s}_{m} = {f}_{1} + \cdots + {f}_{n} \). That is, 
\[  \sum_{ n=1  }^{ \infty  } {f}_{n} = f \implies {s}_{m} \to f \ \text{uniformly} \implies f \ \text{is continuous} \]
since \( {s}_{m}  \) is continuous.
\end{proof}

\begin{theorem}[Term-by-Term Differentiability Theorem]
    Assume for each \( n \in \N  \), \( {f}_{n} : [a,b] \to \R  \) is a sequence of differentiable functions, \( \sum_{ n=1  }^{ \infty  } {f}_{n} = f  \) pointwise on \( [a,b] \), and \( \sum_{ n=1  }^{ \infty  } {f}_{n}'  \) converges uniformly on \( [a,b] \). Then
    \( f  \) is differentiable on \( [a,b] \) and 
    \[  \Big(  \sum_{ n=1  }^{ \infty  } {f}_{n} \Big)' = \sum_{ n=1  }^{ \infty  } {f}_{n}'. \]
\end{theorem}
\begin{proof}
Apply the corresponding theorem for sequences of functions to the sequence of partial sums \( {s}_{m} = {f}_{1} + \cdots + {f}_{m} \).
\end{proof}

\begin{theorem}[Term-by-Term Integrability]
    Let \( \alpha: [a,b] \to \R  \) is an increasing function, for each \( n \geq 1  \), \( {f}_{n} \in {R}_{\alpha}[a,b] \), and \( \sum_{ n=1  }^{ \infty  } {f}_{n} \) converges uniformly to \( f: [a,b] \to \R   \). Then 
    \[  f \in {R}_{\alpha}[a,b] \ \ \text{and} \ \ \int_{ a }^{ b }  \sum_{ n=1  }^{ \infty  } {f}_{n} \ d \alpha = \sum_{ n=1  }^{ \infty  } \int_{ a }^{ b }  {f}_{n} \ d \alpha. \]
\end{theorem}
\begin{proof}
Apply the corresponding theorem for sequences of functions to the sequence of partial sums \( {s}_{m} = {f}_{1} + \cdots + {f}_{m} \).
\end{proof}

\begin{theorem}[Cauchy Criterion for Uniform Convergence of Series of Functions]
    Let \( A  \) be a nonempty set and suppose for each \( k \in \N  \), \( {f}_{k } : A \to \R  \). Then
    \begin{center}
        \( \displaystyle \sum_{ k=1  }^{ \infty  } {f}_{k } \) converges uniformly if and only if for all \(  \epsilon > 0  \), there exists an \( N  \) such that for all \( n > m > N  \) and for all \( x \in A  \), \( \displaystyle \Big| \sum_{ k=1  }^{ n } {f}_{k } (x) \Big| < \epsilon \).
    \end{center}
\end{theorem}

\begin{theorem}[Weierstrass M-Test]
    Let \( A  \) be a nonempty set, for all \( n \in \N  \) \( {f}_{n} : A \to \R  \), for all \(  n \in \N  \), there exists \( {M}_{n} \) such that for all \( x \in A  \), \( | {f}_{n}(x) |  \leq {M}_{n} \), and \( \displaystyle \sum_{ n=1  }^{ \infty  } {M}_{n}  \) converges. Then 
    \[  \sum_{ n=1  }^{ \infty  } {f}_{n} \ \ \text{converges uniformly on} \ A.  \]
\end{theorem}
\begin{proof}
    According to the Cauchy Criterion for uniform convergence of series of functions, it suffices to show that for all \( \epsilon > 0  \), there exists \( N  \) such that for all \( n > m > N  \)  and for all \( x \in A  \)
    \[  \Big| \sum_{ k=m+1  }^{ n  } {f}_{k } (x) \Big|  < \epsilon. \tag{*} \]
    Let \( \epsilon > 0 \).
    Note, by assumption, \( \displaystyle \sum_{ n=1  }^{ \infty  } {M}_{n} \) converges. Thus, for our given \( \epsilon  \), there exists \( \hat{N} \) such that 
    \[  \forall m > m > \hat{N} \ \ \Big| \sum_{ k = m + 1  }^{ n } {M}_{k }  \Big|  < \epsilon. \]
    We claim that we can use this \( \hat{N} \) as the \( N  \) that we were looking for. Indeed, if we let \( N = \hat{N} \), then (*) will hold because for all \( n > m > \hat{N} \) and for all \( x \in A  \)
    \[ \Big| \sum_{ k= m+1 }^{  n  } {f}_{k }(x) \Big|  \leq \sum_{ k = m + 1  }^{ n } | {f}_{k }(x) | \leq \sum_{ k= m + 1  }^{ n } {M}_{k } < \epsilon  \]
    as desired.
\end{proof}

\section{Lectures 20-21}

\subsection{Plan}

\begin{enumerate}
    \item[(1)] Dini's Theorem
\end{enumerate}

\begin{theorem}[Rudin 7.13]
    Let \( (X,d) \) be a metric space, let \( K \subseteq  X   \) be a compact set, and suppose for each \( n \in \N  \), \( {f}_{n} : K \to \R  \) is continuous. Assume further that \( {f}_{n} \to f  \) pointwise on \( K  \) where \( f : K \to \R  \) is continuous, and that for all \( n \in \N  \), \( {f}_{n+1} \leq {f}_{n} \). Then \( {f}_{n} \to f  \) uniformly on \( K  \). 
\end{theorem}

\begin{proof}
Let \( \epsilon > 0 \) be given. Our goal is to show, there exists \( N  \) such that for all \( n > N  \) and for all \(  \in K  \), we have 
\[  | {f}_{n}(x) - f(x) < \epsilon. |  \]
For each \( n \in \N  \), let \( {g}_{n} = {f}_{n} - f  \). So, our goal is to show that there exists an \( N  \) such that 
\[  \forall n > N \ \forall x \in K \ | {g}_{n}(x) |  < \epsilon. \]
First, we observe that for all \( {g}_{n} \geq 0  \). Indeed, we see that for each \( x \in K  \), \( ({f}_{n}(x))_{n \geq 1 } \) is a decreasing sequence of real numbers that converges to \( f(x) \). It follows from the Monotone Convergence Theorem that \( f(x) = \inf_{n \in \N} {f}_{n}(x) \). Thus, for all \( n \in \N  \), we have
\[  f(x) \leq {f}_{n}(x). \]
Therefore, for all \( n \in \N  \), \( {g}_{n} \geq  0  \). To get our desired result, all we need to show is that there exists an \( N  \) such that for all \( n > N  \) and for all \( x \in K  \), \( {g}_{n}(x) < \epsilon \). We can reframe our desired conclusion in the following way: 
\[  \exists N \in \N \ \text{such that} \ \forall n > N \ \ \{ x \in K : {g}_{n}(x) \geq \epsilon  \}  = \epsilon. \] Let \( {K}_{n} = {g}_{n}^{-1}([\epsilon, \infty ]) \) for each \( n \in \N  \). Our goal is to show that for all \(  n > N  \), \( {K}_{n} = \emptyset \). Observe further that for each \( n \in \N  \), \( {K}_{n} \) is a compact set. Indeed, we see that for each \( n \in \N  \), \( {g}_{n} : K \to \R  \) is continuous and \( [\epsilon, \infty) \) is a closed set in \( \R  \). From this, we can see that \( {K}_{n} = {g}_{n}^{-1}([\epsilon,\infty) )  \) is closed in \( K  \) because preimages of closed sets under a continuous map is closed. Thus, we can see that each \( {K}_{n} \) must be compact because \( K  \) is compact, \( {K}_{n} \subseteq  K \)  and \( {K}_{n}  \) is closed. 
\end{proof}

