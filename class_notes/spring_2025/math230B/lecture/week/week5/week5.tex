\section{Lectures 11-12}

\subsection{Plan}

\begin{enumerate}
    \item[(1)] Sequential Criterion for integrability;
    \item[(2)] Algebraic properties of R.S integral;
    \item[(3)] Order properties of R.S integrals;
    \item[(4)] Mean Value Theorem and Generalized Mean Value Theorem for integrals;
    \item[(5)] Additivity for R.S integrals.
\end{enumerate}

\begin{theorem}[Sequential Criterion for R.S Integrability]
    Let \( f: [a,b] \to \R  \) be a bounded function and \( \alpha: [a,b] \to \R  \) is an increasing function. Then
    \begin{enumerate}
        \item[(1)] If \( f \in {R}_{\alpha}[a,b]  \), then there exists a sequence of partitions \( ({P}_{n})_{n \geq 1}  \) in \( \Pi [a,b] \) such that \( \lim_{ n \to \infty  }  [U(f,\alpha, {P}_{n}) - L(f,\alpha, {P}_{n}) ] = 0  \).
        \item[(2)] If there exists a sequence of partitions \( ({P}_{n})_{n \geq 1} \) in \( \Pi [a,b] \) such that \( \lim_{ n \to \infty  } [U(f,\alpha, {P}_{n}) - L(f,\alpha, {P}_{n})] = 0  \), then \( f \in {R}_{\alpha}[a,b] \), and 
            \[  \int_{ a }^{ b }  f  \ d \alpha = \lim_{ n \to \infty k U(f,\alpha, {P}_{n}) }  = \lim_{ n \to \infty  }  L(f,\alpha, {P}_{n}).  \]
    \end{enumerate}
\end{theorem}
\begin{proof}
\begin{enumerate}
    \item[(1)] Using the Cauchy Criterion, we see that \( f \in {R}_{\alpha}[a,b] \) if and only if for all \( \epsilon > 0  \), there exists \( {P}_{\epsilon} \in \Pi [a,b] \) such that 
        \[  U(f,\alpha,{P}_{\epsilon}) - L(f,\alpha, {P}_{\epsilon})  < \epsilon. \]
        In particular, we can inductively construct a sequence of partitions \( ({P}_{n})_{n \geq 1 } \) in the following way: for all \( n \in \N \), let \( \epsilon = \frac{ 1 }{ n }  \). Then there exists \( {P}_{n} \in \Pi  \) such that  
        \[   0 \leq U(f,\alpha, {P}_{n}) - L(f,\alpha, {P}_{n}) < \frac{ 1 }{ n }.  \]
       From the squeeze theorem, it follows that  
       \[  \lim_{ n \to \infty  } [U(f,\alpha, {P}_{n}) - L(f,\alpha, {P}_{n})] = 0.  \]
    \item[(2)] According to Cauchy Criterion, it suffices to show that 
        \[  \forall \epsilon > 0 \ \exists P \in \Pi [a,b] \ \text{such that} \ U(f,\alpha,P) - L(f,\alpha,P) < \epsilon. \]
        Since \( \lim_{ n \to \infty  } [U(f,\alpha, {P}_{n}) - L(f,\alpha, {P}_{n})] = 0  \), there exists an \( N \in \N \) such that 
        \[  \forall n > N \ \ U(f,\alpha,{P}_{n}) - L(f,\alpha,{P}_{n}) < \epsilon. \]
        In particular, \( {P}_{N+1} \) can be used as the \( P  \) that we were looking for. It remains to show that 
        \[  \int_{ a }^{ b }  f  \ d \alpha = \lim_{ n \to \infty  }  U(f,\alpha,{P}_{n}) = \lim_{ n \to \infty  }  L(f,\alpha, {P}_{n}). \]
        We have for all \( n \geq 1  \), 
        \[  0 \leq U(f,\alpha,{P}_{n}) - U(f,\alpha) \leq U(f,\alpha, {P}_{n}) - L(f,\alpha) \leq U(f,\alpha, {P}_{n}) - L(f,\alpha, {P}_{n}). \]
        Using the squeeze theorem on the inequality above, we have 
        \[  \lim_{ n \to \infty  } [L(f,\alpha) - L(f,\alpha,{P}_{n})] = 0.  \]
        So, 
        \[  \lim_{ n \to \infty  } L(f,\alpha,{P}_{n}) = L(f,\alpha) = \int_{ a }^{ b }  f  \ d \alpha. \]
\end{enumerate}
\end{proof}

\begin{theorem}[Algebraic Properties of R.S Integral]
    Assume \( f,g \in {R}_{\alpha}[a,b] \). Then
    \begin{enumerate}
        \item[(i)] \( \forall k \in \R  \), \( kf \in {R}_{\alpha}[a,b] \) with \[ \int_{ a }^{ b }  kf  \ d \alpha = k \int_{ a }^{ b }  f  \ d \alpha \];
        \item[(ii)] \( f + g \in {R}_{\alpha}[a,b] \) with 
            \[  \int_{ a }^{ b }  f + g  \ d \alpha = \int_{ a }^{ b }  f  \ d \alpha  + \int_{ a }^{ b }  g  \ d \alpha. \]
        \item[(iii-1)]  \( f^{2} \in {R}_{\alpha}[a,b] \);
        \item[(iii-2)] \( fg \in {R}_{\alpha}[a,b] \);
        \item[(iv-1)] if \( g \neq 0  \) on \( [a,b] \) and \( \frac{ 1 }{ g }  \) is bounded on \( [a,b] \), then \( \frac{ 1 }{ g }  \in {R}_{\alpha}[a,b] \);
        \item[(iv-2)] if \( g \neq 0  \) on \( [a,b] \) and \( \frac{ 1 }{ g }   \) is bounded on \( [a,b] \), then \( \frac{ 1 }{ g }  \in {R}_{\alpha}[a,b] \).
    \end{enumerate}
\end{theorem}

\begin{lemma}[lemma 3]\label{lemma 3}
   Let \( A  \) be a subset of \( \R  \) and \( f,g : A \in \R  \) be two bounded functions. Then 
   \begin{enumerate}
       \item[(i)] \( \sup_{A} (f+g) \leq \sup_{A} f + \sup_{A} g \);
       \item[(ii)] \( \inf_{A} (f+g) \geq \inf_{A} f + \inf_{A} g  \);
       \item[(iii-1)] \( \forall k \geq 0  \), \( \sup_{A} (kf) = k \sup_{A} f \);
       \item[(iii-2)] \(  \forall k \geq 0  \) \( \inf_{A} kf = k \inf_{A} f \);
       \item[(iv-1)] \( \forall k < 0  \) \( \sup_{A} kf = k \inf_{A} f \); 
       \item[(iv-2)] \( \forall k < 0  \) \( \inf_{A} kf = k \sup_{A} f \);
       \item[(v)] \( \sup_{x,y \in A}| f(x) - f(y) | = \sup_{A} f - \inf_{A} f \);
        \item[(vi)] If there exists a constant \( k > 0  \) such that 
            \[  \forall z,w \in A \ \ | f(z) - f(w)  | \leq k | g(z) - g(w) |,  \]
            then
            \[  \sup_{A} f - \inf_{A} f \leq k [\sup_{A} g - \inf_{A}  g].  \]
   \end{enumerate}  
\end{lemma}

\begin{lemma}[lemma 4]\label{lemma 4}
    Let \( f,g: [a,b] \to \R  \) be two bounded functions, \( \alpha: [a,b] \to \R  \) is an increasing function, and \( P \in \Pi[a,b] \). Then
    \begin{enumerate}
        \item[(i)] \( U(f + g,\alpha, P) \leq U(f,\alpha,P) + U(g,\alpha,P) \);
        \item[(ii)] \( L(f+g, \alpha, P) \geq L(f,\alpha,P) + U(g,\alpha, P) \);
        \item[(iii-1)] \( \forall k \geq 0  \) \( U(kf, \alpha, P) = k U(f,\alpha,P) \)
        \item[(iii-2)] \( \forall k \geq 0  \), \( L(kf,\alpha,P) = k L(f,\alpha,P) \);
        \item[(iv-1)] \( \forall k < 0  \) \( U(f,\alpha,P) = k L(f,\alpha,P) \)
        \item[(iv-2)] \( \forall k < 0  \) \( L(kf,\alpha, P) = k U(f,\alpha,P) \).
    \end{enumerate}
\end{lemma}

\begin{theorem}[Order Properties of R.S Integral]
    Assume \( f,g \in {R}_{\alpha}[a,b] \). Then
    \begin{enumerate}
        \item[(i)] If \( m \leq f(x) \leq M  \) for all \( x\in [a,b] \), then 
            \[  m (\alpha(b) - \alpha(a)) \leq \int_{ a }^{ b }  f  \ d \alpha \leq M(\alpha(b) - \alpha(a)). \]
        \item[(ii)] If \( f \leq g  \) on \( [a,b] \), then
            \[  \int_{ a }^{ b }  f  \ d \alpha \leq \int_{ a }^{ b } g   \ d \alpha. \]
    \end{enumerate}
\end{theorem}
\begin{proof}
\begin{enumerate}
    \item[(i)] Note that for any \( P \in \Pi[a,b] \), we have 
        \begin{align*}
            \int_{ a }^{ b }  f  \ d \alpha &= L(f,\alpha) \geq L(f,\alpha,P) \\
            \int_{ a }^{ b }  f \ d \alpha &= U(f,\alpha) \leq U(f,\alpha,P).
        \end{align*}
        In particular, for the partition \( P = \{ a,b \}  \), we have 
        \begin{align*}
            \int_{ a }^{ b }  f  \ d \alpha &\geq L(f,\alpha, P) = \Big(  \inf_{x \in [a,b]} f(x) \Big) (\alpha(b) - \alpha(a)) \geq m (\alpha(b) - \alpha(a)) \tag{1} \\
            \int_{ a }^{ b }  f  \ d \alpha &\leq U(f,\alpha,P) = \Big(  \sup_{x \in [a,b]} f(x) \Big) (\alpha(b) - \alpha(a)) \leq M (\alpha(b) - \alpha(a)). \tag{2}
        \end{align*}
        Using (1) and (2), we obtain our desired result.
    \item[(ii)] Let \( h = g - f  \). We have \( h \geq 0  \), so, by part (i), we have
        \[  0 (\alpha(b) - \alpha(a)) \leq \int_{ a }^{ b }  h  \ d \alpha. \]
        Therefore, 
        \begin{align*}
            0 \leq \int_{ a }^{ b } h \ d \alpha &= \int_{ a }^{ b } g - f  \ d \alpha = \int_{ a }^{ b }  g  \ d \alpha - \int_{ a }^{ b }  f  \ d \alpha \\
                                                 &\implies \int_{ a }^{ b } f  \ d \alpha \leq \int_{ a }^{ b }  g  \ d \alpha.
        \end{align*}
\end{enumerate}
\end{proof}

\begin{theorem}[Triangle Inequality of Integrals]
    Assume \( f \in {R}_{\alpha}[a,b] \). Then
    \begin{enumerate}
        \item[(i)] \( | f |  \in {R}_{\alpha}[a,b] \);
        \item[(ii)] \( \displaystyle \Big| \int_{ a }^{ b }  f  \ d \alpha \Big|  \leq \int_{ a }^{ b }  | f |  \ d \alpha \).
    \end{enumerate}
\end{theorem}
\begin{proof}
\begin{enumerate}
    \item[(i)] Define \( \varphi: \R \to \R  \) by \( \varphi(x) = | x  |  \) which is clearly continuous on \( \R  \). Since \( f \in {R}_{\alpha}[a,b] \), it follows from {\hyperref[Rudin 6.11]{Rudin 6.11}} that \( \varphi \circ f \in {R}_{\alpha}[a,b] \). Hence, we have \( | f |  \in {R}_{\alpha}[a,b]  \).  
    \item[(ii)] Recall that 
        \[  | t  |  \leq s \iff - s \leq t \leq s.  \]
        So, our goal is to show that 
        \[ - \int_{ a }^{ b }  | f  |  \ d \alpha \leq \int_{ a }^{ b }  f  \ d \alpha \leq \int_{ a }^{ b }  | f  |  \ d \alpha. \]
        Also, we have 
        \[ - | f(x) |  \leq f(x) \leq | f(x) |  \ \ \forall x \in [a,b].  \]
        So, 
        \[  -\int_{ a }^{ b }  | f(x) |  \ d \alpha \leq \int_{ a }^{ b }  f(x) \ d \alpha \leq \int_{ a }^{ b }  | f(x) |  \ d \alpha \]
        as desired.
\end{enumerate}
\end{proof}

\begin{theorem}[Mean Value Theorem for Integrals]
    Let \( f:[a,b] \to \R  \) be a continuous function, \( \alpha:[a,b] \to \R  \) is an increasing function and \( \alpha(a) \neq \alpha(b) \). Then there exists \( c \in [a,b] \) such that 
    \[  f(c) = \frac{ 1 }{  \alpha(b) - \alpha(a) }   \int_{ a }^{ b }  f  \ d \alpha. \]
\end{theorem}
\begin{proof}
    Since \( f  \) is continuous on \( [a,b] \) and \( [a,b] \) is a compact interval in \( \R  \), it follows from the Extreme Value Theorem that \( f  \) attains its max and min on \( [a,b] \). Let 
    \[  m = \min_{x \in [a,b]} f(x) \ \ \text{and} \ \ M = \max_{x \in [a,b}] f(x). \]
    We have, for all \( x \in [a,b] \), \( m \leq f(x) \leq M  \). Thus, 
    \[  m(\alpha(b) - \alpha(a)) \leq \int_{ a }^{ b }  f(x) \ d \alpha \leq M(\alpha(b) - \alpha(a)).\]
    Hence, 
    \[  m \leq \frac{ 1 }{  \alpha(b) - \alpha(a)  }  \int_{ a }^{ b }  f  \ d \alpha \leq M.  \]
    Using the Intermediate Value Theorem, we see from the assumption that \( f  \) being continuous on \( [a,b]  \) that
    \[  \exists c \in [a,b] \ \text{such that} \ f(c) = \frac{ 1 }{ \alpha(b) - \alpha(a) } \int_{ a }^{ b }  f \ d \alpha. \]
\end{proof}

\begin{theorem}[Generalized Mean Value Theorme for Integrals]
    Let \( f:[a,b] \to \R  \) be continuous, \( \alpha:[a,b] \to \R  \) is increasing, and \( g \in {R}_{\alpha}[a,b]  \) and either \( g \geq 0  \) on \( [a,b] \) or \( g \leq 0  \) on \( [a,b] \). Then
    \[  \exists c \in [a,b] \ \text{such that} \ \int_{ a }^{ b }  fg  \ d \alpha = f(c) \int_{ a }^{ b }  g  \ d \alpha. \]
\end{theorem}

\begin{theorem}[Additivity for R.S Integrals]
    Let \( f:[a,b]  \to \R \) be continuous, \( \alpha:[a,b] \to \R  \) is increasing, \( c \in (a,b) \). Then
    \[  f \in {R}_{\alpha}[a,b] \iff (f \in {R}_{\alpha}[a,c] \ \text{and} \ f \in {R}_{\alpha}[c,b]). \]
    In this case, we have 
    \[  \int_{ a }^{ b }  f  \ d \alpha = \int_{ a }^{ c }  f  \ d \alpha + \int_{ c }^{ b  }  f \ d \alpha. \]
\end{theorem}

\section{Lectures 13-14}

\subsection{Topics}

\begin{itemize}
    \item Theorem: For "nice" \( \alpha \) we have \( \displaystyle \int_{ a }^{ b }  f  \ d \alpha  = \int_{ a }^{ b }  f(x) \alpha'(x) \ dx \);
    \item Theorem (change of variable)
    \item The Fundamental Theorem of Calculus
    \item Integration By Parts
    \item Unit step function, representing sums by R.S integrals
\end{itemize}

\begin{lemma}
    Let \( f: [a,b] \to \R  \) be a bounded function, \( \alpha: [a,b] \to \R  \) is an increasing function, \( P = \{ {x}_{0}, {x}_{2}, \dots, {x}_{n} \}  \) is a partition of \( [a,b] \) and \( R \in \R  \). Then  
    \begin{enumerate}
        \item[(1)] If for all tags \( ({s}_{k })_{1 \leq k \leq n }  \) of \( P  \), we have \( \sum_{ k=1  }^{ n } f({s}_{k}) \Delta {\alpha}_{k } \leq R  \), then \( U(f,\alpha,P) \leq R  \).
        \item[(2)] If for all tags \( ({s}_{k})_{1 \leq k \leq n } \) of \( P  \), we have \( R \leq \sum_{ k=1  }^{ n } f({s}_{k}) \Delta {\alpha}_{k } \), then \( R \leq L(f,\alpha, P) \).
    \end{enumerate}
\end{lemma}

\begin{proof}
\begin{enumerate}
    \item[(1)] If \(  \alpha \) is constant, then 
        \[  \sum_{ k=1  }^{ n } f({s}_{k}) \Delta {\alpha}_{k } = 0   \]
        which implies 
        \[  U(f,\alpha,P) = \sum_{ k=1  }^{ n } {M}_{k } \Delta {\alpha}_{k } = 0. \]
        So, we may assume that \( \alpha(a) \neq \alpha(b) \). It is suffices to show that 
        \[  \forall \epsilon > 0  \ \ U(f,\alpha, P) \leq R + \epsilon. \]
        Let \( \epsilon > 0  \) be given. For each \( k \in \{ 1, \dots, n \}  \), we have
        \[  {M}_{k } = \sup_{x \in [{x}_{k-1}, {x}_{k}]} f(x) \implies \exists {s}_{k } \in [{x}_{k-1}, {x}_{k}] \ \text{such that} \ {M}_{k }  - \frac{ \epsilon }{  \alpha(b) - \alpha(a) } < f({s}_{k}). \]
        We have
        \begin{align*}
            U(f,\alpha,P) &- \sum_{ k=1  }^{ n } {M}_{k } \Delta  {\alpha}_{k } < \sum_{ k=1  }^{ n } \Big[f({s}_{k}) + \frac{ \epsilon }{  \alpha(b) - \alpha(a) } \Big] \Delta {\alpha}_{k } \\
                          &= \sum_{ k=1  }^{ n } f({s}_{k}) \Delta {\alpha}_{k } + \frac{ \epsilon }{  \alpha(b) - \alpha(a) }  \sum_{ k=1 }^{ n  } \Delta {\alpha}_{k }  \\
                          &\leq R + \epsilon
        \end{align*}
        as desired.
    \item[(2)] Completely analogous to (1).
\end{enumerate}
\end{proof}

\begin{theorem}[Rudin 6.17]\label{Rudin 6.17}
    Let \( f:[a,b] \to \R  \) be a bounded function, \( \alpha: [a,b] \to \R  \) be an increasing function, and \( \alpha' \in R[a,b] \). Then
    \[  f \in {R}_{\alpha}[a,b] \iff f \alpha' \in R[a,b] \]
    and in this case, 
    \[  \int_{ a }^{ b } f  \ d \alpha = \int_{ a }^{ b }  f(x) \alpha'(x) \ dx. \]
\end{theorem}
\begin{proof}
It suffices to show that 
\begin{align*}
    U(f,\alpha) &= U(f \alpha'), \\
    L(f,\alpha) &= L(f \alpha')
\end{align*}
Indeed, if we prove (*), then
\begin{align*}
    f \in {R}_{\alpha}[a,b] &\iff U(f,\alpha) = L(f,\alpha)  \\
                            &\iff U(f \alpha') = L(f \alpha') \\
                            &\iff f \alpha' \in R[a,b].
\end{align*}
Moreover, (*) would imply that
\[  \int_{ a }^{ b }  f  \ d \alpha = U(f,\alpha) = U(f \alpha') = \int_{ a }^{ b }  f(x) \alpha'(x)  \ dx.  \]
In what follows, we will prove \( U(f,\alpha) = U(f \alpha') \). The proof of \( L(f,\alpha) = L(f \alpha') \) is analogous. Let \( P = \{ {x}_{0}, {x}_{1}, \dots , {x}_{n} \}  \) be any partition of \( [a,b] \). Let \( ({s}_{k})  \) be any tag of \( P  \). 
Note that by the Mean Value Theorem, we can find a \( {t}_{k} \in ({x}_{k-1}, {x}_{k}) \) for all \( 1 \leq k \leq n  \) such that
\begin{align*}
    \Delta {\alpha}_{k } &= \alpha({x}_{k}) - \alpha({x}_{k-1}) \\
                         &= \alpha'({t}_{k}) ({x}_{k} - {x}_{k-1}).
\end{align*}
We have
\begin{align*}
    &\Big| \sum_{ k=1  }^{ n } f({s}_{k}) \Delta {\alpha}_{k } - \sum_{ k=1  }^{ n } f({s}_{k}) \alpha'({s}_{k}) \Delta {x}_{k} \Big|   \\
    &= \Big| \sum_{ k=1  }^{ n } f({s}_{k}) \alpha'({t}_{k}) \Delta {x}_{k} - \sum_{ k=1  }^{ n } f({s}_{k}) [\alpha'({t}_{k}) - \alpha'({s}_{k})] \Delta {x}_{k} \Big| \\
    &\leq \sum_{ k=1  }^{ n } | f({s}_{k}) | | \alpha'({t}_{k}) - \alpha'({s}_{k}) |  \Delta {x}_{k} \\
    &\leq \hat{M} \sum_{ k=1  }^{ n } | \alpha'({t}_{k}) - \alpha'({s}_{k}) | \Delta {x}_{k} \tag{\( \hat{M} = \sup_{x \in [a,b]} | f(x) |  \)} \\
    &\leq \hat{M} \sum_{ k=1  }^{ n } [\sup_{{I}_{k}} \alpha' - \inf_{{I}_{k}} \alpha'] \Delta {x}_{k} \tag{lemma 1} \\
    &= \hat{M} [U(\alpha',P) - L(\alpha',P)].
\end{align*}
Hence, we have
\[ \Big| \sum_{ k=1  }^{ n } f({s}_{k}) \Delta {\alpha}_{k } - \sum_{ k=1  }^{ n } f({s}_{k}) \alpha'({s}_{k}) \Delta {x}_{k} \Big| \leq \hat{M} [U(\alpha',P) - L(\alpha', P)]. \]
Therefore, 
\begin{align*}
    \sum_{ k=1  }^{ n } f({s}_{k}) \Delta {\alpha}_{k } &\leq \sum_{ k=1  }^{ n } f({s}_{k}) \alpha'({s}_{k}) \Delta {x}_{k} + \hat{M} [U(\alpha', P) - L(\alpha',P)] \\
                                                        &\leq U(f\alpha', P) + \hat{M} [U(\alpha',P) - L(\alpha',P)]. \tag{1}
\end{align*}
By Lemma 5, we have
\[  U(f \alpha', P) \leq U(f,\alpha, P) + \hat{M} [U(\alpha',P) - L(\alpha',P)]. \tag{3} \]
It follows from (1) and (2) that
\[  | U(f,\alpha,P) - U(f \alpha', P)   |  \leq \hat{M} [U(\alpha',P) - L(\alpha', P )]. \]
Note that 
\begin{align*}
    U(f,\alpha) = \inf_{P \in \Pi} U(f,\alpha,P) &\implies \exists ({P}_{n}^{(1)}) \subseteq  \Pi \ \text{such that} \  \\
    U(f \alpha') = \inf_{P \in \Pi} U(f \alpha', P) &\implies \exists ({P}_{n}^{(2)}) \subseteq \Pi \ \text{such that} \ \lim_{ n \to \infty  }  U(f \alpha', {P}_{n}^{(2)}) = U(f \alpha'). \tag{2}
\end{align*}

Since \( \alpha' \in R[a,b] \), there exists \( ({P}_{n}^{(3)}) \subseteq \Pi  \) such that 
\[  \lim_{ n \to \infty } [U(\alpha', {P}_{n}^{(3)}) - L(\alpha', {P}_{n}^{(3)})]  = 0.   \]
Now, for each \( n \in \N \), let \( {P}_{n} = {P}_{n}^{(1)} \cup {P}_{n}^{(2)} \cup {P}_{n}^{(3)} \). We have
\begin{align*}
    &\forall n \geq 1 \ \  U(f,\alpha) \leq U(f,\alpha,{P}_{n}) \leq U(f,\alpha, {P}_{n}^{(1)}) \implies \lim_{ n \to \infty  } U(f,\alpha, {P}_{n})  = U(f,\alpha) \tag{4} \\
    &\forall n \geq 1 \ \ U(f \alpha') \leq U(f \alpha' , {P}_{n}) \leq U(f \alpha' , {P}_{n}^{2}) \implies \lim_{ n \to \infty  }  U(f \alpha', {P}_{n}) = U(f \alpha'). \tag{5}
\end{align*}
Since \( {P}_{n} \) is a refinement of \( {P}_{n}^{(3)} \), we have
\begin{align*}
     &0 \leq [U(\alpha', {P}_{n}) - L(\alpha',{P}_{n})] \leq U(\alpha' , {P}_{n}^{(3)} - L(\alpha' {P}_{n}^{(3)})) \\
     &\implies \lim_{ n \to \infty  } U(\alpha', {P}_{n}) - L(\alpha', {P}_{n}) = 0. \tag{6} 
\end{align*}
It follows from (3) that
\[  \forall n \geq 1 \ \ 0 \leq U(f,\alpha, {P}_{n}) - U(f \alpha' , {P}_{n}) \leq \hat{M} [U(\alpha', {P}_{n}) - L(\alpha', {P}_{n})] \]
Applying the squeeze theorem as \( n \to \infty   \) to both sides of the inequality above, we have
\begin{align*}
&\lim_{ n \to \infty   } | U(f,\alpha, {P}_{n}) - U(f \alpha' , {P}_{n}) |  = 0   \\
&\implies \Big| \lim_{ n \to \infty  }  (U(f,\alpha, {P}_{n}) - U(f \alpha', {P}_{n})) \Big| = 0 \\
&\implies | U(f,\alpha) - U(f \alpha') |  = 0 \\
&\implies U(f,\alpha) - U(f \alpha') = 0 \\
&\implies U(f,\alpha) = U(f \alpha')
\end{align*}
\end{proof}

\begin{theorem}[Rudin 6.19; Change of Variable]
    Let \( f \in {R}_{\alpha}[a,b]  \) and \( \varphi : [A,B] \to [a,b] \) be an onto and strictly increasing function. If we let \( g = f \circ \varphi  \) and \( \beta = \alpha \circ \varphi  \), then
    \[  g \in {R}_{\beta}[A,B] \ \ \text{and} \ \ \int_{ a }^{ b }  f  \ d \alpha = \int_{ A  }^{  B  }  g  \ d \beta. \]
\end{theorem}
\begin{proof}
    Since \( \alpha:[a,b] \to \R  \) is increasing and \( \varphi : [A,B] \to [a,b]  \) is an increasing, we see that 
    \(  \beta = \alpha \circ \varphi  \)
    is also increasing on \( [A,B] \). Also, note that, there is a one-to-one correspondence between \( \Pi [a,b] \) and \( \Pi [A,B] \):
    \[  H: \Pi [a,b] \to \Pi [A,B] \]
    where \( P = \{ {x}_{0}, {x}_{1}, \dots, {x}_{n} \}  \) corresponding to \( [a,b] \) gets mapped to \( Q = \{ {y}_{0}, {y}_{1}, \dots, {y}_{n} \}  \) corresponding to \( [A,B] \). Under this 1-1 correspondence, we have
    \begin{align*}
        \forall 1 \leq k \leq n \ \ \varphi([{y}_{k-1}, {y}_{k}]) = [{x}_{k-1}, {x}_{k}].
    \end{align*}
    and
    \[  \forall 1 \leq k \leq n \ \ \Delta {\beta}_{k } = \beta({y}_{k}) - \beta({y}_{k-1}) = \alpha(\varphi({y}_{k})) - \alpha(\varphi({y}_{k-1})) = \alpha({x}_{k}) - \alpha({x}_{k-1}) = \Delta {\alpha}_{k }. \]
    and
    \begin{align*}
        \forall 1 \leq k \leq n  \ \ {M}_{k }^{(g)} = \sup_{y \in [{y}_{k-1}, {y}_{k}]} g(y) &= \sup_{y \in [{y}_{k-1}, {y}_{k}]} f \circ \varphi (y) \\
                                                                                             &= \sup_{x \in [{x}_{k-1}, {x}_{k}]} f(x) \\
                                                                                             &= {M}_{k}^{(f)}
    \end{align*}
    Thus, under the correspondence above, we have 
    \begin{align*}
        U(f,\alpha,P) &= U(g, \beta, Q ) \\
        L(f,\alpha,P) &= L(g, \beta, Q )
    \end{align*}
    where \( H(P) = Q  \). In order to show \( g \in {R}_{\beta}[A,B] \), by the Cauchy-Criterion, it suffices to show that 
    \[  \forall \epsilon > 0 \ \ \exists Q \in \Pi [A,B] \ \text{such that} \ U(g,\beta,Q) - L(g,\beta,Q) < \epsilon. \] 
    Let \( \epsilon > 0  \) be given. Since \( f \in {R}_{\alpha}[a,b] \), there exists \( P \in \Pi[a,b] \) such that 
    \[  U(f,\alpha,P) - L(f,\alpha,P) < \epsilon. \]
    We claim that \( Q = H(P) \) can be used as the partition that we were looking for. Indeed, 
    \[  U(g, \beta, H(P)) - L(g, \beta, H(P)) - L(g, \beta, H(P)) = U(f,\alpha,P) - L(f,\alpha,P) < \epsilon \]
    as desired.
    Also, 
    \begin{align*}
        \int_{ a }^{ b }  f  \ d \alpha &= L(f,\alpha) = \sup_{P \in \Pi [a,b]} L(f,\alpha,P) \\
                                        &= \sup_{Q \in \Pi[a,b]} L(g,\beta,Q) \\
                                        &= L(g,\beta) \\
                                        &= \int_{ A }^{ B }  g  \ d \beta.
    \end{align*}
\end{proof}

\begin{theorem}[Fundamental Theorem of Calculus I and II]
    \textbf{(Part I)} Let \( f : [a,b] \to \R  \) be integrable and \( F: [a,b] \to \R  \) satisfies \( F'(x) = f(x) \) for all \( x \in [a,b] \). Then
    \[  \int_{ a }^{ b }  f  \ d x = F(b) - F(a). \]

    \textbf{(Part II)} Let \( g : [a,b] \to \R  \) be Riemann Integrable and \( G: [a,b] \to \R  \) is defined by \( G(x) = \displaystyle \int_{ a }^{ x }  g(t) \ dt  \). Then
    \begin{enumerate}
        \item[(1)] \( G  \) is (uniformly) continuous on \( [a,b] \);
        \item[(2)] If \( g  \) is continuous at a point \( c \in [a,b] \), then \( G  \) is differentiable at the point \( c  \) and \( G'(c) = g(c) \) (In particular, if \( g:[a,b] \to \R  \) is continuous, then \( G(x) = \displaystyle \int_{ a }^{ x }  g(t) \ dt  \) is an antiderivative of \( g  \) on \( [a,b] \)).
    \end{enumerate}


\end{theorem}

    \begin{proof}
    \begin{enumerate}
        \item[(I)] In what follows, we will show that 
            \[  \forall P \in \Pi[a,b] \ \ L(f,P) \leq F(b) - f(a) \leq U(f,P). \tag{*} \]
            Note that as a consequence of (*):
            \begin{enumerate}
                \item[(i)] \( F(b) - F(a) \) is an upper bound for \( \{ L(f,P) : P \in \Pi \}  \). So,
                    \[  \sup_{P \in \Pi} L(f,P) \leq F(b) - F(a) \implies L(f) \leq F(b) - F(a). \]
                \item[(ii)] \( F(b) - F(a)  \) is a lower bound for \( \{ U(f,P) : P \in \Pi \}  \). So, 
                    \[  \inf_{P \in \Pi} U(f,P) \geq F(b) - F(a) \implies U(f) \geq F(b) - F(a). \]
                    Thus, 
                    \[  L(f) \leq F(b) - F(a) \leq U(f). \]
                    Since \( f \in R[a,b] \), \( L(f) = U(f) = \displaystyle \int_{ a }^{ b } f   \), so 
                    \[  \int_{ a }^{ b }  f  \leq F(b) -F(a) \leq \int_{ a }^{ b }  f.   \]
                    Therefore, 
                    \[  \int_{ a }^{ b } f  = F(b) - F(a) \]
                   which is our desired result. 

                   Let \( P = \{ {x}_{0}, {x}_{1}, \dots, {x}_{n} \}  \) be a partition of \( [a,b] \). Since \( F' = f  \) on \( [a,b] \), we can use the Mean Value Theorem to find a \( {t}_{k } \in ({x}_{k-1},{x}_{k}) \) such that 
                   \[  F'({t}_{k}) = f({t}_{k}) = \frac{ f({x}_{k}) - f({x}_{k-1}) }{ {x}_{k} - {x}_{k-1} }.  \]
                   Hence, we have 
                   \begin{align*}
                       F(b) - F(a) &= \sum_{ k=1  }^{ n } [F({x}_{k}) - F({x}_{k-1})] = \sum_{ k=1  }^{ n } F'({t}_{k}) ({x}_{k} - {x}_{k-1}) \\
                                   &= \sum_{ k=1 }^{ n } f({t}_{k}) \Delta {x}_{k}.
                   \end{align*}
                   Therefore, it follows from
                   \[  L(f,P) = \sum_{ k=1  }^{ n } {m}_{k } \Delta {x}_{k} \leq \sum_{ k=1  }^{ n } f({t}_{k}) \Delta {x}_{k} \leq \sum_{ k=1  }^{ n } {M}_{k} \Delta {x}_{k } = U(f,P) \]
                   that 
                   \[  L(f,P) \leq F(b) - F(a) \leq U(f,P). \]
            \end{enumerate}
        \item[(II)] 
            \begin{enumerate}
                \item[(i)] Our goal is to show \( G  \) is uniformly continuous on \( [a,b] \). That is, we need to show 
                    \[  \forall \epsilon > 0 \ \exists \delta > 0 \ \text{such that} \ \forall x,y \in [a,b] \ \text{if} \ | x - y |  < \delta \ \text{then} \ | G(x) - G(y) |  < \epsilon.  \]
                    Let \( \epsilon > 0  \) be given. Let \( x,y \in [a,b] \). If \( x \geq y  \), then 
                    \begin{align*}
                        | G(x) - G(y) | &= \Big| \int_{ a }^{ x }  g(t) \ dt - \int_{ a }^{ y }  g(t) \ dt  \Big| = \Big| \int_{ y }^{ x }  g(t) \ dt  \Big|  \\
                                        &\leq \int_{ y }^{ x }  | g(t) |   \ dt \leq \int_{ y }^{ x }  R  \ dt = R(x-y) = R | x - y |
                    \end{align*}
                    where \( R = \sup_[t \in [a,b]] | g(t) |  \). If \( y > x  \), then
                    \begin{align*}
                        | G(x) - G(y)  | &= | G(y) - G(x) | =  \Big| \int_{ a }^{ y }  g(t) \ dt - \int_{ a }^{ x }  g(t) \ dt  \Big|  \\
                                         &= \Big| \int_{ x }^{ y }  g(t) \ dt  \Big|  \leq \int_{ x }^{ y }  | g(t) |  \ dt \leq \int_{ x }^{ y } R  \ dt \\
                                         &= R(y-x) = R| x - y   |. 
                    \end{align*}
                    Thus, for all \( x,y \in [a,b] \), we have 
                    \[  | G(x) - G(y) | \leq R | x - y  |.  \]
                    Hence, to make sure \( | G(x) - G(y) |  \) is less than \( \epsilon  \), it suffices to make \( R | x - y  |  \) less than \( \epsilon  \), that is, it is enough to ensure that \( | x - y  |  < \frac{ \epsilon }{ R  }  \). This argument shows that \( \delta = \frac{ \epsilon }{ R }  \) does the job. 
                \item[(ii)] Now, suppose \( g  \) is continuous at \( c \in [a,b] \). Our goal is to show that \( G'(c) = g(c) \). That is, we want to show 
                    \[  \lim_{ x \to c  }  \frac{ G(x)  - G(c) }{ x - c  }  = g(c). \]
                    That is, our goal is to show that 
                    \[  \forall \epsilon > 0 \ \exists \delta > 0 \ \text{such that} \ \text{if} \ 0 < | x - c  |  < \delta \ \text{(with \( x \in [a,b] \))} \ \text{then} \ \Big| \frac{ G(x) - G(c) }{ x - c  }  -  g(c) \Big|  < \epsilon. \]
                    Let \( \epsilon > 0  \) be given. Since \( g  \) is continuous at \( c  \), for this given \( \epsilon  \), there exists \( \hat{\delta} > 0  \) such that if
                    \[  | t - c  |  < \hat{\delta} \ \text{(with \( t \in [a,b] \))} \ \text{then} \ | g(t) - g(c) |  < \frac{ \epsilon }{ 2 }. \]
                    We claim that this \( \hat{\delta} \) can be used as the \( \delta  \) that we were looking for. Indeed, let \( \delta = \hat{\delta} \). We will consider the following two cases:
                    \textbf{(Case 1)} Suppose \( 0 < | x - c  |  < \hat{\delta} \), \( x \in [a,b] \), \( x > c  \). We have 
                    \begin{align*}
                        &\Big| \frac{ G(x) - G(c) }{  x-  c  }  - g(c) \Big| = \Big| \frac{ G(x) - G(c) - g(c) (x-c) }{ x - c  }  \Big|  \\
                        &= \Big| \frac{ \int_{ a }^{ x  }  g(t) \ dt - \int_{ a }^{ c }  g(t) \ dt - \int_{ c }^{ x }  g(c) \ dt  }{  x - c  }  \Big|  \\
                        &= \Big| \frac{ 1 }{ x - c  }  \Big(  \int_{ c }^{ x }  g(t) \ dt - \int_{ c }^{ x }  g(c) \ dt  \Big) \Big| \\
                        &= \Big| \frac{ 1 }{ x - c  } \int_{ c  }^{  x  } [g(t) - g(c)] \ dt  \Big| \\
                        &= \frac{ 1 }{ | x - c  |  }  \Big| \int_{ c }^{ x }  [g(t) - g(c)] \ dt   \Big|  \\
                        &\leq \frac{ 1 }{ x - c  }  \int_{ c }^{ x }  | g(t) - g(c) |  \ dt \\
                        &\leq \frac{ 1 }{ x - c  }  \int_{ c }^{ x }  \frac{ \epsilon }{ 2 }   \ dt \\
                        &= \frac{ 1 }{ x - c  }  \frac{ \epsilon }{ 2 }  (x - c ) \\
                        &= \frac{ \epsilon }{ 2 }  \\
                        &< \epsilon
                    \end{align*}
                    as desired. On the other hand, if \( 0 < | x - c  |  < \hat{\delta} \), \( x \in [a,b] \), and \( x < c  \), then a similar argument shows our desired result.

            \end{enumerate}
    \end{enumerate}

    \end{proof}

\begin{theorem}[Integration by Parts]
    Let \( u : [a,b] \to \R  \) and \( v : [a,b] \to \R  \) are differentiable and let \( u' \in R[a,b] \) and \( v' \in R[a,b] \). Then we have  
    \begin{enumerate}
        \item[(1)] \( uv' \in R [a,b] \)
        \item[(2)] \( u' v \in R[a,b] \)
        \item[(3)] \( \displaystyle  \int_{ a }^{ b }  uv'  \ dx = u(b)v(b) - u(a) v(a) - \displaystyle \int_{ a }^{ b } u' v  \ dx  \).
    \end{enumerate}
\end{theorem}

\begin{proof}
\begin{enumerate}
    \item[(1)] Since \( u:[a,b] \to \R  \) is differentiable, we have \( u \in C[a,b] \). So, we have \( u \in R[a,b] \). By assumption, \( v' \in R[a,b] \) and so we can conclude that \( uv' \in R[a,b] \).
    \item[(2)] Using the same argument above, we have \( uv' \in R[a,b] \).
    \item[(3)] By the product rule, we have 
        \[  (uv)' = u' v + u v'. \]
        In particular, since \( (uv)' \) is a sum of integrable functions, it belongs to \( R[a,b] \). Now, we integrate both sides 
        \[  \int_{ a }^{ b }  (uv)'   \ dx = \int_{ a }^{ b } u'v \ dx + \int_{ a }^{ b }  uv' \ dx. \tag{I} \]
        According to FTC I, we have
        \[  \int_{ a }^{ b } (uv)'  \ dx = [uv]_{x = a}^{x =b} = u(b)v(b) - u(a)v(a). \tag{II} \]
        Hence, we have (I) and (II) imply that
        \[  u(b)v(b) - u(a)v(a) = \int_{ a }^{ b } u'v \ dx + \int_{ a }^{ b }  u v' \ dx \]
        which further implies that 
        \[  \int_{ a }^{ b } uv' \ dx = u(b)v(b) - u(a)v(a) - \int_{ a }^{ b } u'v \ dx. \]
\end{enumerate}
\end{proof}

\begin{definition}[Unit Step Function]
    The \textbf{unit step function} \( I: \R \to \R  \) is defined by
    \[  I(x) = 
    \begin{cases}
        0 &\text{if} \ x \leq 0 \\
        1 &\text{if} x > 0 
    \end{cases}. \]
\end{definition}

\begin{remark}
    Note that for all \( s \in |R  \), we have 
    \[  I(x-s) = 
    \begin{cases}
        0 &\text{if} \ x \leq s \\
        1 &\text{if} x > s 
    \end{cases}. \]
    Also, for all \( c \neq 0  \), we have
    \[  c I(x-s) = 
    \begin{cases}
        0 &\text{if} \ x \leq s \\
        c &\text{if} \ x > s.
    \end{cases} \]
\end{remark}

\begin{theorem}[Rudin 6.15]
    Let \( f: [a,b] \to \R  \) be a bounded function, \( s \in (a,b) \), \( f  \) is right continuous at \( s  \), and \( \alpha(x) = I(x-s) \). Then 
    \begin{center}
    \( f \in {R}_{\alpha}[a,b] \) and \( \displaystyle \int_{ a }^{ b }  f  \ d \alpha  = f(s) \).
    \end{center}
\end{theorem}
\begin{proof}
see hw4
\end{proof}

\begin{theorem}[Rudin 6.16]
    Suppose for all \( n \geq 1  \), \( {c}_{n} \geq 0  \), \( \displaystyle \sum_{ n=1  }^{ N  } {c}_{n} \) converges, \( {s}_{1} < {s}_{2} < {s}_{3} < \cdots   \) are points in \( (a,b) \), \( \alpha:[a,b] \to \R  \) is continuous. Then
    \begin{center}
        \( f \in {R}_{\alpha}[a,b] \) and \( \displaystyle \int_{ a }^{ b } f  \ d \alpha = \displaystyle \sum_{ n=1  }^{ \infty  } {c}_{n} f({s}_{n})  \).
    \end{center}
\end{theorem}
\begin{proof}
See hw4.
\end{proof}

