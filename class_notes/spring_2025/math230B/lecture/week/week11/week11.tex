\subsection{Lecture 19}

\subsubsection{Topics}

\begin{itemize}
    \item Series of functions
    \item Cauchy Criterion for Uniform Convergence of Series
    \item Weiertstrass M-Test
\end{itemize}


\begin{definition}[Infinite Series of Funtions]\label{Infinite Series of Functions}
    Let \( ({f}_{n} : A \to \R )_{n \geq 1} \) be a sequence of functions on a nonempty set \( A  \). 
    \begin{enumerate}
        \item[(*)] An expression of the form \( \sum_{ n=1  }^{ \infty  } {f}_{n} = {f}_{1} + {f}_{2} + {f}_{3} + \cdots  \) is called an \textbf{infinite series of functions}.
        \item[(*)] The functions \( {f}_{1}, {f}_{2}, \dots \) are the terms of the series. 
        \item[(*)] The corresponding  sequence of partial sums is definitely 
            \[  {s}_{m} = \text{(finite) sum of the first \( m  \) terms of the series} \]
            that is, 
            \begin{align*}
                {s}_{1}(x) &= {f}_{1}(x) \\
                {s}_{2}(x) &= {f}_{1}(x) + {f}_{2}(x) \\
                {s}_{3}(x) &= {f}_{1}(x) + {f}_{2}(x) + {f}_{3}(x) \\
                           &\vdots \\
                {s}_{m}(x) &= {f}_{1}(x) + \cdots + {f}_{m}(x) \\
                           &\vdots 
            \end{align*}
        \item[(*)] We say that the series \textbf{converges pointwise} on \( A  \) to \( f: A \to \R  \) (and we write \( \sum_{ n=1  }^{ \infty  } {f}_{n} = f  \) pointwise) if \( \lim_{ m \to \infty  }  {s}_{m} = f  \) pointwise on \( A \).
        \item[(*)] We say that the series \textbf{converges uniformly} on \( A  \) to \( f: A \to \R  \) (and we write \( \sum_{ n=1  }^{ \infty  } {f}_{n} = f  \) uniformly) if \( {s}_{m} \to f  \) uniformly on \( A  \). 
    \end{enumerate}
\end{definition}


\begin{theorem}[Term-by-Term Continuity Theorem]\label{Term-by-Term Continuity Theorem}
Let \( A \subseteq  (X,d)  \) be nonempty. Suppose for all \( n \in \N \) \( {f}_{n} : A \to \R  \) is a sequence of continuous functions, and \( \sum_{ n=1  }^{ \infty  } {f}_{n} \) converges uniformly to \( f: A \to \R  \). Then \( f: A \to \R  \) is continuous.    
\end{theorem}
\begin{proof}
Applying the corresponding theorem for sequences of functions to the sequence of partial sums \( {s}_{m} = {f}_{1} + \cdots + {f}_{n} \). That is, 
\[  \sum_{ n=1  }^{ \infty  } {f}_{n} = f \implies {s}_{m} \to f \ \text{uniformly} \implies f \ \text{is continuous} \]
since \( {s}_{m}  \) is continuous.
\end{proof}

\begin{theorem}[Term-by-Term Differentiability Theorem]\label{Term-by-Term Differentiability Theorem}
    Assume for each \( n \in \N  \), \( {f}_{n} : [a,b] \to \R  \) is a sequence of differentiable functions, \( \sum_{ n=1  }^{ \infty  } {f}_{n} = f  \) pointwise on \( [a,b] \), and \( \sum_{ n=1  }^{ \infty  } {f}_{n}'  \) converges uniformly on \( [a,b] \). Then
    \( f  \) is differentiable on \( [a,b] \) and 
    \[  \Big(  \sum_{ n=1  }^{ \infty  } {f}_{n} \Big)' = \sum_{ n=1  }^{ \infty  } {f}_{n}'. \]
\end{theorem}
\begin{proof}
Apply the corresponding theorem for sequences of functions to the sequence of partial sums \( {s}_{m} = {f}_{1} + \cdots + {f}_{m} \).
\end{proof}

\begin{theorem}[Term-by-Term Integrability]\label{Term-by-Term Integrability}
    Let \( \alpha: [a,b] \to \R  \) is an increasing function, for each \( n \geq 1  \), \( {f}_{n} \in {R}_{\alpha}[a,b] \), and \( \sum_{ n=1  }^{ \infty  } {f}_{n} \) converges uniformly to \( f: [a,b] \to \R   \). Then 
    \[  f \in {R}_{\alpha}[a,b] \ \ \text{and} \ \ \int_{ a }^{ b }  \sum_{ n=1  }^{ \infty  } {f}_{n} \ d \alpha = \sum_{ n=1  }^{ \infty  } \int_{ a }^{ b }  {f}_{n} \ d \alpha. \]
\end{theorem}
\begin{proof}
Apply the corresponding theorem for sequences of functions to the sequence of partial sums \( {s}_{m} = {f}_{1} + \cdots + {f}_{m} \).
\end{proof}

\begin{theorem}[Cauchy Criterion for Uniform Convergence of Series of Functions]\label{Cauchy Criterion for Uniform Convergence of Series of Functions}
    Let \( A  \) be a nonempty set and suppose for each \( k \in \N  \), \( {f}_{k } : A \to \R  \). Then
    \begin{center}
        \( \displaystyle \sum_{ k=1  }^{ \infty  } {f}_{k } \) converges uniformly if and only if for all \(  \epsilon > 0  \), there exists an \( N  \) such that for all \( n > m > N  \) and for all \( x \in A  \), \( \displaystyle \Big| \sum_{ k=1  }^{ n } {f}_{k } (x) \Big| < \epsilon \).
    \end{center}
\end{theorem}

\begin{theorem}[Weierstrass M-Test]\label{Weierstrass M-Test}
    Let \( A  \) be a nonempty set, for all \( n \in \N  \) \( {f}_{n} : A \to \R  \), for all \(  n \in \N  \), there exists \( {M}_{n} \) such that for all \( x \in A  \), \( | {f}_{n}(x) |  \leq {M}_{n} \), and \( \displaystyle \sum_{ n=1  }^{ \infty  } {M}_{n}  \) converges. Then 
    \[  \sum_{ n=1  }^{ \infty  } {f}_{n} \ \ \text{converges uniformly on} \ A.  \]
\end{theorem}
\begin{proof}
    According to the Cauchy Criterion for uniform convergence of series of functions, it suffices to show that for all \( \epsilon > 0  \), there exists \( N  \) such that for all \( n > m > N  \)  and for all \( x \in A  \)
    \[  \Big| \sum_{ k=m+1  }^{ n  } {f}_{k } (x) \Big|  < \epsilon. \tag{*} \]
    Let \( \epsilon > 0 \).
    Note, by assumption, \( \displaystyle \sum_{ n=1  }^{ \infty  } {M}_{n} \) converges. Thus, for our given \( \epsilon  \), there exists \( \hat{N} \) such that 
    \[  \forall m > m > \hat{N} \ \ \Big| \sum_{ k = m + 1  }^{ n } {M}_{k }  \Big|  < \epsilon. \]
    We claim that we can use this \( \hat{N} \) as the \( N  \) that we were looking for. Indeed, if we let \( N = \hat{N} \), then (*) will hold because for all \( n > m > \hat{N} \) and for all \( x \in A  \)
    \[ \Big| \sum_{ k= m+1 }^{  n  } {f}_{k }(x) \Big|  \leq \sum_{ k = m + 1  }^{ n } | {f}_{k }(x) | \leq \sum_{ k= m + 1  }^{ n } {M}_{k } < \epsilon  \]
    as desired.
\end{proof}

\begin{center}
    \textit{End of Lecture 19} 
\end{center}



