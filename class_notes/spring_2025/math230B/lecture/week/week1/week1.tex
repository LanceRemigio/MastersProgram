% \documentclass[a4paper]{article}
% \input{../../../../../../newpreamble.tex}
% \begin{document}

\section{Lecture 1}

\subsection{Topics}

\begin{itemize}
    \item The derivative
    \item Continuity and Differentiability
    \item Differentiability Rules
\end{itemize}

\begin{definition}[Differentiability]
    \begin{enumerate}
        \item[(*)] Let \( I \subseteq  \R  \) be an interval, \( f: I \to \R  \), \( c \in I  \). We say \( f  \) is \textbf{differentiable} at \( c  \) if 
    \[  \lim_{ x \to c }  \frac{ f(x) -f(c) }{ x - c  } \]
    exists (that is, it equals a real number).
    \item[(*)] In this case, the quantity \( \lim_{ x \to c }  \frac{ f(x) - f(c) }{ x - c  }  \) is called the derivative of \( f  \) at \( c  \) and is denoted by
        \[  f'(c), \frac{ df }{ dx }(c), \frac{ df }{ dx } \Bigg|_{x = c} \]
    \item[(*)] If \( f: I \to \R  \) is differentiable at every point \( c \in I  \), we say \( f \) is differentiable (on \( I  \)).
    \end{enumerate}
\end{definition}

\begin{remark}
    The following are equivalent characterizations of the differentiability:
    \begin{align*}
        f'(c) = L &\iff \lim_{ x \to c }  \frac{ f(x) - f(c) }{ x - c  }  = L  \\
                  &\iff \forall \epsilon > 0 \ \exists \delta > 0 \ \text{such that} \ \text{if} \ 0 < | x - c  | < \delta \ \text{then} \ \Big| \frac{ f(x) - f(c) }{ x - c  }  - L  \Big|  < \epsilon \\
                  &\iff \forall \epsilon > 0 \ \exists \delta > 0 \ \text{such that if} \ 0 < | h  |  < \delta \ \text{then} \ \Big| \frac{ f(c+h) - f(c) }{  h  }  - L  \Big|  < \epsilon \\
                  &\iff \lim_{ h \to 0 }  \frac{ f(c+h) - f(c) }{  h  }  = L 
    \end{align*}
\end{remark}

\begin{theorem}[Differentiability Implies Continuous]
    Let \( I \subseteq  \R  \), \( c \in I  \), and \( f: I \to \R  \) is differentiable at \( c  \). Then \( f \) is continuous at \( c  \).
\end{theorem}

\begin{proof}
It suffices to show that \( \lim_{ x \to c } f(x) = f(c) \). Note that 
\begin{align*}
    \lim_{ x \to c } (f(x) - f(c)) &= \lim_{ x \to c }  \Big[ \frac{ f(x) - f(c) }{  x - c  }  \Big] (x - c) \\
                                   &= \Big[ \lim_{ x \to c }  \frac{ f(x) - f(c) }{  x - c  }  \Big] \Big[ \lim_{ x \to c }  (x - c) \Big] \\
                                   &= (f'(c)) (0) \\
                                   &= 0.
\end{align*}
So, we have 
\begin{align*}
    \lim_{ x \to c } f(x) &= \lim_{ x \to c }  [f(x)  - f(c) + f(c)] \\ 
                          &= \lim_{ x \to c } (f(x) - f(c)) + \lim_{ x \to c } f(c) \\   
                          &= 0 + \lim_{ x \to c } f(c) \\
                          &= 0 + f(c) \\
                          &= f(c).
\end{align*}
\end{proof}

\begin{corollary}
    If \( f: I \to \R  \) is NOT continuous at \( c \in I  \), then \( f \) is NOT differentiable at \( c  \).
\end{corollary}

\begin{eg}
    Let \( f: \R \to \R  \) be defined by 
    \[  f(x) = 
    \begin{cases}
        x^{2} &\text{if} \  x \in \Q \\
        0 &\text{if} \ x \notin \Q 
    \end{cases}. \]
\end{eg}

\begin{enumerate}
    \item[(i)] Prove that \( f  \) is continuous at \( 0  \).
        \begin{proof}
        Our goal is to show that 
        \[  \forall \epsilon > 0 \ \exists \delta > 0 \ \text{such that if} \ | x |  < \delta \ \text{then} \ | f(x) - f(c) | < \epsilon.  \]
        Let \( \epsilon > 0  \) be given. Note that if \( x \notin \Q  \),   
       \[  | f(x) |  = | 0  |  < \epsilon. \] 
       Otherwise, we have \( | f(x)  |  = | x^{2} |  = | x |^{2} \). IN this case, we claim that \( \delta = \sqrt{ \epsilon }  \) will work. Indeed, if \( | x  |  < \delta \), then we have 
       \[  | f(x) |  = | x |^{2} < (\sqrt{ \epsilon } )^{2} = \epsilon. \]
        \end{proof}
    \item[(ii)] Prove \( f \) is discontinuous at all \( x \neq 0  \).
        \begin{proof}
        Let \( c \neq 0  \). Our goal is to show that \( f  \) is discontinuous at \( c  \). By the sequential criterion for continuity, it suffices to find a sequence \( ({a}_{n}) \) such that \( {a}_{n} \to c  \) but \( f({a}_{n}) \not\to f(c) \). We will consider two cases; that is, we could either have \( c \notin \Q  \) or \( c \in \Q  \). 

        Suppose \( c \notin \Q  \). Since \( \Q  \) is dense in \( \R  \), there exists a sequence of rational numbers \( ({r}_{n}) \) such that \( {r}_{n} \to c  \). Note that \( f({r}_{n}) = {r}_{n}^{2} \to c^{2} \neq 0   \), but \( f(c) = 0  \). Clearly, \( f({r}_{n}) \not\to f(c) \) and so \( f  \) must be discontinuous at \( c  \).

        Suppose \( c \in \Q  \). Since the set of irrational numbers is also dense in \( \R  \), we can find a sequence \( ({s}_{n}) \) such that \( {s}_{n} \to c  \). Note that \( f({s}_{n})  =  0  \), but \( f(c) = c^{2} \neq 0  \). Thus, \( f({s}_{n}) \not\to f(c) \). Therefore, \( f  \) must be discontinuous at \( c  \). 
        \end{proof}
    \item[(iii)] Prove that \( f  \) is nondifferentiable at all \( x \neq 0  \).
        \begin{proof}
        Let \( c \neq 0  \). Since \( f  \) is discontinuous at \( c  \), we can conclude that \( f  \) is not differentiable at \( c  \).
        \end{proof}
    \item[(iv)] Prove that \( f'(0) = 0   \).
        \begin{proof}
        We need to show 
        \[  \lim_{ x \to c } \frac{ f(x) - f(0) }{ x - 0  }  = \frac{ f(x) }{ x  } = 0. \]
        \end{proof}
\end{enumerate}

\begin{theorem}[Algebraic Differentiability Theorem]
    Assume that \( f: I \to \R  \) and \( g : I \to \R  \) are differentiable at \( c \in I  \) where (\( I  \) is an interval on \( \R  \)). Then
    \begin{enumerate}
        \item[(i)] For all \( k \in \R  \), \( kf  \) is differentiable at \( c  \), and
            \[  (kf)'(c) = k f'(c) \]
        \item[(ii)] \( f+ g  \) is differentiable at \( c  \), and 
            \[  (f+gk)'(c) = f'(c) + g'(c) \]
        \item[(iii)] \( fg  \) is differentiable at \( c  \), and 
            \[  (fg)'(c) = f'(c) g(c) + f(c) g'(c) \]
        \item[(iv)] \( \frac{ f }{ g }  \) is differentiable at \( c  \) provided that \( g(c) \neq 0  \). Then
            \[  \Big(  \frac{ f }{ g }  \Big)' (c) = \frac{ f'(c) g(c) - f(c) g'(c) }{ [g(c)]^{2} }. \]
    \end{enumerate} 
\end{theorem}

\begin{theorem}[Chain Rule]
    Let \( {I}_{1} \subseteq  \R  \) and \( {I}_{2} \subseteq  \R   \) be two intervals, \( f: {I}_{1} \to \R  \) and \( g : {I}_{2} \to \R  \) be two functions, \( f({I}_{1}) \subseteq  {I}_{2} \), \( f \) is differentiable at \( c \in {I}_{1} \) and \( g  \) is differentiable at \( f(c) \in {I}_{2} \). Then the function \( g \circ f : {I}_{1} \to \R  \) is differentiable at \( c \in {I}_{1} \) and 
    \[  (g \circ f)'(c) = g'(f(c)) \cdot f'(c). \]
\end{theorem}
First, we will prove the theorem incorrectly and then show give three criterion to prove the theorem correctly.  

\begin{proof}
Observe that 
\begin{align*}
    \lim_{ x \to c  }  \frac{ (g \circ f)(x) - (g \circ f)(c) }{x - c } &= \lim_{ x \to c  }  \frac{ g(f(x)) - g(f(c)) }{  x - c  }  \\
                                                                        &= \lim_{ x \to c  }  \Big[ \frac{ g(f(x)) - g(f(c)) }{  f(x) - f(c) } \cdot \frac{ f(x) - f(c) }{  x - c   } \Big] \\
                                                                        &= \underbrace{\Big[ \lim_{ x \to c  }  \frac{ g(f(x)) - g(f(c)) }{  f(x) - f(c) } \Big]}_{g'(f(c))} \underbrace{\Big[ \lim_{ x \to c  }  \frac{ f(x)  - f(c) }{  x - c  } \Big]}_{f'(c)} \\
\end{align*}
\end{proof}
What is the problem with this proof? By the definition of a limit of a function, when you take \( \lim_{ x \to c  }  \), it is guaranteed that \( x - c \neq 0   \); however, for \( x  \) close to \( c  \) (as \( x  \) approaches to \( c  \)), \( f(x)- f(c) \) might be zero, so dividing by \( f(x) - f(c) \) is not legitimate. The following proof fixes this issue by introducing a new function \( d(f(x)) \) which is defined by
\begin{enumerate}
    \item[(i)] \( d(f(x)) \displaystyle \frac{ g(f(x)) - g(f(c)) }{  f(x) - f(c) }    \) when \( f(x) \neq f(c) \) 
    \item[(ii)] \( d(f(x))  \) is defined even when \( f(x) = f(c) \)  
    \item[(iii)] \( \displaystyle d(f(x)) \cdot \frac{ f(x) - f(c) }{  x - c  }  = \frac{ g(f(x)) - g(f(c)) }{  x - c  }   \) for all \( x \in {I}_{\kappa} \) where \( x \neq c   \).
\end{enumerate}

\begin{proof}
Let \( d: {I}_{2} \to \R  \) be defined by 
\[  d(y) = 
\begin{cases}
    \frac{ g(y) - g(f(c)) }{ y - f(c) }  &\text{if} \ y \neq f(c) \\
    \lim_{ y \to f(c) } \frac{ g(y) - g(f(c)) }{  y - f(c) }  = g'(f(c)) &\text{if} \ y = f(c)
\end{cases}. \]
Note that this function satisfies the requirements in (i) and (ii) outlined above. We make the following observations: 
\begin{enumerate}
    \item[(1)] \( d  \) is continuous at \( f(c) \). Indeed, we can see that 
        \[  \lim_{ y \to f(c) }  d(y) = \lim_{ y  \to  f(c) }  \frac{ g(y) - g(f(c)) }{ y - f(c) }  = d(f(c)). \]
    \item[(2)] For all \( x \in {I}_{1} \) and \( x \neq c  \), we have 
        \[  d(f(x)) \cdot \frac{ f(x) - f(c) }{  x -c  }  = \frac{ g(f(x)) - g(f(c))  }{  x - c  }. \tag{*}  \]
        We will show that this holds by considering two cases; either \( f(x) \neq f(c) \) or \( f(x) = f(c) \). If \( f(x) \neq f(c) \), then
        \[  \text{LHS} = d(f(x))\cdot \frac{ f(x) - f(c) }{ x - c  } = \frac{ g(f(x)) - g(f(c)) }{  f(x) - f(c) }  \cdot \frac{ f(x) - f(c) }{  x - c  } = \frac{ g(f(x)) - g(f(c))  }{  x - c  }  = \text{RHS}.    \]
        Now, suppose \( f(x) = f(c) \). Then we have 
        \begin{align*}
            \text{LHS} &= d(f(x)) \cdot \frac{ f(x) - f(c) }{  x - c  }  = d(f(c)) \cdot \frac{ f(x) - f(c) }{ x - c  }  = g'(f(c)) \cdot \frac{ 0  }{ x -c  }  = 0  \\
            \text{RHS} &= \frac{ g(f(x)) - g(f(c)) }{  x - c  }  = \frac{ g(f(c)) - g(f(c)) }{ x - c  }  = \frac{ 0  }{ x -c  }   =0.
        \end{align*}
        Thus, we see that the left hand side equals the right hand side of (*).
\end{enumerate}
Now, note that since \( f  \) is continuous at \( c  \) and \( d \) is continuous at \( f(c) \), their composition \( d \circ f  \) is continuous at \( c  \) and so,
\[  \lim_{ x \to c  }  (d \circ f)(x) = (d \circ f)(c). \]

\begin{align*}
    &\lim_{ x \to c  } \frac{ (g \circ f)(x) - (g \circ f)(c) }{ x - c  } = \lim_{ x \to c  }  \frac{ g(f(x)) - g(f(c)) }{ x - c  }  \\
                                            &= \Big[ \lim_{ x  \to c  }  (d \circ f ) (x) \Big] \cdot \Big[ \lim_{ x \to  c  }  \frac{ f(x) - f(c) }{  x - c  } \Big] \\
                                            &= [(d \circ f)(c)] \cdot f'(c) \\
                                            &= [d(f(c))] \cdot f'(c) \\
                                            &= g'(f(c)) \cdot f'(c).
\end{align*}
\end{proof}

\section{Lecture 2-4}

\subsection{Topics}

\begin{enumerate}
    \item[(1)] Local Maxima and minima
    \item[(2)] Interior Extremum Theorem (Theorem 5.8)
    \item[(3)] Darboux's Theorem (Theorem 5.12)
    \item[(4)] Some observations
    \item[(5)] Rolle's Theorem
    \item[(6)] Mean Value Theorem
\end{enumerate}

\begin{theorem}[Interior Extremum Theorem]\label{Theorem 5.8}
    Let \( I \subseteq  \R   \) be an interval and \( f : I \to \R  \). Suppose \( c  \) is an interior point of \( I  \) and \( f  \) is differentiable at \( c  \). Then
    \begin{enumerate}
        \item[(1)] If \( f  \) has a local max at \( c  \), then \( f'(c)  = 0 \);
        \item[(2)] If \( f  \) has a local min at \( c  \), then \( f'(c) = 0  \).
    \end{enumerate}
\end{theorem}

Before we prove this theorem, we will first go over an important lemma that is used in the main proof. 

\begin{lemma}
    Suppose \( \lim_{ x \to c  }  g(x)  \) and \( \lim_{ x \to c  }  h(x) \) both exist. 
    \begin{enumerate}
        \item[(1)] If there exists \( \delta > 0  \) such that \( h(x) \leq g(x) \) for all \( x \in (c -  \delta, c ) \), then \( \lim_{ x \to c  }  h(x) \leq \lim_{ x \to c  }  g(x) \).
        \item[(2)] If there exists \( \delta > 0  \) such that \( h(x) \leq g(x) \) for all \( x \in (c , c + \delta) \), then \( \lim_{ x \to c  }  h(x) \leq \lim_{ x \to c  }  g(x) \).
    \end{enumerate}
\end{lemma}
\begin{proof}
Here we will prove (1). The proof of (2) is analogous. Let \( ({a}_{n}) \) be a sequence in \( (c - \delta, c) \) such that \( {a}_{n} \to c  \). By the Sequential Criterion for limits of functions, we have \( {a}_{n} \to c  \) implies
\( \lim_{ n \to \infty  }  g({a}_{n}) = \lim_{ x \to c  }  g(x) \) and \( \lim_{ n \to \infty \infty   } h({a}_{n}) = \lim_{ x \to c  }  h(x) \). Also, note that from the Order Limit Theorem for sequences, we can see that   
\begin{align*}
    \forall n \ {a}_{n} \in (c - \delta, c ) &\implies \forall n \ h({a}_{n}) \leq g({a}_{n}) \\
                                             &\implies \lim_{ n \to \infty  }  h({a}_{n}) \leq \lim_{ n \to \infty  }  g({a}_{n}).
\end{align*}
Hence, we can see from these two observations that 
\[  \lim_{ x \to c  }  h(x) \leq \lim_{ x \to c  }  g(x). \]
\end{proof}

\subsection{Proof of the Interior Extremum Theorem}

\begin{proof}
Here we will prove (1). Suppose \( f \) has a local max at \( c  \). Then
\begin{itemize}
    \item If \( f  \) has a local max at \( c  \), then there exists \( {\delta}_{1} > 0  \) such that for all \( x \in (c - {\delta}_{1}, c + {\delta}_{1}) \cap I  \) \( f(x) \leq f(c) \).
    \item If \( c  \) is an interior point of \( I  \), then there exists \( {\delta}_{2} > 0  \) such that \( (c - {\delta}_{2}, c + {\delta}_{2}) \subseteq  I  \). So, if we let \( \delta = \min \{ {\delta}_{1}, {\delta}_{2} \}  \), then
        \[  \forall x \in (c - \delta, c + \delta )  \ \ f(x) \leq f(c). \]
\end{itemize}
We have 
\begin{enumerate}
    \item[(I)] For all \( x \in (c - \delta , c ) \), we see that \( x - c < 0  \) and \( f(x) \leq f(c) \) implies that 
        \[  \frac{ f(x) - f(c) }{ x - c  } \geq 0. \]
        By the Order Limit Theorem for functions, we have 
        \[  \lim_{ x \to c  }  \frac{ f(x) - f(c) }{ x - c  }  \geq \lim_{ x \to c  } 0 \implies f'(c) \geq 0.  \]
    \item[(II)] For all \( x \in (c, c + \delta) \). Since \( x - c > 0  \) and \( f(x) \leq f(c) \), we have 
        \[  \frac{ f(x) - f(c) }{ x - c  }  \leq 0.  \]
        Using the Order Limit Theorem again, we have 
        \[ \lim_{ x \to c  } \frac{ f(x) - f(c) }{  x-  c  }  \leq \lim_{ x \to c  }  0 \implies f'(c) \leq 0.   \]
\end{enumerate}
From (I) and (II), we can see that \( f'(c) \leq 0  \) and \( f'(c) \geq 0  \). Thus, \( f'(c) = 0  \).
\end{proof}

\begin{theorem}[Darboux's Theorem]
    Let \( f: [a,b] \to \R  \) be differentiable and \( f'(a) < f'(b) \) (or \( f'(b) < f'(a) \)). Let \( \alpha \in \R  \) be such that \( f'(a) < \alpha < f'(b)  \) (or \( f'(b) < \alpha < f'(a) \)). Then there exists \( c \in (a,b) \) such that \( f'(c) = \alpha \).
\end{theorem}
\begin{proof}
    Let \( g:[a,b] \to \R  \) be defined by \( g(x) = f(x) - \alpha x  \). It follows from the Algebraic Differentiability Theorem that \( g  \) is differentiable on \( [a,b] \) and so it is continuous on \( [a,b] \). It suffices to show that there is a point \( c \in (a,b) \) such that \( g'(c) = 0 \). To this end, it is enough to show that there exists a point \( c \in (a,b) \) at which \( g  \) has a local min. Since \( g  \) is continuous on \( [a,b] \) and \( [a,b]  \) is compact, \( g  \) attains its minimum on \( [a,b] \). Let \( \hat{c} \) be a point at which \( g  \) attains a minimum. IN what follows we will show that \( \hat{c} \in (a,b) \) and so it can be used as the same \( c  \) that we were looking for. Note that  
    \begin{align*}
        g'(a) = f'(a) - \alpha &<  0 \\
        g'(b) = f'(b) - \alpha > 0. 
    \end{align*}
    We make two claims; namely, \( \hat{c} \neq a  \) (\( \hat{c}  \) cannot be the left-endpoint of \( [a,b] \)) and \( \hat{c} \neq b  \) (the right-endpoint of \( [a,b] \)). Suppose for contradiction that \( \hat{c}= a  \). Then for all \( x \in [a,b] \) \( g(x) \geq g(a) \). So, for all \( x \in (a,b) \), we have \( g(x) - g(a) \geq 0  \) and \( x - a > 0  \). Thus, for all \( x \in (a,b) \)
    \[  \frac{ g(x) - g(a) }{ x - a  }  \geq 0.  \]
    Thus, 
    \[  \lim_{ x \to a }  \geq \frac{ g(x) - g(a) }{  x - a  }  \geq 0 \implies \lim_{ x \to a  }  0.   \]
    That is, \( g'(a) \geq 0  \). This contradicts the fact that \( g'(a) < 0 \) and this proves the first claim.
    A similar argument shows that \( g'(b) \leq 0  \), contradicting the fact that \( g'(b) > 0  \).
\end{proof}

\begin{theorem}[Rolle's Theorem]
    Let \( f: [a,b] \to \R  \) be continuous, let \( f  \) be differentiable on \( (a,b) \), and \( f(a) = f(b) \). Then there exists a point \( c \in (a,b) \) such that \( f'(c) = 0  \). 
\end{theorem}

\begin{proof}
    It suffices to show that there exists a point \( c \in (a,b)  \) for which \( f  \) has a local max or a local min. Since \( f  \) is continuous on \( [a,b] \) and \( [a,b]  \) is compact, we see that \( f  \) must attain its max and min on \( [a,b] \). Thus, we may consider two cases; namely, both \( \max f  \) and \( \min f  \) occur at the endpoints and or they occur in the interior of \( [a,b] \). Considering the first case, we see that from the fact that \( f(a) = f(b) \) that   
    \[  \max_{a \leq x \leq b } f(x) = \min_{a \leq x \leq b } f(x). \]
    Hence, \( f \) must be constant on \( [a,b] \) and so, 
    \[  \forall  x \in [a,b] \ \ f'(x) = 0.  \]
    In this case, we can choose \( c  \) to be any point we like in \( (a,b) \).  

    Now, assume the second case. Using the Interior Extremum Theorem, we have \( f'(c) = 0  \) and this concludes the proof.  
\end{proof}

\begin{theorem}[Mean Value Theorem]
    Let \( f: [a,b] \to \R  \) be a continuous function and \( f  \) is differentiable in \( (a,b) \). Then there exists a \( c \in (a,b)  \) such that 
    \[  f'(c) = \frac{ f(b) - f(a) }{  b - a  }. \]
\end{theorem}
\begin{proof}
    Let \( g:[a,b] \to \R  \) be defined by
    \[  g(x) = f(x) - \frac{ f(b) - f(a)  }{  b - a  }  x.  \]
    By the Algebraic Continuity Theorem, \( g  \) must be continuous on \( [a,b] \). Also, the Algebraic Differentiability Theorem implies that \( g  \) is differentiable on \( (a,b) \). Then we have  
    \[  g(a) = f(a) - \frac{ f(b) - f(a) }{  b -a  } = \frac{ b f(a) - a f(a) + a f(a)  }{  b - a  }  = \frac{ bf(a) - a f(b)  }{  b - a  }  \]
    and 
    \[  g(b) = f(b) - \frac{ f(b) - f(a) }{ b - a  } b = \frac{ b f(a) - a f(b) - b f(b) + b f(a) }{  b -a   }  = \frac{ b f(a) - a f(b)  }{  b -a  }. \]
    Using Rolle's Theorem, it follows that 
    \[  \exists c \in (a,b) \ \ g'(c) = 0.  \]
    Note that 
    \[  g'(x) = f'(x) - \frac{ f(b) - f(a) }{ b - a  }. \]
    Thus, 
    \[  g'(c) = 0  \iff f'(c) - \frac{ f(b) - f(a) }{  b -a  }  = 0  \iff f'(c) = \frac{ f(b) -f(a)  }{ b - a  }. \]
\end{proof}  

\begin{theorem}[Generalized Mean Value Theorem]
    Let \( f: [a,b] \to \R  \) and \( g : [a,b] \to \R  \) are continuous functions and suppose \( f  \) and \( g  \) are differentiable in \( (a,b) \). Then there exists a point \( c \in (a,b) \) such that 
    \[  f'(c) [g(b) - g(a)] = g'(c) [f(b) - f(a)]. \]
\end{theorem}
\begin{proof}
    \textbf{See proof in hw1.}
\end{proof}

\begin{remark}
    Note that, if \( g'  \) is never zero in \( (a,b) \), then we may rewrite the claim of GMVT as follows:
    \[  \exists c \in (a,b) \ \ \text{such that} \ \frac{ f'(c) }{  g'(c) } =  \frac{ f(b) - f(a) }{  g(b) - g(a) }. \]
\end{remark}

\begin{corollary}[Corollary 1]
    Let \( I \subseteq  \R   \) be an interval, \( f : I \to \R  \) be diffferentiable, and \( f'(x) = 0  \) for all \( x \in I  \). Then \( f  \) is a constant function on \( I  \), that is, there exists \( k \in \R  \) such that for all \( x \in I  \), \( f(x) = k  \). 
\end{corollary}

\begin{proof}
    Let \( x,y \in I  \) with \( x , y  \). It suffices to show that \( f(x) = f(y) \). To this end, we will apply the MVT to \( f  \) on the interval \( [x,y] \). By the Mean Value Theorem, there exists a \( c \in (x,y) \) such that 
    \[  f'(c) = \frac{ f(y) - f(x)  }{  y - x  }. \]
    Since \( f'(x) = 0  \) for all \( x \in I  \), we have \( f'(c) = 0  \) and so
    \begin{align*}
        0 = \frac{ f(y) - f(x) }{  y - x  }  &\implies 0 = f(y) - f(x)   \\
                                             &\implies f(y) = f(x).
    \end{align*}
\end{proof}

\begin{corollary}
    Let \( I \subseteq  \R   \) is an interval, \( f: I \to \R  \) and \( g: I \to \R  \) are differentiable, and \( f'(x) = g'(x) \) for all \( x \in I  \). Then there exists \( k \in \R  \) such that for all \( x \in I  \), \( f(x) = g(x) + k  \).
\end{corollary}

\begin{proof}
Let \( h = f - g  \). We have 
\[  \forall x \in I \ \ h'(x) = (f - g )'(x) = f'(x) - g'(x) = 0.  \]
Using Corollary 1, we have  
\[  \exists k \in \R \ \text{such that} \ \forall x \in I \ h(x) = k.  \]
Hence, we have 
\[  \exists k \in \R \  \text{such that} \ \forall x \in I \ f(x) - g(x) = k  \]
and so \( f(x) = g(x) + k  \) as desired.
\end{proof}

\begin{theorem}[Parts (a) and (c)]
   Let \( I \subseteq \R   \) is an interval and suppose \( f : I \to \R  \) is differentiable. Then 
   \begin{enumerate}
       \item[(1)] \( f  \) is increasing on \( I  \) if and only if \( \forall c \in I  \) \( f'(c) \geq 0  \).
        \item[(2)] \( f  \) is decreasing on \( I  \) if and only if \( c \in I   \) \( f'(c) \leq 0  \).
   \end{enumerate}
\end{theorem}
\begin{proof}
Here we will prove (1). The proof of (2) is similar. 
\( (\Longrightarrow) \) Let \( f  \) be increasing on \( I  \). Let \( c \in I  \). Note that for all \( x \in I  \) with \( x \neq c  \). Then we have 
\[  \frac{ f(x) - f(c) }{ x - c  }  \geq 0.  \]
Indeed, we can see that if \( x > c  \), then \( x - c > 0  \) and \( f(x) \geq f(c) \). Thus, we have that 
\[  \frac{ f(x) - f(c) }{ x - c  }  \geq 0.  \]
If \( x < c  \), then \( x - c < 0  \) and \( f(x) \leq f(c) \). Then 
\[  \frac{ f(x) - f(c) }{ x -c  }  \geq 0. \]
It follows from the Order Limit Theorem that 
\[  \lim_{ x \to c  }  \frac{ f(x) - f(c) }{  x-  c  }  \geq \lim_{ x \to c  }  0  = 0. \]
Hence, \( f'(c) \geq 0 \) as desired.

\( \Longleftarrow \) Suppose for all \( c \in I  \), \( f'(c) \geq 0  \). To show that \( f  \) is increasing on \( I  \), we need to show that for all \( {x}_{1}, {x}_{2} \in I   \) with \( {x}_{1} < {x}_{2} \). It suffices to show that \( f({x}_{1}) \leq f({x}_{2}) \). To this end, we will apply the Mean Value Theorem to the function \( f  \) on \( [{x}_{1}, {x}_{2}] \). That is, there exists a \( c \in ({x}_{1}, {x}_{2}) \) such that 
\[  f'(c) = \frac{ f({x}_{2})- f({x}_{1}) }{  {x}_{2} - {x}_{1} }. \]
Thus, 
\[  f({x}_{2}) - f({x}_{1}) = f'(c) ({x}_{2} - {x}_{1}). \]
Since \( f'(c) \geq 0  \) and \( {x}_{2} - {x}_{1} \geq 0  \), we can see that 
\[  f({x}_{2}) \geq f({x}_{1}) \]
as desired.
\end{proof}

\begin{theorem}[L'Hopital's Rule]
    Let \( I \subseteq  \R   \) be an interval, \( a \in I  \), and \( f: I \to \R  \) and \( g: I \to \R  \) are continuous functions. Suppose \( f  \) and \( g  \) are differentiable at all points in \( I \setminus  \{  a  \}  \) with \( f(a) = g(a) = 0  \) and 
    \[  \lim_{ x \to a }  \frac{ f'(x) }{ g'(x) }  = L \in \R.  \]
    Then 
    \[  \lim_{ x \to a }  \frac{ f(x) }{ g(x) }  = L. \]
\end{theorem}
\begin{proof}
Our goal is to show that 
\[  \forall \epsilon > 0 \ \exists \delta > 0 \ \text{such that if} \ 0 < | x - c  | < \delta \ \text{with (\( x \in I \)) then} \ \Big| \frac{ f(x) }{  g(x) }  - L  \Big|  < \epsilon. \tag{*} \]
Let \( \epsilon > 0  \) be given. Since, by assumption,  
\[  \lim_{ x \to a }  \frac{ f'(x) }{  g'(x) }  = L,  \]
for the given \( \epsilon > 0  \), there exists \( \hat{\delta} > 0 \) such that if \( 0 < |  x - a  |  < \hat{\delta} \) (with \( x \in I  \), then
\[  \Big| \frac{ f'(x) }{ g'(x) }  - L  \Big| < \epsilon. \]
We claim that this \( \hat{\delta} \) works as the \( \delta  \) we were looking for. Indeed, if we let \( \delta = \hat{\delta} \), then (*) will hold. The reason is as follows: suppose \( x \in I  \) and \(  0 < |  x - a  |  < \delta \). In what follows, we will show that  
\[  \Big| \frac{ f(x) }{ g(x) }  - L  \Big|  < \epsilon. \]
We may consider the following cases:
\begin{enumerate}
    \item[(1)] (\( x > a  \)) (that is, \( x \in I  \) and \( x \in (a, a + \hat{\delta}) \)). We apply the Generalized Mean Value Theorem to \( f  \) and \( g  \) on the interval \( [a,x] \). That is,  
        \[  \exists c \in (a,x) \ \text{such that} \ \frac{ f'(c) }{ g'(c) }  = \frac{ f(x) - f(a) }{  g(x) - g(a) }. \]
        Since \( f(a) = g(a) = 0  \), we can conclude that 
        \[  \frac{ f'(c) }{  g'(c) }  = \frac{ f(x) }{ g(x) }. \]
        It follows that 
        \[  \Big| \frac{ f(x) }{  g(x) }  - L  \Big|  = \Big| \frac{ f'(c) }{ g'(c) }  - L  \Big| < \epsilon. \]
        Note that the latter inequality is true because \( 0 < |  c - a  |  \leq |  x - a  |  < \hat{\delta} \).
    \item[(2)] (\( x < a  \)) (that is, \( x \in I  \) and \( x \in (a - \hat{\delta} , a ) \)) We apply the Generalized Mean Value Theorem to \( f  \) and \( g  \) on \( [x,a] \). That is, there exists \( c \in (x,a) \) such that  
        \[  \frac{ f'(c) }{  g'(c) }  = \frac{ f(a) - f(x) }{  g(a) - g)x }  \]
        and so,
        \[  \frac{ f'(c) }{ g'(c) }  = \frac{ f(x) }{ g(x) }. \]
        It follows that 
        \[  \Big| \frac{ f(x) }{ g(x) }  - L  \Big|  = \Big| \frac{ f'(c) }{ g'(c) }  - L  \Big|  < \epsilon. \]
        Note that the latter inequality is true because \( 0 < |  c- a  |  \leq |  x - a  |  < \hat{\delta}. \)

\end{enumerate}
\end{proof}

