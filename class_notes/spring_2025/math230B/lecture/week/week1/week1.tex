\section{Lecture 1}

\subsection{Topics}

\begin{itemize}
    \item The derivative
    \item Continuity and Differentiability
    \item Differentiability Rules
\end{itemize}

\subsection{The Derivative}

We will also discuss several useful theorems, including: 

\begin{enumerate}
    \item[(*)] Darboux's Theorem
    \item[(*)] Mean Value Theorem
    \item[(*)] L'Hopital's Rule
\end{enumerate}

\begin{definition}[Differentiability]
    \begin{enumerate}
        \item[(*)] Let \( I \subseteq  \R  \) be an interval, \( f: I \to \R  \), \( c \in I  \). We say \( f  \) is \textbf{differentiable} at \( c  \) if 
    \[  \lim_{ x \to c }  \frac{ f(x) -f(c) }{ x - c  } \]
    exists (that is, it equals a real number).
    \item[(*)] In this case, the quantity \( \lim_{ x \to c }  \frac{ f(x) - f(c) }{ x - c  }  \) is called the derivative of \( f  \) at \( c  \) and is denoted by
        \[  f'(c), \frac{ df }{ dx }(c), \frac{ df }{ dx } \Bigg|_{x = c} \]
    \item[(*)] If \( f: I \to \R  \) is differentiable at every point \( c \in I  \), we say \( f \) is differentiable (on \( I  \)).
    \end{enumerate}
\end{definition}


\begin{remark}
    The following are equivalent characterizations of the differentiability:
    \begin{align*}
        f'(c) = L &\iff \lim_{ x \to c }  \frac{ f(x) - f(c) }{ x - c  }  = L  \\
                  &\iff \forall \epsilon > 0 \ \exists \delta > 0 \ \text{such that} \ \text{if} \ 0 < | x - c  | < \delta \ \text{then} \ \Big| \frac{ f(x) - f(c) }{ x - c  }  - L  \Big|  < \epsilon \\
                  &\iff \forall \epsilon > 0 \ \exists \delta > 0 \ \text{such that if} \ 0 < | h  |  < \delta \ \text{then} \ \Big| \frac{ f(c+h) - f(c) }{  h  }  - L  \Big|  < \epsilon \\
                  &\iff \lim_{ h \to 0 }  \frac{ f(c+h) - f(c) }{  h  }  = L 
    \end{align*}
\end{remark}

\begin{eg}
    Let \( I \subseteq  \R   \) be an interval and \( f: I \to \R  \) be given by \( f(x) = x^{2} \). Prove that \( f  \) is differentiable on \( I  \) and find the derivative. 

    For all \( c \in I  \), 
    \begin{align*} 
        \lim_{ x \to c  }  \frac{ f(x) - f(c) }{ x - c  } = \lim_{ x \to c  }  \frac{ x^{2} - c^{2} }{  x - c  }  &= \lim_{ x \to c  }  \frac{ (x - c ) (x + c) }{  x - c  }  =  x + c  \\
                                                                                                                  &= \lim_{ x \to c } (x + c) = 2c
    \end{align*}
    So, for all \( c \in I  \) \( f'(c) = 2c  \). Hence, \( f': I \to \R  \) and so \( f'(x) = 2x  \) for all \( x \in  I \).
\end{eg}

In this example, we used the continuity of the polynomial \( x + c  \) to get the final equality. Next, we prove a more general result.

\begin{eg}
    Let \( I \subseteq  \R  \) be an interval and \( f: I \to \R  \) be given by \( f(x) = x^{n} \), where \( n \in \N  \), \( n \geq 3  \). Prove that \( f  \) is differentiable on \( I  \) and find the derivative.   
    \begin{align*}
        \lim_{ x \to c  }  \frac{ f(x) - f(c) }{ x - c  }  &= \lim_{ x \to c  }  \frac{ x^{n} - c^{n}  }{  x - c  }  = \lim_{ x \to c  }  \frac{ (x - c ) (x^{n-1} + c x^{n-2 } + \cdots + c^{n-1}) }{  x  -c  }  \\
                                                           &= \lim_{ x \to c  }  [x^{n-1} + c x^{n-2 } + \cdots  + c^{n-1} ] \\
                                                           &= c^{n-1 } + c^{n-1} + \cdots + c^{n-1} \\
                                                           &= n c^{n-1}.
    \end{align*}
    So, for all \( c \in I  \), \( f'(c) = n c^{n-1} \). Hence, \( f': I \to \R  \) where \( f'(x) = n x^{n-1 } \).
\end{eg}

\begin{eg}[Continuity does NOT imply differentiability]
Prove that \( f: \R \to \R  \), \( f(x) = | x  |  \) is NOT differentiable at \( c = 0  \). Indeed, we just need to show that the limit
\[  \lim_{ x \to 0  }  \frac{ f(x) - f(0) }{ x - 0  }   \]
does not exist. Note that 
\[  \frac{ f(x) - f(0) }{ x - 0 }  = \frac{ | x  |  - | 0  |  }{ x - 0  } = \frac{ | x  |  }{ x  }. \]
Let \( g(x) = \frac{ | x  |  }{ x  }  \). Our goal is to show that \( \lim_{ x \to 0  } g(x) \) does NOT exist. By the sequential criterion for limits of functions, it suffices to find two sequences \( ({a}_{n}) \) and \( ({b}_{n}) \) in \( \R \setminus  \{  0  \}  \) such that \( {a}_{n} \to 0  \) and \( {b}_{n} \to 0  \) but \( \lim_{ n \to \infty  }  g({a}_{n}) \neq \lim_{ n \to \infty  }  g({b}_{n}) \). Let \( {a}_{n}= \frac{ -1  }{ n }  \) and \( {b}_{n} = \frac{ 1 }{ n }  \). Clearly, \( {a}_{n} \to 0  \) and \( {b}_{n} \to 0  \). Hence, 
\[  \lim_{ n \to \infty  }  g({a}_{n}) = \lim_{ n \to \infty  }  \frac{ | {a}_{n} |  }{ {a}_{n} }  = \lim_{ n \to \infty  }  \frac{ | \frac{ -1 }{ n }  |  }{ \frac{ -1 }{ n }  } = 1  \]
and
\[  \lim_{ n \to \infty  }  g({b}_{n}) = \lim_{ n \to \infty  }  \frac{ | {b}_{n} |   }{ {b}_{n}  } = 1.  \]
\end{eg}


\begin{theorem}[Differentiability Implies Continuous]
    Let \( I \subseteq  \R  \), \( c \in I  \), and \( f: I \to \R  \) is differentiable at \( c  \). Then \( f \) is continuous at \( c  \).
\end{theorem}

\begin{proof}
It suffices to show that \( \lim_{ x \to c } f(x) = f(c) \). Note that 
\begin{align*}
    \lim_{ x \to c } (f(x) - f(c)) &= \lim_{ x \to c }  \Big[ \frac{ f(x) - f(c) }{  x - c  }  \Big] (x - c) \\
                                   &= \Big[ \lim_{ x \to c }  \frac{ f(x) - f(c) }{  x - c  }  \Big] \Big[ \lim_{ x \to c }  (x - c) \Big] \\
                                   &= (f'(c)) (0) \\
                                   &= 0.
\end{align*}
So, we have 
\begin{align*}
    \lim_{ x \to c } f(x) &= \lim_{ x \to c }  [f(x)  - f(c) + f(c)] \\ 
                          &= \lim_{ x \to c } (f(x) - f(c)) + \lim_{ x \to c } f(c) \\   
                          &= 0 + \lim_{ x \to c } f(c) \\
                          &= 0 + f(c) \\
                          &= f(c).
\end{align*}
\end{proof}

\begin{corollary}
    If \( f: I \to \R  \) is NOT continuous at \( c \in I  \), then \( f \) is NOT differentiable at \( c  \).
\end{corollary}

\begin{eg}
    Let \( f: \R \to \R  \) be defined by 
    \[  f(x) = 
    \begin{cases}
        x^{2} &\text{if} \  x \in \Q \\
        0 &\text{if} \ x \notin \Q 
    \end{cases}. \]
\end{eg}

\begin{enumerate}
    \item[(i)] Prove that \( f  \) is continuous at \( 0  \).
        \begin{proof}
        Our goal is to show that 
        \[  \forall \epsilon > 0 \ \exists \delta > 0 \ \text{such that if} \ | x |  < \delta \ \text{then} \ | f(x) - f(c) | < \epsilon.  \]
        Let \( \epsilon > 0  \) be given. Note that if \( x \notin \Q  \),   
       \[  | f(x) |  = | 0  |  < \epsilon. \] 
       Otherwise, we have \( | f(x)  |  = | x^{2} |  = | x |^{2} \). IN this case, we claim that \( \delta = \sqrt{ \epsilon }  \) will work. Indeed, if \( | x  |  < \delta \), then we have 
       \[  | f(x) |  = | x |^{2} < (\sqrt{ \epsilon } )^{2} = \epsilon. \]
        \end{proof}
    \item[(ii)] Prove \( f \) is discontinuous at all \( x \neq 0  \).
        \begin{proof}
        Let \( c \neq 0  \). Our goal is to show that \( f  \) is discontinuous at \( c  \). By the sequential criterion for continuity, it suffices to find a sequence \( ({a}_{n}) \) such that \( {a}_{n} \to c  \) but \( f({a}_{n}) \not\to f(c) \). We will consider two cases; that is, we could either have \( c \notin \Q  \) or \( c \in \Q  \). 

        Suppose \( c \notin \Q  \). Since \( \Q  \) is dense in \( \R  \), there exists a sequence of rational numbers \( ({r}_{n}) \) such that \( {r}_{n} \to c  \). Note that \( f({r}_{n}) = {r}_{n}^{2} \to c^{2} \neq 0   \), but \( f(c) = 0  \). Clearly, \( f({r}_{n}) \not\to f(c) \) and so \( f  \) must be discontinuous at \( c  \).

        Suppose \( c \in \Q  \). Since the set of irrational numbers is also dense in \( \R  \), we can find a sequence \( ({s}_{n}) \) such that \( {s}_{n} \to c  \). Note that \( f({s}_{n})  =  0  \), but \( f(c) = c^{2} \neq 0  \). Thus, \( f({s}_{n}) \not\to f(c) \). Therefore, \( f  \) must be discontinuous at \( c  \). 
        \end{proof}
    \item[(iii)] Prove that \( f  \) is nondifferentiable at all \( x \neq 0  \).
        \begin{proof}
        Let \( c \neq 0  \). Since \( f  \) is discontinuous at \( c  \), we can conclude that \( f  \) is not differentiable at \( c  \).
        \end{proof}
    \item[(iv)] Prove that \( f'(0) = 0   \).
        \begin{proof}
        We need to show 
        \[  \lim_{ x \to c } \frac{ f(x) - f(0) }{ x - 0  }  = \frac{ f(x) }{ x  } = 0. \]
        \end{proof}
\end{enumerate}

\begin{theorem}[Algebraic Differentiability Theorem]
    Assume that \( f: I \to \R  \) and \( g : I \to \R  \) are differentiable at \( c \in I  \) where (\( I  \) is an interval on \( \R  \)). Then
    \begin{enumerate}
        \item[(i)] For all \( k \in \R  \), \( kf  \) is differentiable at \( c  \), and
            \[  (kf)'(c) = k f'(c) \]
        \item[(ii)] \( f+ g  \) is differentiable at \( c  \), and 
            \[  (f+gk)'(c) = f'(c) + g'(c) \]
        \item[(iii)] \( fg  \) is differentiable at \( c  \), and 
            \[  (fg)'(c) = f'(c) g(c) + f(c) g'(c) \]
        \item[(iv)] \( \frac{ f }{ g }  \) is differentiable at \( c  \) provided that \( g(c) \neq 0  \). Then
            \[  \Big(  \frac{ f }{ g }  \Big)' (c) = \frac{ f'(c) g(c) - f(c) g'(c) }{ [g(c)]^{2} }. \]
    \end{enumerate} 
\end{theorem}


