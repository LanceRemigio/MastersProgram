\subsection{Lecture 10}

\subsubsection{Plan}

\begin{itemize}
    \item Discuss properties of Riemann-Stieljtes Integrals
\end{itemize}


\begin{theorem}[Rudin 6.9]\label{Rudin 6.9}
    Let \( \alpha: [a,b] \to \R  \) be increasing and continuous. Then 
    \begin{enumerate}
        \item[(1)] If \( f:[a,b] \to \R  \) is increasing, then \( f \in {R}_{\alpha}[a,b] \).
        \item[(2)] If \( f: [a,b] \to \R  \) is increasing
    \end{enumerate}
\end{theorem}
\begin{proof}
Here we will prove (1). The proof of (2) is analogous. First, note that
\[  \forall x \in [a,b] \ \ f(a) \leq f(x) \leq f(b)  \implies f \ \text{is bounded on} \ [a,b].\]
If \( \alpha(a) = \alpha(b) \), then we previously proved \( f \in {R}_{\alpha}[a,b] \) and \( \int_{ a }^{ b }  f  \ d \alpha = 0  \). So, it remains to prove the claim for the case where \( \alpha(a) \neq \alpha(b) \). According to the Cauchy Criterion for integrability, in order to show that \( f \in {R}_{\alpha}[a,b]  \), it suffices to show that
\[  \forall \epsilon > 0 \ \exists P \in \Pi \ \text{such that} \ U(f,\alpha, P) - L(f,\alpha, P) < \epsilon.  \]
Let \( \epsilon > 0  \) be given. Choose \( n \in \N \) be large enough so that \( \frac{ \alpha(b) - \alpha(a)  }{ n  }  [f(b) - f(a)] < \epsilon \). Let \( \tilde{P} = \{  x_{0}, {x}_{1}, \dots, {x}_{n} \}  \) be a partition of \( [a,b] \) such that 
\[  \forall 1 \leq k \leq n \ \ \Delta {\alpha}_{k} = \alpha({x}_{k}) - \alpha({x}_{k-1}) = \frac{ \alpha(b) - \alpha(a) }{  n  }. \]
We claim that \( \tilde{P} \) can be used as the \( P  \) that we were looking for. Now, since \( f  \) is increasing, we know that for each \( 1 \leq  k \leq n  \)
\[  {M}_{k} = \sup_{x \in [{x}_{k-1}, {x}_{k}]} f(x) = f({x}_{k}) \]
and
\[  {m}_{k } = \inf_{x \in [{x}_{k-1}, {x}_{k}]} f(x) = f({x}_{k-1}). \]
Hence, we see that 
\begin{align*}
    U(f,\alpha, \tilde{P}) - L(f,\alpha, \tilde{P})  &= \sum_{ k=1  }^{ n } ({M}_{k } - {m}_{k}) \Delta {\alpha}_{k }  \\
                                                     &= \frac{ \alpha(b) - \alpha(a) }{ n }  \sum_{ k=1  }^{ n } [f({x}_{k}) - f({x}_{k-1})] \\
                                                     &= \frac{ \alpha(b) - \alpha(a) }{  n }  [f(b) - f(a)] < \epsilon
\end{align*}
as desired.
\end{proof}

\begin{theorem}[Rudin 6.10]\label{Rudin 6.10}
    Let \( f: [a,b] \to \R  \) be a bounded function. Suppose that \( f  \) has only finitely many points of discontinuity  
    \[  {y}_{1} < {y}_{2} < \cdots < {y}_{m}  \]
    and \( \alpha:[a,b] \to \R \) is increasing and \( \alpha \) is continuous at \( {y}_{1}, {y}_{2}, \dots, {y}_{m} \). Then \( f \in {R}_{\alpha}[a,b] \).
\end{theorem}
\begin{proof}
According to the Cauchy Criterion, it suffices to show that
\[  \forall \epsilon > 0 \ \exists P \in \Pi \ \text{such that} \ U(f,\alpha,P) - L(f,\alpha,P) < \epsilon. \]
Let \( \epsilon > 0 \) be given. Let \( \tilde{M} = \sup_{x \in [a,b]} | f(x) |  \). Let 
\[  \hat{\epsilon} = \frac{ \epsilon }{ [\alpha(b) - \alpha(a) + 2 \tilde{M} + 1] }.  \]
We will make the following two claims:
\begin{enumerate}
    \item[(1)] There exists many disjoint intervals \( [{u}_{1}, {v}_{1}] , \dots, [{u}_{m}, {v}_{m}] \) such that 
        \begin{enumerate}
            \item[(I)] \( \forall 1 \leq j \leq m  \) \( {y}_{j} \in [{u}_{j}, {v}_{j}] \).
            \item[(II)] \( \forall 1 \leq j \leq m  \) if \( {y}_{j} \notin \{ a,b \}  \), then \( {y}_{j} \in ({u}_{j}, {v}_{j}) \)
            \item[(III)] \( \forall 1 \leq j \leq m  \) \( \alpha({v}_{j}) - \alpha({u}_{j}) < \frac{ \hat{\epsilon} }{ m }  \) and so
                \[  \sum_{ j=1  }^{ m } \alpha({v}_{j}) - \alpha({u}_{j}) < \hat{\epsilon}. \]
        \end{enumerate}
    \item[(2)] Let \( K = [a,b] \setminus  \bigcup_{ j=1  }^{ m  }  ({u}_{j}, {v}_{j}) \). Then \( f  \) is uniformly continuous on \( K  \).
\end{enumerate}
The two claims above will be proven as lemmas after the proof of this theorem. For now, we will assume that the two claims hold.

By claim 2, we know there exists \( \delta > 0  \) such that for all \( s,t \in K  \) if \( | s - t  |  < \delta  \), then
\[  | f(s) - f(t) |  < \hat{\epsilon}. \]
Now, we form a partition \( \tilde{P}  \) of \( [a,b] \) as follows:
\begin{enumerate}
    \item[(i)] \( \forall 1 \leq j \leq m  \) \( {u}_{j}, {v}_{j} \in \tilde{P} \).
    \item[(ii)] \( \forall 1 \leq j \leq m  \) no point of the segment \( ({u}_{j}, {v}_{j}) \) is in \( \tilde{P} \)
    \item[(iii)] If \( 1 \leq k \leq m  \) is such that \( {x}_{k-1} \notin \{  {u}_{1}, \dots, {u}_{m} \}  \), then we will choose \( {x}_{k} \) such that \( {x}_{k } - {x}_{k-1} < \delta \). 
\end{enumerate}
We claim that this \( \tilde{P} \) can be used as the \( P  \) that we were looking for. Indeed, define the two sets
        \[ A = \{ k : {x}_{k-1} \notin \{ {u}_{1}, \dots, {u}_{m} \}  \} \ \ \text{and} \ \  B = \{ 1, \dots, n \}  \setminus  A.   \]
        For the case that \( k \in A  \), \( {x}_{k} - {x}_{k-1} < \delta \), so
        for all \( s,t \in [{x}_{k-1}, {x}_{k}] \), if \( | s- t  |  < \delta \), then \( | f(s) - f(t)  | < \hat{\epsilon} \). Then taking the supremum, we have 
        \[  \sup_{s,t \in [{x}_{k-1}, {x}_{k}]} | f(s) - f(t) |  \leq \hat{\epsilon} \]
        and so from {\hyperref[lemma 2]{lemma 2}}, we have  
        \[  {M}_{k } - {m}_{k } \leq \hat{\epsilon}. \]

    If \( k \in B \), then
    \[  {M}_{k } - {m}_{k } = \sup_{s,t \in [{x}_{k-1}, {x}_{k}]} | f(s) - f(t) |  \leq 2 \tilde{M}. \]
    Therefore, 
    \begin{align*}
        U(f,\alpha, P) - L(f,\alpha, P) &= \sum_{ k=1  }^{ n } ({M}_{k } - {m}_{k } ) \Delta {\alpha}_{k} \\
                                        &= \sum_{k \in A} ({M}_{k } - {m}_{k}) \Delta {\alpha}_{k } + \sum_{k \in B} ({M}_{k } - {m}_{k}) \Delta {\alpha}_{k } \\
                                        &\leq \sum_{k \in A} \hat{\epsilon} \Delta {\alpha}_{k } + 2 \tilde{M} \sum_{k \in B} \Delta {\alpha}_{k } \\ 
                                        &\leq \hat{\epsilon} [\alpha(b) - \alpha(a)] + 2 \tilde{M} \hat{\epsilon} \\
                                        &= [\alpha(b) - \alpha(a) + 2 \tilde{M}] \hat{\epsilon} \\
                                        &< \epsilon.
    \end{align*}
\end{proof}

\begin{lemma}\label{Claim 1}
   There exists finitely many disjoint intervals 
   \[  [{u}_{1}, {v}_{1}], \dots, [{u}_{m}, {v}_{m}] \]
   in \( [a,b] \) such that
   \begin{enumerate}
       \item[(1)] \( \forall 1 \leq j \leq m  \) \( {y}_{j} \in [{u}_{j}, {v}_{j}] \);
        \item[(2)] \( \forall 1 \leq j \leq m  \) if \( {y}_{j} \notin \{ a,b \}  \) then \( {y}_{j} \in ({u}_{j}, {v}_{j}) \);
        \item[(3)] \( \forall  1 \leq j \leq m  \) \( \alpha({v}_{j}) - \alpha({u}_{j}) < \frac{ \hat{\epsilon} }{ m }  \) and so 
            \[  \sum_{ j=1  }^{ m } [\alpha({v}_{j}) - \alpha({u}_{j})] < \hat{\epsilon}. \]
   \end{enumerate}
\end{lemma}
\begin{proof}
Since for each \( 1 \leq j \leq m  \), \( \alpha \) is continuous at \( {y}_{j} \), we can choose \( {\delta}_{j} > 0  \) such that  
\[  \text{if} \ | y  - {y}_{j} |  < {\delta}_{j} , \ \text{then} \ | \alpha(y) - \alpha({y}_{j}) | < \frac{ \hat{\epsilon} }{ 2 m }. \]
Now, let 
\[  \tilde{\delta} = \frac{ 1 }{ 4 }  \min \{ {\delta}_{1}, {\delta}_{2}, \dots, {\delta}_{m}, {y}_{2} - {y}_{1}, {y}_{3} - {y}_{2}, \dots, {y}_{m} - {y}_{m-1} \}. \]
For each \( 1 \leq j \leq m  \), we define
\begin{enumerate}
    \item[(1)] If \( {y}_{j} \notin \{ a,b \}  \), then \( [{u}_{j}, {v}_{j}] = [{y}_{j} - \hat{\delta}, {y}_{j} + \hat{\delta}] \)
    \item[(2)] If \( {y}_{j} = a \), then \( [{u}_{j}, {v}_{j}] = [a, a + \hat{\delta}] \)
    \item[(3)] If \( {y}_{j} = b  \), then \( [{u}_{j}, {v}_{j}] = [b - \hat{\delta}, b ] \).
\end{enumerate}
Clearly, there intervals satisfy all the requirements, in particular, 
\begin{align*}
    \alpha({v}_{j}) - \alpha({u}_{j}) &= | \alpha({v}_{j}) - \alpha({u}_{j}) |  \\
                                      &\leq | \alpha({v}_{j}) - \alpha({y}_{j}) | + | \alpha({y}_{j}) - \alpha({u}_{j}) |  \\
                                      &< \frac{ \hat{\epsilon} }{ 2m }  + \frac{ \hat{\epsilon} }{ 2m }  \\
                                      &= \frac{ \hat{\epsilon} }{ m }
\end{align*}
where \( | {v}_{j} - {y}_{j} | \leq \hat{\delta} < {\delta}_{j} \) and \( | {u}_{j} - {y}_{j} |  \leq \hat{\delta} < {\delta}_{j} \).
\end{proof}

\begin{lemma}[Claim 2]
    Let \( K = [a,b] \setminus  \bigcup_{ j=1 }^{ m } ({u}_{j}, {v}_{j})  \). Then \( f \) is uniformly continuous on \( K  \).
\end{lemma}
\begin{proof}
Note that \( \bigcup_{ k=1  }^{ m } ({u}_{j}, {v}_{j})   \) is open. Hence, 
\[  K = [a,b] \setminus  \bigcup_{ j=1  }^{ m }  ({u}_{j}, {v}_{j}) = [a,b] \cap \Big[ \bigcup_{ j=1  }^{ m }  ({u}_{j}, {v}_{j}) \Big]^{c} \]
is closed. Since \( K \subseteq  [a,b] \), \( K  \) is closed, and \( [a,b]  \) is compact, it follows from the fact that closed subsets of a compact set are compact that \( K  \) is compact. Since \( f: K \to \R  \) is continuous and \( K  \) is compact, we can conclude that \( f  \) {\hyperref[is uniformly continuous]{is uniformly continuous}} on \( K  \).
\end{proof}

\begin{remark}[Why is \( f: K \to \R  \) is continuous?]\label{is uniformly continuous}
   We will consider four claims:
   \begin{enumerate}
       \item[(1)] Suppose \( f  \) is continuous at \( a \) and \( b  \). In this case by removing \( \bigcup_{ j=1  }^{ m }  ({u}_{j}, {v}_{j})  \), the discontinuities of \( f  \) will be removed.
        \item[(2)] Since \( f  \) is discontinuous at \( a \), but continuous at \( b  \). In this case, by removing \( \bigcup_{ j=1 }^{ m } ({u}_{j},{v}_{j}) \) all discontinuities will be removed except \( a \). In this case, removing \( ({u}_{1}, {v}_{1}) \) makes \( a \) an isolated point of \( K  \). Every function is continuous at every is isolated point of its domain.
        \item[(3)] Suppose \( f \) is continuous at \( a \) and discontinuities at \( b \).
        \item[(4)] Suppose \( f \) is both discontinuous at \( a \) and \( b \).
   \end{enumerate}
   Case (3) and (4) follows similarly from case (2).
\end{remark}

\begin{center}
    \textit{End of Lecture 10} 
\end{center}


%end of lecture 10

\subsection{Lecture 11}


\begin{theorem}[Rudin 6.11]\label{Rudin 6.11}
    Let \( f \in {R}_{\alpha}[a,b] \), for all \( x \in [a,b] \) \( m \leq f(x) \leq M  \), \( \varphi: [m,M] \to \R  \) is continuous. Then \( h: \varphi \circ f : [a,b] \to \R  \), then \( h \in {R}_{\alpha}[a,b] \).
\end{theorem}
\begin{proof}
    Firs note that a composition of bounded functions is bounded. So \( h: \varphi \circ f  \) is a bounded function on \( [a,b] \). According to the Cauchy criterion, in order to show \( h \in {R}_{\alpha}[a,b] \), it suffices to show that for all \( \epsilon > 0  \), there exists \( P \in \Pi \) such that  
    \[  U(f,\alpha, P) - L(h,\alpha,P) < \epsilon. \]
    Let \( \epsilon > 0  \) be given. Let \( \tilde{M} = \sup_{x \in [a,b]} | h(x) |  \). Let 
    \[  \hat{\epsilon} = \frac{ \epsilon }{ [\alpha(b) - \alpha(a) + 2 \tilde{M} + 1] }. \] 
    We have 
    \begin{enumerate}
        \item[(I)] Since \( \varphi  \) is continuous in \( [m,M] \) and \( [m,M] \) is compact, it follows that \( \varphi  \) is uniformly continuous on \( [,m,M] \). So,
            \[ \exists 0 < \delta < \hat{\epsilon} \ \text{such that} \ \forall s,t \in [m,M] \ \text{if} | s -t  |  < \delta \ \text{then} \ | \varphi(s) - \varphi(t) | < \hat{\epsilon}. \]
        \item[(II)] Since \( f \in {R}_{\alpha}[a,b] \), we know from the CauchyCriterion that 
            \[ \exists \tilde{P} \in \Pi \ \text{such that} \ U(f,\alpha,\tilde{P}) - L(f,\alpha, \tilde{P}) = \sum_{ k=1  }^{ n } ({M}_{k } - {m}_{k}) \Delta {\alpha}_{k } < \delta^{2}.  \]
    \end{enumerate}
    We claim that this \( \tilde{P} \) can be used as the \( P  \) that we were looking for. Indeed, let for all \( 1 \leq k \leq n  \)
    \[ {m}_{k }^{*} = \inf_{x \in [{x}_{k-1}, {x}_{k}]} h(x) \ \ \text{and} \ \ {M}_{k }^{*} = \sup_{x \in [{x}_{k-1}, {x}_{k}]} h(x). \]
    Note that
    \[  U(h,\alpha,\tilde{P}) - L(h,\alpha, \tilde{P}) = \sum_{ k=1  }^{ n } ({M}_{k }^{*} - {m}_{k }^{*}) \Delta {\alpha}_{k }.  \]
    In what follows, we will show that the sum above is less than \( \epsilon \). Divide the indices \( 1, \dots, n \) in two classes, namely
    \[  A = \{  k : {M}_{k } - {m}_{k } < \delta  \}  \ \ \text{and} \ \ B = \{  k : {M}_{k } - {m}_{k } \geq \delta \}. \]
    We have 
    \[  U(h,\alpha,\tilde{P}) - L(h,\alpha, \tilde{P}) = \sum_{ k=1  }^{ n } ({M}_{k }^{*} - {m}_{k }^{*}) \Delta {\alpha}_{k } = \sum_{k \in A} ({M}_{k }^{*} - {m}_{k }^{*}) \Delta {\alpha}_{k } + \sum_{k \in B} ({M}_{k}^{*} - {m}_{k }^{*}) \Delta {\alpha}_{k }. \tag{1}  \]
    \begin{enumerate}
        \item[(*)] If \( k \in A \), then for all \( x,y \in [{x}_{k-1}, {x}_{k}] \), we have
            \begin{align*}
                {M}_{k } - {m}_{k } < \delta &\implies \sup_{x,y \in [{x}_{k-1}, {x}_{k}]} | f(x) - f(y) |  \\
                                             &\implies | f(x) - f(y) |  < \delta \\
                                             &\implies | \varphi(f(x)) - \varphi(f(y)) |  < \hat{\epsilon} \\
                                             &\implies | h(x) - h(y) | < \hat{\epsilon} \\
                                             &\implies \sup_{x,y \in [{x}_{k-1}, {x}_{k}]} | h(x) - h(y) |  \leq \hat{\epsilon} \\
                                             &\implies {M}_{k }^{*} - {m}_{k }^{*} \leq \hat{\epsilon}. \tag{2}
            \end{align*}
        \item[(*)] For \( k \in B \), 
            \begin{align*}
                \delta \sum_{k \in B} \Delta {\alpha}_{k } = \sum_{k \in B} \delta \Delta {\alpha}_{k } &\leq \sum_{k \in B} ({M}_{k } - {m}_{k}) \Delta {\alpha}_{k } \\ 
                                                                                                        &\leq \sum_{ k=1  }^{ n }({M}_{k } -{m}_{k }) \Delta {\alpha}_{k } = U(f,\alpha, \tilde{P}) -  L(f,\alpha, \tilde{P}) < \delta^{2}. \tag{3}
            \end{align*}
            It follows from (1), (2), and (3) that 
            \begin{align*}
                \sum_{ k=1  }^{ n } ({M}_{k }^{*} - {m}_{k }^{*}) \Delta {\alpha}_{k } &= \sum_{k \in A} ({M}_{k }^{*} - {m}_{k }^{*}) \Delta {\alpha}_{k } + \sum_{k \in B} ({M}_{k }^{*} -{m}_{k }^{*}) \Delta {\alpha}_{k } \\
                                                                                       &\leq \sum_{k \in A} \hat{\epsilon} \Delta {\alpha}_{k } + \sum_{k \in B} 2 \tilde{M} \Delta {\alpha}_{k } \\
                                                                                       &= \hat{\epsilon} \sum_{ k=1  }^{ n } \Delta {\alpha}_{k } + 2 \tilde{M} \hat{\epsilon} \\
                                                                                       &= \hat{\epsilon} [\alpha(b) - \alpha(a)] + 2 \tilde{M} \hat{\epsilon} \\
                                                                                       &= [\alpha(b) - \alpha(a) + 2 \tilde{M}] \hat{\epsilon} \\
                                                                                       &= [\alpha(b) - \alpha(a) + 2 \tilde{M}] \cdot \frac{ \epsilon }{ \alpha(b) - \alpha(a) + 2 \tilde{M} + 1  }  < \epsilon
            \end{align*}
            as desired.
    \end{enumerate}
\end{proof}

\subsection{Plan}

\begin{itemize}
    \item {\hyperref[Sequential Criterion for integrability]{Sequential Criterion for integrability}};
    \item {\hyperref[Algebraic properties of R.S integral]{Algebraic properties of R.S integral}};
    \item {\hyperref[Order properties of R.S integrals]{Order properties of R.S integrals}};
    \item {\hyperref[Mean Value Theorem and Generalized Mean Value Theorem for integrals]{Mean Value Theorem and Generalized Mean Value Theorem for integrals}};
    \item {\hyperref[Additivity for R.S integrals]{Additivity for R.S integrals}}.
\end{itemize}

\begin{theorem}[Sequential Criterion for R.S Integrability]\label{Sequential Criterion for integrability}
    Let \( f: [a,b] \to \R  \) be a bounded function and \( \alpha: [a,b] \to \R  \) is an increasing function. Then
    \begin{enumerate}
        \item[(1)] If \( f \in {R}_{\alpha}[a,b]  \), then there exists a sequence of partitions \( ({P}_{n})_{n \geq 1}  \) in \( \Pi [a,b] \) such that 
            \[ \lim_{ n \to \infty  }  [U(f,\alpha, {P}_{n}) - L(f,\alpha, {P}_{n}) ] = 0.  \]
        \item[(2)] If there exists a sequence of partitions \( ({P}_{n})_{n \geq 1} \) in \( \Pi [a,b] \) such that \[ \lim_{ n \to \infty  } [U(f,\alpha, {P}_{n}) - L(f,\alpha, {P}_{n})] = 0,  \] then \( f \in {R}_{\alpha}[a,b] \), and 
            \[  \int_{ a }^{ b }  f  \ d \alpha = \lim_{ n \to \infty  } k U(f,\alpha, {P}_{n}) = \lim_{ n \to \infty  }  L(f,\alpha, {P}_{n}).  \]
    \end{enumerate}
\end{theorem}
\begin{proof}
\begin{enumerate}
    \item[(1)] Using the Cauchy Criterion, we see that \( f \in {R}_{\alpha}[a,b] \) if and only if for all \( \epsilon > 0  \), there exists \( {P}_{\epsilon} \in \Pi [a,b] \) such that 
        \[  U(f,\alpha,{P}_{\epsilon}) - L(f,\alpha, {P}_{\epsilon})  < \epsilon. \]
        In particular, we can inductively construct a sequence of partitions \( ({P}_{n})_{n \geq 1 } \) in the following way: for all \( n \in \N \), let \( \epsilon = \frac{ 1 }{ n }  \). Then there exists \( {P}_{n} \in \Pi  \) such that  
        \[   0 \leq U(f,\alpha, {P}_{n}) - L(f,\alpha, {P}_{n}) < \frac{ 1 }{ n }.  \]
       From the squeeze theorem, it follows that  
       \[  \lim_{ n \to \infty  } [U(f,\alpha, {P}_{n}) - L(f,\alpha, {P}_{n})] = 0.  \]
    \item[(2)] According to Cauchy Criterion, it suffices to show that 
        \[  \forall \epsilon > 0 \ \exists P \in \Pi [a,b] \ \text{such that} \ U(f,\alpha,P) - L(f,\alpha,P) < \epsilon. \]
        Since \( \lim_{ n \to \infty  } [U(f,\alpha, {P}_{n}) - L(f,\alpha, {P}_{n})] = 0  \), there exists an \( N \in \N \) such that 
        \[  \forall n > N \ \ U(f,\alpha,{P}_{n}) - L(f,\alpha,{P}_{n}) < \epsilon. \]
        In particular, \( {P}_{N+1} \) can be used as the \( P  \) that we were looking for. It remains to show that 
        \[  \int_{ a }^{ b }  f  \ d \alpha = \lim_{ n \to \infty  }  U(f,\alpha,{P}_{n}) = \lim_{ n \to \infty  }  L(f,\alpha, {P}_{n}). \]
        We have for all \( n \geq 1  \), 
        \[  0 \leq U(f,\alpha,{P}_{n}) - U(f,\alpha) \leq U(f,\alpha, {P}_{n}) - L(f,\alpha) \leq U(f,\alpha, {P}_{n}) - L(f,\alpha, {P}_{n}). \]
        Using the squeeze theorem on the inequality above, we have 
        \[  \lim_{ n \to \infty  } [L(f,\alpha) - L(f,\alpha,{P}_{n})] = 0.  \]
        So, 
        \[  \lim_{ n \to \infty  } L(f,\alpha,{P}_{n}) = L(f,\alpha) = \int_{ a }^{ b }  f  \ d \alpha. \]
\end{enumerate}
\end{proof}

\begin{theorem}[Algebraic Properties of R.S Integral]\label{Algebraic properties of R.S integral}
    Assume \( f,g \in {R}_{\alpha}[a,b] \). Then
    \begin{enumerate}
        \item[(i)] \( \forall k \in \R  \), \( kf \in {R}_{\alpha}[a,b] \) with \[ \int_{ a }^{ b }  kf  \ d \alpha = k \int_{ a }^{ b }  f  \ d \alpha \];
        \item[(ii)] \( f + g \in {R}_{\alpha}[a,b] \) with 
            \[  \int_{ a }^{ b }  f + g  \ d \alpha = \int_{ a }^{ b }  f  \ d \alpha  + \int_{ a }^{ b }  g  \ d \alpha. \]
        \item[(iii-1)]  \( f^{2} \in {R}_{\alpha}[a,b] \);
        \item[(iii-2)] \( fg \in {R}_{\alpha}[a,b] \);
        \item[(iv-1)] if \( g \neq 0  \) on \( [a,b] \) and \( \frac{ 1 }{ g }  \) is bounded on \( [a,b] \), then \( \frac{ 1 }{ g }  \in {R}_{\alpha}[a,b] \);
        \item[(iv-2)] if \( g \neq 0  \) on \( [a,b] \) and \( \frac{ 1 }{ g }   \) is bounded on \( [a,b] \), then \( \frac{ 1 }{ g }  \in {R}_{\alpha}[a,b] \).
    \end{enumerate}
\end{theorem}


% end of lecture 11

\begin{center}
    \textit{End of Lecture 11} 
\end{center}
