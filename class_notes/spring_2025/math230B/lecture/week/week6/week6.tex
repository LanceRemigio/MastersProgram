\section{Plan}

\begin{itemize}
    \item Uniform convergence
    \item Uniform convergence and boundedness
    \item Uniform convergence and continuity
    \item Cauchy criterion for uniform convergence
    \item Uniform convergence and differentiability
    \item Uniform Convergence and integrability
\end{itemize}

\begin{definition}[Uniform Convergence]
    We say (\( {f}_{n} : A \to \R  \)) \textbf{converges uniformly} to \( f: A \to \R  \) if 
    \[  \forall \epsilon > 0 \ \exists {N}_{\epsilon} \ \text{such that} \ \forall n > {N}_{\epsilon} \ \forall x \in A \ | {f}_{n}(x) - f(c) |  < \epsilon. \]
\end{definition}

\begin{theorem}[Uniform Convergence Preserves Boundedness]
    Let \( A \neq \emptyset  \), for each \( n \in \N \), \( {f}_{n}: A \to \R  \) is bounded, and \( {f}_{n} \to f  \) uniformly on \( A  \). Then \( f : A \to \R  \) is bounded.
\end{theorem}

\begin{remark}
    Please make a clear distinction between the following statements:
    \begin{enumerate}
        \item[(1)] For all \( n \in \N \), \( {f}_{n}: A \to \R  \) is bounded:
            \[  \forall n \in \N \ \exists \hat{M}_n \ \text{such that} \ | {f}_{n}(c) |  \leq \hat{M}_n. \]
        \item[(2)] For all \( ({f}_{n})_{n \geq 1 } \) is uniformly bounded:
            \[  \exists M \ \text{such that} \ \forall n \geq 1 \ \forall x \in A \ \ | {f}_{n}(x) |  \leq M.  \]
        \item[(3)] \( ({f}_{n})_{n \geq 1 } \) is pointwise bounded:
            \[ \forall x \in A \ ({f}_{n}(x))_{n \geq 1 } \ \text{is bounded}. \]
    \end{enumerate}
\end{remark}

\begin{proof}
Our goal is to show that 
\[  \exists M \in \R \ \text{such that} \ \forall x \in A \ | f(x) |  \leq M.   \]
Since \( {f}_{n} \to f  \) uniformly on \( A  \), we have for all \( \epsilon > 0  \), there exists \( N  \) such that for all \( n > N  \) and for all \( x \in A  \), 
\begin{align*}
&| {f}_{n}(x) - f(x) | < \epsilon   \\
&\implies | f(x)  | - | {f}_{n}(x) |  < \epsilon \\
&\implies | f(x) |  < \epsilon + | {f}_{n}(x) |.
\end{align*}
In particular, for \( \epsilon  = 1  \), there exists \( N \in \N \) such that 
\[  \forall n > N \ \forall x \in A \ | f(x) |  < 1 + | {f}_{n}(x) |. \]
Now, if we let \( n = N + 1  \), we get
\[  \forall x \in A \ \ | f(x) |  < 1 + | {f}_{N+1}(x) |. \tag{1} \]
Since, by assumption, \( {f}_{N+1}  \) is bounded, there exists a number \( {\hat{M}}_{N+1} \) such that 
\[  \forall x \in A \ \ | {f}_{N+1}(x) | \leq {\hat{M}}_{N+1}. \tag{2} \]
It follows from (1) and (2) that
\[  \forall x \in A \ \ | f(x)| < 1 + {\hat{M}}_{N+1}.  \]
Clearly, we can use \( 1  + {\hat{M}}_{N+1} \) as the same \( M  \) we were looking for.
 \end{proof}

 \begin{theorem}[Rudin 7.12]
     Let \( A \subseteq  (X,d)  \) and \( x \in A  \). Suppose for all \( n \in \N  \), \( {f}_{n}: A \to \R  \) is continuous at \( c  \) and \( {f}_{n} \to f  \) uniformly on \( A  \). Then \( f: A \to \R  \) is continuous at \( c  \).
 \end{theorem}
 \begin{proof}
 Our goal is to show that 
 \[  \forall \epsilon > 0 \ \exists \delta  > 0 \ \text{if} \ d(x,c)  <\delta \ \text{then} \ | f(x) - f(c) | < \epsilon.  \]
 Let \( \epsilon > 0  \) be given. Since \( {f}_{n} \to f  \) uniformly on \( A  \), there exists \( N \in \N  \) such that for all \( n  > N  \) and for all \( z \in A  \), we have
 \[  | {f}_{n}(z) - f(z) |  < \frac{ \epsilon }{ 3 }. \tag{1} \]
 Also, since \( {f}_{N+1}  \) is continuous at \( c  \), 
 \[  \exists \hat{\delta} > 0 \ \text{such that} \ \forall x \in {N}_{\hat{\delta}}(c) \cap A \ \ | {f}_{N+1}(x) - {f}_{N+1}(c) |  < \frac{ \epsilon }{ 3 }. \tag{2} \]
 We claim that \( \hat{\delta} > 0  \) can be used as the same \( \hat{\delta} \) that we were looking for. Indeed, for all \( x \in {N}_{\hat{\delta}}(c) \cap A  \), we have 
 \begin{align*}
    | f(x)  - f(c)|  &\leq | f(x) - {f}_{N+1}(x) | + | {f}_{N+1}(x) - {f}_{N+1}(c) | {f}_{N+1}  (c) - f(c) |   \\
                     &< \frac{ \epsilon }{ 3 }  + \frac{ \epsilon }{ 3 }  + \frac{ \epsilon }{ 3 }  \\
                     &= \epsilon
 \end{align*}
 as desired.
 \end{proof}

 \begin{remark}[A Useful Observation]\label{Useful Observation}
    Let \( ({a}_{n}) \) be a sequence of real numbers. Suppose \( {a}_{n} \to a  \) in \( \R  \). Suppose there exists \( N \) such that  
    \[  \forall m, n > N  \ \ | {a}_{n} - {a}_{m} | < \frac{ 1 }{ 3 }. \]
    So, by taking the limit as \( n \to \infty   \), it follows from the order limit theorem that for each \( n > N  \), we have 
    \[  \lim_{ m \to \infty  }  | {a}_{n} - {a}_{m} |  \leq \lim_{ m \to \infty  }  \frac{ 1 }{ 3 } = \frac{ 1 }{ 3 }. \]
    More generally, given \( \epsilon > 0  \), if there exists \( N  \) such that
    \[  \forall n,m > N \ \ | {a}_{n} - {a}_{m} |  \leq \epsilon. \]
 \end{remark}

 We will use the remark above to prov the following theorem: 

 \begin{theorem}[Cauchy Criterion for Uniform Convergence]\label{Cauchy Criterion for Uniform Convergence}
     Let \( A \neq \emptyset  \) and suppose for each \( n \in \N  \), \( {f}_{n}: A \to \R  \) is a sequence of functions. Then \( ({f}_{n})_{n \geq 1 } \) converges uniformly on \( A  \) if and only if for all \( \epsilon > 0 \), there exists \( N  \) such that for all \( m,n > N  \) and for all \( x \in A  \), \( | {f}_{n}(x) - {f}_{m}(x) |  < \epsilon. \)
 \end{theorem}
 \begin{proof}
     (\( \Longrightarrow \)) Suppose there exists \( f: A \to \R  \) such that \( {f}_{n} \to f  \) uniformly on \( A  \). Our goal is to find an \( N  \) such that for all \( m,n >  N  \) and for all \( x \in A  \)
 \[ | {f}_{n}(x) - {f}_{m}(x) | < \epsilon. \tag{*} \]
 Since \( {f}_{n} \to f  \) uniformly on \( A  \), for the given \( \epsilon > 0  \), there exists \( \hat{N} \) such that 
 \[  \forall k > \hat{N} \ \ \forall x \in A \ \ | {f}_{k } (x) - f(x) | < \frac{ \epsilon }{ 2 }.   \]do
 We claim that this \( \hat{N} \) can be used as the \( N  \) that we were looking for. Indeed, if \( m,n > \hat{N} \) and \( x \in A  \), then
 \begin{align*}
     | {f}_{n}(x) - {f}_{m}(x) | &\leq | {f}_{n}(x)- f(x) |  + | {f}_{m}(x) - f(x) |  \\
                                 &< \frac{ \epsilon }{ 2 }  + \frac{ \epsilon }{ 2 } \\
                                 &= \epsilon
 \end{align*}
 as desired.

    (\( (\Longleftarrow) \)) Suppose for all \( \epsilon > 0  \), there exists \( N  \) such that for all \( n,m > N  \) and for all \( x \in A  \)
    \[  | {f}_{n}(x) - {f}_{m}(x) |  < \epsilon. \]
    Our goal is to show that \( ({f}_{n})_{n \geq 1 } \) converges uniformly on \( A  \). It follows from the assumption that at each point \( x \in A  \), the sequence of real numbers \( ({f}_{n}(x))_{n \geq 1 } \) is a Cauchy sequence in \( \R  \). Since \( \R \) is complete, we can conclude that at each point \( x \in A  \), the sequence of real numbers \( ({f}_{n}(x))_{n \geq 1} \) that converges. This tells us that the sequence of functions \( ({f}_{n})_{n \geq 1 } \) is pointwise convergent on \( A  \). Let's denote the pointwise limit of \( ({f}_{n})_{n \geq 1 }  \) by \( f: A \to \R  \). In what follows, we will prove that \( {f}_{n} \to f  \) uniformly on \( A  \). To this end, we need to show   
    \[  \forall \epsilon > 0  \ \exists N  \ \text{such that} \ \forall n > N \ \forall x \in A \ | {f}_{n}(x) - f(x) |  < \epsilon. \] 
    Let \( \epsilon > 0  \) be given. It follows from the assumption that for the given \( \epsilon> 0  \), there exists \( \hat{N} \) such that 
    \[  \forall m,n > \hat{N} \ \forall x \in A \ | {f}_{n}(x) - {f}_{m}(x) |  < \frac{ \epsilon }{ 2 }. \]
    We claim that this \( \hat{N} \) can be used as the \( N  \) we were looking for. Indeed, if \( n > \hat{N} \) and \( x \in A  \), then 
    \[  \forall m > \hat{N} \ \ | {f}_{n}(x) - {f}_{m}(x) |  < \frac{ \epsilon }{ 2 }.  \]
So, by taking the limit as \( m \to \infty   \) (using the {\hyperref[Useful Observation]{Useful Observation}}), we have 
\[ | {f}_{n}(x) - f(x) |  \leq \frac{ \epsilon }{ 2 }  < \epsilon  \]
as desired.
\end{proof}

\begin{theorem}[Rudin 7.17]
    Suppose for each \( n \in \N \), \( {f}_{n} : [a,b] \to \R  \) is a sequence of differentiable functions and \( {f}_{n} \to f  \) pointwise. Assume that \( {f}_{n}'  \) converges uniformly to a function \( g  \) on \( [a,b] \). Then \( f \) is differentiable to a function \( g  \) on \( [a,b] \).
\end{theorem}

\begin{proof}
    Our goal is to show that for all \( c \in [a,b] \), \( f'(c) = g(c) \). Let \( c \in [a,b] \). We want to show that 
    \[  \lim_{ x \to c  }  \frac{ f(x)  - f(c) }{ x - c  }  = g(c). \]
    That is, we want to show that for all \( \epsilon > 0  \), there exists \( \delta > 0  \) such that if \( 0 < |  x - c  |  < \delta  \) (with \( x \in [a,b] \)), then 
    \[  \Big| \frac{ f(x) - f(c) }{  x - c  }  - g(c) \Big| < \epsilon. \]
    To this end, let \( \epsilon > 0  \) be given. Since \( {f}_{n}' \to g  \) uniformly, we can find an \( {N}_{1} \) such that for all \( n > {N}_{1}  \), for all \( z \in [a,b] \), \( | {f}_{n}' - g(z) |  < \frac{ \epsilon }{ 3 }  \). This tells us that \( ({f}_{n}') \) fulfills the {\hyperref[Cauchy Criterion for Uniform Convergence]{Cauchy Criterion for Uniform Convergence}} and so there exists an \( {N}_{2}  \) such that for all \( m,n > {N}_{2} \) and for all \( z \in [a,b] \), we have  
    \[  | {f}_{n}'(z) - {f}_{m}'(z) |  < \frac{ \epsilon }{ 3 }. \]
    Let \( N = \max \{ {N}_{2}, {N}_{2} \}  + 1  \). Also, \( {f}_{N}  \) is differentiable at \( c  \), so for our given \( \epsilon  \), there exists \( \hat{\delta} > 0  \) such that if \( 0 < |  x - c  |  < \hat{\delta} \) (with \( x \in [a,b] \))  
    \[ \Big| \frac{ {f}_{N}(x) - {f}_{N}(c) }{ x - c  } - {f}_{N}'(c) \Big|  < \frac{ \epsilon }{ 3 }.  \]
    We claim that this \( \hat{\delta} \) can be used as the \( \delta  \) that we were looking for. Indeed, if \( x \in [a,b] \) and \(  0 < |  x - c  |  < \hat{\delta} \), then 
    \begin{align*}
        \Big| &\frac{ f(x) - f(c) }{  x - c  }  - g(c) \Big|  = \Big| \frac{ f(x) - f(c) }{  x - c  } - \frac{ {f}_{N}(x) - {f}_{N}(c) }{  x - c  }  + \frac{ {f}_{N}(x) - {f}_{N}(c) }{  x - c  } - {f}_{N}'(c) + {f}_{N}'(c) - g(c) \Big|   \\
              &\leq \Big| \frac{ f(x) - f(c) }{  x - c  }  - \frac{ {f}_{N}(x) - {f}_{N}(c) }{   x - c  }  \Big|  + \Big| \frac{ {f}_{N}(x)- {f}_{N}(c) }{  x - c  }  - {f}_{N}'(c)  \Big| 
              + | {f}_{N}'(c) - g(c) |  \\
              &< \Big| \frac{ (f - {f}_{N})(x) - (f - {f}_{N})(c) }{  x - c  }  \Big|  + \frac{ \epsilon }{ 3 }  + \frac{ \epsilon }{ 3 }. 
    \end{align*}
    To complete the proof, it suffices to show that the first term on the second inequality above is less than \( \frac{ \epsilon }{ 3 }  \).
    Suppose without loss of generality that \( x < c  \) where \( x \in [a,b] \). Then for every \( m > N  \), we can apply the Mean Value Theorem to the function \( {f}_{m} - {f}_{N} \) on the interval \( [x,c] \). That is, for all \( m > N  \), there exists \( {\alpha}_{m} \in (x,c) \) such that 
    \[  ({f}_{m}- {f}_{N})'({\alpha}_{m}) = \frac{ ({f}_{m}- {f}_{N})(c) - ({f}_{m}- {f}_{N})(x) }{  c - x  }.  \]
    By (2), we know that \( | {f}_{m}'({\alpha}_{m}) - {f}_{N}'({\alpha}_{m}) |  < \frac{ \epsilon }{ 3 } \). So, we have 
    \[  \Big| \frac{ ({f}_{m} - {f}_{N})(c)- ({f}_{m} - {f}_{N})(x) }{ c - x  }  \Big|  < \frac{ \epsilon }{ 3 }. \]
    By taking the limit as \( m \to \infty   \), we get
    \[  \Big| \frac{ (f - {f}_{N})(c) - (f - {f}_{N})(x) }{  c - x  }    \Big| \leq \frac{ \epsilon }{ 3 }.  \]
    So, 
    \[  \Big| \frac{ (f - {f}_{N})(x) - (f - {f}_{N})(c) }{  x - c  }   \Big|  \leq \frac{ \epsilon }{ 3 } \]
    as desired.
\end{proof}

\begin{lemma}[lemma 1]
    Let \( A  \) be nonempty. Let \( f: A \to \R  \). Suppose \( ({f}_{n}: A \to \R ) \) is a sequence of functions. The following statements are equivalent:
    \begin{enumerate}
        \item[(1)] \( {f}_{n} \to f  \) uniformly on \( A  \);
        \item[(2)] \( \lim_{ n \to \infty  }  \sup_{x \in A } | {f}_{n}(x) - f(x) | = 0   \)
    \end{enumerate}
\end{lemma}

\begin{lemma}[lemma 2]
    Let \( \alpha: [a,b] \to \R  \) is increasing, \( f: [a,b] \to \R  \) and \( g: [a,b] \to \R  \) are bounded, and \( f \leq g  \). Then     
    \begin{center}
        \( L(f,\alpha) \leq L(g, \alpha) \) and \( U(f,\alpha) \leq U(g,\alpha) \).
    \end{center}
\end{lemma}

\begin{theorem}[Rudin 7.16]
    Let \( \alpha: [a,b] \to \R  \) is increasing, for each \( n \in \N  \) \( {f}_{n} \in {R}_{\alpha} [a,b] \) , and \( {f}_{n} \to f  \) uniformly on \( [a,b] \). Then \( f \in {R}_{\alpha}[a,b] \) and 
    \[  \int_{ a }^{ b } f  \ d \alpha = \lim_{ n \to \infty  }  \int_{ a }^{ b } {f}_{n}  \ d \alpha. \]
\end{theorem}
\begin{proof}
    Since uniform convergence preserves boundedness, we can conclude that \( f: [a,b] \to \R  \) is bounded. Now, in order to show that \( f \in {R}_{\alpha}[a,b] \), it suffices to show that \( L(f,\alpha) = U(f,\alpha) \). For each \( n \in \N \), let 
    \[  {r}_{n} = \sup_{x \in [a,b]} | {f}_{n}(x) - f(x) |.  \]
    Since \( {f}_{n} \to f  \) uniformly, we know that \( \lim_{ n \to \infty  } {r}_{n} = 0  \). For each \( n \in \N  \), we have 
    \[  {r}_{n} = \sup_{x \in [a,b]} | {f}_{n}(x) - f(x) | \implies | f(x) - {f}_{n}(x) |  \leq {r}_{n} \forall x \in [a,b]. \]
    Hence, 
    \[  \forall x \in [a,b] \ \ \ {f}_{n}(x) - {r}_{n} \leq f(x) \leq {f}_{n}(x) + {r}_{n}. \tag{*}  \]
    So, it follows from {\hyperref[lemma 2]{lemma 2}} that 
    \[  L({f}_{n}- {r}_{n}, \alpha) \leq L(f,\alpha) \leq U(f,\alpha) \leq U({f}_{n} + {r}_{n}). \]
    Thus, 
    \[  0 \leq U(f,\alpha) - L(f,\alpha) \leq U({f}_{n} + {r}_{n}, \alpha) - L({f}_{n}-{r}_{n}, \alpha ). \]
    Note that 
    \begin{align*}
        U({f}_{n} + {r}_{n}, \alpha) - L({f}_{n}- {r}_{n}, \alpha) &= \int_{ a }^{ b } ({f}_{n} + {r}_{n})   \ d \alpha - \int_{ a }^{ b }  ({f}_{n} - {r}_{n}) \ d \alpha  \\
                                                                   &= \int_{ a }^{ b }  \Big[ ({f}_{n} + {r}_{n}) - ({f}_{n} - {r}_{n}) \Big]  \ d \alpha \\
                                                                   &= \int_{ a }^{ b } 2 {r}_{n}  \ d \alpha \\
                                                                   &= 2 {r}_{n} [\alpha(b) - \alpha(a)].
    \end{align*}
    So, 
    \[  0 \leq U(f,\alpha) - L(f,\alpha) \leq 2 {r}_{n} [\alpha(b) - \alpha(a)]. \]
    Using the Squeeze Theorem, we have \( U(f,\alpha) = L(f,\alpha) \) (by applying the limit as \( n \to \infty  \)). Now, it follows from (*) that 
    \[  \int_{ a }^{ b }  ({f}_{n}- {r}_{n})  \ d \alpha \leq \int_{ a }^{ b }  f  \ d \alpha \leq \int_{ a }^{ b }  ({f}_{n} + {r}_{n})   \ d \alpha. \]
    So, 
    \[  \int_{ a }^{ b } (-{r}_{n})  \ d \alpha \leq \int_{ a }^{ b }  f  \ d \alpha - \int_{ a }^{ b }  {f}_{n} \ d \alpha \leq \int_{ a }^{ b }  {r}_{n} \ d \alpha. \]
    Thus, 
    \[  - {r}_{n} [\alpha(b) - \alpha(a)] \leq \int_{ a }^{ b }  f  \ d \alpha - \int_{ a }^{ b }  {f}_{n}  \ d \alpha \leq {r}_{n} [\alpha(b) - \alpha(a)]. \]
    Using the Squeeze Theorem as \( n \to \infty   \), we have
    \[  \lim_{ n \to \infty  }  \Big[ \int_{ a }^{ b }  f  \ d \alpha - \int_{ a }^{ b }  {f}_{n} \ d \alpha \Big] = 0.  \]
    That is, 
    \[ \lim_{ n \to \infty   } \int_{ a }^{ b }  {f}_{n} \ d \alpha = \int_{ a }^{ b }  f  \ d \alpha. \]
    \[   \]
\end{proof}
