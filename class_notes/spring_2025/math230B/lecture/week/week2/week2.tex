
\section{Lecture 2}

\begin{theorem}[Chain Rule]
    Let \( {I}_{1} \subseteq  \R  \) and \( {I}_{2} \subseteq  \R   \) be two intervals, \( f: {I}_{1} \to \R  \) and \( g : {I}_{2} \to \R  \) be two functions, \( f({I}_{1}) \subseteq  {I}_{2} \), \( f \) is differentiable at \( c \in {I}_{1} \) and \( g  \) is differentiable at \( f(c) \in {I}_{2} \). Then the function \( g \circ f : {I}_{1} \to \R  \) is differentiable at \( c \in {I}_{1} \) and 
    \[  (g \circ f)'(c) = g'(f(c)) \cdot f'(c). \]
\end{theorem}
First, we will prove the theorem incorrectly and then show give three criterion to prove the theorem correctly.  

\begin{proof}
Observe that 
\begin{align*}
    \lim_{ x \to c  }  \frac{ (g \circ f)(x) - (g \circ f)(c) }{x - c } &= \lim_{ x \to c  }  \frac{ g(f(x)) - g(f(c)) }{  x - c  }  \\
                                                                        &= \lim_{ x \to c  }  \Big[ \frac{ g(f(x)) - g(f(c)) }{  f(x) - f(c) } \cdot \frac{ f(x) - f(c) }{  x - c   } \Big] \\
                                                                        &= \underbrace{\Big[ \lim_{ x \to c  }  \frac{ g(f(x)) - g(f(c)) }{  f(x) - f(c) } \Big]}_{g'(f(c))} \underbrace{\Big[ \lim_{ x \to c  }  \frac{ f(x)  - f(c) }{  x - c  } \Big]}_{f'(c)} \\
\end{align*}
\end{proof}
What is the problem with this proof? By the definition of a limit of a function, when you take \( \lim_{ x \to c  }  \), it is guaranteed that \( x - c \neq 0   \); however, for \( x  \) close to \( c  \) (as \( x  \) approaches to \( c  \)), \( f(x)- f(c) \) might be zero, so dividing by \( f(x) - f(c) \) is not legitimate. The following proof fixes this issue by introducing a new function \( d(f(x)) \) which is defined by
\begin{enumerate}
    \item[(i)] \( d(f(x)) \displaystyle \frac{ g(f(x)) - g(f(c)) }{  f(x) - f(c) }    \) when \( f(x) \neq f(c) \) 
    \item[(ii)] \( d(f(x))  \) is defined even when \( f(x) = f(c) \)  
    \item[(iii)] \( \displaystyle d(f(x)) \cdot \frac{ f(x) - f(c) }{  x - c  }  = \frac{ g(f(x)) - g(f(c)) }{  x - c  }   \) for all \( x \in {I}_{\kappa} \) where \( x \neq c   \).
\end{enumerate}

\begin{proof}
Let \( d: {I}_{2} \to \R  \) be defined by 
\[  d(y) = 
\begin{cases}
    \frac{ g(y) - g(f(c)) }{ y - f(c) }  &\text{if} \ y \neq f(c) \\
    \lim_{ y \to f(c) } \frac{ g(y) - g(f(c)) }{  y - f(c) }  = g'(f(c)) &\text{if} \ y = f(c)
\end{cases}. \]
Note that this function satisfies the requirements in (i) and (ii) outlined above. We make the following observations: 
\begin{enumerate}
    \item[(1)] \( d  \) is continuous at \( f(c) \). Indeed, we can see that 
        \[  \lim_{ y \to f(c) }  d(y) = \lim_{ y  \to  f(c) }  \frac{ g(y) - g(f(c)) }{ y - f(c) }  = d(f(c)). \]
    \item[(2)] For all \( x \in {I}_{1} \) and \( x \neq c  \), we have 
        \[  d(f(x)) \cdot \frac{ f(x) - f(c) }{  x -c  }  = \frac{ g(f(x)) - g(f(c))  }{  x - c  }. \tag{*}  \]
        We will show that this holds by considering two cases; either \( f(x) \neq f(c) \) or \( f(x) = f(c) \). If \( f(x) \neq f(c) \), then
        \[  \text{LHS} = d(f(x))\cdot \frac{ f(x) - f(c) }{ x - c  } = \frac{ g(f(x)) - g(f(c)) }{  f(x) - f(c) }  \cdot \frac{ f(x) - f(c) }{  x - c  } = \frac{ g(f(x)) - g(f(c))  }{  x - c  }  = \text{RHS}.    \]
        Now, suppose \( f(x) = f(c) \). Then we have 
        \begin{align*}
            \text{LHS} &= d(f(x)) \cdot \frac{ f(x) - f(c) }{  x - c  }  = d(f(c)) \cdot \frac{ f(x) - f(c) }{ x - c  }  = g'(f(c)) \cdot \frac{ 0  }{ x -c  }  = 0  \\
            \text{RHS} &= \frac{ g(f(x)) - g(f(c)) }{  x - c  }  = \frac{ g(f(c)) - g(f(c)) }{ x - c  }  = \frac{ 0  }{ x -c  }   =0.
        \end{align*}
        Thus, we see that the left hand side equals the right hand side of (*).
\end{enumerate}
Now, note that since \( f  \) is continuous at \( c  \) and \( d \) is continuous at \( f(c) \), their composition \( d \circ f  \) is continuous at \( c  \) and so,
\[  \lim_{ x \to c  }  (d \circ f)(x) = (d \circ f)(c). \]

\begin{align*}
    &\lim_{ x \to c  } \frac{ (g \circ f)(x) - (g \circ f)(c) }{ x - c  } = \lim_{ x \to c  }  \frac{ g(f(x)) - g(f(c)) }{ x - c  }  \\
                                            &= \Big[ \lim_{ x  \to c  }  (d \circ f ) (x) \Big] \cdot \Big[ \lim_{ x \to  c  }  \frac{ f(x) - f(c) }{  x - c  } \Big] \\
                                            &= [(d \circ f)(c)] \cdot f'(c) \\
                                            &= [d(f(c))] \cdot f'(c) \\
                                            &= g'(f(c)) \cdot f'(c).
\end{align*}
\end{proof}

\section{Lecture 3}

\subsection{Topics}

\begin{enumerate}
    \item[(1)] Local Maxima and minima
    \item[(2)] Interior Extremum Theorem (Theorem 5.8)
    \item[(3)] Darboux's Theorem (Theorem 5.12)
    \item[(4)] Some observations
    \item[(5)] Rolle's Theorem
    \item[(6)] Mean Value Theorem
\end{enumerate}

\begin{theorem}[Interior Extremum Theorem]\label{Theorem 5.8}
    Let \( I \subseteq  \R   \) be an interval and \( f : I \to \R  \). Suppose \( c  \) is an interior point of \( I  \) and \( f  \) is differentiable at \( c  \). Then
    \begin{enumerate}
        \item[(1)] If \( f  \) has a local max at \( c  \), then \( f'(c)  = 0 \);
        \item[(2)] If \( f  \) has a local min at \( c  \), then \( f'(c) = 0  \).
    \end{enumerate}
\end{theorem}

Before we prove this theorem, we will first go over an important lemma that is used in the main proof. 

\begin{lemma}
    Suppose \( \lim_{ x \to c  }  g(x)  \) and \( \lim_{ x \to c  }  h(x) \) both exist. 
    \begin{enumerate}
        \item[(1)] If there exists \( \delta > 0  \) such that \( h(x) \leq g(x) \) for all \( x \in (c -  \delta, c ) \), then \( \lim_{ x \to c  }  h(x) \leq \lim_{ x \to c  }  g(x) \).
        \item[(2)] If there exists \( \delta > 0  \) such that \( h(x) \leq g(x) \) for all \( x \in (c , c + \delta) \), then \( \lim_{ x \to c  }  h(x) \leq \lim_{ x \to c  }  g(x) \).
    \end{enumerate}
\end{lemma}
\begin{proof}
Here we will prove (1). The proof of (2) is analogous. Let \( ({a}_{n}) \) be a sequence in \( (c - \delta, c) \) such that \( {a}_{n} \to c  \). By the Sequential Criterion for limits of functions, we have \( {a}_{n} \to c  \) implies
\( \lim_{ n \to \infty  }  g({a}_{n}) = \lim_{ x \to c  }  g(x) \) and \( \lim_{ n \to \infty \infty   } h({a}_{n}) = \lim_{ x \to c  }  h(x) \). Also, note that from the Order Limit Theorem for sequences, we can see that   
\begin{align*}
    \forall n \ {a}_{n} \in (c - \delta, c ) &\implies \forall n \ h({a}_{n}) \leq g({a}_{n}) \\
                                             &\implies \lim_{ n \to \infty  }  h({a}_{n}) \leq \lim_{ n \to \infty  }  g({a}_{n}).
\end{align*}
Hence, we can see from these two observations that 
\[  \lim_{ x \to c  }  h(x) \leq \lim_{ x \to c  }  g(x). \]
\end{proof}

\subsection{Proof of the Interior Extremum Theorem}

\begin{proof}
Here we will prove (1). Suppose \( f \) has a local max at \( c  \). Then
\begin{itemize}
    \item If \( f  \) has a local max at \( c  \), then there exists \( {\delta}_{1} > 0  \) such that for all \( x \in (c - {\delta}_{1}, c + {\delta}_{1}) \cap I  \) \( f(x) \leq f(c) \).
    \item If \( c  \) is an interior point of \( I  \), then there exists \( {\delta}_{2} > 0  \) such that \( (c - {\delta}_{2}, c + {\delta}_{2}) \subseteq  I  \). So, if we let \( \delta = \min \{ {\delta}_{1}, {\delta}_{2} \}  \), then
        \[  \forall x \in (c - \delta, c + \delta )  \ \ f(x) \leq f(c). \]
\end{itemize}
We have 
\begin{enumerate}
    \item[(I)] For all \( x \in (c - \delta , c ) \), we see that \( x - c < 0  \) and \( f(x) \leq f(c) \) implies that 
        \[  \frac{ f(x) - f(c) }{ x - c  } \geq 0. \]
        By the Order Limit Theorem for functions, we have 
        \[  \lim_{ x \to c  }  \frac{ f(x) - f(c) }{ x - c  }  \geq \lim_{ x \to c  } 0 \implies f'(c) \geq 0.  \]
    \item[(II)] For all \( x \in (c, c + \delta) \). Since \( x - c > 0  \) and \( f(x) \leq f(c) \), we have 
        \[  \frac{ f(x) - f(c) }{ x - c  }  \leq 0.  \]
        Using the Order Limit Theorem again, we have 
        \[ \lim_{ x \to c  } \frac{ f(x) - f(c) }{  x-  c  }  \leq \lim_{ x \to c  }  0 \implies f'(c) \leq 0.   \]
\end{enumerate}
From (I) and (II), we can see that \( f'(c) \leq 0  \) and \( f'(c) \geq 0  \). Thus, \( f'(c) = 0  \).
\end{proof}

\begin{theorem}[Darboux's Theorem]
    Let \( f: [a,b] \to \R  \) be differentiable and \( f'(a) < f'(b) \) (or \( f'(b) < f'(a) \)). Let \( \alpha \in \R  \) be such that \( f'(a) < \alpha < f'(b)  \) (or \( f'(b) < \alpha < f'(a) \)). Then there exists \( c \in (a,b) \) such that \( f'(c) = \alpha \).
\end{theorem}
\begin{proof}
    Let \( g:[a,b] \to \R  \) be defined by \( g(x) = f(x) - \alpha x  \). It follows from the Algebraic Differentiability Theorem that \( g  \) is differentiable on \( [a,b] \) and so it is continuous on \( [a,b] \). It suffices to show that there is a point \( c \in (a,b) \) such that \( g'(c) = 0 \). To this end, it is enough to show that there exists a point \( c \in (a,b) \) at which \( g  \) has a local min. Since \( g  \) is continuous on \( [a,b] \) and \( [a,b]  \) is compact, \( g  \) attains its minimum on \( [a,b] \). Let \( \hat{c} \) be a point at which \( g  \) attains a minimum. IN what follows we will show that \( \hat{c} \in (a,b) \) and so it can be used as the same \( c  \) that we were looking for. Note that  
    \begin{align*}
        g'(a) = f'(a) - \alpha &<  0 \\
        g'(b) = f'(b) - \alpha > 0. 
    \end{align*}
    We make two claims; namely, \( \hat{c} \neq a  \) (\( \hat{c}  \) cannot be the left-endpoint of \( [a,b] \)) and \( \hat{c} \neq b  \) (the right-endpoint of \( [a,b] \)). Suppose for contradiction that \( \hat{c}= a  \). Then for all \( x \in [a,b] \) \( g(x) \geq g(a) \). So, for all \( x \in (a,b) \), we have \( g(x) - g(a) \geq 0  \) and \( x - a > 0  \). Thus, for all \( x \in (a,b) \)
    \[  \frac{ g(x) - g(a) }{ x - a  }  \geq 0.  \]
    Thus, 
    \[  \lim_{ x \to a }  \geq \frac{ g(x) - g(a) }{  x - a  }  \geq 0 \implies \lim_{ x \to a  }  0.   \]
    That is, \( g'(a) \geq 0  \). This contradicts the fact that \( g'(a) < 0 \) and this proves the first claim.
    A similar argument shows that \( g'(b) \leq 0  \), contradicting the fact that \( g'(b) > 0  \).
\end{proof}


% end of week 2


% end of lecture 4


\begin{theorem}[Taylor's Theorem with Lagrange Remainder]
    Let \( I \subseteq  \R   \) be an open interval, \( {x}_{0} \in I  \), and \( n \in \N \cup \{ 0  \}  \). Suppose \( f: I \to \R  \) has \( n + 1  \) derivatives. Then for each point \( x \neq {x}_{0} \) in \( I  \), there is a point \( c  \) strictly between \( x  \) and \( {x}_{0} \) such that 
    \[  f(x) = \sum_{ k=0  }^{  n } \frac{ f^{(k)}({x}_{0})  }{ n!  } (x - {x}_{0})^{k} + \frac{ f^{(n+1)}(c) }{  (n+1)! } (x - {x}_{0})^{n+1} \]
    where 
    \begin{align*}
        {T}_{n,{x}_{0}}(x) &= \sum_{ k=0  }^{ n } \frac{ f^{(k)}({x}_{0}) }{ n!  }  ( x - {x}_{0})^{k} \\
        {R}_{n,{x}_{0}}(x) &= \frac{ f^{(n+1)}(c) }{  (n+1)! }  (x - {x}_{0})^{n+1}.
    \end{align*}
\end{theorem}

\begin{remark}
    \begin{itemize}
        \item Note that clearly the equality above holds at \( x = {x}_{0} \) too (for any value of \( c  \)).
        \item Recall that for any fixed number \( R  \)
            \[ \lim_{ n \to \infty  }  \frac{ R^{n+1} }{ (n+1)! }  = 0.   \]
            However, \( f^{(n+1)}(c) \) may become very large.
    \end{itemize}
\end{remark}

\subsection{Proof of Taylor's Theorem}
\begin{proof}
Let \( {R}_{n,{x}_{0}}(x) = f(x) - {T}_{n,{x}_{0}}(x)  \). Our goal is to show that 
\[  {R}_{n,{x}_{0}}(x) = \frac{ f^{(n+1)}(c) }{ (n+1)! } (x - {x}_{0})^{n+1} \]
for some \( c  \) between \( x  \) and \( {x}_{0} \). Note that 
\begin{enumerate}
    \item[(i)] By assumption, \( f \) contains \( n + 1  \) derivatives and \( {T}_{n,{x}_{0}}  \) is a polynomial which also contains \( n + 1  \) derivatives. Also, we have 
        \[  {R}_{n,{x}_{0}} = f - {T}_{n,{x}_{0}}. \]
        Thus, \( {R}_{n,{x}_{0}}  \) must have \( n + 1  \) derivatives.
    \item[(ii)] For all \( 0 \leq k \leq n  \),
        \[  {R}_{n, {x}_{0}}^{(k)}({x}_{0}) = f^{(k)}({x}_{0}) - {T}_{n,{x}_{0}}^{(k)}({x}_{0}) = 0. \]
\end{enumerate}
Using (i), (ii), and corollary 5, we can see that for each \( x \neq {x}_{0} \), we have 
\[  R_{n,{x}_{0}}(x) = \frac{ {R}_{n,{x}_{0}}^{(n+1)}(c) }{ (n+1)! }  (x - {x}_{0})^{n+1} \tag{I} \]
for some \( c  \) strictly between \( x  \) and \( {x}_{0} \). Now, note that 
\[  {R}_{n,{x}_{0}}^{(n+1)}(c) = f^{(n+1)}(c) - {T}_{n,{x}_{0}}^{(n+1)}(c) = f^{(n+1)}(c). \tag{II}  \]
Using (I) and (II), we have 
\[  {R}_{n,{x}_{0}} = \frac{ f^{(n+1)}(c) }{ (n+1)! } (x - {x}_{0})^{n+1}.\]
\end{proof}

