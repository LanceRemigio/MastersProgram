\section{Lecture 15}

\begin{theorem}[Integration by Parts]
    Let \( u : [a,b] \to \R  \) and \( v : [a,b] \to \R  \) are differentiable and let \( u' \in R[a,b] \) and \( v' \in R[a,b] \). Then we have  
    \begin{enumerate}
        \item[(1)] \( uv' \in R [a,b] \)
        \item[(2)] \( u' v \in R[a,b] \)
        \item[(3)] \( \displaystyle  \int_{ a }^{ b }  uv'  \ dx = u(b)v(b) - u(a) v(a) - \displaystyle \int_{ a }^{ b } u' v  \ dx  \).
    \end{enumerate}
\end{theorem}

\begin{proof}
\begin{enumerate}
    \item[(1)] Since \( u:[a,b] \to \R  \) is differentiable, we have \( u \in C[a,b] \). So, we have \( u \in R[a,b] \). By assumption, \( v' \in R[a,b] \) and so we can conclude that \( uv' \in R[a,b] \).
    \item[(2)] Using the same argument above, we have \( uv' \in R[a,b] \).
    \item[(3)] By the product rule, we have 
        \[  (uv)' = u' v + u v'. \]
        In particular, since \( (uv)' \) is a sum of integrable functions, it belongs to \( R[a,b] \). Now, we integrate both sides 
        \[  \int_{ a }^{ b }  (uv)'   \ dx = \int_{ a }^{ b } u'v \ dx + \int_{ a }^{ b }  uv' \ dx. \tag{I} \]
        According to FTC I, we have
        \[  \int_{ a }^{ b } (uv)'  \ dx = [uv]_{x = a}^{x =b} = u(b)v(b) - u(a)v(a). \tag{II} \]
        Hence, we have (I) and (II) imply that
        \[  u(b)v(b) - u(a)v(a) = \int_{ a }^{ b } u'v \ dx + \int_{ a }^{ b }  u v' \ dx \]
        which further implies that 
        \[  \int_{ a }^{ b } uv' \ dx = u(b)v(b) - u(a)v(a) - \int_{ a }^{ b } u'v \ dx. \]
\end{enumerate}
\end{proof}

\begin{definition}[Unit Step Function]
    The \textbf{unit step function} \( I: \R \to \R  \) is defined by
    \[  I(x) = 
    \begin{cases}
        0 &\text{if} \ x \leq 0 \\
        1 &\text{if} x > 0 
    \end{cases}. \]
\end{definition}

\begin{remark}
    Note that for all \( s \in |R  \), we have 
    \[  I(x-s) = 
    \begin{cases}
        0 &\text{if} \ x \leq s \\
        1 &\text{if} x > s 
    \end{cases}. \]
    Also, for all \( c \neq 0  \), we have
    \[  c I(x-s) = 
    \begin{cases}
        0 &\text{if} \ x \leq s \\
        c &\text{if} \ x > s.
    \end{cases} \]
\end{remark}

\begin{theorem}[Rudin 6.15]
    Let \( f: [a,b] \to \R  \) be a bounded function, \( s \in (a,b) \), \( f  \) is right continuous at \( s  \), and \( \alpha(x) = I(x-s) \). Then 
    \begin{center}
    \( f \in {R}_{\alpha}[a,b] \) and \( \displaystyle \int_{ a }^{ b }  f  \ d \alpha  = f(s) \).
    \end{center}
\end{theorem}
\begin{proof}
see hw4
\end{proof}

\begin{theorem}[Rudin 6.16]
    Suppose for all \( n \geq 1  \), \( {c}_{n} \geq 0  \), \( \displaystyle \sum_{ n=1  }^{ N  } {c}_{n} \) converges, \( {s}_{1} < {s}_{2} < {s}_{3} < \cdots   \) are points in \( (a,b) \), \( \alpha:[a,b] \to \R  \) is continuous. Then
    \begin{center}
        \( f \in {R}_{\alpha}[a,b] \) and \( \displaystyle \int_{ a }^{ b } f  \ d \alpha = \displaystyle \sum_{ n=1  }^{ \infty  } {c}_{n} f({s}_{n})  \).
    \end{center}
\end{theorem}
\begin{proof}
See hw4.
\end{proof}



