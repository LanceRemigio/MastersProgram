\subsection{Lecture 26}

\begin{itemize}
    \item {\hyperref[Holder Continuous functions]{Holder Continuous functions}} 
    \item {\hyperref[Absolutely Continuous Functions]{Absolutely Continuous Functions}}  
    \item {\hyperref[Continuity and oscillation]{Continuity and oscillation}} 
\end{itemize}

\begin{theorem}[Riesz Theorem]
    Let \( X = C([a,b], \R ) \) equipped with \( {d}_{\infty } \) metric. Let \( L : X \to \R  \) be a continuous linear function. Then there exists \( \alpha \in BV ([a,b]) \) such that 
    \[  L(f) = \int_{ a }^{ b } f(x) \ dx \ \ \forall f \in X. \]
\end{theorem}

\begin{remark}[Refining the Notion of Uniform Continuity]
    \begin{enumerate}
        \item[(*)] Roughly speaking, the notion of uniform continuity of \( f:[a,b] \to \R  \) can be thought of in the following way: For \( x  \) close enough to \( y \), the size of \( | f(x) - f(y) |  \) can be controlled by the size of \( | x - y  |  \) (independently of \( x  \) and \( y \)).
        \item[(*)] How can we make this notion of continuity more refined or stronger?
        \item[(*)] One way to do this is to jump to differentiability!
    \end{enumerate}
\end{remark}

\begin{remark}[Uniform Continuity in terms of differentiability]
   \begin{enumerate}
       \item[(*)] However, there are other things we can consider before jumping to differentiability. In particular, we can ask to what degree \( | f(x) - f(y) |  \) is controlled by \( | x - y  | \). How fast does \( f(y) \) approach \( f(x) \) as \( y  \) approaches \( x  \)?
        \item[(*)] for a constant function \( f  \), the fastest \( f(y) \) can approach \( f(x) \) is proportional to \( | x - y  |^{2} \). In fact, it can be shown that if \( | f(x) - f(y) |  \leq M |  x - y  |^{\theta} \) for some \( \theta > 1  \), then \( f  \) must be a constant function!
   \end{enumerate} 
\end{remark}

\begin{definition}[Holder Continuous functions Lipschitz Continuous Functions]\label{Holder Continous functions}
    Let \( 0 < \theta \leq 1  \). A function \( f: [a,b] \to \R  \) is said to be \textbf{Holder continuous with exponent \( \theta \)} if there exists a number \( M > 0  \) such that 
    \[  \forall x ,y \in [a,b] \ \ | f(x) - f(y) |  \leq M |  x-  y  |^{\theta}. \]
\end{definition}

\begin{enumerate}
    \item[(*)] Note that for a fixed \( \theta  \), the smaller the \( M  \), the better continuity.
    \item[(*)] For Holder continuous functions with exponent \( \theta = 1  \) are called Lipschitz continuous.    
    \item[(*)] Any Holder continuous function is uniformly continuous.
    \item[(*)] \( C^{0,\theta} ([a,b] ; \R ) = \{ f: [a,b] \to \R : \text{\( f  \) is Holder Continuous with exponent \( \theta \)} \}  \). We can equip the above vector space of functions with the following norm: 
        \[  \|f \|_{C^{0,\theta}} = \sup_{x \in [a,b]} | f(x) | + \sup_{x \neq y } |f(x) - f(y)| \] 
        where 
        \[  d(f,g) = \|f - g\|_{C^{0,\theta}}. \]
\end{enumerate}

\begin{remark}
    Let \( \mathcal{F} \) be a collection of function in \( C^{0,\theta}([a,b]; \R) \). If \( \mathcal{F} \) is bounded in this metric space, then 
    \begin{enumerate}
        \item[(1)] \( \mathcal{F} \) is uniformly bounded
        \item[(2)] \( \mathcal{F} \) is equicontinuous.
    \end{enumerate}
    So, every sequence in \( \mathcal{F} \) has a uniformly convergent subsequence.
\end{remark}

\begin{remark}[The Case Where \( \theta >  1  \)]
    Suppose there exists \( M > 0  \) and \( \theta > 1  \) such that 
    \[  \forall x,y \in [a,b] \ \ | f(x) - f(y) |  \leq M | x- y  |^{\theta}.  \]
    Then 
    \[  \forall x,y \in [a,b], x \neq y \ \ 0 \leq \Big| \frac{ f(x) - f(y) }{ x - y  }  \Big|  \leq M |  x -y  |^{\theta - 1}.  \]
    Therefore, 
    \[  \forall x \in [a,b] \ \ \lim_{ y \to x }  \Big| \frac{ f(x) - f(y) }{ x - y  }  \Big|  = 0.   \]
    Thus, 
    \[  \forall x \in [a,b] \ \ f'(x ) = 0.  \]
    This tells us that \( f  \) is constant on \( [a,b] \).
\end{remark}

\begin{definition}[Absolutely Continuous Functions]\label{Absolutely Continuous Functions}
    A function \( f:[a,b] \to \R  \) is said to be \textbf{absolutely continuous} if for every \( \epsilon > 0  \), there exists \( \delta > 0  \) such that for every collection of pairwise disjoint intervals \( \{ ({c}_{k }, {d}_{k}) : 1 \leq k \leq n  \}  \) from \( [a,b] \) with 
    \[  \sum_{ k=1  }^{ n } ({d}_{k} - {c}_{k}) < \delta \]
    we have 
    \[  \sum_{ k=1  }^{ n } | f({d}_{k}) - f({c}_{k}) |  < \epsilon. \]
\end{definition}

Why should we care about absolute continuity? For one, we have the following fact:

The statements below are equivalent
\begin{enumerate}
    \item[(1)] \( f : [a,b] \to \R  \) is absolutely continuous
    \item[(2)] \( f  \) is diffrentiable everywhere on \( [a,b] \) except at with respect to a set of measure zero and 
        \[  \forall x \in [a,b] \ \ f(x) - f(a) = \int_{ a }^{ x }  f'(t) \ dt. \]
\end{enumerate}

\begin{theorem}[Lipschitz versus absolutely continuous versus uniformly continuous]
   Lipschitz continuous implies absolutely continuous implies uniformly continuous.  
\end{theorem}

\begin{eg}
    \begin{enumerate}
        \item[(1)] Let \( f : [0,1] \to \R  \) be defined by \( f(x) = \sqrt{ x  }  \). We see that \( f  \) is absolutely continuous, but not Lipschitz continuous.
        \item[(2)] The \textbf{cantor function} \( C : [0,1] \to \R  \) is uniformly continuous, but not absolutely continuous.
     \end{enumerate}
\end{eg}

\subsubsection{Continuity and oscillation}\label{Continuity and oscillation}

\begin{definition}[Oscillation of a Function]
    Let \( (X,d) \) and \( (Y,\tilde{d}) \) be metric spaces and \( f: X \to Y  \). For any \( r > 0  \) and any \( x \in X  \), we define the \textbf{oscillation of \( f  \) on the set \( {N}_{r}(x) \)} as 
    \[  \text{osc}(f, x, r) = \sup \{ \tilde{d}(f({x}_{1}), f({x}_{2}) : {x}_{1}, {x}_{2} \in {N}_{r}(x)) \}.  \]

    We define the \textbf{oscillation of \( f  \) at \( x  \)} as 
    \[  \text{osc}(f,x) = \lim_{ r \to  0^{+} }  \text{osc}(f,x,r) = \inf_{r > 0 } \text{osc}(f,x,r). \]
    \[   \]
\end{definition}

\begin{remark}
    In the first definition above, if \( 0 < {r}_{1} < {r}_{2} \), then 
    \[  \text{osc}(f,x,{r}_{1}) \leq \text{osc}(f,x,{r}_{2}). \]
\end{remark}

\begin{eg}
    Let \( f:[-1,1] \to \R  \) be defined by 
    \[  f(x) = 
    \begin{cases}
        1 &\text{if} \ -1 \leq x < 0 \\ 
        3 &\text{if} \ 0 \leq x \leq 1 
    \end{cases}. \]
    Let's study the oscillation at \( x = 0  \). We have 
    \begin{align*}
        \text{osc}(f,0,r) &=  \sup_{{x}_{2},{x}_{1} \in {N}_{r}(0) } | f({x}_{2}) - f({x}_{1}) | \\
                          &= |  3 - 1  |  = 2.
    \end{align*}
\end{eg}

\begin{theorem}[ ]
    Let \( (X,d) \) and \( (Y, \tilde{d}) \) be two metric spaces, \( f: X \to Y  \), and \( x \in X  \). Then we have
    \begin{center}
        \( f  \) is continuous at \( x  \) if and only if \( \text{osc}(f,x) = 0 \).
    \end{center}
\end{theorem}
\begin{proof}
\( (\Longrightarrow) \) It suffices to show that 
\[  \forall \epsilon > 0 \ \ \text{osc}(f,x) \leq \epsilon. \]
Let \( \epsilon > 0  \) be given. Since \( f  \) is continuous at \( x  \), there exists \( \delta > 0  \) such that  
\[  \forall y \in {N}_{\delta}(x) \ \ \tilde{d}(f(x), f(y)) < \frac{ \epsilon }{ 2 }. \]
Thus, for any \( {x}_{1} \) and \( {x}_{2}  \) in \( {N}_{\delta}(x)  \), we have 
\begin{align*}
    \tilde{d}(f({x}_{1}), f({x}_{2})) &\leq \tilde{d(f({x}_{1}), f(x)} ) + \tilde{d}(f(x), f({x}_{2})) \\
                                      &< \frac{ \epsilon }{ 2 }  + \frac{ \epsilon }{ 2 }  = \epsilon.
\end{align*}
Therefore, 
\[  \text{osc}(f,x,\delta) = \sup_{{x}_{1}, {x}_{2} \in {N}_{\delta}(x)} \tilde{d}(f({x}_{1}), f({x}_{2})) \leq \epsilon. \]
Finally, we have 
\[  \text{osc}(f,x) = \inf_{r > 0} \text{osc}(f,x,r) \leq \text{osc}(f,x,\delta) \leq \epsilon \]
as desired.

\( (\Longleftarrow) \) Suppose \( \text{osc}(f,x) = 0  \). Our goal is to show that 
\[  \forall \epsilon > 0 \ \exists \delta > 0 \ \text{such that} \ \forall y \in {N}_{\delta}(x) \ \tilde{d}(f(x) , f(y)) < \epsilon. \tag{*} \]
Let \( \epsilon > 0  \). For the given \( \epsilon  \), we can find a \( \hat{\delta} > 0  \) such that  
\[  \forall 0 < r < \hat{\delta} \ \ \text{osc}(f,x,r) < \epsilon. \]
Thus, 
\[  \sup_{{x}_{1}, {x}_{2} \in {N}_{r}(x)} \tilde{d}(f({x}_{1}, f({x}_{2})))  < \epsilon. \]
Therefore,  
\[  \forall y \in {N}_{r}(x) \ \ \tilde{d} (f(y) , f(x)) < \epsilon \]
Hence, any positive number less than \( \hat{\delta}  \) can be used as the \( \delta  \) that we were looking for.
\end{proof}

\begin{theorem}[ ]
    Let \( (X,d) \) and \( (Y,\tilde{d}) \) be two metric spaces, \( f : X \to Y  \), and \( \gamma > 0  \) is a fixed number. Then the set \( {D}_{\gamma} = \{  x \in X : \text{osc}(f,x) \geq \gamma \}  \).
\end{theorem}
\begin{proof}
    It suffices to show that \( {D}_{\gamma}^{c} = \{  x \in X : \text{osc}(f,x) < \gamma \}  \) is open. Let \( x \in {D}_{\gamma}^{c} \) (we will show that \( x  \) is an interior point). Hence, by definition, \( \text{osc}(f,x) < \gamma \). In what follows, we will prove that \( {N}_{\frac{ \hat{r} }{ 2  } }(x) \subseteq {D}_{\gamma}^{c}  \) (and so \( x  \) is an interior point). Suppose \( z \in {N}_{\frac{ \hat{r} }{ 2 } }(x) \). Then \( {N}_{\frac{ \hat{r} }{ 2 } }(z) \subseteq  {N}_{\hat{r}}(x) \). Thus,    
    \[  \text{osc}(f,z,\frac{ \hat{r} }{ 2 } ) \leq \text{osc}(f,x,\hat{r}) < \gamma. \]
    Therefore, 
    \[  \text{osc}(f,z) = \inf_{r > 0 } \text{osc}(f,z,r) \leq \text{osc}(f,z,\frac{ \hat{r} }{ 2 } ) < \gamma. \]
    hence, \( z \in {D}_{\gamma}^{c} \).
\end{proof}

\begin{remark}
    Suppose \( f: (X,d) \to (Y,\tilde{d}) \). For each \( n \in \N \), let 
    \[  {A}_{n} = \Big\{  x \in X : \text{osc}(f,x) \geq \frac{ 1 }{ n }  \Big\}.  \]
    Let \( D = \{ \text{the set of points at which \( f  \) is continuous} \}  \). Then 
    \[  D = \bigcup_{ n =1  }^{ \infty  }  {A}_{n}. \]
    Indeed, \( x \in D  \) implies \( \text{osc}(f,x) > 0  \) and so there exists \(N  \) such that \( \text{osc}(f,x) > \frac{ 1 }{ N }  \). Thus, \( x \in {A}_{N} \) and so \( x \in \bigcup_{ n=1  }^{ \infty  }  {A}_{n} \). On the other hand, if \( x \in \bigcup_{ n=1  }^{ \infty  }  {A}_{n} \), then there exists an \( N  \) such that \( x \in {A}_{N} \). Thus, \( \text{osc}(f,x) \geq \frac{ 1 }{ N }  \) and so \( \text{osc}(f,x) \neq 0  \). Thus, \( f \in D  \).
\end{remark}

\begin{definition}[Sets of First Category]
    Let \( (X,d) \) be a metric space.  
    \begin{enumerate}
        \item[(*)] A set \( A \subseteq  X   \) is said to be \textbf{nowhere dense} if \( (\overline{A})^{\circ} = \emptyset \). 
        \item[(*)] A set \( A \subseteq X  \) is said to be of \textbf{First Category in \( X  \)} if it can be written as a countable union of nowhere dense sets.
        \item[(*)] A set \( A \subseteq  X   \) is said to be of \textbf{Second Category of \( X  \)} if it is NOT of first category. 
    \end{enumerate}
\end{definition}

\begin{theorem}[ ]
    Let \( (X,d) \) and \( (Y,\tilde{d}) \) be two metric spaces. Suppose \( E  \) is dense in \( X  \). If \( f: X \to Y  \) is a function that is continuous at each point of \( E  \), then the set of points at which \( f  \) is discontinuous is of \textbf{first category} in \( X  \).
\end{theorem}
\begin{proof}
    Let \( D = \{ \text{the set of discontinuities of} f \}  \) and for all \( n \geq 1  \) 
    \[  {A}_{n} = \{  x \in X : \text{osc}(f,x) \geq \frac{1  }{ n } \}. \]
    As we proved, 
    \[  D = \bigcup_{ n=1  }^{ \infty  }  {A}_{n}. \]
    It remains to show that each \( {A}_{n} \) is nowhere dense, that is, we need to show that \( (\overline{{A}_{n}})^{\circ} = \emptyset \). Since each \( {A}_{n} \) is closed (Theorem 2), we need to show \( {A}_{n}^{\circ} = \emptyset \). Assume for contradiction that \( {A}_{n}^{\circ} \neq \emptyset \). Since \( {A}_{n}^{\circ}  \) is a nonempty open set and \( E  \) is dense in \( X  \), we have (by exercise 12, hw4) 
    \begin{align*}
         &E \cap {A}_{n}^{\circ} \neq \emptyset \\
         &\implies E \cap {A}_{n} \neq \emptyset \\
         &\implies \exists  p \  \text{such that} \ p \in E \ \text{and} \ p \in {A}_{n}.
    \end{align*}

\end{proof}

\begin{remark}
    There is no function \( f: \R \to \R  \) such that \( f  \) is continuous at every rational number and discontinuous at every irrational number. Indeed, because \( \Q  \) is dense in \( \R  \), if \( f  \) is continuous on \( \Q  \), then the set of discontinuities of \( f  \) must be of first category. But the set of irrational numbers is of second category. 
\end{remark}

\begin{center}
    \textit{End of Lecture 26} 
\end{center}
