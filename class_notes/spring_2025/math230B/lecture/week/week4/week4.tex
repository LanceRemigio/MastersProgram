\section{Lecture 6}
\subsection{Topics}

\begin{enumerate}
    \item[(1)] The definition of Riemann-Stieltjes integral
    \item[(2)] Refinement of partitions 
\end{enumerate}

\begin{definition}[Almost Disjoint Intervals]
    We say that two intervals \( I  \) and \( J  \) are \textbf{almost disjoint} if either \( I \cap J  \) is empty or \(  I \cap J  \) has exactly one point.
\end{definition}

\begin{definition}[Partition]
    A partition \( P  \) of an interval \( [a,b] \) is a finite set of points in \( [a,b] \) that includes both \( a  \) and \( b  \). We always list the points of a partition \( P = \{  {x}_{0} , {x}_{1}, {x}_{2}, \dots, {x}_{n} \}  \) in an increasing order; so, 
    \[  a = {x}_{0} < {x}_{1}< \cdots < {x}_{n} = b.  \]
\end{definition}

\begin{remark}
    A partition of \( P  \) of an interval \( [a,b] \) is a finite collection of almost disjoint (nonempty) compact intervals whose union is \( [a,b] \): 
    \[  P = {I}_{1}, {I}_{2}, \dots, {I}_{n} \]
    where
    \[  {I}_{1} = [{x}_{0}, {x}_{1}], \ \ {I}_{2} = [{x}_{1}, {x}_{2}], \ \ \cdots \ \ {I}_{n} = [{x}_{n-1}, {x}_{n}]. \]
    Again, we denote \( {x}_{0} = a  \) and \( {x}_{n} = b  \).
\end{remark}

\begin{definition}[Lower Sum, Upper Sum]
    Let \( f: [a,b] \to \R  \) be bounded, \( \alpha : [a,b] \to \R  \) be increasing, and \( P = \{ {x}_{0}, {x}_{2}, \dots, {x}_{n} \}  \) be a partition of \( [a,b] \). Let \( \Delta {\alpha}_{k} = \alpha({x}_{k}) - \alpha({x}_{k-1}) \). 
    \begin{enumerate}
        \item[(i)] The \textbf{Lower Riemann-Stieltjes Sum} of \( f  \) with respect to the integrator \( \alpha \) for the partition \( P  \) is defined by 
            \[  L(f, \alpha, P ) = \sum_{ k=1  }^{ n } {m}_{k} (\alpha({x}_{k}) - \alpha({x}_{k-1})) = \sum_{ k=1  }^{ n } {m}_{k} \Delta {\alpha}_{ k }. \] 
        \item[(ii)] The upper \textbf{Riemann-Stieltjes sum} of \( f  \) with respect to the integrator \( \alpha  \) for the partition \( P  \) is defined by
            \[  U(f, \alpha, P) = \sum_{ k=1  }^{ n } {M}_{k} (\alpha({x}_{k}) - \alpha({x}_{k-1})) = \sum_{ k=1  }^{ n } {M}_{k } \Delta {\alpha}_{k }. \]
    \end{enumerate}
\end{definition}

\begin{definition}[Upper R.S Integral, Lower R.s Integral]
    Let \( f: [a,b] \to \R  \) be bounded, \( \alpha: [a,b] \to \R  \) be increasing. Then 
    \begin{enumerate}
        \item[(i)] The \textbf{Upper R.S integral} of \( f  \) with respect to \( \alpha \) (on \( [a,b] \)) is defined by 
            \[  U(f,\alpha) = \inf_{P \in \Pi} U(f,\alpha, P).   \]
            Note that the set \( \{  U(f,\alpha, P) : P \in \Pi \}   \) is bounded below by \( m (\alpha(b) - \alpha(a)) \). So the infimum above is a real number. 
        \item[(ii)] The \textbf{Lower R.S Integral} of \( f  \) with respect to \( \alpha \) (on \( [a,b] \)) is defined by
            \[  L(f,\alpha) = \sup_{P \in \Pi} L(f,\alpha, P ). \]
            Note that the set \( \{  L(f,\alpha, P ) : P \in \Pi \} \)
            the lower sums is bounded above by \( M(\alpha(b) - \alpha(a)) \). So, the supremum above is a real number.
    \end{enumerate}
\end{definition}

\begin{definition}[Riemann-Stieltjes integrable functions]
    Let \( \alpha : [a,b] \to \R  \) be an increasing function. A function \( f: [a,b] \to \R  \) is said to be \textbf{Riemann-Stieltjes integrable} (on \( [a,b] \)) if 
    \begin{enumerate}
        \item[(i)] \( f \) is bounded
        \item[(ii)] \( L(f,\alpha) = U(f,\alpha) \).
    \end{enumerate}
    In this case, the R.S integral of \( f  \) with respect to \( \alpha \), denoted by
    \[ \int_{ a }^{ b }  f  \ d \alpha \ \ \text{or} \ \ \int_{ a }^{ b }  f(x)  \ d \alpha(x) \ \ \text{or} \ \ \int_{ [a,b] } f   \ d \alpha  \]
    is the common value of \( L(f,\alpha) \) and \( U(f,\alpha) \). That is, 
    \[  \int_{ a }^{ b }  f  \ d \alpha = L (f,\alpha) = U(f,\alpha). \]
\end{definition}

\section{Lecture 8-9-10}

\begin{theorem}[Rudin 6.4]
    Let \( f: [a,b] \to \R  \) be bounded, \( \alpha : [a,b] \to \R  \) is increasing, \( P  \) is a partition of \( [a,b] \), and \( Q  \) is a refinement of \( P  \). Then
    \begin{enumerate}
        \item[(1)] \( L(f,\alpha,P) \leq L(f,\alpha, Q) \)
        \item[(2)] \( U(f,\alpha, P) \geq U(f,\alpha, Q ) \)
    \end{enumerate}
\end{theorem}
\begin{proof}
Here we will prove (1). The proof of (2) is completely analagous. We proceed via induction on \( \ell = \card (Q \setminus  P )  \) (the number of points in \( Q \setminus  P  \)). Let \( P = \{  {x}_{0}, {x}_{1}, \dots, {x}_{n} \}  \). 

If \( \ell = 0  \), then \( P \subseteq  Q   \) and \( \card Q = \card P  \) implies that \( P = Q  \). Thus, \( L(f,\alpha, P) = L(f,\alpha, Q ) \).

If \( \ell = 1  \), then \( Q  \) has exactly one extra point. Let's call this point \( z  \). So, \( \{ z  \}  = Q \setminus  P  \). Note that \( z \in [a,b] \) and \( P  \) is a partition of \( [a,b] \). Hence, there exists \( 1 \leq i \leq n  \) such that \( z \in ({x}_{i-1}, {x}_{i}) \). Let 
\begin{align*}
    {m}_{i}' &= \inf_{x \in [{x}_{i-1}, z]} f(x)  \\
    {m}_{i}'' &= \inf_{x \in [z,{x}_{i}]} f(x)
\end{align*}
Recall that if \( A \subseteq  B  \), then \( \inf A \geq \inf B  \). Hence, \( {m}_{i}' \geq {m}_{i} \) and \( {m}_{i}'' \geq {m}_{i} \).
We have 
\begin{align*}
    L(f,\alpha, P ) &= \sum_{ k=1  }^{ n } {m}_{k } (\alpha({x}_{k})) \\ 
                    &= \Big[ \sum_{ k \neq i  }^{  } {m}_{k } (\alpha({x}_{k}) - \alpha({x}_{k-1})) \Big] + {m}_{i} (\alpha({x}_{i}) - \alpha(z) + \alpha(z) - \alpha({x}_{i-1})) \\
                    &= \Big[ \sum_{ k \neq i  }^{  } {m}_{k } (\alpha({x}_{k}) - \alpha({x}_{k-1}))  \Big] +  {m}_{i}(\alpha(z) - \alpha({x}_{i-1})) + {m}_{i} (\alpha({x}_{i}) - \alpha(z) ) \\
                    &\leq \Big[\sum_{ k \neq i  }^{  } {m}_{k } (\alpha({x}_{k}) - \alpha({x}_{k-1})) \Big] + {m}_{i}' (\alpha(z) - \alpha({x}_{i-1})) + {m}_{i}''(\alpha({x}_{i}) - \alpha(z)) \\
                    &= L(f,\alpha, Q ).
\end{align*} 
So, we have \( L(f,\alpha,P) \leq L(f,\alpha,Q ) \).

Now, suppose the claim is true for \( \ell = r \geq 1  \). Our goal is to show that the claim holds for \( \ell = r + 1  \). Suppose \( \card (Q \setminus  P ) = r + 1  \). Let
\[  Q \setminus  P = \{  {z}_{1}, {z}_{2}, \dots, {z}_{r}, {z}_{r+1} \}.  \]
Let \( \hat{Q} = P \cup \{ {z}_{1}, {z}_{2}, \dots, {z}_{r} \}  \). We have
\[  L(f,\alpha, P) \leq L(f,\alpha, \hat{Q}) \leq L(f,\alpha, Q ) \]
where the first inequality holds due to our induction hypothesis and the second inequality holds because \( Q \setminus  \hat{Q}  \) contains only one point. So, we have 
\[  L(f,\alpha, P )  \leq L(f,\alpha, Q ).\]
\end{proof}

\begin{theorem}[ ]\label{Theorem 2}
    Let \( f:[a,b] \to \R  \) be a bounded function, \( \alpha: [a,b] \to \R  \) is increasing. Let \( {P}_{1} \) and \( {P}_{2} \) are any two partition of \( [a,b] \). Then 
    \[  L(f,\alpha, {P}_{1}) \leq U(f,\alpha, {P}_{2}). \]
\end{theorem}
\begin{proof}
Let \( Q = {P}_{1} \cup {P}_{2} \) be the common refinement of \( {P}_{1} \) and \( {P}_{2} \). Applying the previous theorem, we can see that \( {P}_{1} \subseteq  {P}_{1} \cup {P}_{2} \) and \( {P}_{2} \subseteq {P}_{1} \cup {P}_{2}   \) implies
\[  L(f,\alpha, {P}_{1}) \leq L(f,\alpha, Q ) \leq U(f,\alpha, Q ) \leq U(f,\alpha, {P}_{2}) \]
\end{proof}

For the following theorem, we will use the lemma below.

\begin{lemma}
    Suppose \( A  \) and \( B  \) are nonempty subsets of \( \R  \). If 
    \[  \forall a \in A \ \forall b \in B \ \ a \leq b  \]
    then \( \sup A \leq \inf B  \).
\end{lemma}

\begin{theorem}[Rudin 6.5]
    Let \( f: [a,b] \to \R  \) be a bounded function and \( \alpha: [a,b] \to \R  \) is an increasing function. Then \( L(f,\alpha) \leq U(f,\alpha) \). 
\end{theorem}
\begin{proof}
Let \( A = \{  L(f,\alpha, P) : P \in \Pi \}  \) and \( B = \{  U(f,\alpha, P) : P \in \Pi \}  \). Using the lemma above and {\hyperref[Theorem 2]{Theorem 2}}, we can see that for all \( a \in A  \) and for all \( b \in B  \), it follows that \( \sup A \leq \inf B  \); that is, \( L(f,\alpha) \leq U(f,\alpha) \).
\end{proof}

\begin{theorem}[Cauchy Criterion for Riemann-Stieltjes Integrability Rudin 6.6]\label{Rudin 6.6}
    Let \( f: [a,b] \to \R  \) be a bounded function, \( \alpha : [a,b] \to \R  \) be an increasing function. Then
    \[  f \in {R}_{\alpha}[a,b] \iff \forall \epsilon > 0 \ \exists {P}_{\epsilon} \in \Pi [a,b] \ \text{such that} \ U(f,\alpha, {P}_{\epsilon}) - L(f,\alpha, {P}_{\epsilon}) < \epsilon.  \]
\end{theorem}
\begin{proof}
    (\( \Longleftarrow \)) Our goal is to show that \( L(f,\alpha) = U(f,\alpha) \). Note that \( L(f,\alpha) \leq U(f,\alpha) \) implies \( U(f,\alpha) - L(f,\alpha) \geq 0  \). Hence, it suffices to show that for all \( \epsilon > 0  \), 
    \[  U(f,\alpha) - L(f,\alpha) < \epsilon. \]
    Let \( \epsilon > 0  \) be given. By assumption, there exists \( {P}_{\epsilon} \in \Pi \) such that 
    \[  U(f,\alpha, {P}_{\epsilon}) - L(f,\alpha, {P}_{\epsilon}) < \epsilon. \]
    We have
    \begin{align*}
        U(f,\alpha) &= \inf_{P \in \Pi} U(f,\alpha,P) \leq U(f,\alpha, {P}_{\epsilon}) \\
        L(f,\alpha) &= \sup_{P \in \Pi} L(f,\alpha,P)  \geq L(f,\alpha, {P}_{\epsilon})
    \end{align*}
    Using {\hyperref[Theorem 3]{Rudin 6.5}}, we can see that 
    \[ L(f,\alpha,{P}_{\epsilon}) \leq L(f,\alpha) \leq U(f,\alpha) \leq U(f,\alpha, {P}_{\epsilon}).  \]
    So, the interval \( [L(f,\alpha), U(f,\alpha)] \) is contained in the interval \( [L(f,\alpha, {P}_{\epsilon}), U(f,\alpha, {P}_{\epsilon})] \). Thus, 
    \[ U(f,\alpha) - L(f,\alpha) \leq U(f,\alpha, {P}_{\epsilon}) - L(f,\alpha, {P}_{\epsilon}) < \epsilon \]
    as desired.
    
    (\( \Longrightarrow \)) Our goal is to show that for any \(  \epsilon > 0  \), there exists a partition \( {P}_{\epsilon} \in \Pi \) such that 
    \[  U(f,\alpha,{P}_{\epsilon}) - L(f,\alpha, {P}_{\epsilon}) < \epsilon.  \]
    Note that 
    \begin{align*}
        U(f,\alpha)  &= \inf_{P \in \Pi} U(f,\alpha, P) \implies \exists {P}_{1} \in \Pi \ \text{such that} \ U(f,\alpha, {P}_{1}) < U(f,\alpha) + \frac{ \epsilon }{ 2 }  \\
        L(f,\alpha) &= \sup_{P \in \Pi} L(f,\alpha,P) \implies \exists {P}_{2} \in \Pi \ \text{such that} \ L(f,\alpha) - \frac{ \epsilon }{ 2 }  < L(f,\alpha, {P}_{2})
    \end{align*}
    Let \( {P}_{\epsilon} = {P}_{1} \cup {P}_{2} \) (we claim that this partition can be used as the one that we were looking for).
    \[  L(f,\alpha) - \frac{ \epsilon }{ 2 }  < L(f,\alpha, {P}_{2}) \leq L(f,\alpha, {P}_{\epsilon}) \leq U(f,\alpha,{P}_{\epsilon}) \leq U(f,\alpha, {P}_{1}) < U(f,\alpha) + \frac{ \epsilon }{ 2 }. \]
    Thus, we have
    \begin{align*}
        U(f,\alpha, {P}_{\epsilon}) - L(f,\alpha, {P}_{\epsilon}) &< \Big[ \Big( U(f,\alpha) + \frac{ \epsilon }{ 2 }  \Big) - \Big(  L(f,\alpha) - \frac{ \epsilon }{ 2 }  \Big) \Big] \\
                                                                  &= U(f,\alpha) - L(f,\alpha) + \epsilon \\
                                                                  &= 0 + \epsilon = \epsilon
    \end{align*}
    as desired.
\end{proof}

\begin{theorem}[Rudin 6.7]\label{Rudin 6.7}
    Let \( f:[a,b] \to \R  \) be a bounded function, \( \alpha:[a,b] \to \R  \) is an increasing function, fix \( \epsilon > 0  \), \( P = \{ {x}_{0}, {x}_{1}, \dots, {x}_{n} \}  \) is a partition of \( [a,b] \), and 
    \[  U(f,\alpha, P) - L(f,\alpha, P) < \epsilon. \]
    Then
    \begin{enumerate}
        \item[(1)] If \( Q  \) is any refinement of \( P  \), then \( U(f,\alpha, Q ) - L(f,\alpha, Q ) < \epsilon \).
        \item[(2)] If for every \( 1 \leq k \leq n  \), \( {t}_{k } \) and \( {s}_{k } \) are arbitrary points in \( [{x}_{k-1}, {x}_{k}] \), then
            \[  \sum_{ k=1  }^{ n } | f({s}_{k}) - f({t}_{k}) | \Delta {\alpha}_{k } < \epsilon. \]
        \item[(3)] If \( f \in {R}_{\alpha}[a,b] \) and for each \( 1 \leq k \leq n  \), \( {s}_{k} \) is a point in \( [{x}_{k-1}, {x}_{k}] \), then
            \[  \Big| \sum_{ k=1  }^{ n } f({s}_{k}) \Delta {\alpha}_{k } - \int_{ a }^{ b } f \ d \alpha \Big|  < \epsilon. \]
    \end{enumerate}
\end{theorem} 
\begin{proof}
\begin{enumerate}
    \item[(1)] We have 
        \[  L(f,\alpha, P) \leq L(f,\alpha, Q ) \leq U(f,\alpha, Q) \leq U(f,\alpha, P). \]
        Therefore, 
        \[  U(f,\alpha, Q ) - L(f,\alpha,Q) \leq U(f,\alpha, P) - U(f,\alpha, P) < \epsilon. \]
    \item[(2)] For each \( 1 \leq k \leq n  \), we have 
        \begin{align*}
            {m}_{k } &\leq f({s}_{k}) \leq {M}_{k} \\
            {m}_{k}&\leq f({t}_{k}) \leq {M}_{k} \implies - {M}_{k } \leq - f({t}_{k}) \leq - {m}_{k}.
        \end{align*}
        So, we have 
        \[  {m}_{k } - {M}_{k } \leq f({s}_{k}) - f({t}_{k}) \leq {M}_{k } - {m}_{k}. \]
        That is, 
        \[  - ({M}_{k } - {m}_{k}) \leq f({s}_{k}) - f({t}_{k}) \leq {M}_{k } - {m}_{k}. \]
        Therefore, 
        \[  | f({s}_{k}) - f({t}_{k})  | \leq {M}_{k } - {m}_{k}. \]
        Hence, we have 
        \[ \sum_{ k=1  }^{ n } | f({s}_{k}) - f({t}_{k}) |  \Delta {\alpha}_{k } \leq \sum_{ k=1  }^{ n } ({M}_{k } - {m}_{k}) \Delta {\alpha}_{k } = U(f,\alpha, P) - L(f,\alpha,P) < \epsilon. \]
    \item[(3)] For all \( 1 \leq k \leq n  \), we have 
        \[  {m}_{k } \leq f({s}_{k}) \leq {M}_{k }. \]
        So, 
        \[  \sum_{ k=1  }^{ n } {m}_{k } \Delta {\alpha}_{k } \leq \sum_{ k=1  }^{ n } f({s}_{k}) \Delta {\alpha}_{k } \leq \sum_{ k=1  }^{ n } {M}_{k } \Delta {\alpha}_{k }. \]
        Therefore, 
        \[  L(f,\alpha, P) \leq \sum_{ k=1  }^{ n } f({s}_{k}) \leq U(f,\alpha,P) \tag{I} \]
        Also, note that 
        \[  L(f,\alpha,P) \leq \int_{ a }^{ b }  f  \ d \alpha \leq U(f,\alpha, P). \tag{II} \]
        Hence,
        \[  \Big| \sum_{ k=1  }^{ n } f({s}_{k}) \Delta {\alpha}_{k } - \int_{ a }^{ b }  f \ d \alpha \Big|  \leq U(f,\alpha, P) - L(f,\alpha, P) < \epsilon \]
        as desired.
\end{enumerate}
\end{proof}

\begin{lemma}\label{lemma 1}
    Let \( f:[a,b] \to \R  \) be a bounded function. Let \( P = \{  {x}_{0}, {x}_{1}, \dots, {x}_{n} \}  \) be a partition of \( [a,b] \). Then 
    \[  \forall 1 \leq k \leq n \ \ \sup_{s,t \in [{x}_{k-1}, {x}_{k}]} | f(s) -f(t) | = {M}_{k } -{m}_{k }. \]
\end{lemma}
\begin{proof}
Let \( k \in \{  1, 2 , \dots, n \}  \). We need to show
\begin{enumerate}
    \item[(1)] \( \forall s,t \in [{x}_{k-1}, {x}_{k}] \) \( | f(s) - f(t) |  \leq {M}_{k } - {m}_{k} \).
    \item[(2)] \( \forall \epsilon >0  \), \( \exists \hat{s} \hat{t} \in [{x}_{k-1}, {x}_{k}]  \) such that \( {M}_{k } - {m}_{k } - \epsilon < | f(\hat{s}) - f(\hat{t}) |  \). 
\end{enumerate}
Note that we have already shown (1) in our discussion of {\hyperref[Rudin 6.7]{Theorem 6.7}}. 

Let \( \epsilon > 0  \) be given. Then we have
\begin{align*}
    {m}_{k } = \inf_{t \in [{x}_{k-1}, {x}_{k}]} f(t) &\implies \hat{t} \in [{x}_{k-1}, {x}_{k}] \ \text{such that} \ f(\hat{t}) < {m}_{k } + \frac{ \epsilon }{ 2 }  \\
    {M}_{k }  = \sup_{t \in [{x}_{k-1}, {x}_{k}]} f(t) &\implies \hat{s} \in [{x}_{k-1}, {x}_{k}] \ \text{such that} \ {M}_{k }  - \frac{ \epsilon }{ 2 }  < f(\hat{s}).
\end{align*}
Adding the inequalities above, we get
\[  {M}_{k } - {m}_{k } - \epsilon < f(\hat{s}) - f(\hat{t}) \leq | f(\hat{s}) - f(\hat{t}) |. \]
\end{proof}

\begin{theorem}[Rudin 6.8]\label{Rudin 6.8}
    Let \( f:[a,b] \to \R  \) be a continuous function and \( \alpha: [a,b] \to \R  \) is an increasing function. Then \( f \in {R}_{\alpha}[a,b] \). 
\end{theorem}

\begin{proof} 
    Since \( f: [a,b] \to \R  \) is a continuous function and \( [a,b]  \) is compact, it follows from the Extreme Value Theorem that \( f  \) is bounded on \( [a,b] \). Now, according to the Cauchy Criterion for Riemann-Stieltjes integrability, it suffices to show that 
    \[  \forall \epsilon > 0 \ \exists P \in \Pi  \ \text{such that} \ U(f,\alpha,P) - L(f,\alpha,P) < \epsilon. \tag{*}\]
    Let \( \epsilon > 0  \) be given. By the same reasoning that showed \( f  \) is bounded on \( [a,b] \), it follows that \( f  \) is uniformly continuous on \( [a,b] \). For the given \( \epsilon  \), there exists a \( \delta > 0  \) such that for all \( s,t \in [a,b] \):
    \[  \text{if} \ | s - t  |  < \delta \ \text{then} \ | f(s) - f(t) | < \frac{ \epsilon }{  2 [\alpha(b) - \alpha(a) + 1 ] }. \]
    Let \( P = \{  {x}_{0}, {x}_{1}, \dots, {x}_{n} \}  \) be any partition of \( [a,b] \) such that \( \|P\| < \delta \). We claim (*) holds for such a partition. Indeed, for all \( k \in \{  1, 2, \dots, n  \}  \) and for all \( s,t \in [{x}_{k-1}, {x}_{k}] \), if \( | s-  t  |  < \delta \), then
    \[  | f(s) - f(t) |  < \frac{ \epsilon }{ 2 [\alpha(b) - \alpha(a) + 1 ] }. \]
    Hence, 
    \[  \sup_{s,t \in [{x}_{k-1}, {x}_{k}]} | f(s) - f(t) |  \leq \frac{ \epsilon }{ 2 [\alpha(b) - \alpha(a) + 1] }. \]
    Thus, 
    \[  {M}_{k } - {m}_{k } \leq \frac{ \epsilon }{ 2 [\alpha(b) - \alpha(a) + 1] }. \]
    Therefore, 
    \begin{align*}
        U(f,\alpha,P) - L(f,\alpha,P) &= \sum_{ k=1  }^{ n } ({M}_{k } - {m}_{k}) \Delta {\alpha}_{k } \\
                                      &\leq \sum_{ k=1  }^{ n } \frac{ \epsilon }{  2 [\alpha(b)- \alpha(a) + 1] }  \Delta {\alpha}_{k } \\
                                      &= \frac{ \epsilon }{ 2 [\alpha(b) - \alpha(a) + 1] }  \sum_{ k=1  }^{ n } \Delta {\alpha}_{k } \\
                                      &= \frac{ \epsilon }{ 2 [\alpha(b) - \alpha(a) + 1] }  \cdot [\alpha(b) - \alpha(a)] \\
                                      &\leq \frac{ \epsilon }{ 2 }  \\
                                      &< \epsilon
    \end{align*}
    as desired.


\end{proof}

\begin{lemma}\label{lemma 2}
    Let \( \alpha:[a,b] \to \R  \) be an increasing and continuous function and \( \alpha(a) < \alpha(b) \). Then for each \( n \in \N \), there exists a partition \( P = \{  {x}_{0}, {x}_{1}, \dots, {x}_{n} \}  \) such that   
    \[  \forall 1 \leq k \leq n \ \ \Delta \alpha_k = \alpha({x}_{k}) - \alpha({x}_{k-1}) = \frac{ \alpha(b) - \alpha(a) }{  n  }. \]
\end{lemma}
\begin{proof}
    Let \( n \in \N \). Divide the interval \( [\alpha(a), \alpha(b)] \) into \( n  \) subintervals of equal length: \( \frac{ \alpha(b) - \alpha(a) }{ n  }  \). For each \( 1 \leq k \leq n  \), we have \( {y}_{k } \in (\alpha(a), \alpha(b)) \). Hence, the Intermediate Value Theorem implies that 
    \[  \exists {x}_{k } \in (a,b) \ \text{such that} \ {y}_{k } = \alpha({x}_{k}). \]
    Since \( \alpha  \) is increasing, we have 
    \[  a = {x}_{0} < {x}_{1} < {x}_{2} < \cdots < {x}_{n} = b.   \]
    This tells us that \( P = \{  {x}_{0},{x}_{1}, \dots, {x}_{n} \}  \) will be partition of \( [a,b] \) such that
    \[  \forall 1 \leq k \leq n  \ \ \Delta {\alpha}_{k } = \alpha({x}_{k}) - \alpha({x}_{k-1}) = {y}_{k } - {y}_{k-1} = \frac{ \alpha(b) - \alpha(a) }{ n }. \]
\end{proof}

\begin{theorem}[Rudin 6.9]\label{Rudin 6.9}
    Let \( \alpha: [a,b] \to \R  \) be increasing and continuous. Then 
    \begin{enumerate}
        \item[(1)] If \( f:[a,b] \to \R  \) is increasing, then \( f \in {R}_{\alpha}[a,b] \).
        \item[(2)] If \( f: [a,b] \to \R  \) is increasing
    \end{enumerate}
\end{theorem}
\begin{proof}
Here we will prove (1). The proof of (2) is analogous. First, note that
\[  \forall x \in [a,b] \ \ f(a) \leq f(x) \leq f(b)  \implies f \ \text{is bounded on} \ [a,b].\]
If \( \alpha(a) = \alpha(b) \), then we previously proved \( f \in {R}_{\alpha}[a,b] \) and \( \int_{ a }^{ b }  f  \ d \alpha = 0  \). So, it remains to prove the claim for the case where \( \alpha(a) \neq \alpha(b) \). According to the Cauchy Criterion for integrability, in order to show that \( f \in {R}_{\alpha}[a,b]  \), it suffices to show that
\[  \forall \epsilon > 0 \ \exists P \in \Pi \ \text{such that} \ U(f,\alpha, P) - L(f,\alpha, P) < \epsilon.  \]
Let \( \epsilon > 0  \) be given. Choose \( n \in \N \) be large enough so that \( \frac{ \alpha(b) - \alpha(a)  }{ n  }  [f(b) - f(a)] < \epsilon \). Let \( \tilde{P} = \{  x_{0}, {x}_{1}, \dots, {x}_{n} \}  \) be a partition of \( [a,b] \) such that 
\[  \forall 1 \leq k \leq n \ \ \Delta {\alpha}_{k} = \alpha({x}_{k}) - \alpha({x}_{k-1}) = \frac{ \alpha(b) - \alpha(a) }{  n  }. \]
We claim that \( \tilde{P} \) can be used as the \( P  \) that we were looking for. Now, since \( f  \) is increasing, we know that for each \( 1 \leq  k \leq n  \)
\[  {M}_{k} = \sup_{x \in [{x}_{k-1}, {x}_{k}]} f(x) = f({x}_{k}) \]
and
\[  {m}_{k } = \inf_{x \in [{x}_{k-1}, {x}_{k}]} f(x) = f({x}_{k-1}). \]
Hence, we see that 
\begin{align*}
    U(f,\alpha, \tilde{P}) - L(f,\alpha, \tilde{P})  &= \sum_{ k=1  }^{ n } ({M}_{k } - {m}_{k}) \Delta {\alpha}_{k }  \\
                                                     &= \frac{ \alpha(b) - \alpha(a) }{ n }  \sum_{ k=1  }^{ n } [f({x}_{k}) - f({x}_{k-1})] \\
                                                     &= \frac{ \alpha(b) - \alpha(a) }{  n }  [f(b) - f(a)] < \epsilon
\end{align*}
as desired.
\end{proof}

\begin{theorem}[Rudin 6.10]\label{Rudin 6.10}
    Let \( f: [a,b] \to \R  \) be a bounded function. Suppose that \( f  \) has only finitely many points of discontinuity  
    \[  {y}_{1} < {y}_{2} < \cdots < {y}_{m}  \]
    and \( \alpha:[a,b] \to \R \) is increasing and \( \alpha \) is continuous at \( {y}_{1}, {y}_{2}, \dots, {y}_{m} \). Then \( f \in {R}_{\alpha}[a,b] \).
\end{theorem}
\begin{proof}
According to the Cauchy Criterion, it suffices to show that
\[  \forall \epsilon > 0 \ \exists P \in \Pi \ \text{such that} \ U(f,\alpha,P) - L(f,\alpha,P) < \epsilon. \]
Let \( \epsilon > 0 \) be given. Let \( \tilde{M} = \sup_{x \in [a,b]} | f(x) |  \). Let 
\[  \hat{\epsilon} = \frac{ \epsilon }{ [\alpha(b) - \alpha(a) + 2 \tilde{M} + 1] }.  \]
We will make the following two claims:
\begin{enumerate}
    \item[(1)] There exists many disjoint intervals \( [{u}_{1}, {v}_{1}] , \dots, [{u}_{m}, {v}_{m}] \) such that 
        \begin{enumerate}
            \item[(I)] \( \forall 1 \leq j \leq m  \) \( {y}_{j} \in [{u}_{j}, {v}_{j}] \).
            \item[(II)] \( \forall 1 \leq j \leq m  \) if \( {y}_{j} \notin \{ a,b \}  \), then \( {y}_{j} \in ({u}_{j}, {v}_{j}) \)
            \item[(III)] \( \forall 1 \leq j \leq m  \) \( \alpha({v}_{j}) - \alpha({u}_{j}) < \frac{ \hat{\epsilon} }{ m }  \) and so
                \[  \sum_{ j=1  }^{ m } \alpha({v}_{j}) - \alpha({u}_{j}) < \hat{\epsilon}. \]
        \end{enumerate}
    \item[(2)] Let \( K = [a,b] \setminus  \bigcup_{ j=1  }^{ m  }  ({u}_{j}, {v}_{j}) \). Then \( f  \) is uniformly continuous on \( K  \).
\end{enumerate}
The two claims above will be proven as lemmas after the proof of this theorem. For now, we will assume that the two claims hold.

By claim 2, we know there exists \( \delta > 0  \) such that for all \( s,t \in K  \) if \( | s - t  |  < \delta  \), then
\[  | f(s) - f(t) |  < \hat{\epsilon}. \]
Now, we form a partition \( \tilde{P}  \) of \( [a,b] \) as follows:
\begin{enumerate}
    \item[(i)] \( \forall 1 \leq j \leq m  \) \( {u}_{j}, {v}_{j} \in \tilde{P} \).
    \item[(ii)] \( \forall 1 \leq j \leq m  \) no point of the segment \( ({u}_{j}, {v}_{j}) \) is in \( \tilde{P} \)
    \item[(iii)] If \( 1 \leq k \leq m  \) is such that \( {x}_{k-1} \notin \{  {u}_{1}, \dots, {u}_{m} \}  \), then we will choose \( {x}_{k} \) such that \( {x}_{k } - {x}_{k-1} < \delta \). 
\end{enumerate}
We claim that this \( \tilde{P} \) can be used as the \( P  \) that we were looking for. Indeed, define the two sets
        \[ A = \{ k : {x}_{k-1} \notin \{ {u}_{1}, \dots, {u}_{m} \}  \} \ \ \text{and} \ \  B = \{ 1, \dots, n \}  \setminus  A.   \]
        For the case that \( k \in A  \), \( {x}_{k} - {x}_{k-1} < \delta \), so
        for all \( s,t \in [{x}_{k-1}, {x}_{k}] \), if \( | s- t  |  < \delta \), then \( | f(s) - f(t)  | < \hat{\epsilon} \). Then taking the supremum, we have 
        \[  \sup_{s,t \in [{x}_{k-1}, {x}_{k}]} | f(s) - f(t) |  \leq \hat{\epsilon} \]
        and so from {\hyperref[lemma 2]{lemma 2}}, we have  
        \[  {M}_{k } - {m}_{k } \leq \hat{\epsilon}. \]

    If \( k \in B \), then
    \[  {M}_{k } - {m}_{k } = \sup_{s,t \in [{x}_{k-1}, {x}_{k}]} | f(s) - f(t) |  \leq 2 \tilde{M}. \]
    Therefore, 
    \begin{align*}
        U(f,\alpha, P) - L(f,\alpha, P) &= \sum_{ k=1  }^{ n } ({M}_{k } - {m}_{k } ) \Delta {\alpha}_{k} \\
                                        &= \sum_{k \in A} ({M}_{k } - {m}_{k}) \Delta {\alpha}_{k } + \sum_{k \in B} ({M}_{k } - {m}_{k}) \Delta {\alpha}_{k } \\
                                        &\leq \sum_{k \in A} \hat{\epsilon} \Delta {\alpha}_{k } + 2 \tilde{M} \sum_{k \in B} \Delta {\alpha}_{k } \\ 
                                        &\leq \hat{\epsilon} [\alpha(b) - \alpha(a)] + 2 \tilde{M} \hat{\epsilon} \\
                                        &= [\alpha(b) - \alpha(a) + 2 \tilde{M}] \hat{\epsilon} \\
                                        &< \epsilon.
    \end{align*}
\end{proof}

\begin{lemma}\label{Claim 1}
   There exists finitely many disjoint intervals 
   \[  [{u}_{1}, {v}_{1}], \dots, [{u}_{m}, {v}_{m}] \]
   in \( [a,b] \) such that
   \begin{enumerate}
       \item[(1)] \( \forall 1 \leq j \leq m  \) \( {y}_{j} \in [{u}_{j}, {v}_{j}] \);
        \item[(2)] \( \forall 1 \leq j \leq m  \) if \( {y}_{j} \notin \{ a,b \}  \) then \( {y}_{j} \in ({u}_{j}, {v}_{j}) \);
        \item[(3)] \( \forall  1 \leq j \leq m  \) \( \alpha({v}_{j}) - \alpha({u}_{j}) < \frac{ \hat{\epsilon} }{ m }  \) and so 
            \[  \sum_{ j=1  }^{ m } [\alpha({v}_{j}) - \alpha({u}_{j})] < \hat{\epsilon}. \]
   \end{enumerate}
\end{lemma}
\begin{proof}
Since for each \( 1 \leq j \leq m  \), \( \alpha \) is continuous at \( {y}_{j} \), we can choose \( {\delta}_{j} > 0  \) such that  
\[  \text{if} \ | y  - {y}_{j} |  < {\delta}_{j} , \ \text{then} \ | \alpha(y) - \alpha({y}_{j}) | < \frac{ \hat{\epsilon} }{ 2 m }. \]
Now, let 
\[  \tilde{\delta} = \frac{ 1 }{ 4 }  \min \{ {\delta}_{1}, {\delta}_{2}, \dots, {\delta}_{m}, {y}_{2} - {y}_{1}, {y}_{3} - {y}_{2}, \dots, {y}_{m} - {y}_{m-1} \}. \]
For each \( 1 \leq j \leq m  \), we define
\begin{enumerate}
    \item[(1)] If \( {y}_{j} \notin \{ a,b \}  \), then \( [{u}_{j}, {v}_{j}] = [{y}_{j} - \hat{\delta}, {y}_{j} + \hat{\delta}] \)
    \item[(2)] If \( {y}_{j} = a \), then \( [{u}_{j}, {v}_{j}] = [a, a + \hat{\delta}] \)
    \item[(3)] If \( {y}_{j} = b  \), then \( [{u}_{j}, {v}_{j}] = [b - \hat{\delta}, b ] \).
\end{enumerate}
Clearly, there intervals satisfy all the requirements, in particular, 
\begin{align*}
    \alpha({v}_{j}) - \alpha({u}_{j}) &= | \alpha({v}_{j}) - \alpha({u}_{j}) |  \\
                                      &\leq | \alpha({v}_{j}) - \alpha({y}_{j}) | + | \alpha({y}_{j}) - \alpha({u}_{j}) |  \\
                                      &< \frac{ \hat{\epsilon} }{ 2m }  + \frac{ \hat{\epsilon} }{ 2m }  \\
                                      &= \frac{ \hat{\epsilon} }{ m }
\end{align*}
where \( | {v}_{j} - {y}_{j} | \leq \hat{\delta} < {\delta}_{j} \) and \( | {u}_{j} - {y}_{j} |  \leq \hat{\delta} < {\delta}_{j} \).
\end{proof}

\begin{lemma}[Claim 2]
    Let \( K = [a,b] \setminus  \bigcup_{ j=1 }^{ m } ({u}_{j}, {v}_{j})  \). Then \( f \) is uniformly continuous on \( K  \).
\end{lemma}
\begin{proof}
Note that \( \bigcup_{ k=1  }^{ m } ({u}_{j}, {v}_{j})   \) is open. Hence, 
\[  K = [a,b] \setminus  \bigcup_{ j=1  }^{ m }  ({u}_{j}, {v}_{j}) = [a,b] \cap \Big[ \bigcup_{ j=1  }^{ m }  ({u}_{j}, {v}_{j}) \Big]^{c} \]
is closed. Since \( K \subseteq  [a,b] \), \( K  \) is closed, and \( [a,b]  \) is compact, it follows from the fact that closed subsets of a compact set are compact that \( K  \) is compact. Since \( f: K \to \R  \) is continuous and \( K  \) is compact, we can conclude that \( f  \) {\hyperref[is uniformly continuous]{is uniformly continuous}} on \( K  \).
\end{proof}

\begin{remark}[Why is \( f: K \to \R  \) is continuous?]\label{is uniformly continuous}
   We will consider four claims:
   \begin{enumerate}
       \item[(1)] Suppose \( f  \) is continuous at \( a \) and \( b  \). In this case by removing \( \bigcup_{ j=1  }^{ m }  ({u}_{j}, {v}_{j})  \), the discontinuities of \( f  \) will be removed.
        \item[(2)] Since \( f  \) is discontinuous at \( a \), but continuous at \( b  \). In this case, by removing \( \bigcup_{ j=1 }^{ m } ({u}_{j},{v}_{j}) \) all discontinuities will be removed except \( a \). In this case, removing \( ({u}_{1}, {v}_{1}) \) makes \( a \) an isolated point of \( K  \). Every function is continuous at every is isolated point of its domain.
        \item[(3)] Suppose \( f \) is continuous at \( a \) and discontinuities at \( b \).
        \item[(4)] Suppose \( f \) is both discontinuous at \( a \) and \( b \).
   \end{enumerate}
   Case (3) and (4) follows similarly from case (2).
\end{remark}

\begin{theorem}[Rudin 6.11]\label{Rudin 6.11}
    Let \( f \in {R}_{\alpha}[a,b] \), for all \( x \in [a,b] \) \( m \leq f(x) \leq M  \), \( \varphi: [m,M] \to \R  \) is continuous. Then \( h: \varphi \circ f : [a,b] \to \R  \), then \( h \in {R}_{\alpha}[a,b] \).
\end{theorem}
\begin{proof}
    Firs note that a composition of bounded functions is bounded. So \( h: \varphi \circ f  \) is a bounded function on \( [a,b] \). According to the Cauchy criterion, in order to show \( h \in {R}_{\alpha}[a,b] \), it suffices to show that for all \( \epsilon > 0  \), there exists \( P \in \Pi \) such that  
    \[  U(f,\alpha, P) - L(h,\alpha,P) < \epsilon. \]
    Let \( \epsilon > 0  \) be given. Let \( \tilde{M} = \sup_{x \in [a,b]} | h(x) |  \). Let 
    \[  \hat{\epsilon} = \frac{ \epsilon }{ [\alpha(b) - \alpha(a) + 2 \tilde{M} + 1] }. \] 
    We have 
    \begin{enumerate}
        \item[(I)] Since \( \varphi  \) is continuous in \( [m,M] \) and \( [m,M] \) is compact, it follows that \( \varphi  \) is uniformly continuous on \( [,m,M] \). So,
            \[ \exists 0 < \delta < \hat{\epsilon} \ \text{such that} \ \forall s,t \in [m,M] \ \text{if} | s -t  |  < \delta \ \text{then} \ | \varphi(s) - \varphi(t) | < \hat{\epsilon}. \]
        \item[(II)] Since \( f \in {R}_{\alpha}[a,b] \), we know from the CauchyCriterion that 
            \[ \exists \tilde{P} \in \Pi \ \text{such that} \ U(f,\alpha,\tilde{P}) - L(f,\alpha, \tilde{P}) = \sum_{ k=1  }^{ n } ({M}_{k } - {m}_{k}) \Delta {\alpha}_{k } < \delta^{2}.  \]
    \end{enumerate}
    We claim that this \( \tilde{P} \) can be used as the \( P  \) that we were looking for. Indeed, let for all \( 1 \leq k \leq n  \)
    \[ {m}_{k }^{*} = \inf_{x \in [{x}_{k-1}, {x}_{k}]} h(x) \ \ \text{and} \ \ {M}_{k }^{*} = \sup_{x \in [{x}_{k-1}, {x}_{k}]} h(x). \]
    Note that
    \[  U(h,\alpha,\tilde{P}) - L(h,\alpha, \tilde{P}) = \sum_{ k=1  }^{ n } ({M}_{k }^{*} - {m}_{k }^{*}) \Delta {\alpha}_{k }.  \]
    In what follows, we will show that the sum above is less than \( \epsilon \). Divide the indices \( 1, \dots, n \) in two classes, namely
    \[  A = \{  k : {M}_{k } - {m}_{k } < \delta  \}  \ \ \text{and} \ \ B = \{  k : {M}_{k } - {m}_{k } \geq \delta \}. \]
    We have 
    \[  U(h,\alpha,\tilde{P}) - L(h,\alpha, \tilde{P}) = \sum_{ k=1  }^{ n } ({M}_{k }^{*} - {m}_{k }^{*}) \Delta {\alpha}_{k } = \sum_{k \in A} ({M}_{k }^{*} - {m}_{k }^{*}) \Delta {\alpha}_{k } + \sum_{k \in B} ({M}_{k}^{*} - {m}_{k }^{*}) \Delta {\alpha}_{k }. \tag{1}  \]
    \begin{enumerate}
        \item[(*)] If \( k \in A \), then for all \( x,y \in [{x}_{k-1}, {x}_{k}] \), we have
            \begin{align*}
                {M}_{k } - {m}_{k } < \delta &\implies \sup_{x,y \in [{x}_{k-1}, {x}_{k}]} | f(x) - f(y) |  \\
                                             &\implies | f(x) - f(y) |  < \delta \\
                                             &\implies | \varphi(f(x)) - \varphi(f(y)) |  < \hat{\epsilon} \\
                                             &\implies | h(x) - h(y) | < \hat{\epsilon} \\
                                             &\implies \sup_{x,y \in [{x}_{k-1}, {x}_{k}]} | h(x) - h(y) |  \leq \hat{\epsilon} \\
                                             &\implies {M}_{k }^{*} - {m}_{k }^{*} \leq \hat{\epsilon}. \tag{2}
            \end{align*}
        \item[(*)] For \( k \in B \), 
            \begin{align*}
                \delta \sum_{k \in B} \Delta {\alpha}_{k } = \sum_{k \in B} \delta \Delta {\alpha}_{k } &\leq \sum_{k \in B} ({M}_{k } - {m}_{k}) \Delta {\alpha}_{k } \\ 
                                                                                                        &\leq \sum_{ k=1  }^{ n }({M}_{k } -{m}_{k }) \Delta {\alpha}_{k } = U(f,\alpha, \tilde{P}) -  L(f,\alpha, \tilde{P}) < \delta^{2}. \tag{3}
            \end{align*}
            It follows from (1), (2), and (3) that 
            \begin{align*}
                \sum_{ k=1  }^{ n } ({M}_{k }^{*} - {m}_{k }^{*}) \Delta {\alpha}_{k } &= \sum_{k \in A} ({M}_{k }^{*} - {m}_{k }^{*}) \Delta {\alpha}_{k } + \sum_{k \in B} ({M}_{k }^{*} -{m}_{k }^{*}) \Delta {\alpha}_{k } \\
                                                                                       &\leq \sum_{k \in A} \hat{\epsilon} \Delta {\alpha}_{k } + \sum_{k \in B} 2 \tilde{M} \Delta {\alpha}_{k } \\
                                                                                       &= \hat{\epsilon} \sum_{ k=1  }^{ n } \Delta {\alpha}_{k } + 2 \tilde{M} \hat{\epsilon} \\
                                                                                       &= \hat{\epsilon} [\alpha(b) - \alpha(a)] + 2 \tilde{M} \hat{\epsilon} \\
                                                                                       &= [\alpha(b) - \alpha(a) + 2 \tilde{M}] \hat{\epsilon} \\
                                                                                       &= [\alpha(b) - \alpha(a) + 2 \tilde{M}] \cdot \frac{ \epsilon }{ \alpha(b) - \alpha(a) + 2 \tilde{M} + 1  }  < \epsilon
            \end{align*}
            as desired.
    \end{enumerate}
\end{proof}

