
\subsection{Basics/Definitions}

\begin{definition}[Length of Open Interval; \( \ell(I) \)]
   The \textbf{length \( \ell(I) \)} of an open interval \( I  \) is defined by 
   \[  \ell(I) = 
   \begin{cases}
       b - a &\text{if} \ I = (a,b) \ \text{for some} \ a,b \in \R \ \text{with} \ a < b \\
       0 &\text{if} \ I = \emptyset \\
       \infty &\text{if} \ I = (-\infty ,a) \ \text{or} \ I = (a,\infty ) \ \text{for some} \ a \in \R \\
       \infty  &\text{if} \ I = (-\infty , \infty )
   \end{cases} \]
\end{definition}

\begin{definition}[Outer Measure; \( |A| \)]
    The \textbf{outer measure} \( | A  |  \) of a set \( A \subseteq  \R    \) is defined by
    \[  | A  |  = \inf \Big\{ \sum_{ k=1  }^{ \infty  } \ell({I}_{k}) : {I}_{1}, {I}_{2}, \dots \ \text{are open intervals such that} \ A \subset \bigcup_{ k = 1  }^{ \infty  }  {I}_{k}   \Big\}.  \]
\end{definition}


\subsection{Good Properties of Outer Measure}

\begin{prop}[Countable Sets Have Outer Measure \( 0 \)]
   Every countable subset of \( \R  \) has outer measure \( 0  \). 
\end{prop}

\begin{prop}[Outer Measure Preserves Order]
   Suppose \( A  \) and \( B  \) are subsets of \( \R  \) with \( A \subset B  \). Then \( | A  |  \leq | B  |  \). 
\end{prop}
\begin{proof}

\end{proof}

\begin{definition}[Translation; \( t + A  \)]
    If \( t \in \R   \) and \( A \subset \R  \), then the \textbf{translation} \( t +  A  \) is defined by 
    \[  t + A = \{  t + a : a \in A  \}. \]
\end{definition}

\begin{prop}[Outer Measure is Translation Invariant]
    Suppose \( t \in \R  \) and \( A \subset \R  \). Then \( | t + A  |  = | A  |  \).
\end{prop}


\begin{prop}[Countable Subaddivity of Outer Measure]
    Suppose \( {A}_{1}, {A}_{2}, \dots  \) is a sequence of subsets of \( \R  \). Then
    \[  \Big| \bigcup_{ k=1  }^{ \infty  }  {A}_{k }  \Big|  \leq \sum_{ k=1  }^{ \infty  } | {A}_{k } |. \]
\end{prop}
\begin{proof}

\end{proof}

\begin{definition}[Open Cover]
    Suppose \( A \subseteq  \R .  \)    
    \begin{itemize}
        \item A collection \( \{  {O}_{\alpha} \}_{\alpha \in \Lambda} \) of open subsets of \( \R  \) is called an \textbf{open cover} of \( A  \) if \( A  \) is contained in the union of all the sets in \( \{ {O}_{\alpha} \}_{\alpha \in \Lambda} \).
        \item An open \( \{ {O}_{\alpha} \}_{\alpha \in \Lambda} \) of \( A  \) is said to have a \textbf{finite subcover} if \( A  \) is contained in the union of some finite list of sets in \( \{ {O}_{\alpha} \}_{\alpha \in \Lambda}\).
    \end{itemize}
\end{definition}

\begin{prop}[Heine-Borel Theorem]
Every open cover of a closed bounded subset of \( \R  \) has a finite subcover.
\end{prop}


\subsection{Outer Measure of Closed Bounded Interval}


\begin{prop}[Outer Measure of a Closed Interval]
    Suppose \( a,b \in \R  \), with \( a < b  \). Then \( | [a,b] |  = b - a  \). 
\end{prop}

\begin{prop}[Nontrivial Intervals are Uncountable]
    Every interval in \( \R  \) that contains at least two distinct elements is uncountable.
\end{prop}


\subsection{Outer Measure is Not Additive}

\begin{prop}[Nonadditivity of Outer Measure]
    There exist disjoint subsets \( A  \) and \( B  \) of \( \R  \) such that
    \[  | A \cup B  |  \neq | A  |  + | B  |. \]
\end{prop}
