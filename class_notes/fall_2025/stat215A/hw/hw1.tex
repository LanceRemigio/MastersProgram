\documentclass[a4paper]{article} 
\usepackage{standalone}
\usepackage{import}
\usepackage[utf8]{inputenc}
\usepackage[T1]{fontenc}
% \usepackage{fourier}
\usepackage{textcomp}
\usepackage{hyperref}
\usepackage[english]{babel}
\usepackage{url}
% \usepackage{hyperref}
% \hypersetup{
%     colorlinks,
%     linkcolor={black},
%     citecolor={black},
%     urlcolor={blue!80!black}
% }
\usepackage{graphicx} \usepackage{float}
\usepackage{booktabs}
\usepackage{enumitem}
% \usepackage{parskip}
% \usepackage{parskip}
\usepackage{emptypage}
\usepackage{subcaption}
\usepackage{multicol}
\usepackage[usenames,dvipsnames]{xcolor}
\usepackage{ocgx}
% \usepackage{cmbright}


\usepackage[margin=1in]{geometry}
\usepackage{amsmath, amsfonts, mathtools, amsthm, amssymb}
\usepackage{thmtools}
\usepackage{mathrsfs}
\usepackage{cancel}
\usepackage{bm}
\newcommand\N{\ensuremath{\mathbb{N}}}
\newcommand\R{\ensuremath{\mathbb{R}}}
\newcommand\Z{\ensuremath{\mathbb{Z}}}
\renewcommand\O{\ensuremath{\emptyset}}
\newcommand\Q{\ensuremath{\mathbb{Q}}}
\newcommand\C{\ensuremath{\mathbb{C}}}
\newcommand\F{\ensuremath{\mathbb{F}}}
% \newcommand\P{\ensuremath{\mathbb{P}}}
\DeclareMathOperator{\sgn}{sgn}
\DeclareMathOperator{\diam}{diam}
\DeclareMathOperator{\LO}{LO}
\DeclareMathOperator{\UP}{UP}
\DeclareMathOperator{\card}{card}
\DeclareMathOperator{\Arg}{Arg}
\DeclareMathOperator{\Dom}{Dom}
\DeclareMathOperator{\Log}{Log}
\DeclareMathOperator{\dist}{dist}
% \DeclareMathOperator{\span}{span}
\usepackage{systeme}
\let\svlim\lim\def\lim{\svlim\limits}
\renewcommand\implies\Longrightarrow
\let\impliedby\Longleftarrow
\let\iff\Longleftrightarrow
\let\epsilon\varepsilon
\usepackage{stmaryrd} % for \lightning
\newcommand\contra{\scalebox{1.1}{$\lightning$}}
% \let\phi\varphi
\renewcommand\qedsymbol{$\blacksquare$}

% correct
\definecolor{correct}{HTML}{009900}
\newcommand\correct[2]{\ensuremath{\:}{\color{red}{#1}}\ensuremath{\to }{\color{correct}{#2}}\ensuremath{\:}}
\newcommand\green[1]{{\color{correct}{#1}}}

% horizontal rule
\newcommand\hr{
    \noindent\rule[0.5ex]{\linewidth}{0.5pt}
}

% hide parts
\newcommand\hide[1]{}

% si unitx
\usepackage{siunitx}
\sisetup{locale = FR}
% \renewcommand\vec[1]{\mathbf{#1}}
\newcommand\mat[1]{\mathbf{#1}}

% tikz
\usepackage{tikz}
\usepackage{tikz-cd}
\usetikzlibrary{intersections, angles, quotes, calc, positioning}
\usetikzlibrary{arrows.meta}
\usepackage{pgfplots}
\pgfplotsset{compat=1.13}

\tikzset{
    force/.style={thick, {Circle[length=2pt]}-stealth, shorten <=-1pt}
}

% theorems
\makeatother
\usepackage{thmtools}
\usepackage[framemethod=TikZ]{mdframed}
\mdfsetup{skipabove=1em,skipbelow=1em}

\theoremstyle{definition}

\declaretheoremstyle[
    headfont=\bfseries\sffamily\color{ForestGreen!70!black}, bodyfont=\normalfont,
    mdframed={
        linewidth=1pt,
        rightline=false, topline=false, bottomline=false,
        linecolor=ForestGreen, backgroundcolor=ForestGreen!5,
    }
]{thmgreenbox}

\declaretheoremstyle[
    headfont=\bfseries\sffamily\color{NavyBlue!70!black}, bodyfont=\normalfont,
    mdframed={
        linewidth=1pt,
        rightline=false, topline=false, bottomline=false,
        linecolor=NavyBlue, backgroundcolor=NavyBlue!5,
    }
]{thmbluebox}

\declaretheoremstyle[
    headfont=\bfseries\sffamily\color{NavyBlue!70!black}, bodyfont=\normalfont,
    mdframed={
        linewidth=1pt,
        rightline=false, topline=false, bottomline=false,
        linecolor=NavyBlue
    }
]{thmblueline}

\declaretheoremstyle[
    headfont=\bfseries\sffamily, bodyfont=\normalfont,
    numbered = no,
    mdframed={
        rightline=true, topline=true, bottomline=true,
    }
]{thmbox}

\declaretheoremstyle[
    headfont=\bfseries\sffamily, bodyfont=\normalfont,
    numbered=no,
    % mdframed={
    %     rightline=true, topline=false, bottomline=true,
    % },
    qed=\qedsymbol
]{thmproofbox}

\declaretheoremstyle[
    headfont=\bfseries\sffamily\color{NavyBlue!70!black}, bodyfont=\normalfont,
    numbered=no,
    mdframed={
        rightline=false, topline=false, bottomline=false,
        linecolor=NavyBlue, backgroundcolor=NavyBlue!1,
    },
]{thmexplanationbox}

\declaretheorem[
    style=thmbox, 
    % numberwithin = section,
    numbered = no,
    name=Definition
    ]{definition}

\declaretheorem[
    style=thmbox, 
    name=Example,
    ]{eg}

\declaretheorem[
    style=thmbox, 
    % numberwithin = section,
    name=Proposition]{prop}

\declaretheorem[
    style = thmbox,
    numbered=yes,
    name =Problem
    ]{problem}

\declaretheorem[style=thmbox, name=Theorem]{theorem}
\declaretheorem[style=thmbox, name=Lemma]{lemma}
\declaretheorem[style=thmbox, name=Corollary]{corollary}

\declaretheorem[style=thmproofbox, name=Proof]{replacementproof}

\declaretheorem[style=thmproofbox, 
                name = Solution
                ]{replacementsolution}

\renewenvironment{proof}[1][\proofname]{\vspace{-1pt}\begin{replacementproof}}{\end{replacementproof}}

\newenvironment{solution}
    {
        \vspace{-1pt}\begin{replacementsolution}
    }
    { 
            \end{replacementsolution}
    }

\declaretheorem[style=thmexplanationbox, name=Proof]{tmpexplanation}
\newenvironment{explanation}[1][]{\vspace{-10pt}\begin{tmpexplanation}}{\end{tmpexplanation}}

\declaretheorem[style=thmbox, numbered=no, name=Remark]{remark}
\declaretheorem[style=thmbox, numbered=no, name=Note]{note}

\newtheorem*{uovt}{UOVT}
\newtheorem*{notation}{Notation}
\newtheorem*{previouslyseen}{As previously seen}
% \newtheorem*{problem}{Problem}
\newtheorem*{observe}{Observe}
\newtheorem*{property}{Property}
\newtheorem*{intuition}{Intuition}

\usepackage{etoolbox}
\AtEndEnvironment{vb}{\null\hfill$\diamond$}%
\AtEndEnvironment{intermezzo}{\null\hfill$\diamond$}%
% \AtEndEnvironment{opmerking}{\null\hfill$\diamond$}%

% http://tex.stackexchange.com/questions/22119/how-can-i-change-the-spacing-before-theorems-with-amsthm
\makeatletter
% \def\thm@space@setup{%
%   \thm@preskip=\parskip \thm@postskip=0pt
% }
\newcommand{\oefening}[1]{%
    \def\@oefening{#1}%
    \subsection*{Oefening #1}
}

\newcommand{\suboefening}[1]{%
    \subsubsection*{Oefening \@oefening.#1}
}

\newcommand{\exercise}[1]{%
    \def\@exercise{#1}%
    \subsection*{Exercise #1}
}

\newcommand{\subexercise}[1]{%
    \subsubsection*{Exercise \@exercise.#1}
}


\usepackage{xifthen}

\def\testdateparts#1{\dateparts#1\relax}
\def\dateparts#1 #2 #3 #4 #5\relax{
    \marginpar{\small\textsf{\mbox{#1 #2 #3 #5}}}
}

\def\@lesson{}%
\newcommand{\lesson}[3]{
    \ifthenelse{\isempty{#3}}{%
        \def\@lesson{Lecture #1}%
    }{%
        \def\@lesson{Lecture #1: #3}%
    }%
    \subsection*{\@lesson}
    \testdateparts{#2}
}

% \renewcommand\date[1]{\marginpar{#1}}


% fancy headers
\usepackage{fancyhdr}
\pagestyle{fancy}

\makeatother

% notes
\usepackage{todonotes}
\usepackage{tcolorbox}

\tcbuselibrary{breakable}
\newenvironment{verbetering}{\begin{tcolorbox}[
    arc=0mm,
    colback=white,
    colframe=green!60!black,
    title=Opmerking,
    fonttitle=\sffamily,
    breakable
]}{\end{tcolorbox}}

\newenvironment{noot}[1]{\begin{tcolorbox}[
    arc=0mm,
    colback=white,
    colframe=white!60!black,
    title=#1,
    fonttitle=\sffamily,
    breakable
]}{\end{tcolorbox}}

% figure support
\usepackage{import}
\usepackage{xifthen}
\pdfminorversion=7
\usepackage{pdfpages}
\usepackage{transparent}
\newcommand{\incfig}[1]{%
    \def\svgwidth{\columnwidth}
    \import{./figures/}{#1.pdf_tex}
}

% %http://tex.stackexchange.com/questions/76273/multiple-pdfs-with-page-group-included-in-a-single-page-warning
\pdfsuppresswarningpagegroup=1


\usepackage{fancyhdr}
\pagestyle{fancy}
\title{Stat 215A Homework 1}
\author{Lance Remigio}

\begin{document}
\maketitle   

\lhead{Lance Remigio}
\rhead{\thepage}
\begin{prop}[A.1.1]
   For all sets \( A, B , C \subseteq \Omega \). 
   \begin{enumerate}
        \item[(1)] Union and intersection commutative and distributive:
            \begin{enumerate}
                \item[(i)] \( A \cup B   = B \cup A \) 
                \item[(ii)] \( A \cap B = B \cap A  \)
                \item[(iii)] \( (A \cup B) \cup C  = A \cup (B \cup C) \)
                \item[(iv)] \( (A \cap B) \cap C  = A \cap (A \cap C) \)
                \item[(v)] \( A \cap (B \cup C ) = (A \cap B) \cup (A \cap C) \)
                \item[(vi)] \( A \cup (B \cap C) = (A \cup B) \cap (A \cup C) \)
            \end{enumerate}
        \item[(2)] \( (A^{c})^{c} = A  \), \( \emptyset^{c} = \Omega\), and \( \Omega^{c} = \emptyset \);
        \item[(3)] \( \emptyset \subseteq  A  \);
        \item[(4)] \( A \subseteq  A  \);
        \item[(5)] \( A \subseteq  B   \) and \( B \subseteq A  \) implies \( A \subseteq  C  \);
        \item[(6)] \( A \subseteq  B  \) if and only if \( B^{c} \subseteq  A^{c} \);
        \item[(7)] \( A \cup A = A = A \cap A  \);
        \item[(8)] \( A \cup \Omega = \Omega \) and \( A \cap \Omega = A  \);
        \item[(9)] \( A \cup \emptyset = A  \) and \( A \cap \emptyset = \emptyset \).
   \end{enumerate}
\end{prop}

\subsection*{Part (1)}

\begin{enumerate}
    \item[(i)] Our goal is to show that \( A \cup B = B \cup A  \). It suffices to show the following two containments:
        \[  A \cup B \subseteq  B \cup A \tag{*}  \]
        and
        \[  B \cup A \subseteq  A \cup B. \tag{**} \]
        We will first show (*). Let \( x \in A \cup B  \) be arbitrary. Then either \( x \in A  \), \( x \in B  \), or in both. If \( x \in A  \), then \( x \in B \cup A  \). If \( x \in B  \), then \( x \in B \cup A  \). If \( x  \) is in both \( A  \) and \( B  \), then \( x \in B \cup A  \). Hence, in all three cases, \( x \in B \cup A  \) and so \( A \cup B \subseteq  B \cup A  \), satisfying (*). To show (**), let \( x \in B \cup A  \) be arbitrary. Then either \( x \in B  \), \( x \in A  \), or \( x  \) is in both \( A  \) and \( B  \). If \( x \in B  \), then \( x \in B \cup A  \) by definition. If \( x \in A   \), then \( x \in A \cup  B \). If \( x  \) is in both \( A  \) and \( B  \), then \( x \in A \cup B \). Thus, in all three cases, \( x \in A \cup B \).

    \item[(ii)] Our goal is to show that \( A \cap B = B \cap A  \). It suffices to show the following two containments:
        \[ A \cap B \subseteq  B \cap A  \tag{*}  \]
        and
        \[  B \cap A \subseteq  A \cap B. \tag{**}  \]
        To show (*), let \( x \in A \cap B  \) be arbitrary. Then this holds if and only if \( x \in A  \) and \( x \in B \). That is, \( x \in B  \) and \( x \in A  \). Thus, \( x \in B \cap A   \). Hence, \( A \cap B \subseteq B \cap A  \), proving (*). Let \( x \in B \cap A  \) be arbitrary. Then both \( x \in B  \) and \( x \in A  \). Thus, \( x \in A  \) and \( x \in B  \). Therefore, \( x \in A \cap B  \) and so \(  B \cap A \subseteq  A \cap B  \), proving (**). From (*) and (**), we get \( A \cap B = B \cap A \).
    \item[(iii)] Our goal is to show that \( A \cap (B \cap C) = (A \cap B) \cap C  \). We will show the following two containments:
        \[  A \cup (B \cup C) \subseteq  (A \cup B) \cup C \tag{*} \]
        and
        \[  (A \cup B) \cup C \subseteq A \cup (B \cup C). \tag{**} \]
        To show (*), let \( x \in A \cup (B \cup C) \) be arbitrary. Then either \( x \in A  \), \( x \in B \cup C  \), or in both. If \( x \in A  \), then \( x \in A \cup B \). Hence, \( x \in (A \cup B) \cup C  \), by definition. If \( x \in B \cup C  \), then either \( x \in B  \), \( x \in C  \), or \( x \in B \cup C  \). If \( x \in B  \), then \( x \in A \cup B  \) and thus, \( x \in (A \cup B) \cup C  \). If \( x \in C  \), then \( x \in (A \cup B) \cup C  \) by definition. If \( x  \) is in both, then immediately \( x \in (A \cup B) \cup C  \). Now, if \( x \in A  \) and \( x \in B \cup C  \), then we also have \( x \in (A \cup B) \cup C  \). Thus, we have \( A \cup (B \cup C ) \subseteq (A \cup B) \cup C  \).

        To show (**), let \( x \in (A \cup B) \cup C  \) be arbitrary. Then either \( x \in A \cup B \), \( x \in C  \) or both. If \( x \in A \cup B \), then either \( x \in A    \) or \( x \in B \). If \( x \in A  \), then \( x \in A \cup (B \cup C ) \). If \( x \in B \), then \( x \in B \cup C  \). So, \( x \in A \cup (B \cup C) \). Now, if \( x \in C  \), then \( x \in B \cup C  \). By definition, this tells us that \( x \in A \cup (B \cup C ) \). If \( x  \) is in both, then immediately we have \( x \in A \cup (B \cup C ) \) (since it is in all of them and we only require \( x  \) to be in one of them at least). Thus, we have \( (A \cup B) \cup C \subseteq A \cup (B \cup C ) \).
    
    \item[(iv)] Our goal is to show that \( (A \cap B) \cap C  = A \cap (A \cap C) \). We will show the following two containments:
        \[  (A \cap B) \cap C \subseteq  A \cap (B \cap C) \tag{*} \]
        and
        \[  A \cap (B \cap C) \subseteq  (A \cap B) \cap C.  \tag{**} \]
        To show (*), let \( x \in (A \cap B) \cap C  \) be arbitrary. Then \( x \in A \cap B \) and \( x \in C  \). Thus, \( x \in A  \), \( x \in B  \) and \( x \in C  \). Thus, \( x \in A  \) and \( x \in B \cap C  \). By definition, \( x \in A \cap (B \cap C) \). Thus, \( (A \cap B) \cap C \subseteq A \cap (B \cap C) \), proving (*).

        To show (**), let \( x \in A \cap (B \cap C)  \) be arbitrary. Then \( x \in A  \) and \( x \in B \cap C \). Thus, \( x \in A  \), \( x \in B  \), and \( x \in C  \). Now, \( x \in A \cap B  \) and \( x \in C  \) and so \( x \in (A \cap B) \cap C  \), by definition. Therefore, \( A \cap (B \cap C) \subseteq  (A \cap B) \cap C  \), proving (**). Thus, (*) and (**) implies that \( (A \cap B) \cap C = A \cap (B \cap C) \).
       
    \item[(v)] Our goal is to show that \( A \cap (B \cup C) = (A \cap B) \cup (A \cup C) \). We will show the following two containments:   
        \[  A \cap (B \cup C ) \subseteq  (A \cap B) \cup (A \cap C) \tag{*} \]
        and
        \[  (A \cap B) \cup (A \cap C) \subseteq A \cap (B \cup C). \tag{**} \]
        Starting with (*), let \( x \in A \cap (B \cup C) \) be arbitrary. Then \( x \in A  \) and \( x \in B \cup C \). Since \( x \in B \cup C  \), then either \( x \in B  \) or \( x \in C \). Now, if \( x \in B \), then since \( x \in A  \) as well, we have \( x \in A \cap B \). But now \( x  \) lies in at least one of the sets in the union \( (A \cap B) \cup (A \cap C) \). Hence, \( x \in (A \cap B) \cup (A \cap C) \) and so \( A \cap (B \cup C) \subseteq  (A \cap B) \cup (A \cap C) \). Likewise, if \( x \in C  \), then since \( x \in A  \) as well, we have \( x \in A \cap C  \). By definition of union, \( x \in (A \cap  B) \cup (B \cup C) \). Thus, \( A \cap (B \cup C) \subseteq (A \cap B) \cup (A \cap C)  \), proving (*).

        With (**), let \( x \in (A \cap B) \cup (A \cap C) \). Then either \( x \in A \cap B \) or \( x \in A \cap C  \) or both. If \( x \in A \cap B \), then \( x \in A  \) and \( x \in B  \). Since \( x \in B \), it must lie in \( B \cup C  \) because it is contained in at least one of the sets within that union. Thus, we have \( x \in A  \) and \( x \in B \cup C  \) and so \( x \in A \cap (B \cup C) \). Therefore, \( (A \cap B) \cup (A \cap C) \subseteq A \cap (B \cup C) \). If \( x \in A \cap C \), then \( x \in A  \) and \( x \in C  \). Since \( x \in C  \), it follows that \( x \in B \cup C  \) by the same reasoning as before. So, \( x \in A  \) and \( x \in B \cup C  \). Then \( x \in A \cap (B \cup C) \) and so \( (A \cap B) \cup (A \cap C) \subseteq A \cap (B \cup C) \), proving (**).

        From (*) and (**), we have our desired result. 
    \item[(vi)] Our goal is to show that \( A \cup (B \cap C) = (A \cup B) \cap (A \cup C) \). It suffices to show the following two containments:
        \[  A \cup  (B \cap C) \subseteq (A \cup B) \cap (B \cup C) \tag{*} \]
        and
        \[  (A \cup B) \cap (B \cup C) \subseteq  A \cup (B \cap C). \tag{**} \]
        Starting with (*), let \( x \in A \cup (B \cap C) \) be arbitrary. Then \( x \in A  \) or \( x \in B \cap C \). If \(x \in A  \), then \( x \in A \cup B \) because \( x  \) is contained in at least one of the sets in \( A \cup B  \) (of course, it is \( A  \)). But we also have that \( x \in A \cup C  \) by the same reasoning. Hence, \( x \in A \cup B  \) and \( x \in A \cup C  \). So, \( x \in (A \cup B) \cap (A \cup C) \) and so \( A \cup (B \cap C) \subseteq (A \cup B) \cap (B \cup C)  \), proving (*).

       With (**), let \( x \in (A \cup B) \cap (B \cup C) \) be arbitrary. Then \( x \in A \cup B \) and \( x \in A \cup C  \). Then \( x \in A  \) or \( x \in B \) or \( x  \) is in both and \( x \in A  \) or \( x \in C \) or \( x  \) is in both.   
\end{enumerate}


\subsection*{Part (3)}

\begin{proof}
    Our goal is to show that \( \emptyset \subseteq  A  \). Let \( x \in \emptyset \) be arbitrary. Since \( x \in A  \) is a vacuously true statement (by definition of the emptyset), it follows that \( \emptyset \subseteq  A  \).
\end{proof}


\subsection*{Part (2)}

\begin{proof}
    Our goal is to show that 
    \[ (A^{c})^{c} \subseteq A \tag{1}  \] and 
    \[ (A^{c})^{c} \supseteq A \tag{2}. \]

    Let \( x \in (A^{c})^{c} \) be arbitrary. Since \( A^{c} = \Omega \setminus  A  \), we have  
    \[  (A^{c})^{c} =  \Omega \setminus  A^{c}. \]
    Hence, \( x \in \Omega \), but \( x \notin A^{c} \). However, \( x \notin A^{c} \) implies that \( x \notin \Omega \) or \( x \in A  \). Note that the former yields a contradiction because \( x \in \Omega  \) from an earlier statement. Thus, it must be the case that \( x \in A \). Hence, \( (A^{c})^{c} \subseteq  A   \).

    For the containment in (2), assume for contradiction that \( A \not\subseteq (A^{c})^{c} \). Hence, there exists an \( x \in A  \) such that \( x \notin (A^{c})^{c} \). By definition of complement with respect to \( \Omega \), we have \( (A^{c})^{c} = \Omega \setminus  A^{c} \). Since \( x \notin (A^{c})^{c} \), then either \( x \notin \Omega \) or \( x \in A^{c}  \). If \( x \notin \Omega \), then we have a contradiction because we assumed that \( x \in A  \) earlier. If \( x \in A^{c} = \Omega \setminus  A  \), then we are also saying that \( x \in \Omega \) but \( x \notin A  \) which contradicts our earlier assumption that \( x \in A  \). Hence, we must have that \( A \subseteq  (A^{c})^{c} \).
\end{proof}

\begin{proof}
Our goal is to show that \( \emptyset^{c} = \Omega \). Note that the complement of \( \emptyset  \) with respect to \( \Omega \) is \( \emptyset^{c} = \Omega \setminus  \emptyset = \Omega \). Hence, \( \emptyset^{c} = \Omega \).
\end{proof}

\begin{proof}
    Note that, from part (3), we have \( \emptyset \subseteq \Omega^{c} \). Let \( x \in \Omega^{c} \). Then \( x \in \Omega \setminus \Omega \). Hence, \( x \in \Omega \), but \( x \notin \Omega  \). This tells us that \( x \in \emptyset \). Hence, we conclude that \( \Omega^{c} = \emptyset \).
\end{proof}

\subsection*{Part (4)}

\begin{proof}
Let \( x \in A  \) be arbitrary. Since \( A \subseteq \Omega  \) and \( \Omega \neq \emptyset \), we have that \( x \in A  \). Hence, \( A \subseteq A  \).
\end{proof}

\subsection*{Part (5)}

\begin{proof}
    Suppose \( A \subseteq  B  \) and \( B \subseteq  C  \). Our goal is to show that \( A \subseteq  C  \); that is, for all \( x \in A  \), \( x \in C  \). To this end, let \( x \in A  \). Since \( A \subseteq B  \), we have \( x \in B \). Since \( B \subseteq C  \), we have \( x \in C  \). Thus, \( A \subseteq  C  \).
\end{proof}

\subsection*{Part (6)}

\begin{proof}
\( (\Longrightarrow) \) Suppose \( A \subseteq  B  \). Our goal is to show that \( B^{c} \subseteq A^{c} \). Suppose for contradiction that \( B^{c} \not\subseteq A^{c} \). Then there exists an \( x \in B^{c} \) such that \( x \notin A^{c}  \). Since \( x \notin A^{c}  \), it follows that \( x \in (A^{c})^{c} \). But from part (2), we have \( (A^{c})^{c} =A  \). Thus, \( x \in A  \). Since \( A \subseteq B \), we have \( x \in B \) which is a contradiction.

\( (\Longleftarrow) \) Suppose \( B^{c} \subseteq  A^{c} \). Our goal is to show that \( A \subseteq B \). Suppose for contradiction that \( A \not\subseteq B \). Then there exists an \( x \in A  \) such that \( x \notin B \). Then \( x \in B^{c} \). But \( B^{c} \subseteq A^{c}  \), and so \( x \in A^{c} \). Thus, \( x \notin A   \) which is a contradiction. Thus, \( A \subseteq B \).  
\end{proof}

\subsection*{Part (7)}

\begin{proof}
Our goal is to show that \( A \cup A = A = A \cap A  \). First, we will show that \( A \cup A = A  \). We will show the following containments; \( A \cup A \subseteq A  \) and \( A \subseteq  A \cup A  \). Starting with the first containment, let \( x \in A \cup A  \) be arbitrary. Then \( x \in A  \) or \( x \in A  \) or \( x  \) in both. In either case, \( x \in A  \) and so \( A \cup A \subseteq A  \) because \( A \subseteq  A  \) in part (4). If \( x  \) is in both, then \( x \in A  \) by using the same fact. Hence, \( A \cup A \subseteq  A  \). For the second containment, let \( x \in A  \) be arbitrary. Immediately, \( x \in A  \) or \( x \in A  \) since \( A \subseteq  A  \) and \( x  \) lies in all the sets in the union \( A \cup A  \). Thus, \( x \in A \cup A  \). Hence, \( A \cup A = A  \).   

Second, we will show that \( A \cap A = A  \). Let \( x \in A \cap A  \) be arbitrary. Then \( x \in A  \) and \( x \in A  \). Hence, \( x \in A  \) since \( A \subseteq  A   \) in part (4) and so \( A \cap A \subseteq  A  \). Let \( x \in A  \) be arbitrary. Then immediately \( x \in A  \) and \( x \in A  \) by using part (4) again. Hence, \( x \in A \cap A  \) and so \( A \subseteq A \cap A  \).
\end{proof}

\subsection*{Part (8)}

\begin{proof}
Our goal is to show that \( A \cup \Omega = \Omega \) and \( A \cap \Omega = A  \). Starting with the first equation, it suffices to show that 
\[  A \cup \Omega \subseteq \Omega \tag{1} \]
and
\[  \Omega \subseteq A \cup \Omega. \tag{2} \]
For (1), let \( x \in A \cup \Omega \) be arbitrary. Then either \( x \in A  \) or \( x \in \Omega \) or \( x  \) is in both. If \( x \in A  \), we have
\[  A \subseteq \Omega \implies x \in \Omega. \]
Clearly, we see that \( x \not\in \emptyset  \) because both \( A  \) and \( \Omega \) are non-empty sets. 
So, \( A \cup \Omega \subseteq \Omega \). On the other hand, if \( x \in \Omega \), we are done. If \( x  \) is in both, then we have \( x \in \Omega \) and \( x \in A  \). Since \( A \subseteq \Omega \), we have 
\[  A \cup \Omega \subseteq \Omega \cup \Omega = \Omega  \]
by part (4). Thus, \( A \cup \Omega \subseteq  \Omega \). 

For (2), let \( x \in \Omega \) be arbitrary. Since \( \Omega \subseteq \Omega \), it follows that \( x  \) is contained in the union \( A \cup \Omega \) containing \( \Omega \). Hence, \( x \in A \cup \Omega \) and so \( A \subseteq A \cup \Omega \).

Now, we will show \( A \cap \Omega = A  \). We will first show \( A \cap \Omega \subseteq  A  \). Let \( x \in A \cap \Omega \). Then \( x \in A  \) and \( x \in \Omega \). Since \( x \in A  \), we have \( A \cap \Omega \subseteq  A  \). Let \( x \in A  \) be arbitrary. Since \( A \subseteq  \Omega \), we have \( x \in \Omega \). Since \( x \in A  \) and \( x \in \Omega \), we have \( x \in A \cap \Omega \). Thus, \( A \subseteq  A \cap \Omega \). 
\end{proof}

\subsection*{Part (9)}

\begin{proof}
Our goal is to show the following two equations:
\[  A \cup \emptyset = A \tag{1} \]
and 
\[  A \cap \emptyset = \emptyset \tag{2}  \]

First, we show (1). It suffices to show that \[ A \cup \emptyset \subseteq A \tag{*}   \]  
and 
\[ A \subseteq  A \cup \emptyset. \tag{**} \] To show the first containment, we use the fact that \( A \cup A = A  \), \( A \subseteq  A  \) and \( \emptyset \subseteq A  \) to get  
\[  A \cup \emptyset \subseteq  A \cup A  = A.  \]
Hence, the first containment is proved.

To show the second containment, suppose for contradiction that \( A \not\subseteq A \cup \emptyset \). Then there exists an \( x \in A  \) such that \( x \not\in A \cup \emptyset \). Then \( x \in (A \cup \emptyset)^{c} \). That is, \( x \in A^{c} \cap \emptyset^{c} \). But from part (2), \( \emptyset^{c} = \Omega \). Hence, \( x \in A^{c} = \Omega \setminus  A  \), but \( x \in \Omega \). That is, \( x \notin A  \), but \( x \in \Omega \) which contradicts the assumption that \( x \in A  \). Therefore, we must have \( A  \subseteq  A \cup \emptyset  \).  

\end{proof}





\end{document}
