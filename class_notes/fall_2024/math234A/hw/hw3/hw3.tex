\documentclass[a4paper]{article}
\usepackage{standalone}
\usepackage{import}
\usepackage[utf8]{inputenc}
\usepackage[T1]{fontenc}
% \usepackage{fourier}
\usepackage{textcomp}
\usepackage{hyperref}
\usepackage[english]{babel}
\usepackage{url}
% \usepackage{hyperref}
% \hypersetup{
%     colorlinks,
%     linkcolor={black},
%     citecolor={black},
%     urlcolor={blue!80!black}
% }
\usepackage{graphicx} \usepackage{float}
\usepackage{booktabs}
\usepackage{enumitem}
% \usepackage{parskip}
% \usepackage{parskip}
\usepackage{emptypage}
\usepackage{subcaption}
\usepackage{multicol}
\usepackage[usenames,dvipsnames]{xcolor}
\usepackage{ocgx}
% \usepackage{cmbright}


\usepackage[margin=1in]{geometry}
\usepackage{amsmath, amsfonts, mathtools, amsthm, amssymb}
\usepackage{thmtools}
\usepackage{mathrsfs}
\usepackage{cancel}
\usepackage{bm}
\newcommand\N{\ensuremath{\mathbb{N}}}
\newcommand\R{\ensuremath{\mathbb{R}}}
\newcommand\Z{\ensuremath{\mathbb{Z}}}
\renewcommand\O{\ensuremath{\emptyset}}
\newcommand\Q{\ensuremath{\mathbb{Q}}}
\newcommand\C{\ensuremath{\mathbb{C}}}
\newcommand\F{\ensuremath{\mathbb{F}}}
% \newcommand\P{\ensuremath{\mathbb{P}}}
\DeclareMathOperator{\sgn}{sgn}
\DeclareMathOperator{\diam}{diam}
\DeclareMathOperator{\LO}{LO}
\DeclareMathOperator{\UP}{UP}
\DeclareMathOperator{\card}{card}
\DeclareMathOperator{\Arg}{Arg}
\DeclareMathOperator{\Dom}{Dom}
\DeclareMathOperator{\Log}{Log}
\DeclareMathOperator{\dist}{dist}
% \DeclareMathOperator{\span}{span}
\usepackage{systeme}
\let\svlim\lim\def\lim{\svlim\limits}
\renewcommand\implies\Longrightarrow
\let\impliedby\Longleftarrow
\let\iff\Longleftrightarrow
\let\epsilon\varepsilon
\usepackage{stmaryrd} % for \lightning
\newcommand\contra{\scalebox{1.1}{$\lightning$}}
% \let\phi\varphi
\renewcommand\qedsymbol{$\blacksquare$}

% correct
\definecolor{correct}{HTML}{009900}
\newcommand\correct[2]{\ensuremath{\:}{\color{red}{#1}}\ensuremath{\to }{\color{correct}{#2}}\ensuremath{\:}}
\newcommand\green[1]{{\color{correct}{#1}}}

% horizontal rule
\newcommand\hr{
    \noindent\rule[0.5ex]{\linewidth}{0.5pt}
}

% hide parts
\newcommand\hide[1]{}

% si unitx
\usepackage{siunitx}
\sisetup{locale = FR}
% \renewcommand\vec[1]{\mathbf{#1}}
\newcommand\mat[1]{\mathbf{#1}}

% tikz
\usepackage{tikz}
\usepackage{tikz-cd}
\usetikzlibrary{intersections, angles, quotes, calc, positioning}
\usetikzlibrary{arrows.meta}
\usepackage{pgfplots}
\pgfplotsset{compat=1.13}

\tikzset{
    force/.style={thick, {Circle[length=2pt]}-stealth, shorten <=-1pt}
}

% theorems
\makeatother
\usepackage{thmtools}
\usepackage[framemethod=TikZ]{mdframed}
\mdfsetup{skipabove=1em,skipbelow=1em}

\theoremstyle{definition}

\declaretheoremstyle[
    headfont=\bfseries\sffamily\color{ForestGreen!70!black}, bodyfont=\normalfont,
    mdframed={
        linewidth=1pt,
        rightline=false, topline=false, bottomline=false,
        linecolor=ForestGreen, backgroundcolor=ForestGreen!5,
    }
]{thmgreenbox}

\declaretheoremstyle[
    headfont=\bfseries\sffamily\color{NavyBlue!70!black}, bodyfont=\normalfont,
    mdframed={
        linewidth=1pt,
        rightline=false, topline=false, bottomline=false,
        linecolor=NavyBlue, backgroundcolor=NavyBlue!5,
    }
]{thmbluebox}

\declaretheoremstyle[
    headfont=\bfseries\sffamily\color{NavyBlue!70!black}, bodyfont=\normalfont,
    mdframed={
        linewidth=1pt,
        rightline=false, topline=false, bottomline=false,
        linecolor=NavyBlue
    }
]{thmblueline}

\declaretheoremstyle[
    headfont=\bfseries\sffamily, bodyfont=\normalfont,
    numbered = no,
    mdframed={
        rightline=true, topline=true, bottomline=true,
    }
]{thmbox}

\declaretheoremstyle[
    headfont=\bfseries\sffamily, bodyfont=\normalfont,
    numbered=no,
    % mdframed={
    %     rightline=true, topline=false, bottomline=true,
    % },
    qed=\qedsymbol
]{thmproofbox}

\declaretheoremstyle[
    headfont=\bfseries\sffamily\color{NavyBlue!70!black}, bodyfont=\normalfont,
    numbered=no,
    mdframed={
        rightline=false, topline=false, bottomline=false,
        linecolor=NavyBlue, backgroundcolor=NavyBlue!1,
    },
]{thmexplanationbox}

\declaretheorem[
    style=thmbox, 
    % numberwithin = section,
    numbered = no,
    name=Definition
    ]{definition}

\declaretheorem[
    style=thmbox, 
    name=Example,
    ]{eg}

\declaretheorem[
    style=thmbox, 
    % numberwithin = section,
    name=Proposition]{prop}

\declaretheorem[
    style = thmbox,
    numbered=yes,
    name =Problem
    ]{problem}

\declaretheorem[style=thmbox, name=Theorem]{theorem}
\declaretheorem[style=thmbox, name=Lemma]{lemma}
\declaretheorem[style=thmbox, name=Corollary]{corollary}

\declaretheorem[style=thmproofbox, name=Proof]{replacementproof}

\declaretheorem[style=thmproofbox, 
                name = Solution
                ]{replacementsolution}

\renewenvironment{proof}[1][\proofname]{\vspace{-1pt}\begin{replacementproof}}{\end{replacementproof}}

\newenvironment{solution}
    {
        \vspace{-1pt}\begin{replacementsolution}
    }
    { 
            \end{replacementsolution}
    }

\declaretheorem[style=thmexplanationbox, name=Proof]{tmpexplanation}
\newenvironment{explanation}[1][]{\vspace{-10pt}\begin{tmpexplanation}}{\end{tmpexplanation}}

\declaretheorem[style=thmbox, numbered=no, name=Remark]{remark}
\declaretheorem[style=thmbox, numbered=no, name=Note]{note}

\newtheorem*{uovt}{UOVT}
\newtheorem*{notation}{Notation}
\newtheorem*{previouslyseen}{As previously seen}
% \newtheorem*{problem}{Problem}
\newtheorem*{observe}{Observe}
\newtheorem*{property}{Property}
\newtheorem*{intuition}{Intuition}

\usepackage{etoolbox}
\AtEndEnvironment{vb}{\null\hfill$\diamond$}%
\AtEndEnvironment{intermezzo}{\null\hfill$\diamond$}%
% \AtEndEnvironment{opmerking}{\null\hfill$\diamond$}%

% http://tex.stackexchange.com/questions/22119/how-can-i-change-the-spacing-before-theorems-with-amsthm
\makeatletter
% \def\thm@space@setup{%
%   \thm@preskip=\parskip \thm@postskip=0pt
% }
\newcommand{\oefening}[1]{%
    \def\@oefening{#1}%
    \subsection*{Oefening #1}
}

\newcommand{\suboefening}[1]{%
    \subsubsection*{Oefening \@oefening.#1}
}

\newcommand{\exercise}[1]{%
    \def\@exercise{#1}%
    \subsection*{Exercise #1}
}

\newcommand{\subexercise}[1]{%
    \subsubsection*{Exercise \@exercise.#1}
}


\usepackage{xifthen}

\def\testdateparts#1{\dateparts#1\relax}
\def\dateparts#1 #2 #3 #4 #5\relax{
    \marginpar{\small\textsf{\mbox{#1 #2 #3 #5}}}
}

\def\@lesson{}%
\newcommand{\lesson}[3]{
    \ifthenelse{\isempty{#3}}{%
        \def\@lesson{Lecture #1}%
    }{%
        \def\@lesson{Lecture #1: #3}%
    }%
    \subsection*{\@lesson}
    \testdateparts{#2}
}

% \renewcommand\date[1]{\marginpar{#1}}


% fancy headers
\usepackage{fancyhdr}
\pagestyle{fancy}

\makeatother

% notes
\usepackage{todonotes}
\usepackage{tcolorbox}

\tcbuselibrary{breakable}
\newenvironment{verbetering}{\begin{tcolorbox}[
    arc=0mm,
    colback=white,
    colframe=green!60!black,
    title=Opmerking,
    fonttitle=\sffamily,
    breakable
]}{\end{tcolorbox}}

\newenvironment{noot}[1]{\begin{tcolorbox}[
    arc=0mm,
    colback=white,
    colframe=white!60!black,
    title=#1,
    fonttitle=\sffamily,
    breakable
]}{\end{tcolorbox}}

% figure support
\usepackage{import}
\usepackage{xifthen}
\pdfminorversion=7
\usepackage{pdfpages}
\usepackage{transparent}
\newcommand{\incfig}[1]{%
    \def\svgwidth{\columnwidth}
    \import{./figures/}{#1.pdf_tex}
}

% %http://tex.stackexchange.com/questions/76273/multiple-pdfs-with-page-group-included-in-a-single-page-warning
\pdfsuppresswarningpagegroup=1



\pagestyle{fancy}
\fancyhf{}

\title{Math 234A: Homework 3}
\author{Lance Remigio}

\begin{document}
\maketitle    
\lhead{Math 234A: Homework 3}
\chead{Lance Remigio}
\rhead{\thepage}

\begin{problem}[Complex Logarithms]
    Compute the following:
    \begin{enumerate}
        \item[(i)] \( \Log(i)  \) and \( \log(i) \)
        \item[(ii)] \( \Log(1+i)   \) and \( \log(1+i) \).
        \item[(iii)] \( \Log(-1)  \) and \( \log(-1) \).
    \end{enumerate}
    Note: for \( z \in \C^{\bullet} \), \( \log z  \) is a set not a single number.
\end{problem}

\begin{solution}
\begin{enumerate}
    \item[(i)] Observe that 
        \begin{align*}
            \Log(i) &= \ln | i |  + i \Arg(i) \\
                    &= \ln(1) + i \frac{ \pi }{ 2 }  \\
                    &= i \frac{ \pi }{ 2 } 
        \end{align*}
        and
        \begin{align*}
            \log(i) &= \{ \ln | i |  + i (\Arg(i) + 2\pi i k): k \in \Z  \} \\
                    &= \Big\{ \ln(1) + i \Big(  \frac{ \pi  }{  2  }  + 2 \pi i k\Big) : k \in \Z  \Big\} \\
                    &= \Big\{ i \Big(  \frac{ \pi }{ 2 }  + 2 \pi k  \Big): k \in \Z \Big\}. 
        \end{align*}
    \item[(ii)] Notice that 
        \begin{align*}
            \Log(1+i) &= \ln | 1 + i  | + i \Arg(1+i) \\
                      &= \ln(\sqrt{ 2 }) + i \frac{ \pi }{ 4 }
        \end{align*}
        and
        \begin{align*}
            \log(1+i) &= \Big\{\ln(\sqrt{ 2 }) + i \Big(\frac{ \pi }{ 4 }  + 2 \pi k   \Big) : k \in \Z \Big\}.   
        \end{align*}
    \item[(iii)] Observe that
        \begin{align*}
            \Log(-1) &= \ln | -1 |  + i \Arg(-1) \\
                     &= \ln | 1 |  + i \pi \\  
                     &= i \pi
        \end{align*}
        and
        \[ \log(-1) = \{ i (\pi + 2 \pi k ) : k \in \Z  \}.  \]
\end{enumerate}    
\end{solution}

\begin{problem}[Complex Powers]
   Compute the following: 
   \begin{enumerate}
       \item[(i)] \( (1 + i)^{3 +i} \)
        \item[(ii)] \( \Big(  \frac{ 1 + i  }{  1 - i  }  \Big)^{i} \)
        \item[(iii)] \( (-e)^{i/2} \).
   \end{enumerate}
\end{problem}
\begin{solution}
    \begin{enumerate}
        \item[(i)] Observe that 
            \[  (1+i)^{3+i} = \exp((3+i)\log(1+i)). \]
            Note that from part (ii) in problem 1, we have  
            \[  \log(1+i) = \Big\{ \ln(\sqrt{ 2 } ) + i \Big(  \frac{ \pi }{ 4 }  + 2 \pi k  \Big); k \in \Z \Big\}.  \]
            Then we see that
            \begin{align*}
                \exp((3+i)\log(1+i)) &= \exp(3+i) \exp(\log(1+i)) \\
                                     &= \Big\{ e^{3} \cdot e^{i} \cdot e^{\ln \sqrt{ 2 } } \cdot e^{i \Big(  \frac{ \pi  }{ 4 }  + 2 \pi k \Big)} : k \in \Z  \Big\} \ \\
                                     &= \Big\{ e^{3} \sqrt{ 2 } \cdot e^{i \Big(  \frac{ 5 \pi  }{ 4  }  + 2 \pi k  \Big)}: k \in \Z \Big\}.  
            \end{align*}
        \item[(ii)] Notice that  
            \[  \Big(  \frac{ 1 + i  }{  1 - i  }   \Big)^{i} = i^{i}.  \]
            So, we must have
            \begin{align*}
                i^{i} &= \exp(i \log(i)) \\
                      &= \Big\{ \exp \Big(  i \Big(  \frac{ \pi }{ 2 }  i + 2 \pi k i  \Big) \Big) : k \in \Z  \Big\} \\
                      &= \Big\{ \exp \Big(  - \Big(  \frac{ \pi }{ 2 }  + 2 \pi k  \Big) \Big): k \in \Z \Big\}.  
            \end{align*}
        \item[(iii)] Observe that 
            \[  (-e)^{i/2} = (-1)^{1/2} \exp(i) = i \exp(i).  \]
    \end{enumerate}
\end{solution}

\begin{problem}
   \begin{enumerate}
       \item[(a)] Let \( A \subseteq  \C  \). Show that the following statements are equivalent.
           \begin{enumerate}
               \item[(i)] \( A  \) is closed.
                \item[(ii)] For any sequence \( ({a}_{n}) \) in \( A  \) such that \( {a}_{n} \to a \in \C  \) implies that \( a \in A  \).
                \item[(iii)] \( A  \) contains all its accumulation points;  that is, if \( a \in \C  \) is an accumulation point of \( A  \), then \( a \in A  \). 
           \end{enumerate}
        \item[(b)] Given a set \( A \subseteq \C \), we define 
            \[  {\mathcal{F}}_{A} = \{ F \subseteq \C : F \ \text{is closed and} \ A \subseteq F  \}. \]
            Define \( \overline{A} = \bigcup_{ F \in {\mathcal{F}}_{A} }^{  }  F  \). Show that \( \overline{A} = A \cup A' \) where 
            \[  A' = \{ z \in \C : z \ \text{is an accumulation point} \}.  \]
   \end{enumerate} 
\end{problem}
\begin{proof}
\begin{enumerate}
    \item[(a)] To show that all the statements are equivalent, we will show that \( (i) \implies (ii) \implies (iii) \implies (i) \).  

        \( (i) \implies (ii) \) Suppose \( A  \) is closed. Let \( ({a}_{n}) \) be a sequence in \( A  \) where \( {a}_{n} \to a \in \C  \) (note that \( {a}_{n} \neq a  \). Our goal is to show that \( a \in A  \). Suppose for sake of contradiction that \( a \notin A  \). Then there exists some \( \epsilon > 0  \) such that \( {N}_{\epsilon}(a) \cap E = \emptyset  \); that is, \( {N}_{\epsilon}(a) \subseteq A^{c} \). Hence, we have that \( a \in A^{c} \). But \( ({a}_{n}) \to a \in \C  \) implies that there exists at least one \( {a}_{n} \neq a   \) such that \( {a}_{n} \in A^{c} \). However, the sequence \( ({a}_{n}) \) must be entirely contained in \( A  \) by assumption which is a contradiction. Thus, \( a \in A   \).

        \( (ii) \implies (iii) \) Let \( ({a}_{n}) \) be a sequence in \( A  \) where \( {a}_{n} \neq a \in \C  \) where \( a \in A  \). Our goal is to show that \( A  \) contains all of its limit points. Let \( a  \) be a limit point of \( A  \). Choose \( \epsilon = 1/n \) and choose \( {a}_{n} \neq a  \) to be a sequence of points in \( A  \). Then by assumption, the sequence \( ({a}_{n}) \to a \in \C   \) implies that \( a \in A  \); that is, we have that 
        \[  {N}_{1/n}(a) \cap A  \neq \emptyset.   \]
        Because \( a \in A  \), we can conclude that \( A  \) must contain all of it's accumulation points.
       
        \( (iii) \implies (i) \) Suppose \( A  \) contains all of its accumulation points. Our goal is to show that \( A  \) is closed. It suffices to show that \( A^{c} \) is an open set; that is, we need to find an \( \delta > 0  \) such that \( {N}_{\delta}(x) \subseteq A^{c} \) for all \( x \in A^{c} \). To this end, let \( x \in A^{c} \). Then \( x \notin A  \). This tells us that \( x  \) cannot be a limit point of \( A  \). That is, there exists an \( \delta > 0  \) such that \( {N}_{\delta}(x) \cap A = \emptyset \). This implies that \( {N}_{\delta}(x) \subseteq A^{c} \) for some \( \delta > 0  \), and so \( A^{c} \) must be open. Hence, \( A  \) must be closed.
    \item[(b)] Our goal is to show that \( \overline{A} = A \cup A' \). First, we would like to show two lemmas:   
        \begin{enumerate}
            \item[(*)] \( A \cup A' \) is a closed set.
            \item[(**)] If \( F  \) is a closed set and \( A \subseteq F \), then \( A \cup A' \subseteq F  \) as well. 
        \end{enumerate}
        To show that (*) holds, let \( x  \) be a accumulation point of \( A \cup A' \). Our goal is to show that this accumulation point is contained in \( A \cup A' \). By definition, we see that for all \( \epsilon > 0  \), we have 
        \[  {B}(x,\epsilon) \cap ((A \cup A') \setminus  \{ x \} ) \neq \emptyset. \]
        To this end, pick a point in this intersection, say, \( a  \) such that \( a \in {B(x,\epsilon})  \) and \( a \in (A \cup A') \setminus  \{ x \}  \). That is, we have \( a \in A  \) or \( a \in A' \). If \( a \in A  \), then \( x  \) is a accumulation point of \( A  \), and so \( x \in A \cup A' \). If \( a \in A' \), then \( a  \) is a accumulation point of \( A' \). That is, for all \( \delta> 0  \), we have 
        \[ B(a,\delta) \cap A' \setminus  \{ a \} \neq \emptyset.  \]
        Pick a point in this intersection, say, \( p \neq a  \) such that \( p \in A' \). But this implies that \( x  \) must be a limit point of \( A  \), and so \( x  \in A'  \) and thus \( A \cup A' \) must be a closed set.

        To show that (**) holds, suppose \( F  \) is a closed set and that \( A \subseteq  F  \). Our goal is to show that \( A \cup A' \subseteq F  \). Let \( x \in A \cup A' \). Then either \( x \in A  \) or \( x \in A' \). If \( x \in A  \), then \( x \in F  \) since \( A \subseteq F  \). On the other hand, if \( x \in A' \), then \( x  \) is a limit point of \( A  \). That is, for all \( \delta > 0  \), we have 
        \[  B(x,\delta) \cap (A \setminus  \{ x \} ) \neq \emptyset. \]
         Since \( A \subseteq F  \), we can see that 
         \[  B(x,\delta) \cap (F \setminus  \{ x \} ) \neq \emptyset \]
         which implies that \( x  \) is a limit point of \( F  \). But \( F  \) is closed, so \( x  \) must be contained in \( F  \). Thus, we have \( A \cup A' \subseteq  F  \) in both cases.

    In what follows, we will show that \( \overline{A} = A \cup A' \). To do this, we need to show two inclusions:
    \begin{enumerate}
        \item[(1)] \( \overline{A} \subseteq  A \cup A'  \)
        \item[(2)] \( A \cup A'  \subseteq \overline{A}\).
    \end{enumerate}
    Starting with (1), we see that \( A \cup A'  \subseteq F \) by (*). But this implies that \( A \cup A' \) is the smallest closed set containing \( F  \), we must have that 
    \[  A \cup A' \subseteq  \bigcap_{ F \in {\mathcal{F}}_{A} }^{  }  F = \overline{A} \]
    which satisfies (1).
    
    With (2), we want to show that \( \overline{A} \subseteq A \cup A' \). Note that \( A \cup A' \) is a closed set and \( A \subseteq  A \cup A' \). Then immediately we see that \( \overline{A} \subseteq A \cup A' \), satisfying (2). Thus, we conclude that \( \overline{A} = A \cup A' \).
\end{enumerate}
\end{proof}

\begin{problem}[Discontinuity of "Arg" Function]
   \begin{enumerate}
       \item[(i)] Consider the sequence \( ({z}_{n}) \) with \( {z}_{n} = -1 + \frac{ i }{ n }  \). Show that \( {z}_{n} \to -1 \).
       \item[(ii)] Consider the sequence \( ({w}_{n}) \) with \( {w}_{n} = - 1 - \frac{ i }{ n }  \). Show that \( {w}_{n} \to -1  \) as well.
       \item[(iii)] Show that \( \Arg({z}_{n})  \to \pi \) and \( \Arg({w}_{n}) \to - \pi \). 
        \item[(iv)] What did you observe from part (iii)?
   \end{enumerate} 
\end{problem}

\begin{proof}
    \begin{enumerate}
        \item[(i)] Consider the real and imaginary part of \( {z}_{n} \) 
            \[  \Re({z}_{n}) = -1 \ \text{and} \ \Im({z}_{n}) = \frac{ 1 }{ n }.  \]
            Clearly, \( \Re({z}_{n}) \to -1  \) and \( \Im({z}_{n}) \to 0  \) as \( n \to \infty  \). Thus, \( ({z}_{n}) \to -1 + i 0 = -1 \).
        \item[(ii)] Similarly, notice that 
            \begin{center}
                \( \Re({w}_{n}) \to -1 \) and \( \Im({w}_{n}) = \frac{ -1 }{ n }  \to 0   \)
            \end{center}
            as \( n \to \infty  \). So, \( ({w}_{n}) \to -1 \) as well.
        \item[(iii)] From parts (i), we see that
            \[  \Arg({z}_{n}) \to \Arg(-1) = \pi. \]
            Consider \( \Arg\Big(-1 + \frac{ i }{ n } \Big) \). Then we see that 
            \[   \Arg({w}_{n}) = \Arg\Big( -1 + \frac{ i }{ n } \Big) = \Arg(-1) + \Arg \Big(  \frac{ i }{ n }  \Big) = -\pi + \tan^{-1} \Big(  \frac{ i }{ n }  \Big). \tag{1} \] 
            If we take the limit as \( n \to \infty   \) of (1), then we obtain
            \[  \Arg({w}_{n}) \to -\pi \]
            which is different result from part (i).
        \item[(iv)] I observed that by the sequential criterion of continuity, the argument function \( \Arg(z)  \) is not a continuous function.
    \end{enumerate}
\end{proof}

\begin{problem}
    \begin{enumerate}
        \item[(i)] Let \( a \in \C  \) and \( \epsilon > 0  \). Show that 
            \[  \overline{B(a,\epsilon)} = \overline{B}(a,\epsilon). \]
        \item[(ii)] Let \( A = \{ x + iy : x,y \in \Q  \}. \)
            Show that \( \overline{A} = \C  \). (Hint: 3(b) can be useful here)
        \item[(iii)] Let \( A,B \subseteq \C  \). Show that \( \overline{A \cup B} \subseteq  \overline{A} \cup \overline{B} \).
    \end{enumerate}
\end{problem}
\begin{proof}
\begin{enumerate}
    \item[(i)] We will show that \( \overline{B(a,\epsilon)} = \overline{B}(a,\epsilon) \). It suffices to show two inclusions:
        \begin{enumerate}
            \item[(1)] \( \overline{B(a,\epsilon)} \subseteq  \overline{B}(a,\epsilon) \)
            \item[(2)] \( \overline{B}(a,\epsilon) \subseteq  \overline{B(a,\epsilon)} \).
        \end{enumerate}
    With (1), observe that \( B(a,\epsilon) \subseteq \overline{B}(a,\epsilon) \). Since \( \overline{B}(a,\epsilon) \) is closed, we know by the hint given in part (b) of problem 3 that \( \overline{B(a,\epsilon)} \subseteq \overline{B}(a,\epsilon) \).
    
    With (2), let \( x \in \overline{B}(a,\epsilon) \). By definition of \( \overline{B}(a,\epsilon) \), we have \( d(x,a) \leq \epsilon \). Then either \( d(x,a) < \epsilon \) or \( d(x,a) = \epsilon \). If \( d(x,a) < \epsilon  \), then \( x  \) is contained in \( B(a,\epsilon) \), and so \( x \in \overline{B(a,\epsilon)} \). Now, suppose \( d(x,a) = \epsilon \). Observe that the closure \( \overline{B(a,\epsilon)} \) contains its boundary points. Thus, \( x \in \overline{B(a,\epsilon)} \). Thus, \( \overline{B}(a,\epsilon) \subseteq  \overline{B(a,\epsilon)} \). 

    We conclude that (1) and (2) imply \( \overline{B(a,\epsilon)} = \overline{B}(a,\epsilon) \).
\item[(ii)] Let \( A = \{ x + iy : x,y \in \Q  \}  \). Our goal is to show that \( \overline{A } = \C  \). Note that, by problem 3(b), we see that \( \overline{A} = A \cup A'  \). We need to show the following two inclusions:
    \begin{enumerate}
        \item[(1)] \( A \cup A' \subseteq  \C  \)
        \item[(2)] \( \C \subseteq  A \cup A' \).
    \end{enumerate}
    Starting with (1), suppose \( z \in A \cup A' \). Then either \( z \in A  \) or \( z \in A' \). If \( z \in A  \), then \( z = x + iy  \) with \( x,y \in \Q  \). Since \( \Q \subseteq  \R  \), we see that \( x,y \in \R  \) and so \( z \in \C  \). If \( z \in A' \), then \( z  \) is a limit point of \( A \). That is, for all \( \epsilon > 0  \) 
    \[  B(z,\epsilon) \cap (A \setminus \{ z \} ) \neq \emptyset.  \]
    Since \( \Q  \) is dense in \( \R  \), we know that every limit point of \( \Q  \) is contained in \( \R  \). Hence, \( z  \) must be contained in \( \C  \). So, \( A \cup A' \subseteq  \C  \). 

    Let \( z \in \C  \). Then \( z = \alpha + i \beta  \) with \( \alpha, \beta \in \R  \). Our goal is to show that \( z \in A \cup A' \); that is, either \( z  \) is a limit point of \( A  \) or is an element of \( A  \). To this end, suppose that \( z  \) is not an element of \( A  \). Note that \( \alpha, \beta \in \R  \) which are limit points of \( \Q  \). Hence, \( z  \) must be a limit point of \( A  \). Thus, \( z \in A' \) and so, \( z \in A \cup A' \).

    With (1) and (2), we can conclude that \( \overline{A} =\C  \).

\item[(iii)] Our goal is to show that \( \overline{A \cup B} = \overline{A} \cup \overline{B} \); that is, we need to show that  
    \begin{enumerate}
        \item[(1)] \( \overline{A \cup B} \subseteq  \overline{A} \cup \overline{B} \)
        \item[(2)] \( \overline{A} \cup \overline{B} \subseteq \overline{A \cup B} \).
    \end{enumerate}

    Starting with (1), suppose \( x \in \overline{A \cup B} \). Then either \( x \in A \cup B  \) or \( x \in (A \cup B)' \). If \( x \in A \cup B   \), then either \( x \in A  \) or \(  x \in B  \). If \( x \in A  \), then \( x \in \overline{A}  \) since \( A \subseteq \overline{A}     \) and so \( x \in \overline{A} \cup \overline{B} \). Likewise, if \( x \in B  \), then \( x \in \overline{B} \) since \( B \subseteq  \overline{B} \). Thus, \( x \in \overline{A} \cup \overline{B} \). If \( x \in (A \cup B)' \), then \( x  \) is a limit point of \( A \cup B  \); that is, for all \( \epsilon > 0  \) 
    \[ B(x,\epsilon) \cap ((A \cup B) \setminus  \{ x \} ) \neq \emptyset. \]
    Hence, there exists \( q \in B(x,\epsilon) \cap ((A \cup B) \setminus  \{ x \} ) \). Thus, we have \( q \in A \cup B  \); that is, either \( q \in A  \) or \( q \in B  \). If \( q \in A  \), then \( x  \) is a limit point of \( A  \). Thus, \( x \in A' \) and so \( x \in \overline{A} \). Hence, \( x \in \overline{A} \cup \overline{B} \). If \( q \in B  \), then \( x  \) is a limit point of \( B \). Hence, \( x \in B' \) and so \( x \in \overline{B} \). Thus, \( x \in \overline{A} \cup \overline{B} \). Thus, we see that \( \overline{A \cup B} \subseteq \overline{A} \cup \overline{B} \) which shows (1). 

    Now, we will show (2). Let \( x \in \overline{A} \cup \overline{B} \). Then either \( x \in \overline{A} \) or \( x \in \overline{B} \). If \( x \in \overline{A} \), then \( x \in A  \) or \( x \in A' \). If \( x \in A  \), then \( x \in A \cup B  \). Thus, \( x \in \overline{A \cup B} \). If \( x \in A'  \), then \( x  \) is a limit point of \( A  \); that is, for all \( \delta > 0 \) 
    \[  B(x,\delta) \cap (A \setminus \{ x \}   ) \neq \emptyset. \]
    Since \( A \subseteq  A \cup B   \), we know that
    \[  B(x,\delta) \cap ((A \cup B) \setminus  \{ x \} ) \neq \emptyset. \]
    Hence, \( x  \) is a limit point of \( A \cup B  \) and so \( x \in (A \cup B)' \); that is, \( x \in \overline{A \cup B} \). On the other hand, if \(  x \in \overline{B} \), then the proof is analogous to the case that \( x \in A' \). Thus, \( \overline{A} \cup \overline{B} \subseteq  \overline{A \cup B} \).

    Together with (1) and (2), we have that \( \overline{A \cup B} = \overline{A} \cup \overline{B}  \).
\end{enumerate} 
\end{proof}

\begin{problem}
    \begin{enumerate}
        \item[(i)] Show that \( \mathbb{H} := \{ z = x + iy : y > 0  \}  \) is open.
        \item[(ii)] Show that \( {Q}_{1} := \{ z = x + iy : x > 0, y > 0  \}   \) is open.
        \item[(iii)] Show that \( S = \{ x + iy: -\pi < y < \pi \}  \)
    \end{enumerate}
\end{problem}
\begin{proof}
\begin{enumerate}
    \item[(i)] Our goal is to show that \( \mathbb{H} \) is an open set. It suffices to show that \( \mathbb{H}^{c} \) is a closed set. Let \( ({z}_{n}) \) be a sequence in \( \mathbb{H}^{c} \) such that \( ({z}_{n}) \to z \in \C  \). Our goal is to show that \( z \in \mathbb{H}^{c} \). Note that for \( z \in \mathbb{H}^{c} \), \( z  \) must have the property that \( \Im(z) \leq 0 \). Let \( \epsilon > 0  \). Choose \( N \in \N  \) such that for any \( n \geq N  \), we have    
        \[  | \Im({z}_{n})  - \Im(z) | < \epsilon \implies | \Im(z) | < | \Im({z}_{n}) | + \epsilon \leq \epsilon. \tag{\( | \Im({z}_{n}) | \leq 0 \)}  \]
        Since \( \epsilon > 0 \) is arbitrary, we have that \( \Im(z) \leq 0  \). Hence, \( z \in \mathbb{H}^{c} \). 
    \item[(ii)] Our goal is to show that \( {Q}_{1} \) is open by showing that \( {Q}_{1}^{c} \) is closed; that is, we need to show that for any sequence \( ({z}_{n}) \subseteq  {Q}_{1}^{c}\), \( ({z}_{n}) \to z \in \C   \) with \( z \in {Q}_{1}^{c} \). Note that \( z \in {Q}_{1}^{c} \) if \( \Im(z) \leq 0 \) and \( \Re(z) \leq 0  \). To this end, let \( ({z}_{n})  \) be a sequence in \( {Q}_{1}^{c} \) that converges to \( z \in \C  \). Let \( \epsilon > 0  \). Our goal is to show that \( z \in {Q}_{1}^{c} \). Since \( ({z}_{n}) \to z  \), we know that the real and imaginary part must converge. Thus, choose \( N \in \N  \) such that for any \( n \geq N   \), we have
        \[  | \Im({z}_{n}) - \Im(z) | < \epsilon \tag{1} \]
        and 
        \[  | \Re({z}_{n}) - \Re(z) | < \epsilon.  \tag{2}\]
        Since \( \Re({z}_{n}) \leq 0  \) and \( \Im({z}_{n}) \leq 0  \), (1) and (2) imply that
        \[ | \Im(z) | < | \Im({z}_{n}) |  + \epsilon \leq \epsilon  \]
        and 
        \[  | \Re(z) | < | \Re({z}_{n}) | + \epsilon \leq \epsilon. \]
        Since \( \epsilon > 0  \) is arbitrary, we conclude that \( | \Re(z) | \leq 0  \) and \( | \Im(z) | \leq 0  \). Thus, \( z \in {Q}_{1}^{c} \).
    \item[(iii)] Our goal is to show that \( S  \) is open; that is, we need to find \( \delta > 0  \) such that \( B(z,\delta) \subseteq  S  \) for any \( z \in S  \). To this end , let \( z \in S  \). Choose \( \delta = \frac{ 1 }{ 2 }  \min \{ \Im(z) - (-\pi), \pi - \Im(z) \}  \). Let \( w \in B(z,\delta) \). By the way we chose \( \delta \), we have
        \begin{align*}
            | z - w  | < \delta &\iff | \Im(z-w) | < \delta  \\
                                &\iff | \Im(w) | < | \Im(z) | + \delta \\
                                &\iff | \Im(w) | < \pi.
        \end{align*}
        Hence, we see that \( w \in S  \) and so, \( S  \) is open.

\end{enumerate}
\end{proof}

\begin{problem}
    \begin{enumerate}
        \item[(i)] Consider the sequence \( ({z}_{n}) \) defined by \( {z}_{n} = - 1 + \frac{ i }{ n }  \). Compute \( \lim_{ n \to \infty  }  \Log({z}_{n}) \). 
        \item[(ii)] Consider the sequence \( ({w}_{n}) \) defined by \( {w}_{n} = - 1 - \frac{ i }{ n }  \). Compute \( \lim_{ n \to \infty  }  \Log({w}_{n}) \).
        \item[(iii)] What did you observe from (i) and (ii)?
    \end{enumerate}
\end{problem}

\begin{solution}
    \begin{enumerate}
        \item[(i)] Note that \( | {z}_{n} |  = \sqrt{ 1 + \frac{ 1 }{ n^{2} }   } \to 1  \) as \( n \to \infty   \). Then observe that 
            \begin{align*}
                \lim_{ n \to \infty   } \Log({z}_{n}) &= \lim_{ n \to \infty   } [ \ln | {z}_{n} | + i \Arg({z}_{n}) ]   \\
                                                      &=  \lim_{ n \to \infty   }  \ln | {z}_{n} |  + i \lim_{ n \to \infty  }  \Arg({z}_{n}) \\
                                                      &= \ln(1) + i \pi \tag{4-(iii)} \\
                                                      &= \pi i.
            \end{align*}
        \item[(ii)] Note that \( | {w}_{n}  |  = \sqrt{  1 + \frac{ 1 }{ n^{2} }  } \to 1  \) as \( n \to \infty   \). Then we see that 
            \begin{align*}
                \lim_{ n \to \infty  }  \Log({w}_{n}) &= \lim_{ n \to \infty  }  [ \ln | {w}_{n} |  + i \Arg({w}_{n}) ] \\
                                                      &= \lim_{ n \to \infty  }  \ln | {w}_{n} |  + i \lim_{ n \to \infty  }  \Arg({w}_{n}) \\
                                                      &= \ln(1) - \pi \tag{4-(iii)} \\
                                                      &= -\pi  i
            \end{align*}
        \item[(iii)] I observed that \( \Log(z) \) is discontinuous by the Sequential Criterion of Continuity.
    \end{enumerate}
\end{solution}




\end{document}
