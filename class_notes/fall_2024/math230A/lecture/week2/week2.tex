\documentclass[a4paper]{report}
\usepackage{standalone}
\usepackage{import}
\usepackage[utf8]{inputenc}
\usepackage[T1]{fontenc}
% \usepackage{fourier}
\usepackage{textcomp}
\usepackage{hyperref}
\usepackage[english]{babel}
\usepackage{url}
% \usepackage{hyperref}
% \hypersetup{
%     colorlinks,
%     linkcolor={black},
%     citecolor={black},
%     urlcolor={blue!80!black}
% }
\usepackage{graphicx} \usepackage{float}
\usepackage{booktabs}
\usepackage{enumitem}
% \usepackage{parskip}
% \usepackage{parskip}
\usepackage{emptypage}
\usepackage{subcaption}
\usepackage{multicol}
\usepackage[usenames,dvipsnames]{xcolor}
\usepackage{ocgx}
% \usepackage{cmbright}


\usepackage[margin=1in]{geometry}
\usepackage{amsmath, amsfonts, mathtools, amsthm, amssymb}
\usepackage{thmtools}
\usepackage{mathrsfs}
\usepackage{cancel}
\usepackage{bm}
\newcommand\N{\ensuremath{\mathbb{N}}}
\newcommand\R{\ensuremath{\mathbb{R}}}
\newcommand\Z{\ensuremath{\mathbb{Z}}}
\renewcommand\O{\ensuremath{\emptyset}}
\newcommand\Q{\ensuremath{\mathbb{Q}}}
\newcommand\C{\ensuremath{\mathbb{C}}}
\newcommand\F{\ensuremath{\mathbb{F}}}
% \newcommand\P{\ensuremath{\mathbb{P}}}
\DeclareMathOperator{\sgn}{sgn}
\DeclareMathOperator{\diam}{diam}
\DeclareMathOperator{\LO}{LO}
\DeclareMathOperator{\UP}{UP}
\DeclareMathOperator{\card}{card}
\DeclareMathOperator{\Arg}{Arg}
\DeclareMathOperator{\Dom}{Dom}
\DeclareMathOperator{\Log}{Log}
\DeclareMathOperator{\dist}{dist}
% \DeclareMathOperator{\span}{span}
\usepackage{systeme}
\let\svlim\lim\def\lim{\svlim\limits}
\renewcommand\implies\Longrightarrow
\let\impliedby\Longleftarrow
\let\iff\Longleftrightarrow
\let\epsilon\varepsilon
\usepackage{stmaryrd} % for \lightning
\newcommand\contra{\scalebox{1.1}{$\lightning$}}
% \let\phi\varphi
\renewcommand\qedsymbol{$\blacksquare$}

% correct
\definecolor{correct}{HTML}{009900}
\newcommand\correct[2]{\ensuremath{\:}{\color{red}{#1}}\ensuremath{\to }{\color{correct}{#2}}\ensuremath{\:}}
\newcommand\green[1]{{\color{correct}{#1}}}

% horizontal rule
\newcommand\hr{
    \noindent\rule[0.5ex]{\linewidth}{0.5pt}
}

% hide parts
\newcommand\hide[1]{}

% si unitx
\usepackage{siunitx}
\sisetup{locale = FR}
% \renewcommand\vec[1]{\mathbf{#1}}
\newcommand\mat[1]{\mathbf{#1}}

% tikz
\usepackage{tikz}
\usepackage{tikz-cd}
\usetikzlibrary{intersections, angles, quotes, calc, positioning}
\usetikzlibrary{arrows.meta}
\usepackage{pgfplots}
\pgfplotsset{compat=1.13}

\tikzset{
    force/.style={thick, {Circle[length=2pt]}-stealth, shorten <=-1pt}
}

% theorems
\makeatother
\usepackage{thmtools}
\usepackage[framemethod=TikZ]{mdframed}
\mdfsetup{skipabove=1em,skipbelow=1em}

\theoremstyle{definition}

\declaretheoremstyle[
    headfont=\bfseries\sffamily\color{ForestGreen!70!black}, bodyfont=\normalfont,
    mdframed={
        linewidth=1pt,
        rightline=false, topline=false, bottomline=false,
        linecolor=ForestGreen, backgroundcolor=ForestGreen!5,
    }
]{thmgreenbox}

\declaretheoremstyle[
    headfont=\bfseries\sffamily\color{NavyBlue!70!black}, bodyfont=\normalfont,
    mdframed={
        linewidth=1pt,
        rightline=false, topline=false, bottomline=false,
        linecolor=NavyBlue, backgroundcolor=NavyBlue!5,
    }
]{thmbluebox}

\declaretheoremstyle[
    headfont=\bfseries\sffamily\color{NavyBlue!70!black}, bodyfont=\normalfont,
    mdframed={
        linewidth=1pt,
        rightline=false, topline=false, bottomline=false,
        linecolor=NavyBlue
    }
]{thmblueline}

\declaretheoremstyle[
    headfont=\bfseries\sffamily, bodyfont=\normalfont,
    numbered = no,
    mdframed={
        rightline=true, topline=true, bottomline=true,
    }
]{thmbox}

\declaretheoremstyle[
    headfont=\bfseries\sffamily, bodyfont=\normalfont,
    numbered=no,
    % mdframed={
    %     rightline=true, topline=false, bottomline=true,
    % },
    qed=\qedsymbol
]{thmproofbox}

\declaretheoremstyle[
    headfont=\bfseries\sffamily\color{NavyBlue!70!black}, bodyfont=\normalfont,
    numbered=no,
    mdframed={
        rightline=false, topline=false, bottomline=false,
        linecolor=NavyBlue, backgroundcolor=NavyBlue!1,
    },
]{thmexplanationbox}

\declaretheorem[
    style=thmbox, 
    % numberwithin = section,
    numbered = no,
    name=Definition
    ]{definition}

\declaretheorem[
    style=thmbox, 
    name=Example,
    ]{eg}

\declaretheorem[
    style=thmbox, 
    % numberwithin = section,
    name=Proposition]{prop}

\declaretheorem[
    style = thmbox,
    numbered=yes,
    name =Problem
    ]{problem}

\declaretheorem[style=thmbox, name=Theorem]{theorem}
\declaretheorem[style=thmbox, name=Lemma]{lemma}
\declaretheorem[style=thmbox, name=Corollary]{corollary}

\declaretheorem[style=thmproofbox, name=Proof]{replacementproof}

\declaretheorem[style=thmproofbox, 
                name = Solution
                ]{replacementsolution}

\renewenvironment{proof}[1][\proofname]{\vspace{-1pt}\begin{replacementproof}}{\end{replacementproof}}

\newenvironment{solution}
    {
        \vspace{-1pt}\begin{replacementsolution}
    }
    { 
            \end{replacementsolution}
    }

\declaretheorem[style=thmexplanationbox, name=Proof]{tmpexplanation}
\newenvironment{explanation}[1][]{\vspace{-10pt}\begin{tmpexplanation}}{\end{tmpexplanation}}

\declaretheorem[style=thmbox, numbered=no, name=Remark]{remark}
\declaretheorem[style=thmbox, numbered=no, name=Note]{note}

\newtheorem*{uovt}{UOVT}
\newtheorem*{notation}{Notation}
\newtheorem*{previouslyseen}{As previously seen}
% \newtheorem*{problem}{Problem}
\newtheorem*{observe}{Observe}
\newtheorem*{property}{Property}
\newtheorem*{intuition}{Intuition}

\usepackage{etoolbox}
\AtEndEnvironment{vb}{\null\hfill$\diamond$}%
\AtEndEnvironment{intermezzo}{\null\hfill$\diamond$}%
% \AtEndEnvironment{opmerking}{\null\hfill$\diamond$}%

% http://tex.stackexchange.com/questions/22119/how-can-i-change-the-spacing-before-theorems-with-amsthm
\makeatletter
% \def\thm@space@setup{%
%   \thm@preskip=\parskip \thm@postskip=0pt
% }
\newcommand{\oefening}[1]{%
    \def\@oefening{#1}%
    \subsection*{Oefening #1}
}

\newcommand{\suboefening}[1]{%
    \subsubsection*{Oefening \@oefening.#1}
}

\newcommand{\exercise}[1]{%
    \def\@exercise{#1}%
    \subsection*{Exercise #1}
}

\newcommand{\subexercise}[1]{%
    \subsubsection*{Exercise \@exercise.#1}
}


\usepackage{xifthen}

\def\testdateparts#1{\dateparts#1\relax}
\def\dateparts#1 #2 #3 #4 #5\relax{
    \marginpar{\small\textsf{\mbox{#1 #2 #3 #5}}}
}

\def\@lesson{}%
\newcommand{\lesson}[3]{
    \ifthenelse{\isempty{#3}}{%
        \def\@lesson{Lecture #1}%
    }{%
        \def\@lesson{Lecture #1: #3}%
    }%
    \subsection*{\@lesson}
    \testdateparts{#2}
}

% \renewcommand\date[1]{\marginpar{#1}}


% fancy headers
\usepackage{fancyhdr}
\pagestyle{fancy}

\makeatother

% notes
\usepackage{todonotes}
\usepackage{tcolorbox}

\tcbuselibrary{breakable}
\newenvironment{verbetering}{\begin{tcolorbox}[
    arc=0mm,
    colback=white,
    colframe=green!60!black,
    title=Opmerking,
    fonttitle=\sffamily,
    breakable
]}{\end{tcolorbox}}

\newenvironment{noot}[1]{\begin{tcolorbox}[
    arc=0mm,
    colback=white,
    colframe=white!60!black,
    title=#1,
    fonttitle=\sffamily,
    breakable
]}{\end{tcolorbox}}

% figure support
\usepackage{import}
\usepackage{xifthen}
\pdfminorversion=7
\usepackage{pdfpages}
\usepackage{transparent}
\newcommand{\incfig}[1]{%
    \def\svgwidth{\columnwidth}
    \import{./figures/}{#1.pdf_tex}
}

% %http://tex.stackexchange.com/questions/76273/multiple-pdfs-with-page-group-included-in-a-single-page-warning
\pdfsuppresswarningpagegroup=1



\begin{document}

\section{Lecture 3}

\subsection{Topics}

\begin{itemize}
    \item Review, Existence of Roots.
    \item Function, injective, and surjective.
    \item Equivalent Sets
    \item Finite, Infinite, Countable, At most countable.
\end{itemize}

\subsection{Review, Existence of Roots}

\begin{prop}
   There is no rational number whose square is \( 2 \). 
\end{prop}

\begin{theorem}[ ]
    There is a unique positive real number \( \alpha  \) satisfying \( \alpha^{2} = 2  \). 
\end{theorem}

\begin{proof}
\begin{enumerate}
    \item[(i)] \textbf{Uniqueness:} Suppose there are two of them \( {\alpha}_{1} \) and \( {\alpha}_{2} \). Prove that both \( {\alpha}_{1} < {\alpha}_{2} \) and \( {\alpha}_{1} > {\alpha}_{2} \) lead to a contradiction. Thus, \( {\alpha}_{1} = {\alpha}_{2} \). 
    \item[(ii)] \textbf{Existence:} Show that \( A  \) is nonempty and bounded above. Let \( \alpha = \sup A  \). Prove that both \( \alpha^{2} > 2  \) and \( \alpha^{2} < 2  \) leads to a contradiction. Thus, \( \alpha^{2} = 2  \).
\end{enumerate}
\end{proof}

\begin{remark}
    A similar argument can be used to prove that if \( x > 0  \) and \( m \in \N  \), then t there exists a unique positive real number \( \alpha  \) such that \( \alpha^{m} = x  \). We write
    \begin{center}
        \( \alpha = \sqrt[m]{ x } \) and \( \alpha = x^{1/m} \).
    \end{center}
\end{remark}

\subsection{Functions, Injective, and Surjective}

There are two definitions for functions. The former is the most common way it is defined and the latter is the more rigorous and more "correct" definition. 

\begin{definition}[Usual Way of Defining Functions]
   Let \( A  \) and \( B  \) be two sets. A \textbf{function} from \( A  \) to \( B  \) denoted by \( f: A \to B  \), is a rule that assigns each element \( x \in A  \) a unique element \( f(x) \in B  \).  
\end{definition}

In the definition above, what do we mean by "rule" and "assigning"? Notice how these words are not very mathematically precise. 

\begin{definition}[The Correct Way of Defining Functions]
    Let \( A  \) and \( B  \) be two sets. A function from \( A  \) to \( B  \) is a triple \( (f,A,B) \) where \( f \) is a \textbf{relation} form \( A  \) to \( B  \) satisfying 
    \begin{enumerate}
        \item[(i)] \( \Dom(f) = A   \) 
        \item[(ii)] If \( (x,y) \in f  \) and \( (x,z) \in f  \), then \( y = z  \). (In this case, \( A  \) is called the \textbf{domain} of \( f  \) and \( B  \) is called the \textbf{codomain} of \( f  \))
    \end{enumerate}
\end{definition}

\begin{eg}
    Let \( A = \emptyset  \) and \( B  \) be any set. Clearly, \( \emptyset \times B = \emptyset \). So, the only function from \( A = \emptyset  \) to \( B  \) is the empty function \( (f, \emptyset, B) \).
\end{eg}

\begin{itemize}
    \item The empty function is one-to-one.
    \item The empty function is onto only when \( B = \emptyset  \).
\end{itemize}

\begin{definition}[Image, Range, Onto (Surjective)]
   Consider a function \( f: A \to B  \). Let \( E \subseteq A  \). Define the \textbf{image} of \( f  \) as the set    
   \[  f(E) = \{ f(x) : x \in  E \}  = \{ y \in B : y = f(x) \ \text{for some} \ x \in E  \}. \]
   Define the \textbf{range} of \( f  \) as 
   \[  f(A) = \{ \text{the collection of all the outputs of \( f \)} \}. \]
   If \( f(A) = B  \), then we say \( f \) is \textbf{Onto (Surjective)}.
\end{definition}

\begin{definition}[Preimage]
    Consider a function \( f: A \to B  \). Let \( D \subseteq B  \). Then the \textbf{preimage} of \( D  \) under \( f  \) is denoted by 
    \[  f^{-1}(D) = \{ x \in A : f(x) \in D  \}  \]
\end{definition}

\begin{definition}[One-to-One (Injective)]
   Consider a function \( f: A \to B  \). We call \( f  \) one-to-one if any of the following equivalent conditions hold:  
   \begin{enumerate}
       \item[(i)] For all \( {x}_{1}, {x}_{2} \in A  \), if \( {x}_{1} \neq {x}_{2}  \), then \( f({x}_{1}) \neq f({x}_{2}) \).
        \item[(ii)] For all \( {x}_{1}, {x}_{2} \in A  \), if \( f({x}_{1}) = f({x}_{2})  \), then \( {x}_{1} = {x}_{2} \).
        \item[(iii)] For all \( y \in B \), the set \( f^{-1}(\{ y \} ) \) consists at most one element of \( A  \).
   \end{enumerate}
\end{definition}


\subsection{Equivalent Sets}

\begin{definition}[ ]
   Let \( A  \) and \( B  \) be two sets. We say that \( A  \) and \( B  \) have the same cardinal number, and we write \( A \sim B  \), if there is a function \( f: A \to B  \) that is both injective and surjective.
\end{definition}

\begin{remark}
    \begin{itemize}
        \item An injective and surjective mapping is a bijective mapping.
        \item \( A  \) and \( B  \) have the same cardinal number 
            \begin{align*}
                &= A \  \text{and} \ B \ \text{have the same cardinality} \\
                &= A \  \text{and} \ B \ \text{can be put in the \textbf{one-to-one correspondence}} \\
                &= \card{A} = \card{B} \\
                &= A \  \text{and} \ B \ \text{are equivalent} \\
                &= A \  \text{and} \ B \ \text{are equipotent} \\
            \end{align*}
    \end{itemize}
\end{remark}

\begin{eg}
   Consider \( \{ 1,2,3 \}  \sim \{ a,b,c \}  \). Indeed, the function \( f: \{ 1,2,3 \}  \to \{ a,b,c \}  \) defined by 
   \[ f(1) = a, f(2) = b, f(3) = c  \]
   is a bijection.
\end{eg}

\begin{eg}
    \( \N \sim \{ 2,4,6,\dots \}  \). Indeed, the function \( f: \N \to \{ 2,4,6,\dots \}   \) defined by 
    \[  f(n) = 2n \]
    is a bijection.
\end{eg}

\begin{eg}
    \( \N \sim \Z  \). Indeed, \( f: \N \to \Z  \) defined by
    \[  f(n) = 
    \begin{cases}
        \frac{ n }{ 2 }  &\text{if} \ n \ \text{is even} \\
        -\frac{ n - 1 }{ 2 }  &\text{if} \ n \ \text{is odd} \\
    \end{cases}  \]
    is a bijection.
\end{eg}

\begin{eg}
    \( (-\infty , \infty ) \sim (0,\infty ) \). Indeed, \( f(x) = e^{x}  \) is a bijection between \( (-\infty , \infty ) \) and \( (0,\infty ) \).
\end{eg}

\begin{eg}
    \( (0,\infty ) \sim (0,1) \). Indeed, the function \( f: (0,\infty) \to (0,1) \) defined by 
    \[  f(x) = \frac{ x  }{  x + 1  }  \] is a bijection.
\end{eg}

\begin{eg}
    \( [0,1) \sim (0,1) \). Indeed, the function \( f: [0,1) \to (0,1) \) defined by
    \[  f(x) = 
    \begin{cases}
        \frac{ 1 }{ 2 }  &\text{if} \ x = 0 \\ 
        \frac{ 1 }{ n+1 }  &\text{if} \ x = \frac{ 1 }{ n }  \ \text{for} \ n \geq 2 \\
        x &\text{otherwise}
    \end{cases}  \]
    is a bijection.
\end{eg}

\begin{definition}[\( \sim  \) is an equivalence relation]
    Let \(  A  \) and \( B  \) be two sets. Note that 
    \begin{enumerate}
        \item[(i)] \( A \sim A \) (\( \sim  \) is reflexive) 
        \item[(ii)] If \( A \sim B  \), then \( B \sim A  \) (\( \sim  \) is symmetric) 
        \item[(iii)] If \( A \sim B  \) and \( B \sim C  \), then \( A \sim C  \) (\( \sim  \) is transitive).
    \end{enumerate}
\end{definition}

Observe the following notation
\begin{align*}
    {\N}_{n} &= \{ 1,2,3, \dots, n  \}  \\
    \N &= \{ 1,2,3,\dots  \} 
\end{align*}

\subsection{Finite, Infinite, Countable, At most countable}

\begin{definition}[Finite, Infinite, Countable, At most countable]
    Let \( A  \) be any set.
    \begin{enumerate}
        \item[(a)] We say that \( A  \) is \textbf{finite} if \( A \neq \emptyset  \) or \( A \sim {\N}_{n} \) for some natural number \( n  \).
            \begin{itemize}
            \item[(*)] When \( A \sim {\N}_{n} \), we say \( A  \) has \( n  \) elements and we write \( \card(A) = n \).
            \item[(*)] Also, we set \( \card(\emptyset) = 0 \).
            \end{itemize}
        \item[(b)] The set \( A  \) is said to be \textbf{infinite} if it is not finite.
        \item[(c)] The set \(  A \) is said to be \textbf{countable} if \( A \sim \N  \); that is, there exists \( g: \N \to A  \) is a bijection where \( A = \{ g(1), g(2), g(3), \dots  \}  \).
        \item[(d)] The set \( A  \) is said to be \textbf{uncountable} if it is neither countable or finite. 
        \item[(e)] The set \( A  \) is said to be \textbf{at most countable} if it is either finite or countable.
    \end{enumerate}
\end{definition}

\begin{remark}
   Previously, we shared \( \Z \sim \N  \). Thus, \( \Z  \) is countable. (Also, note that \( \N  \) is a proper subset of \( \Z  \), nevertheless, \( \N \sim \Z  \)) 
\end{remark}

Below are some Basic Theorems 
\begin{enumerate}
    \item[(i)] Every countable set is infinite (There is no bijection \( {\N}_{n} \to \N  \)).
    \item[(ii)] Suppose \( A \sim B  \). Then
        \begin{align*}
            A \  \text{is finite} &\iff B \  \text{is finite} \\
            A \  \text{is countable} &\iff B \  \text{is countable} \\
            A \  \text{is uncountable} &\iff B \  \text{is uncountable} \\
        \end{align*}
    \item[(iii)] The union of two finite sets is finite. If \( A  \) is infinite and \( B  \) is infinite, then \( A \setminus  B   \) is infinite. 
    \item[(iv)] If \textbf{\( A  \) is at most countable}, then there exists a \( 1-1 \) function \( f: A \to \N  \).
\end{enumerate} 


\end{document}
