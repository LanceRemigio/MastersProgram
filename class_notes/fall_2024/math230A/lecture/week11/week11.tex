\documentclass[a4paper]{article}
\usepackage[utf8]{inputenc}
\usepackage[T1]{fontenc}
% \usepackage{fourier}
\usepackage{textcomp}
\usepackage{hyperref}
\usepackage[english]{babel}
\usepackage{url}
% \usepackage{hyperref}
% \hypersetup{
%     colorlinks,
%     linkcolor={black},
%     citecolor={black},
%     urlcolor={blue!80!black}
% }
\usepackage{graphicx} \usepackage{float}
\usepackage{booktabs}
\usepackage{enumitem}
% \usepackage{parskip}
% \usepackage{parskip}
\usepackage{emptypage}
\usepackage{subcaption}
\usepackage{multicol}
\usepackage[usenames,dvipsnames]{xcolor}
\usepackage{ocgx}
% \usepackage{cmbright}


\usepackage[margin=1in]{geometry}
\usepackage{amsmath, amsfonts, mathtools, amsthm, amssymb}
\usepackage{thmtools}
\usepackage{mathrsfs}
\usepackage{cancel}
\usepackage{bm}
\newcommand\N{\ensuremath{\mathbb{N}}}
\newcommand\R{\ensuremath{\mathbb{R}}}
\newcommand\Z{\ensuremath{\mathbb{Z}}}
\renewcommand\O{\ensuremath{\emptyset}}
\newcommand\Q{\ensuremath{\mathbb{Q}}}
\newcommand\C{\ensuremath{\mathbb{C}}}
\newcommand\F{\ensuremath{\mathbb{F}}}
% \newcommand\P{\ensuremath{\mathbb{P}}}
\DeclareMathOperator{\sgn}{sgn}
\DeclareMathOperator{\diam}{diam}
\DeclareMathOperator{\LO}{LO}
\DeclareMathOperator{\UP}{UP}
\DeclareMathOperator{\card}{card}
\DeclareMathOperator{\Arg}{Arg}
\DeclareMathOperator{\Dom}{Dom}
\DeclareMathOperator{\Log}{Log}
\DeclareMathOperator{\dist}{dist}
% \DeclareMathOperator{\span}{span}
\usepackage{systeme}
\let\svlim\lim\def\lim{\svlim\limits}
\renewcommand\implies\Longrightarrow
\let\impliedby\Longleftarrow
\let\iff\Longleftrightarrow
\let\epsilon\varepsilon
\usepackage{stmaryrd} % for \lightning
\newcommand\contra{\scalebox{1.1}{$\lightning$}}
% \let\phi\varphi
\renewcommand\qedsymbol{$\blacksquare$}

% correct
\definecolor{correct}{HTML}{009900}
\newcommand\correct[2]{\ensuremath{\:}{\color{red}{#1}}\ensuremath{\to }{\color{correct}{#2}}\ensuremath{\:}}
\newcommand\green[1]{{\color{correct}{#1}}}

% horizontal rule
\newcommand\hr{
    \noindent\rule[0.5ex]{\linewidth}{0.5pt}
}

% hide parts
\newcommand\hide[1]{}

% si unitx
\usepackage{siunitx}
\sisetup{locale = FR}
% \renewcommand\vec[1]{\mathbf{#1}}
\newcommand\mat[1]{\mathbf{#1}}

% tikz
\usepackage{tikz}
\usepackage{tikz-cd}
\usetikzlibrary{intersections, angles, quotes, calc, positioning}
\usetikzlibrary{arrows.meta}
\usepackage{pgfplots}
\pgfplotsset{compat=1.13}

\tikzset{
    force/.style={thick, {Circle[length=2pt]}-stealth, shorten <=-1pt}
}

% theorems
\makeatother
\usepackage{thmtools}
\usepackage[framemethod=TikZ]{mdframed}
\mdfsetup{skipabove=1em,skipbelow=1em}

\theoremstyle{definition}

\declaretheoremstyle[
    headfont=\bfseries\sffamily\color{ForestGreen!70!black}, bodyfont=\normalfont,
    mdframed={
        linewidth=1pt,
        rightline=false, topline=false, bottomline=false,
        linecolor=ForestGreen, backgroundcolor=ForestGreen!5,
    }
]{thmgreenbox}

\declaretheoremstyle[
    headfont=\bfseries\sffamily\color{NavyBlue!70!black}, bodyfont=\normalfont,
    mdframed={
        linewidth=1pt,
        rightline=false, topline=false, bottomline=false,
        linecolor=NavyBlue, backgroundcolor=NavyBlue!5,
    }
]{thmbluebox}

\declaretheoremstyle[
    headfont=\bfseries\sffamily\color{NavyBlue!70!black}, bodyfont=\normalfont,
    mdframed={
        linewidth=1pt,
        rightline=false, topline=false, bottomline=false,
        linecolor=NavyBlue
    }
]{thmblueline}

\declaretheoremstyle[
    headfont=\bfseries\sffamily, bodyfont=\normalfont,
    numbered = no,
    mdframed={
        rightline=true, topline=true, bottomline=true,
    }
]{thmbox}

\declaretheoremstyle[
    headfont=\bfseries\sffamily, bodyfont=\normalfont,
    numbered=no,
    % mdframed={
    %     rightline=true, topline=false, bottomline=true,
    % },
    qed=\qedsymbol
]{thmproofbox}

\declaretheoremstyle[
    headfont=\bfseries\sffamily\color{NavyBlue!70!black}, bodyfont=\normalfont,
    numbered=no,
    mdframed={
        rightline=false, topline=false, bottomline=false,
        linecolor=NavyBlue, backgroundcolor=NavyBlue!1,
    },
]{thmexplanationbox}

\declaretheorem[
    style=thmbox, 
    % numberwithin = section,
    numbered = no,
    name=Definition
    ]{definition}

\declaretheorem[
    style=thmbox, 
    name=Example,
    ]{eg}

\declaretheorem[
    style=thmbox, 
    % numberwithin = section,
    name=Proposition]{prop}

\declaretheorem[
    style = thmbox,
    numbered=yes,
    name =Problem
    ]{problem}

\declaretheorem[style=thmbox, name=Theorem]{theorem}
\declaretheorem[style=thmbox, name=Lemma]{lemma}
\declaretheorem[style=thmbox, name=Corollary]{corollary}

\declaretheorem[style=thmproofbox, name=Proof]{replacementproof}

\declaretheorem[style=thmproofbox, 
                name = Solution
                ]{replacementsolution}

\renewenvironment{proof}[1][\proofname]{\vspace{-1pt}\begin{replacementproof}}{\end{replacementproof}}

\newenvironment{solution}
    {
        \vspace{-1pt}\begin{replacementsolution}
    }
    { 
            \end{replacementsolution}
    }

\declaretheorem[style=thmexplanationbox, name=Proof]{tmpexplanation}
\newenvironment{explanation}[1][]{\vspace{-10pt}\begin{tmpexplanation}}{\end{tmpexplanation}}

\declaretheorem[style=thmbox, numbered=no, name=Remark]{remark}
\declaretheorem[style=thmbox, numbered=no, name=Note]{note}

\newtheorem*{uovt}{UOVT}
\newtheorem*{notation}{Notation}
\newtheorem*{previouslyseen}{As previously seen}
% \newtheorem*{problem}{Problem}
\newtheorem*{observe}{Observe}
\newtheorem*{property}{Property}
\newtheorem*{intuition}{Intuition}

\usepackage{etoolbox}
\AtEndEnvironment{vb}{\null\hfill$\diamond$}%
\AtEndEnvironment{intermezzo}{\null\hfill$\diamond$}%
% \AtEndEnvironment{opmerking}{\null\hfill$\diamond$}%

% http://tex.stackexchange.com/questions/22119/how-can-i-change-the-spacing-before-theorems-with-amsthm
\makeatletter
% \def\thm@space@setup{%
%   \thm@preskip=\parskip \thm@postskip=0pt
% }
\newcommand{\oefening}[1]{%
    \def\@oefening{#1}%
    \subsection*{Oefening #1}
}

\newcommand{\suboefening}[1]{%
    \subsubsection*{Oefening \@oefening.#1}
}

\newcommand{\exercise}[1]{%
    \def\@exercise{#1}%
    \subsection*{Exercise #1}
}

\newcommand{\subexercise}[1]{%
    \subsubsection*{Exercise \@exercise.#1}
}


\usepackage{xifthen}

\def\testdateparts#1{\dateparts#1\relax}
\def\dateparts#1 #2 #3 #4 #5\relax{
    \marginpar{\small\textsf{\mbox{#1 #2 #3 #5}}}
}

\def\@lesson{}%
\newcommand{\lesson}[3]{
    \ifthenelse{\isempty{#3}}{%
        \def\@lesson{Lecture #1}%
    }{%
        \def\@lesson{Lecture #1: #3}%
    }%
    \subsection*{\@lesson}
    \testdateparts{#2}
}

% \renewcommand\date[1]{\marginpar{#1}}


% fancy headers
\usepackage{fancyhdr}
\pagestyle{fancy}

\makeatother

% notes
\usepackage{todonotes}
\usepackage{tcolorbox}

\tcbuselibrary{breakable}
\newenvironment{verbetering}{\begin{tcolorbox}[
    arc=0mm,
    colback=white,
    colframe=green!60!black,
    title=Opmerking,
    fonttitle=\sffamily,
    breakable
]}{\end{tcolorbox}}

\newenvironment{noot}[1]{\begin{tcolorbox}[
    arc=0mm,
    colback=white,
    colframe=white!60!black,
    title=#1,
    fonttitle=\sffamily,
    breakable
]}{\end{tcolorbox}}

% figure support
\usepackage{import}
\usepackage{xifthen}
\pdfminorversion=7
\usepackage{pdfpages}
\usepackage{transparent}
\newcommand{\incfig}[1]{%
    \def\svgwidth{\columnwidth}
    \import{./figures/}{#1.pdf_tex}
}

% %http://tex.stackexchange.com/questions/76273/multiple-pdfs-with-page-group-included-in-a-single-page-warning
\pdfsuppresswarningpagegroup=1



\begin{document}

\section{Lecture 20-21}
\subsection{Topics}

\begin{itemize}
    \item {\hyperref[Infinite Series]{Infinite Series}}
    \item {\hyperref[Telescoping Series, Geometric Series]{Telescoping Series, Geometric Series}}
    \item {\hyperref[Algebraic Limit Theorem for Series]{Algebraic Limit Theorem for Series}}
    \item {\hyperref[Divergence Test]{Divergence Test}}
    \item {\hyperref[Cauchy Criterion for Series]{Cauchy Criterion for Series}}
    \item {\hyperref[Absolute Convergence Test]{Absolute Convergence Test}}
\end{itemize}

\subsection{Infinite Series}\label{Infinite Series}

Consider the following expression:
\[  \frac{ 1  }{  2  }  + \frac{ 1  }{  4  }  + \frac{ 1  }{  8  }  + \frac{ 1 }{ 16  }  + \cdots \ . \]
How can we make sense of the infinite sum above? More generally, let \( ({a}_{n}) \) be a sequence of real numbers. Then what does the following expression mean?
\[  \sum_{ n=1  }^{ \infty  } {a}_{n} = {a}_{1} + {a}_{2} + {a}_{3} + \cdots \ ? \]
\begin{definition}[Infinite Series]
    Let \( (X, \|\cdot\|) \) be a normed space. Let \( ({x}_{n}) \) be a sequence in \( X  \). 
    \begin{enumerate}
        \item[(*)] An expression of the form 
            \[  \sum_{ n=1  }^{ \infty  } {x}_{n} = {x}_{1} + {x}_{2} + {x}_{3} + \cdots  \]
            is called an \textbf{infinite series}.
        \item[(*)] \( {x}_{1}, {x}_{2}, \dots  \) are called the \textbf{terms} of this infinite series.
        \item[(*)] The corresponding sequence of \textbf{partial sums} is defined by
            \[  \forall m \in \N  \ \ {s}_{m} = \text{(finite) sum of the first \(  m  \) terms of the series}; \]
            that is,
            \begin{align*}
                {s}_{1} &= {x}_{1} \\
                {s}_{2} &= {x}_{1} + {x}_{2} \\
                {s}_{3} &= {x}_{1} + {x}_{2} + {x}_{3} \\
                        &\vdots \\
                {s}_{m} &= {x}_{1} + {x}_{2} + \cdots + {x}_{m} \\
                        &\vdots
            \end{align*}
        \item We say that the infinite series \( \sum_{ n=1  }^{ \infty  } {x}_{n}  \) converges to \( L \in X  \) (and we write \( \sum_{ n=1  }^{ \infty  } {x}_{n} = L  \)) if \( \lim_{ n \to \infty  }  {s}_{m} =  L  \).
        \item We say that the infinite series \textbf{diverges}, if \( ({s}_{m}) \) diverges.
        \item If \( X = \R  \) and \( {s}_{m} \to \infty   \), we write \( \sum_{ n=1  }^{ \infty  } {x}_{n} = \infty  \).
        \item[(*)] If \( X = \R  \) and \( {s}_{m} \to - \infty   \), we write \( \sum_{ n=1  }^{ \infty  } {x}_{n} = - \infty  \).
    \end{enumerate}
\end{definition}

\begin{remark}[1]
    Given an infinite series \( \sum_{ n=1  }^{ \infty  } {x}_{n}  \), it is important to keep a clear distinction between
    \begin{enumerate}
        \item[(a)] the sequence of terms: \( ({x}_{1}, {x}_{2}, {x}_{3}, \dots ) \)
        \item[(b)] the sequence of partial sums: \( ({s}_{1}, {s}_{2}, {s}_{3}, \dots ) \).
    \end{enumerate}
\end{remark}

\begin{remark}[2]
    We may sometimes consider infinite series where the summation begins with \( n = 0  \) or \( n = {n}_{0} \) for some integer \( {n}_{0}  \) different from \( 1  \).
\end{remark}

As we shall see, some of our theorems apply specifically to series in \( \R  \) or to series with terms in \( [0,\infty) \). Also, in our examples, we will primarily focus on series in \( \R  \); however, we will also consider encounter highly useful theorems that hold in more general normed spaces.  

In most cases it is difficult (or even impossible)  to find a simple formula for the partial sum \( {s}_{m} \). However, there are two types of series for which we can easily find a simple formula for the partial sums. These two types are:
\begin{enumerate}
    \item[(1)] Telescoping Series
    \item[(2)] Geometric Series
\end{enumerate}

\subsection{Telescoping Series, Geometric Series}\label{Telescoping Series, Geometric Series}

\begin{eg}
   Consider the following series 
   \[  \sum_{ n=1  }^{ \infty  } \Big(  \frac{ 1 }{ n } - \frac{ 1 }{ n + 1 }  \Big). \]
   Notice that \( {x}_{n} = \frac{ 1 }{ n }  - \frac{ 1 }{ n+1 }  \). The corresponding sequence of partial sums is  
   \begin{align*}
       {s}_{1} &= 1 - \frac{ 1 }{ 2 }  \\
       {s}_{2} &= \Big(  1 - \frac{ 1 }{ 2 }  \Big) + \Big(  \frac{ 1 }{ 2 }  - \frac{ 1 }{ 3 }  \Big) = 1 - \frac{ 1 }{ 3 } \\
       {s}_{3} &= \Big(  1 - \frac{ 1 }{ 2 }  \Big) + \Big(  \frac{ 1 }{ 2 }  - \frac{ 1 }{ 3 }  \Big) + \Big(  \frac{ 1 }{ 3 }  - \frac{ 1 }{ 4 }  \Big) = 1 - \frac{ 1 }{ 4 } \\
               &\vdots \\
       {s}_{m} &= \sum_{ n=1  }^{ m } \Big(  \frac{ 1 }{ n }  - \frac{ 1 }{ n + 1 }  \Big) = \Big(  \sum_{ n=1  }^{ m  } \frac{ 1 }{ n }  \Big) - \Big(  \sum_{ n=1  }^{ m } \frac{ 1 }{ n+1 }  \Big) \\ 
               &= 1 - \frac{ 1 }{ m + 1 }.
   \end{align*}
   Clearly, we see that 
   \[  \lim_{ m \to \infty  }  {s}_{m} = \lim_{ m \to \infty  }  \Big[ 1 - \frac{ 1 }{  m + 1  } \Big] = 1.  \]
   Hence, \( \sum_{ n=1  }^{ \infty   \frac{ 1 }{ n (n+1) }  } \) converges to \( 1  \).
\end{eg}

In general, a telescoping series is an infinite series where partial sums eventually have a finite number of terms after cancellation. For example, if \( ({y}_{n}) \) is a sequence in the normed space \( (X, \|\cdot\|) \), then \( \sum_{ n=1  }^{ \infty   } ({y}_{n} - {y}_{n+1}) \) is a telescoping series; that is, 
\begin{align*}
    {s}_{m} = \sum_{ n=1  }^{ m } ({y}_{n} - {y}_{n+1}) = \Big(  \sum_{ n=1  }^{ m } {y}_{n} \Big) - \Big(  \sum_{ n=1  }^{ m } {y}_{n+1} \Big) &= [{y}_{1} + {y}_{2} + \cdots + {y}_{m}] - [{y}_{2} + {y}_{3} + \cdots + {y}_{m+1}] \\
                                                                                                                                                &= {y}_{1} - {y}_{m}.
\end{align*}

\subsection{Geometric Series}

Let \( k  \) be a fixed integer and let \( r \neq 0  \) be a fixed real number. The infinite series \( \sum_{ n= k  }^{  \infty   } r^{n} = r^{k } + r^{k +1} + r^{k + 2 } + \cdots   \) is called a \textbf{geometric series} with common ration \( "r" \). For example, 
\[ \sum_{ n=1  }^{ \infty   } \Big(  \frac{ 1 }{ 2 }  \Big)^{n} = \frac{ 1 }{ 2 }  + \frac{ 1 }{ 4 }  + \frac{ 1 }{ 8 }  + \cdots \ \text{is a geometric series with common ratio} \  \frac{ 1 }{ 2 }. \]
Another example is that
\[  \sum_{ n=1  }^{ \infty  } \frac{ 7^{n} }{  29^{n} }  \ \text{is a geometric series with common ratio} \ \frac{ 7  }{  2 9 }. \]
A non-example is the following:
\[  \sum_{ n=1  }^{ \infty   } \frac{ 1 }{ n^{2} }. \]
We can easily find a formula for the \( m \)th partial sum of \( \sum_{ n= k  }^{  \infty   } r^{k } \) where
\begin{align*}
    {s}_{1} &= r^{k} \\
    {s}_{2}&= r^{k } + r^{k+1} \\
    {s}_{3} &= r^{k } + r^{k+1} + r^{k+2} \\
            &\vdots \\
    {s}_{m} &= r^{k } + r^{k+1} + \cdots + r^{k + m - 1} \tag{*}
\end{align*}

Now, if \( r = 1  \), we have
\[  {s}_{m} = \underbrace{1 + 1 + \cdots + 1}_{m \text{summands}} = m.  \]

If \( r \neq 1  \), then multiply both sides of (*) by \( r  \):
\[  r {s}_{m} = r^{k+1} + r^{k+2} + \cdots+ r^{k+m} \tag{**}. \]
Subtracting (**) from (*), we get
\[  {s}_{m} - r {s}_{m} = r^{k} - r^{k + m}. \]
Since \( r \neq 1  \), we have
\[  {s}_{m} = \frac{ r^{k } - r^{k + m} }{ 1 - r  }  = \frac{ r^{k } (1 - r^{m}) }{  1 - r  }. \]

Note that
\begin{enumerate}
    \item[(i)] If \( | r  |  < 1  \), then \( \lim r^{m} = 0  \).
    \item[(ii)] If \( | r  |  > 1  \) or \( r = -1  \), then \( \lim_{ n \to \infty  }  r^{m}  \) does not exists.
\end{enumerate}
Hence, we have 
\[  \lim_{ m \to \infty  }  {s}_{m} = 
\begin{cases}
    \frac{ r^{k } }{  1 - r  }  &\text{if} | r  |  < 1 \\
    \text{DNE} &\text{if} | r  |  \geq 1. 
\end{cases} \]
Thus, 
\[  \sum_{ n= k  }^{  \infty   } r^{n } = 
\begin{cases}
    \frac{ r^{k }  }{  1 - r  }  &\text{if} | r  |  < 1 \\
    \text{diverges} &\text{if} | r  |  \geq 1.
\end{cases} \]

\begin{eg}
    \begin{itemize}
        \item \( \sum_{ n=1  }^{ \infty   } \Big(  \frac{ 1 }{ 2 }  \Big)^{n} \)

            Observe that 
            \[  \sum_{ n=1  }^{ \infty  } \Big(  \frac{ 1 }{ 2 }  \Big)^{n} = \frac{ \Big(  \frac{ 1 }{ 2 }  \Big)^{1} }{  1 - \frac{ 1 }{ 2 }  }  = \frac{ \frac{ 1 }{ 2 }  }{ \frac{ 1 }{ 2 }  }  = 1.  \]

        \item \( \sum_{ n=4 }^{ \infty  } \Big(  \frac{ 1 }{ 2 }  \Big)^{n}  \)

            Observe that 
            \[  \sum_{ n=4 }^{ \infty  } \Big(  \frac{ 1 }{ 2 }  \Big)^{n} = \frac{ \Big(  \frac{ 1 }{ 2 }  \Big)^{4} }{  1 - \frac{ 1 }{ 2 }  } = \frac{ \Big(  \frac{ 1 }{ 2 }  \Big)^{4} }{  \Big(  \frac{ 1 }{ 2 }  \Big) }  = \frac{ 1 }{ 8 }. \]
    \end{itemize}
\end{eg}

\subsection{Algebraic Limit Theorem for Series}\label{Algebraic Limit Theorem for Series}

\begin{theorem}[ ]
    Let \( (X , \|\cdot\|) \) be a normed space. Let \( ({a}_{n}) \) and \( ({b}_{n}) \) be two sequence in \( X  \). Suppose that 
    \[  \sum_{ n=1  }^{ \infty  } {a}_{n} = A \ \ (A \in X) , \ \ \sum_{ n=1  }^{ \infty  } {b}_{n} = B \ \ (B \in X). \]
    Then
    \begin{enumerate}
        \item[(i)] For any scalar \( \lambda  \), \( \sum_{ n=1  }^{ \infty   } (\lambda {a}_{n}) = \lambda A  \).
        \item[(ii)] \( \sum_{ n=1  }^{ \infty  } ({a}_{n} + {b}_{n}) = A + B \).
    \end{enumerate}
\end{theorem}
\begin{proof}
Can easily be proven via the Algebraic Limit Theorem for Sequences.
\end{proof}

\subsection{Divergence Test}\label{Divergence Test}

\begin{theorem}[Divergence Test]
   Let \( (X,\|\cdot\|) \) be a normed space. Let \( ({x}_{n}) \) be a sequence in \( X  \). If \( \sum_{ n=1  }^{ \infty  } {x}_{n}  \) converges, then \( \lim_{ n \to \infty  }  {x}_{n} = 0  \). 
\end{theorem}
\begin{proof}
Let \( {s}_{n} = {x}_{1} + \cdots + {x}_{n} \). Let \( L = \sum_{ n=1  }^{ \infty  } {x}_{n} \). Note that 
\[  \sum_{ n=1  }^{ \infty  } {x}_{n} = L \implies \lim_{ n \to \infty  }  {s}_{n} = L.  \]
Also, note that 
\[  \forall n \geq 2 \ \ {x}_{n} = {s}_{n} - {s}_{n-1}. \]
Note that \( \lim {s}_{n} = L  \) and \( \lim {s}_{n-1} = L  \). Therefore, 
\[  \lim_{ n \to \infty  }  {x}_{n} = \lim_{ n \to \infty  }  ({s}_{n} - {s}_{n-1}) = L - L =  0  \]
by the Algebraic Limit Theorem for normed spaces.
\end{proof}
\begin{remark}
    Note that the divergence test is just the contrapositive of the above.
\end{remark}

\begin{eg}
    \begin{itemize}
        \item \( \sum_{ n=1  }^{ \infty  } (-1)^{n} \) diverges because \( \lim_{ n \to \infty  }  (-1)^{n} \) does not exist.
        \item \( \sum_{ n=1  }^{ \infty  } \frac{ 3n+1 }{ 7n-4 }  \) diverges because \( \lim_{ n \to \infty  }  \frac{ 3n + 1 }{ 7 n - 4  }  = \frac{ 3 }{ 7 } \neq 0  \).
    \end{itemize}
\end{eg}

From the above statements, we can now see make two key observations:

\begin{itemize}
    \item If \( \lim_{ n \to \infty  }  {x}_{n} = 0  \), then \( \sum_{ n=1  }^{ \infty  } {x}_{n} \) may or may not converge.
    \item If \( \lim_{ n \to \infty  }  {x}_{n} \neq 0  \), then \( \sum_{ n=1  }^{ \infty  } {x}_{n}  \) diverges.
\end{itemize}

As for the first observation above, we see that \( \sum_{  }^{  } \frac{ 1 }{ n }  \) diverges, but \( \sum_{   }^{  } \frac{ 1 }{ n^{2} }   \) converges.

\subsection{Cauchy Criterion for Series}\label{Cauchy Criterion for Series}


\begin{theorem}[Cauchy Criterion]
    Let \( (X,\|\cdot\|) \) be a complete normed space. Let \( ({x}_{n}) \) be a sequence in \( X  \). Then
    \[  \sum_{ k=1  }^{ \infty  } {x}_{k } \ \text{converges} \ \iff \forall \epsilon > 0 \ \exists N \in \N \ \text{such that} \ \forall n > m > N \ \ \|\sum_{ k=1  }^{ n } {x}_{k } \| < \epsilon. \]
\end{theorem}
\begin{proof}
Let \( {s}_{n} = {x}_{1} + \cdots + {x}_{k} \). Assuming that  
\[  \|{s}_{n} - {s}_{m} \| = \Big\|\sum_{ k=m+1 }^{ n } {x}_{k} \Big\| \]
where \( n > m  \) and from the fact that 
\begin{align*}
    {s}_{n} - {s}_{m} &= ({x}_{1} + \cdots + {x}_{m} + \cdots + {x}_{n}) - ({x}_{1} +  \cdots + {x}_{m}) \\
                      &= \sum_{ k= m+1 }^{ n } {s}_{k}.
\end{align*}
Then we have 
\begin{align*}
    \sum_{ k=1  }^{ \infty  } {x}_{k } \ \text{converges} &\iff ({s}_{k}) \ \text{converges} \\
                                                          &\iff ({s}_{k}) \ \text{is Cauchy} \\
                                                          &\iff \forall \epsilon > 0 \ \exists N \in \N \ \text{such that} \ \forall n > m > N \  \ \|{s}_{n} - {s}_{m} \| < \epsilon \\
                                                          &\iff \forall \epsilon > 0 \ \exists N \in \N \ \text{such that} \ \forall n > m > N \ \ \|\sum_{ k=m+1 }^{ n } {x}_{k} \| < \epsilon
\end{align*}
as desired.
\end{proof}

From here, we will refer to complete normed spaces as Banach spaces.

\subsection{Absolute Convergence Test}\label{Absolute Convergence Test}

\begin{theorem}[Absolute Convergence Test]
    Let \( (X,\|\cdot\|) \) be a Banach Space. Let \( ({x}_{n}) \) be a sequence in \( X  \). If \( \underbrace{\sum_{ n=1  }^{ \infty  } \|{x}_{n}\| }_{\text{a sum in \( \R  \)}} \) converges, then \( \sum_{ n=1  }^{ \infty  } {x}_{n} \) converges.
\end{theorem}
\begin{proof}
    By the Cauchy Criterion for series, it suffices to show that 
    \[  \forall \epsilon > 0 \ \exists N \in \N \ \text{such that} \ \forall n > m > N \Big\|\sum_{ k=m+1 }^{ n } {x}_{k }\Big\| < \epsilon. \tag{*} \]
    Since \( \sum_{ k=1  }^{ \infty   } \|{x}_{k }\|  \) converges, and since \( \R  \) is complete, it follows from the Cauchy Criterion for series that there exists \( \hat{N} \) such that 
    \[ \forall n> m > \hat{N} \ \ \Big| \sum_{ k= m +1 }^{ n  } \|{x}_{k }\| \Big| < \epsilon.  \]
    We claim that \( \hat{N} \) is the same \( N  \) we were looking for. Hence, if \( n > m > \hat{N} \), then we have
    \[  \Big\| \sum_{ k=m+1 }^{ n  } {x}_{k }  \Big\| \leq \sum_{ k= m+1 }^{ n  } \| {x}_{k} \| = \Big|  \sum_{ k= m+1 }^{ n  } \|{x}_{k }\| \Big|  < \epsilon  \]
    as desired.
\end{proof}

Please take note of the following observations: 

\begin{enumerate}
    \item[(1)] If \( \sum_{ n=1  }^{ \infty  } \|{x}_{n}\| \) converges, then \( \sum_{ n=1  }^{ \infty  } {x}_{n} \) converges (in Banach spaces).
    \item[(2)] If \( \sum_{ n=1  }^{ \infty  } \|{x}_{n}\|  \) diverges, then \( \sum_{ n=1  }^{ \infty  } {x}_{n} \) may converge or diverge.
\end{enumerate}

From (2), we shall see (in the next lecture) that
\begin{enumerate}
    \item[(1)] \( \sum_{ n=1  }^{ \infty  } \Big| \frac{ (-1)^{n+1} }{ n }  \Big|   \) diverges but \( \sum_{ n=1  }^{ \infty   } \frac{ (-1)^{n+1} }{ n }   \) converges.
    \item[(2)] \( \sum_{ n=1  }^{ \infty  } | (-1)^{n} |  \) diverges, also \( \sum_{ n=1  }^{ \infty  } (-1)^{n} \) diverges (by the divergence test).
\end{enumerate}

\begin{definition}[Absolute Convergence and Conditional Convergence]
    We say that a series \( \sum_{  }^{  }{x}_{n} \) \textbf{absolutely converges} if \( \sum_{  }^{  } \|{x}_{n}\| \) converges and \( \sum_{  }^{  } {x}_{n} \) converges. We say that \( \sum_{  }^{  }{x}_{n} \) \textbf{conditionally converges} if \( \sum_{  }^{  } \|{x}_{n}\|  \) diverges but \( \sum_{  }^{  } {x}_{n} \) converges. 
\end{definition}

\begin{eg}[Conditionally Convergent]
    Consider \( \sum_{ n=1  }^{ \infty  } \frac{ (-1)^{n+1} }{ n }  \). We see that this series is conditionally convergent since  
    \[  \Big| \frac{ (-1)^{n+1} }{ n }  \Big|  = \frac{ 1 }{ n } \to 0 \]
    which tell us that the above series diverges by the divergence test. But the above series converges via the Leibniz Test (As we shall see in the next lecture).
\end{eg}

\section{Lecture 21-22}

\subsection{Topics}

\begin{itemize}
    \item Cauchy Condensation Test
    \item Comparison Test
    \item More on \( \lim \sup   \) and \( \lim \inf  \)
    \item Root Test
    \item Ratio Test
    \item Dirichlet's Test
\end{itemize}

\begin{theorem}[Cauchy Condensation Test]
    Assume \( {a}_{n} \geq 0  \) for all \( n  \), and \( ({a}_{n}) \) is a decreasing sequence. Then
    \[  \sum_{ n=1  }^{ \infty  } {a}_{n} \ \text{converges} \ \iff \ \sum_{ n=0 }^{ \infty  } 2^{n} {a}_{2^{n}} = {a}_{1} + {2a}_{2} + {4a}_{4} + {8a}_{8} + {16a}_{16} + \cdots \ \text{converges}. \]
\end{theorem} 

\begin{proof}
Let \( {s}_{m} = {a}_{1} + \cdots + {a}_{m} \) and \( {t}_{m} = {a}_{1} + {2a}_{2} + {4a}_{4} + \cdots + 2^{m-1} {a}_{2^{m-1}} \). Using the fact that \(({a}_{n})\) is a decreasing sequence, we can see that
\begin{align*}
    {s}_{2^{k}} &= {a}_{1} = {a}_{2} + ({a}_{3} + {a}_{4}) + ({a}_{5} + {a}_{6} + {a}_{7} + {a}_{8}) + \cdots + ({a}_{2^{k-1} + 1} + \cdots + {a}_{2^{k}}) \\
                &\geq {a}_{1} + {a}_{2} + ({a}_{4} + {a}_{4}) + ({a}_{8} + {a}_{8} + {a}_{8} + {a}_{8}) + \cdots + ({a}_{2^{k}} + \cdots + {a}_{2^{k}}) \\
                &= {a}_{1} + {a}_{2} + 2 {a}_{4} + {4a}_{8} + \cdots + 2^{k-1} {a}_{2^{k}} \\
                &=  {a}_{1} + \frac{ 1 }{ 2 }  \Big[ {t}_{k+1} - {a}_{1} \Big] \\
                &= {a}_{1} + \frac{ 1 }{ 2 }  {t}_{k+1} - \frac{ 1 }{ 2 }  {a}_{1} \\
                &= \frac{ 1 }{ 2 }  ({a}_{1} + {t}_{k+1}) \\  
                &\geq \frac{ 1 }{ 2 }  {t}_{k+1}. 
\end{align*}
Thus, we have  
\[ {s}_{2^{k}} \geq \frac{ 1 }{ 2 }  {t}_{k+1}. \tag{*} \]
Similarly, we have
\begin{align*}
    {s}_{2^{k} - 1} &= {a}_{1} + ({a}_{2} + {a}_{3}) + ({a}_{4} + {a}_{5} + {a}_{6} + {a}_{7}) + \cdots + ({a}_{2^{k-1}} + \cdots + {a}_{2^{k} - 1}) \\
                    &\leq {a}_{1} + ({a}_{2} + {a}_{2}) + ({a}_{4} + {a}_{4} + {a}_{4} + {a}_{4}) + \cdots + ({a}_{2^{k-1}} + \cdots + {a}_{2^{k-1}}) \\ 
                    &= {a}_{1} + {2a}_{2} + {4a}_{4} + \cdots + 2^{k-1} {a}_{2^{k-1}}  \\
                    &= {t}_{k}.
\end{align*}
Thus, we have that 
\[  {s}_{2^{k } - 1} \leq {t}_{k}. \tag{**} \]

\( (\Longleftarrow) \) Assume that \( \sum_{ n=0  }^{ \infty  } 2^{n} {a}_{2^{n}} \) converges (\( ({t}_{m}) \) converges). Our goal is to show that \( \sum_{ n=1  }^{ \infty  } {a}_{n} \) converges; that is, \( ({s}_{m}) \) converges. Note that since \( {a}_{n} \geq 0  \), both \( ({s}_{m}) \) and \( ({t}_{m}) \) are increasing sequences. It follows from the Monotone Convergence Theorem that in order to prove \( ({s}_{n}) \) converges, it suffices to show that \( ({s}_{m}) \) is bounded. 

Since \( ({t}_{m}) \) converges, we have that \( ({t}_{m}) \) is bounded. Hence, there exists \( R > 0  \) such that \( {t}_{m} \leq R  \) for all \( m \in \N \). In what follows, we will show that \( R  \) is an upper bound for \( ({s}_{m}) \) as well. Indeed, let \( m \in \N  \) be given. Choose \( k  \) large enough so that \( m < 2^{k } - 1  \), then 
\[  {s}_{m} \leq s^{2^{k} - 1} \leq {t}_{k} \leq R.  \]
Thus, for all \( m \in \N  \), \( 0 \leq {s}_{m} \leq R  \). Hence, \( ({s}_{m}) \) is bounded and so \( ({s}_{m})  \) converges by MCT.

\( (\Longrightarrow) \) Assume that \( \sum_{ n=1  }^{ \infty  } {a}_{n} \) (\( ({s}_{m}) \) converges). Our goal is to show that \( \sum_{ n=0 }^{ \infty   } 2^{n} {a}_{2^{n}} \) converges; that is, \( ({t}_{m}) \) converges.

We will prove the contrapositive: we will show that if \( ({t}_{m}) \) diverges then \( ({s}_{m}) \) diverges. Suppose \( ({t}_{m}) \) is divergent. Let \( R > )  \) be given. We will show that there is a term in the nonnegative sequence \( ({s}_{m}) \) that is larger than \( R  \). Since \( ({t}_{m}) \) diverges and \( ({t}_{m}) \) is an increasing sequence (where \( {t}_{m} \geq 0  \)), we see that \( ({t}_{m}) \) cannot be bounded above by the Monotone Convergence Theorem. Hence, there exists \( k \in \N  \) such that \( {t}_{k+1} > 2R \). Now, we have  
\[  {s}_{2^{k}} \geq \frac{ 1 }{ 2 }  {t}_{k+1} ? \frac{ 1 }{ 2 }  (2R) = R. \]
Thus, \( ({s}_{m}) \) is not bounded.
\end{proof}

\subsection{Applications of the Cauchy-Condensation Test}

\begin{eg}[P-series Test]
    Let \( p > 0  \). One can show that the sequence \( \Big({a}_{n} = \frac{ 1 }{ n^{p} }  \Big) \) is a decreasing nonnegative sequence. Prove that 
    \[  \sum_{ n=1  }^{ \infty  } \frac{ 1 }{ n^{p} }  \ \text{converges} \iff \ p >  1. \]
\end{eg}
\begin{proof}
Using the Cauchy Condensation test, we have
\begin{align*}
\sum_{ n=1  }^{ \infty  } \frac{ 1 }{ n^{p} }  \ \text{converges} &\iff \sum_{ n=0 }^{ \infty  } 2^{n} \frac{ 1 }{ (2^{n})^{p} } \ \text{converges} \\ 
                                                                  &\iff \sum_{ n=0  }^{ \infty  } \frac{ 1 }{ 2^{np-n} }  \ \text{converges} \\
                                                                  &\iff \sum_{ n=0 }^{ \infty  } \Big(  \frac{ 1  }{ 2^{p-1} }  \Big)^{n}  \ \text{converges} \\
                                                                  &\iff \Big|  \frac{ 1  }{ 2^{p-1} }  \Big| < 1 \\
                                                                  &\iff 1 < 2^{p-1} \\
                                                                  &\iff 0 < p - 1 \\
                                                                  &\iff 1 < p. 
\end{align*}
In the fourth equivalence statement, we used the fact \( \sum_{ n=0 }^{ \infty  } \Big(  \frac{ 1 }{ 2^{p-1} }  \Big)^{n} \) is a geometric series with common ratio \( \frac{ 1 }{ 2^{p-1} }  \). 
\end{proof}

\begin{eg}
    Let \( p > 0  \). One can show that the sequence \( \Big(  {a}_{n} = \frac{ 1  }{ n (\ln n )^{p} }  \Big)_{n \geq 2} \) is a decreasing nonnegative sequence. Prove that 
    \[ \sum_{ n=2 }^{ \infty  } \frac{ 1 }{ n (\ln n )^{p} }  \ \text{converges} \iff p > 1.  \]
\end{eg}
\begin{proof}
By the Cauchy Condensation Test, we see that 
\begin{align*}
    \sum_{ n=2 }^{ \infty  } \frac{ 1 }{ n (\ln n )^{p} } \ \text{converges} &\iff \sum_{ n=1  }^{ \infty  } 2^{n} \frac{ 1 }{ 2^{n} (\ln (2^{n}))^{p} } \ \text{converges} \\
                                                                             &\iff \sum_{ n=1 }^{ \infty  } \frac{ 1 }{ (n \ln 2)^{p} }  \ \text{converges} \\
                                                                             &\iff \frac{ 1 }{ (\ln 2)^{p} } \sum_{ n=1  }^{ \infty  } \frac{ 1 }{ n^{p} }  \ \text{converges} \\
                                                                             &\iff p > 1. 
\end{align*}
\end{proof}

\begin{theorem}[Comparison Test]
    Assume there exists an integer \( {n}_{0}  \) such that \( 0 \leq {a}_{n} \leq {b}_{n} \) for all \( n \geq {n}_{0} \). 
    \begin{enumerate}
        \item[(i)] If \( \sum_{ n=1  }^{ \infty  } {b}_{n} \) converges, then \( \sum_{ n=1  }^{ \infty  } {a}_{n} \) converges.
        \item[(ii)] If \( \sum_{ n=1  }^{ \infty  } {a}_{n} \) diverges, then \( \sum_{ n=1  }^{ \infty  } {b}_{n} \) diverges.
    \end{enumerate}
\end{theorem}
\begin{proof}
Notice that (ii) is just the contrapositive of (i). So, it suffices to show (i).

By the Cauchy Criterion for convergence of series, it is enough to show that 
\[  \forall \epsilon > 0 \ \exists N \in \N \ \text{such that} \ \forall n > m > N \ \ \Big| \sum_{ k=m+1 }^{ n } {a}_{k} \Big|  < \epsilon. \tag{*} \]
To this end, let \( \epsilon > 0  \) be given. Our goal is to find an \( N  \) such that (*) holds. Since \( \sum_{ n=1  }^{ \infty  } {b}_{n} \) converges, it follows from the Cauchy Criterion for series that 
\[  \exists \hat{N} \ \text{such that} \ \forall n > m > \hat{N} \ \ \Big|  \sum_{ k= m +1 }^{  } {b}_{k} \Big|  < \epsilon. \]
Let \(N = \max \{ {n}_{0}, \hat{N} \}  \). If \( k \geq {n}_{0}  \) where \( {a}_{k }, {b}_{k} \geq 0  \), we see that 
\[  \Big|  \sum_{ k=m+1 }^{ n } {a}_{k } \Big| = \sum_{ k=m+1  }^{ n } {a}_{k} \ \text{and} \ \Big| \sum_{ k=m+1 }^{ n } {a}_{k} \Big|  = \sum_{ k=m+1 }^{ n } {b}_{k}. \tag{1} \]
Furthermore, if \( k \geq {n}_{0} \), we have \( {a}_{k } \leq {b}_{k} \), we have
\[  \sum_{ k=m+1 }^{ n } {a}_{k} \leq \sum_{ k=m+1 }^{ n } {b}_{k}. \tag{2} \]
If \( n > m > N \), we see that (1) and (2) imply that  
\[  \Big| \sum_{ k=m+1 }^{ n  } {a}_{k} \Big|  = \sum_{ k= m+1 }^{ n } {a}_{k} \leq \sum_{ k= m +1 }^{ n  } {b}_{k} = \Big|  \sum_{ k=m+1 }^{ n } {b}_{k} \Big|  < \epsilon. \]


\end{proof}

\begin{eg}
    \begin{enumerate}
        \item[(*)] Does \( \sum_{ n=1  }^{ \infty  } \frac{ 1 }{ n + 5^{n } }  \) converge?

            Indeed, for all \( n \in \N \), we have 
            \[  0 \leq \frac{ 1 }{ n+5^{n} }  \leq \frac{ 1 }{ 5^{n} }. \]
            Note that \( \sum_{ n=1  }^{ \infty  } \frac{ 1 }{ 5^{n} }  \) converges (because it is a geometric series). Thus, the comparison test implies that 
\( \sum_{ n=1  }^{ \infty  } \frac{ 1 }{ n+ 5^{n} }  \) converges.

        \item[(*)] Suppose \( {a}_{n } \geq 0 \) and \( \sum_{ n=1  }^{ \infty  } {a}_{n} \) converges. Prove that \( \sum_{ n=1  }^{ \infty  } {a}_{n}^{2} \) converges. 

            Indeed, we see that 
            \[  \sum_{ n=1  }^{ \infty  } {a}_{n} \ \text{converges}  \implies \lim {a}_{n} = 0.  \]
            Hence, there exists \( {n}_{0} \in \N  \) such that for all \( n \geq {n}_{0} \), \( 0 \leq {a}_{n} < 1  \). Thus, 
            \[  \forall n \geq {n}_{0} \ \ 0 \leq {a}_{n}^{2} \leq {a}_{n}. \]
            By the comparison test, we can conclude that \( \sum_{ n=1  }^{ \infty  } {a}_{n}^{2}  \) converges.
    \end{enumerate}
\end{eg}

\begin{remark}[Some useful properties]
    Let \( ({a}_{n}) \) be a sequence of real numbers. Suppose \( \lim_{ n \to \infty  }  {a}_{n} = A \in \R  \).
    \begin{enumerate}
        \item[(i)] If \( A < \beta \), then there exists \( N \in \N  \) such that for all \( n > N  \), \( {a}_{n} < \beta \).
        \item[(ii)] If \( \alpha < A  \), then there exists \( N  \) such that for all \( n > N  \), \( \alpha < {a}_{n} \).
    \end{enumerate}
\end{remark}

\begin{theorem}[ ]
    Let \( ({a}_{n}) \) be a sequence of real numbers. 
    \begin{enumerate}
        \item[(i)] Suppose \( \beta \in \R  \) is such that \( \lim \sup  {a}_{n} < \beta\). Then
            \[  \exists N \in \N \ \text{such that} \ \forall n > N \ \ {a}_{n} < \beta. \]
        \item[(ii)] Suppose \( \alpha \in \R  \) is such that \( \lim \inf {a}_{n} > \alpha \). Then 
            \[  \exists N \in \N \ \text{such that} \ \forall n > N \ \ {a}_{n} > \alpha. \]
    \end{enumerate}
\end{theorem}
\begin{proof}
Here we will prove (i). Since \( \lim \sup {a}_{n} < \beta \), we have \( \lim \sup {a}_{n} \neq \infty  \). We may consider two cases:
\begin{enumerate}
    \item[(1)] Suppose \( \lim \sup {a}_{n} = - \infty  \). Since \( \lim \inf {a}_{n} \leq \lim \sup {a}_{n} \), we can conclude that \( \lim \inf {a}_{n} = - \infty   \). Therefore, \( \lim {a}_{n} = - \infty  \). The claim immediately follows from the definition of \( {a}_{n} \to - \infty  \).
    \item[(2)] Suppose \( A = \lim \sup {a}_{n} \). Let \( A = \lim \sup {a}_{n} \) and \( r = \frac{ \beta - A  }{  2  }  \). Since \( \lim_{ n \to \infty  }  \sup \{ {a}_{k } : k \geq n  \}  = A  \), there exists \( N  \) such that 
        \[  \forall n > N \ \ \sup \{ {a}_{k } : k \geq n  \}  < A + r. \]
        In particular, we have 
        \[  \forall n > N \ \ \sup \{ {a}_{k } : k \geq n  \}  < \beta. \]
        Therefore, we have (noticing that \( {a}_{n} \leq \sup \{ {a}_{k }: k \geq n  \}  \)), 
        \[  \forall n > N \ \ {a}_{n} < \beta. \]
\end{enumerate}
Note that the proof of (ii) is completely analogous to the proof of (i).
\end{proof}

\begin{theorem}[ ]
    Let \( ({a}_{n}) \) be a sequence of real numbers. 
    \begin{enumerate}
        \item[(i)] Suppose \( \lim \sup {a}_{n} > \beta \). Then, for infinitely many \( k  \), we have \( {a}_{k} > \beta \). That is,
            \[  \forall n \in \N \ \ \exists k \geq n \ \text{such that} \ {a}_{k} > \beta. \]
        \item[(ii)] Suppose \( \lim \inf {a}_{n} < \alpha \). Then, for infinitely many \( k  \), \( {a}_{k} < \alpha \). That is, 
            \[  \forall n \in \N \ \ \exists k \geq n \ \text{such that} \ {a}_{k} < \alpha. \]
    \end{enumerate}
\end{theorem}

\begin{proof}
Here we will prove (i) (the proof for (ii) is completely analogous). Assume for contradiction that only for finitely many \( k  \), \( {a}_{k } > \beta  \). Then there exists an \( N \in \N  \) such that for all \( k > N  \), we have \( {a}_{k } \leq \beta \). Therefore, 
\[  \lim \sup {a}_{k } \leq \lim \sup \beta = \lim \beta = \beta \]
which contradicts the assumption that \( \lim \sup {a}_{k } > \beta \).
\end{proof}

\begin{theorem}[Root Test (Version 1)]
    Let \( ({a}_{n}) \) be a sequence of real numbers. Let \( \alpha = \lim \sup \sqrt[n]{ | {a}_{n} |  }  \). 
    \begin{enumerate}
        \item[(i)] If \( \alpha < 1  \), then \( \sum_{ n=1  }^{ \infty  } {a}_{n} \) is absolutely convergent.
        \item[(ii)] If \( \alpha > 1  \), then \( \sum_{ n=1  }^{ \infty  } {a}_{n} \) diverges.
    \end{enumerate}
\end{theorem}

\begin{theorem}[Root Test (Version 2)]
    Let \( ({a}_{n}) \) be a sequence of real numbers. Assume that following limit exists:
    \[  \alpha = \lim_{ n \to \infty  }  \sqrt[n]{ | {a}_{n} |  } \]
    \begin{enumerate}
        \item[(i)] If \( \alpha < 1  \), then \( \sum_{ n=1  }^{ \infty  } {a}_{n} \) is absolutely convergent.
        \item[(ii)] If \( \alpha > 1  \), then \( \sum_{ n=1  }^{ \infty    } {a}_{n} \) diverges.
    \end{enumerate}
\end{theorem}

\begin{proof}[Version 1]
   \begin{enumerate}
       \item[(i)] Choose a number \( \beta \) such that \( \alpha < \beta < 1  \). We have \( \lim \sup \sqrt[n]{| {a}_{n} | } < \beta \). Then there exists an \( N \in \N  \) such that for all \( n > N  \), we have \( \sqrt[n]{| {a}_{n} | }  < \beta  \). Hence, we have for all \( n > N  \), \( 0 \leq | {a}_{n} |  < \beta^{n}  \) and \( \sum_{ n=1  }^{ \infty  } \beta^{n}    \) converges (it is a geometric series with common ratio \( 0 < \beta < 1  \)). As a consequence, we see that \( \sum_{ n=1  }^{ \infty  } | {a}_{n} |  \) converges by the Comparison Test.
       \item[(ii)] Choose a number \( \beta \) such that \( 1 < \beta < \alpha \). We have \( \beta < \lim \sup \sqrt[n]{ | {a}_{n} |  }  \). By Useful Theorem 2, we have for all \( n \in \N  \), we have 
           \begin{align*}
               \exists k \geq n \ \text{such that} \ \sqrt[k]{| {a}_{k} | } > \beta &\implies | {a}_{k} |  > \beta^{k} \\
                                                                                    &\implies \sup \{ | {a}_{m} |  : m \geq n  \}  > \beta^{k}.
           \end{align*}
           Since \( k \geq n  \), we have \( \beta^{k} \geq \beta^{n} \), and so
           \[  \forall n \in \N \sup \{ | {a}_{m} | : m \geq n  \} > \beta^{n}. \]
           Since \( \lim_{ n \to \infty  }  \beta^{n} = \infty  \ (\beta > 1) \), it follows from the order limit theorem (for \( \overline{\R} \)) that \( \lim_{ n \to \infty  }  \sup \{ | {a}_{m} | : m \geq n  \}  = \infty   \). So, \( \lim \sup  | {a}_{n} |  = \infty \). This tells us that \( \lim {a}_{n} \neq 0 \) ({\hyperref[Explanation of Fact]{Explanation of Fact}}). So, \( \sum_{  }^{  } {a}_{n} \) diverges by the Divergence Test.
   \end{enumerate} 
\end{proof}

\begin{remark}\label{Explanation of Fact}
    This is just the contrapositive of the following fact: 
    \begin{center}
        If \( \lim {a}_{n} = 0 \), then \( \lim | {a}_{n} |  = 0  \), so \( \lim \sup  | {a}_{n} |  = 0 \).
    \end{center}
\end{remark}

\begin{theorem}[Ratio Test (Version 1)]
    Let \( ({a}_{n}) \) be a sequence of real numbers.
    \begin{enumerate}
        \item[(i)] If \( \lim \sup \Big| \frac{ {a}_{n+1} }{ {a}_{n} }  \Big|  < 1  \), then \( \sum_{ n=1  }^{ \infty  } {a}_{n} \) converges absolutely.
        \item[(ii)] If \( \Big|  \frac{ {a}_{m+1} }{ {a}_{m} }   \Big| \geq 1  \) for all \( n \geq {n}_{0} \) (some integer \( {n}_{0} \), then \( \sum_{ n=1  }^{ \infty  } {a}_{n}\) diverges).
        \item[(iii)] If \( \lim \inf \Big|  \frac{ {a}_{n+1} }{ {a}_{n} }  \Big|  \geq  1  \), then \( \sum_{ n=1  }^{ \infty  } {a}_{n} \) diverges.
    \end{enumerate}
\end{theorem}
\begin{theorem}[Ratio Test (Version 2)]
    Let \( ({a}_{n}) \) be a sequence of real numbers. Assume that the following limit exists:
    \[  \rho = \lim_{ n \to \infty  }  \Big|  \frac{ {a}_{n+1} }{ {a}_{n} }  \Big|. \]
    \begin{enumerate}
        \item[(i)] If \( \rho < 1 \), then \( \sum_{ n=1  }^{ \infty  } {a}_{n} \) is absolutely convergent.
        \item[(ii)] If \( \rho > 1 \), then \( \sum_{ n=1  }^{ \infty  } {a}_{n} \) diverges.
    \end{enumerate}
\end{theorem}
\begin{proof}
\begin{enumerate}
    \item[(i)] Choose a number \( \beta \) such that \( \rho < \beta < 1 \). We have 
        \[  \lim_{ n \to \infty  }  \Big|  \frac{ {a}_{n+1} }{ {a}_{n} }  = \rho \implies \exists N \in \N \ \text{such that} \ \forall n \geq N \ \ \Big|  \frac{ {a}_{n+1} }{ {a}_{n} }  \Big| < \beta. \]
        Thus, we have 
        \begin{align*}
            | {a}_{N+1} | &< \beta | {a}_{N} |  \\
            | {a}_{N+2} | &< \beta | {a}_{N+1} | < \beta^{2} | {a}_{N} |  \\
            | {a}_{N+3} | &< \beta | {a}_{N+2} | < \beta^{3} | {a}_{N} | \\ 
                          &\vdots 
        \end{align*}
        So, for all \( n \in \N \), \( | {a}_{N +n}  |  < \beta^{n} | {a}_{N} |  \). Now, notice that 
        \[  \sum_{ n=1  }^{ \infty  } \beta^{n} | {a}_{N} | = | {a}_{N} | \sum_{ n=1  }^{ \infty  } \beta^{n} \ \ \text{converges}.  \]
        Now, notice that \( \sum_{ n=1  }^{ \infty  } \beta^{n} | {a}_{N} |  = | {a}_{N} |  \sum_{ n=1  }^{ \infty  } \beta^{n} \) converges (since it is a geometric series with common ratio \( 0 < \beta < 1  \)). It follows from the Comparison Test that \( \sum_{ n=1  }^{ \infty  } | {a}_{N+n} |  \) converges. Considering that \( \sum_{ n=1  }^{ \infty  } | {a}_{N+n} |  = \sum_{ n = N + 1 }^{ \infty  } | {a}_{n} |  \), we can conclude that \( \sum_{ n= N + 1 }^{ \infty  } | {a}_{n} |  \) converges. This immediately implies that \( \sum_{ n=1  }^{ \infty  } | {a}_{n} |  \) converges. 
    \item[(ii)] Choose a number \( \beta  \) such that \( 1 < \beta < \rho \). Then we have 
        \[  \lim_{ n \to \infty  }  \Big|  \frac{ {a}_{n+1} }{ {a}_{n} }  \Big|  = \rho \implies \exists N \in \N \ \text{such that} \ \forall n \geq N \ \ \Big|  \frac{ {a}_{n+1} }{  {a}_{n} }  \Big|  > \beta.   \]
        So, we have 
        \begin{align*}
            | {a}_{N+1} | &> \beta | {a}_{N} |  \\
            | {a}_{N+2} | &> \beta | {a}_{N+1} | > \beta^{2} | {a}_{N} | \\
            | {a}_{N+3} | &> \beta | {a}_{N+2} | > \beta^{3} | {a}_{N} |  \\
                          &\vdots
        \end{align*}
        Thus, for each \( n \in \N \), \( | {a}_{N +n} | > \beta^{n} | {a}_{N} |  \). Since \( \beta > 1  \), \( \lim_{ n \to \infty  }  \beta^{n} | {a}_{N} |  = \infty  \). So, \( \lim_{ n \to \infty  }  | {a}_{N+n} |  = \infty  \). Therefore, \( \lim_{ n \to \infty  }  {a}_{N+n} \neq 0 \). Thus, \( \lim_{ n \to \infty  }  {a}_{n} \neq 0  \) (because \( ({a}_{N+n})_{n \geq1} \)) is a subsequence of \( ({a}_{n})_{n \geq 1} \). So, \( \sum_{ n=1  }^{ \infty  } {a}_{n} \) diverges by the Divergence Test.   
\end{enumerate}
\end{proof}

\begin{eg}
    Let \( R \neq 0  \) be a fixed number. Prove that the series \( \sum_{ n=1  }^{ \infty  } \frac{ R^{n} }{  n!  }   \) converges. Indeed, we have
    \begin{align*}
    \rho = \lim_{ n \to \infty  }  \Big|  \frac{ {a}_{n+1} }{ {a}_{n} }  \Big|  = \lim_{ n \to \infty  }  \Big|  \frac{ \frac{ R^{n+1} }{ (n+1)! }  }{ \frac{ R^{n} }{ n! }  }  \Big|  &= \lim_{ n \to \infty  }  \Big|  \frac{ R^{n+1} n! }{ R^{n} (n+1)! }  \Big| \\
                                                                                                                                                                                       &= \lim_{ n \to \infty  }  \Big|  \frac{ R  }{ n+1 }  \Big|  \\
                                                                                                                                                                                       &=| R | \lim_{ n \to \infty  }  \frac{ 1 }{ n+1 }  \\
                                                                                                                                                                                       &= 0.
\end{align*}
Thus, if \( \rho = 0 < 1  \), then \( \sum_{ n=1  }^{ \infty  } \frac{ R^{n} }{ n! }  \) is absolutely convergent. As a consequence, we have 
\[  \lim_{ n \to \infty  }  \frac{ R^{n} }{ n! }  = 0. \]
\end{eg}

\begin{remark}
    If \( ({a}_{n}) \) is a sequence and \( \lim_{ n \to \infty  }  \Big|  \frac{ {a}_{n+1} }{ {a}_{n} }  \Big|  < 1 \), then \( \lim {a}_{n} = 0 \).
\end{remark}

\begin{theorem}[Dirichlet's Test]
    Let \( ({b}_{n}) \) be the sequence of partial sums of \( \sum_{ n=1  }^{ \infty  } {a}_{n} \) be bounded, is a decreasing sequence of nonnegative numbers (\( {b}_{1} \geq {b}_{2} \geq {b}_{3} \geq \cdots \geq 0  \)), and \( \lim_{ n \to \infty  }  {b}_{n} = 0  \). Then we have \( \sum_{ n=1  }^{ \infty  } {a}_{n} {b}_{n} \) converges.
\end{theorem}

\begin{corollary}[Leibniz Test]
    Let \( ({b}_{n}) \) be a sequence in \( \R  \). Suppose \( {b}_{1} \geq {b}_{2} \geq {b}_{3} \geq \cdots \geq 0 \) and \( \lim_{ n \to \infty  }  {b}_{n} = 0 \). Then we have \( \sum_{ n=1  }^{ \infty  } (-1)^{n+1} {b}_{n}  \) converges.
\end{corollary}

Consider the infinite sum
\[  1 - 1 + \frac{ 1 }{ 2 }  - \frac{ 1 }{ 2 }  + \frac{ 1 }{ 3 }  - \frac{ 1 }{ 3 }  + \frac{ 1 }{ 4 }  - \frac{ 1 }{ 4 }  + \cdots \   \tag{*}\]
and the following questions:
\begin{enumerate}
    \item[(1)] What is \( ({s}_{n}) \)?
    \item[(2)] What is \( \lim_{ n \to \infty  }  {s}_{n} \)?
\end{enumerate}
Define the sequence of partial sums for the series in (*) \( ({s}_{n}) \).

Consider the following partial sums 
\begin{align*}
    {s}_{1} &= 1 \\ \\
    {s}_{2} &= 1 - 1 = 0 \\
    {s}_{3} &= 1 - 1 + \frac{ 1 }{ 2 }  = \frac{ 1 }{ 2 }  \\
    {s}_{4} &= 1 - 1 + \frac{ 1 }{ 2 }  - \frac{ 1 }{ 2 } = 0  \\
    {s}_{5} &= 1 - 1 + \frac{ 1 }{ 2 }  - \frac{ 1 }{ 2 }  + \frac{ 1 }{ 3 }  = \frac{ 1 }{ 3 }  \\
            &\vdots 
\end{align*}
which establishes (1). Looking at the even subsequence of \( ({s}_{n}) \), we can see that for all \( k \in \N  \) \( {s}_{2k} = 0 \) and the odd subsequence \( {s}_{2k-1} = \frac{ 1 }{ k }  \), respectively. Clearly, we can see from these subsequences that  
\[  {s}_{2k} \to 0 \ \ \text{and} \ \ {s}_{2k-1} \to 0 \]
as \( k \to \infty  \). Hence, we can see that \( {s}_{n} \to 0  \) which establishes (2). 

\subsection{Rearrangements}

Consider the following rearrangement of (*)
\[  1 + \frac{ 1 }{ 2 }  - 1 + \frac{ 1 }{ 3 } + \frac{ 1 }{ 4 }  - \frac{ 1 }{ 2 }  + \frac{ 1 }{ 5 }  + \frac{ 1 }{ 6 }  - \frac{ 1 }{ 3 }  + \cdots = \lim_{ n \to \infty   } {s}_{n} = \ln (2).  \]

Consider the sequence of the partial sums of the above: 
\begin{align*}
    {s}_{1} &= 1  \\
    {s}_{2} &= \frac{ 3 }{ 2 }  \\
    {s}_{3} &= \frac{ 1 }{ 2 }  \\
            &\vdots \\
    {s}_{2 \times 10^{2} + 2} &\approx 0.6939 \\
    {s}_{3 \times 10^{4} + 2} &\approx 0.6932 \\
    {s}_{3 \times 10^{6} + 2} &\approx 0.6931 \\
            &\vdots 
\end{align*} 

\begin{theorem}[ ]
    If a series converges absolutely, then for any \( L \in \R  \), there exists some rearrangement of \( \sum_{ n=1  }^{ \infty  } {a}_{n} \) converges to \( L  \).
\end{theorem}




\end{document}
