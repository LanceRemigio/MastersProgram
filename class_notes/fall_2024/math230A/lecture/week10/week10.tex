\documentclass[a4paper]{article}


\usepackage[utf8]{inputenc}
\usepackage[T1]{fontenc}
% \usepackage{fourier}
\usepackage{textcomp}
\usepackage{hyperref}
\usepackage[english]{babel}
\usepackage{url}
% \usepackage{hyperref}
% \hypersetup{
%     colorlinks,
%     linkcolor={black},
%     citecolor={black},
%     urlcolor={blue!80!black}
% }
\usepackage{graphicx} \usepackage{float}
\usepackage{booktabs}
\usepackage{enumitem}
% \usepackage{parskip}
% \usepackage{parskip}
\usepackage{emptypage}
\usepackage{subcaption}
\usepackage{multicol}
\usepackage[usenames,dvipsnames]{xcolor}
\usepackage{ocgx}
% \usepackage{cmbright}


\usepackage[margin=1in]{geometry}
\usepackage{amsmath, amsfonts, mathtools, amsthm, amssymb}
\usepackage{thmtools}
\usepackage{mathrsfs}
\usepackage{cancel}
\usepackage{bm}
\newcommand\N{\ensuremath{\mathbb{N}}}
\newcommand\R{\ensuremath{\mathbb{R}}}
\newcommand\Z{\ensuremath{\mathbb{Z}}}
\renewcommand\O{\ensuremath{\emptyset}}
\newcommand\Q{\ensuremath{\mathbb{Q}}}
\newcommand\C{\ensuremath{\mathbb{C}}}
\newcommand\F{\ensuremath{\mathbb{F}}}
% \newcommand\P{\ensuremath{\mathbb{P}}}
\DeclareMathOperator{\sgn}{sgn}
\DeclareMathOperator{\diam}{diam}
\DeclareMathOperator{\LO}{LO}
\DeclareMathOperator{\UP}{UP}
\DeclareMathOperator{\card}{card}
\DeclareMathOperator{\Arg}{Arg}
\DeclareMathOperator{\Dom}{Dom}
\DeclareMathOperator{\Log}{Log}
\DeclareMathOperator{\dist}{dist}
% \DeclareMathOperator{\span}{span}
\usepackage{systeme}
\let\svlim\lim\def\lim{\svlim\limits}
\renewcommand\implies\Longrightarrow
\let\impliedby\Longleftarrow
\let\iff\Longleftrightarrow
\let\epsilon\varepsilon
\usepackage{stmaryrd} % for \lightning
\newcommand\contra{\scalebox{1.1}{$\lightning$}}
% \let\phi\varphi
\renewcommand\qedsymbol{$\blacksquare$}

% correct
\definecolor{correct}{HTML}{009900}
\newcommand\correct[2]{\ensuremath{\:}{\color{red}{#1}}\ensuremath{\to }{\color{correct}{#2}}\ensuremath{\:}}
\newcommand\green[1]{{\color{correct}{#1}}}

% horizontal rule
\newcommand\hr{
    \noindent\rule[0.5ex]{\linewidth}{0.5pt}
}

% hide parts
\newcommand\hide[1]{}

% si unitx
\usepackage{siunitx}
\sisetup{locale = FR}
% \renewcommand\vec[1]{\mathbf{#1}}
\newcommand\mat[1]{\mathbf{#1}}

% tikz
\usepackage{tikz}
\usepackage{tikz-cd}
\usetikzlibrary{intersections, angles, quotes, calc, positioning}
\usetikzlibrary{arrows.meta}
\usepackage{pgfplots}
\pgfplotsset{compat=1.13}

\tikzset{
    force/.style={thick, {Circle[length=2pt]}-stealth, shorten <=-1pt}
}

% theorems
\makeatother
\usepackage{thmtools}
\usepackage[framemethod=TikZ]{mdframed}
\mdfsetup{skipabove=1em,skipbelow=1em}

\theoremstyle{definition}

\declaretheoremstyle[
    headfont=\bfseries\sffamily\color{ForestGreen!70!black}, bodyfont=\normalfont,
    mdframed={
        linewidth=1pt,
        rightline=false, topline=false, bottomline=false,
        linecolor=ForestGreen, backgroundcolor=ForestGreen!5,
    }
]{thmgreenbox}

\declaretheoremstyle[
    headfont=\bfseries\sffamily\color{NavyBlue!70!black}, bodyfont=\normalfont,
    mdframed={
        linewidth=1pt,
        rightline=false, topline=false, bottomline=false,
        linecolor=NavyBlue, backgroundcolor=NavyBlue!5,
    }
]{thmbluebox}

\declaretheoremstyle[
    headfont=\bfseries\sffamily\color{NavyBlue!70!black}, bodyfont=\normalfont,
    mdframed={
        linewidth=1pt,
        rightline=false, topline=false, bottomline=false,
        linecolor=NavyBlue
    }
]{thmblueline}

\declaretheoremstyle[
    headfont=\bfseries\sffamily, bodyfont=\normalfont,
    numbered = no,
    mdframed={
        rightline=true, topline=true, bottomline=true,
    }
]{thmbox}

\declaretheoremstyle[
    headfont=\bfseries\sffamily, bodyfont=\normalfont,
    numbered=no,
    % mdframed={
    %     rightline=true, topline=false, bottomline=true,
    % },
    qed=\qedsymbol
]{thmproofbox}

\declaretheoremstyle[
    headfont=\bfseries\sffamily\color{NavyBlue!70!black}, bodyfont=\normalfont,
    numbered=no,
    mdframed={
        rightline=false, topline=false, bottomline=false,
        linecolor=NavyBlue, backgroundcolor=NavyBlue!1,
    },
]{thmexplanationbox}

\declaretheorem[
    style=thmbox, 
    % numberwithin = section,
    numbered = no,
    name=Definition
    ]{definition}

\declaretheorem[
    style=thmbox, 
    name=Example,
    ]{eg}

\declaretheorem[
    style=thmbox, 
    % numberwithin = section,
    name=Proposition]{prop}

\declaretheorem[
    style = thmbox,
    numbered=yes,
    name =Problem
    ]{problem}

\declaretheorem[style=thmbox, name=Theorem]{theorem}
\declaretheorem[style=thmbox, name=Lemma]{lemma}
\declaretheorem[style=thmbox, name=Corollary]{corollary}

\declaretheorem[style=thmproofbox, name=Proof]{replacementproof}

\declaretheorem[style=thmproofbox, 
                name = Solution
                ]{replacementsolution}

\renewenvironment{proof}[1][\proofname]{\vspace{-1pt}\begin{replacementproof}}{\end{replacementproof}}

\newenvironment{solution}
    {
        \vspace{-1pt}\begin{replacementsolution}
    }
    { 
            \end{replacementsolution}
    }

\declaretheorem[style=thmexplanationbox, name=Proof]{tmpexplanation}
\newenvironment{explanation}[1][]{\vspace{-10pt}\begin{tmpexplanation}}{\end{tmpexplanation}}

\declaretheorem[style=thmbox, numbered=no, name=Remark]{remark}
\declaretheorem[style=thmbox, numbered=no, name=Note]{note}

\newtheorem*{uovt}{UOVT}
\newtheorem*{notation}{Notation}
\newtheorem*{previouslyseen}{As previously seen}
% \newtheorem*{problem}{Problem}
\newtheorem*{observe}{Observe}
\newtheorem*{property}{Property}
\newtheorem*{intuition}{Intuition}

\usepackage{etoolbox}
\AtEndEnvironment{vb}{\null\hfill$\diamond$}%
\AtEndEnvironment{intermezzo}{\null\hfill$\diamond$}%
% \AtEndEnvironment{opmerking}{\null\hfill$\diamond$}%

% http://tex.stackexchange.com/questions/22119/how-can-i-change-the-spacing-before-theorems-with-amsthm
\makeatletter
% \def\thm@space@setup{%
%   \thm@preskip=\parskip \thm@postskip=0pt
% }
\newcommand{\oefening}[1]{%
    \def\@oefening{#1}%
    \subsection*{Oefening #1}
}

\newcommand{\suboefening}[1]{%
    \subsubsection*{Oefening \@oefening.#1}
}

\newcommand{\exercise}[1]{%
    \def\@exercise{#1}%
    \subsection*{Exercise #1}
}

\newcommand{\subexercise}[1]{%
    \subsubsection*{Exercise \@exercise.#1}
}


\usepackage{xifthen}

\def\testdateparts#1{\dateparts#1\relax}
\def\dateparts#1 #2 #3 #4 #5\relax{
    \marginpar{\small\textsf{\mbox{#1 #2 #3 #5}}}
}

\def\@lesson{}%
\newcommand{\lesson}[3]{
    \ifthenelse{\isempty{#3}}{%
        \def\@lesson{Lecture #1}%
    }{%
        \def\@lesson{Lecture #1: #3}%
    }%
    \subsection*{\@lesson}
    \testdateparts{#2}
}

% \renewcommand\date[1]{\marginpar{#1}}


% fancy headers
\usepackage{fancyhdr}
\pagestyle{fancy}

\makeatother

% notes
\usepackage{todonotes}
\usepackage{tcolorbox}

\tcbuselibrary{breakable}
\newenvironment{verbetering}{\begin{tcolorbox}[
    arc=0mm,
    colback=white,
    colframe=green!60!black,
    title=Opmerking,
    fonttitle=\sffamily,
    breakable
]}{\end{tcolorbox}}

\newenvironment{noot}[1]{\begin{tcolorbox}[
    arc=0mm,
    colback=white,
    colframe=white!60!black,
    title=#1,
    fonttitle=\sffamily,
    breakable
]}{\end{tcolorbox}}

% figure support
\usepackage{import}
\usepackage{xifthen}
\pdfminorversion=7
\usepackage{pdfpages}
\usepackage{transparent}
\newcommand{\incfig}[1]{%
    \def\svgwidth{\columnwidth}
    \import{./figures/}{#1.pdf_tex}
}

% %http://tex.stackexchange.com/questions/76273/multiple-pdfs-with-page-group-included-in-a-single-page-warning
\pdfsuppresswarningpagegroup=1



\begin{document}

\section{Lecture 18-19}


\subsection{Topics}

\begin{itemize}
    \item \( \lim \sup  \) and \( \lim \inf \) (Two equivalent characterizations)
    \item Theorem: \( \lim \inf {a}_{n} \leq \lim \sup {a}_{n}  \).
    \item Theorem: \( \lim_{ n \to \infty  }  {a}_{n} \) exists in \( \overline{\R} \) if and only if \( \lim \inf  {a}_{n} = \lim \sup {a}_{n} \in \overline{\R}\).
    \item Theorem: \( \lim \sup ({a}_{n} + {b}_{n}) \leq \lim \sup {a}_{n} + \lim \sup  {b}_{n}  \) provided that the right-hand side is not \( \infty  - \infty   \).
    \item Some special sequences
\end{itemize}

\subsection{First Characterization of Limsup and Liminf}

Let \( ({x}_{n}) \) be a sequence of real numbers. Let 
\[  S = \{ x \in \overline{\R }  : \text{there exists a subsequence \( ({x}_{{n}_{k }}) \) of \( ({x}_{n}) \) such that \( {x}_{{n}_{k }} \to x  \)} \}.  \]
We define,
\begin{align*}
    \lim \sup {x}_{n} &= \sup S  \\
    \lim \inf {x}_{n} &= \inf S.
\end{align*}

\subsection{Second Characterization of Limsup and Liminf}

Let \( ({x}_{n}) \) be a sequence of real numbers. For each \( n \in \N  \), let \( {F}_{n} = \{ {x}_{k } : k \geq n  \}  \). Clearly, we have 
\[  {F}_{1} \supseteq {F}_{2} \supseteq {F}_{3} \supseteq \dots \ . \]
So, 
\[  \sup {F}_{1} \geq \sup {F}_{2} \geq \sup {F}_{3} \cdots  \tag{A decreasing sequence in \( \overline{\R} \)} \]
and similarly, we have
\[  \inf {F}_{1} \leq \inf {F}_{2} \leq \inf {F}_{3} \leq \dots \ . \tag{An increasng sequence in \( \overline{\R} \)} \]
By the Monotone Convergence Theorem (in \( \overline{\R} \)), we know that \( \lim_{ n \to \infty  }  \sup {F}_{n} \) and \( \lim_{ n \to \infty  }  \inf {F}_{n} \) exists in \( \overline{\R} \). We define
\begin{align*}
   \lim \sup {x}_{n} &= \lim_{ n \to \infty  } \sup {F}_{n} \\
   \lim \inf {x}_{n} &= \lim_{ n \to \infty  }  \inf {F}_{n}.
\end{align*}
That is, we have 
\begin{align*}
    \lim \sup {x}_{n} &= \lim_{ n \to \infty  }  \sup \{ {x}_{k } : k \geq n  \}  = \inf_{n} (\sup {F}_{n}) \\
    \lim \inf {x}_{n} &= \lim_{ n \to \infty  }  \inf \{ {x}_{k } : k \geq n  \}  = \sup_{n} (\inf {F}_{n}).
\end{align*}

Take note of the following notation:
\begin{align*}
    \lim \sup  {x}_{n} &= \lim_{ n \to \infty  }  \sup {x}_{n} = \overline{\lim} {x}_{n} \\
    \lim \inf {x}_{n} &= \lim_{ n \to \infty  }  \inf {x}_{n} = \underline{\lim} {x}_{n}.
\end{align*}

\begin{eg}
    \begin{enumerate}
        \item[(i)] \( {x}_{n} = (-1)^{n} \)

            Notice that 
            \begin{align*}
                \lim \sup {x}_{n} &= \lim_{ n \to \infty  }  \sup \{ {x}_{k } : k \geq n  \} = \lim_{ n \to \infty  }  \sup \{ {x}_{n}, {x}_{n+1}, \dots  \}  = \lim_{ n \to \infty  }  \sup \{ -1,1 \}  = \lim_{ n \to \infty  } 1 = 1 \\
                \lim \inf {x}_{n} &= \lim_{ n \to \infty  }  \inf \{ {x}_{k } : k \geq n  \} = \lim_{ n \to \infty  }  \inf \{ {x}_{n} , {x}_{n+1}, \dots  \}  = \lim_{ n \to \infty  }  \inf \{ -1, 1  \} = \lim_{ n \to \infty   }  - 1 = -1. 
            \end{align*}
        \item[(ii)] Consider \( ({a}_{n}) = (-1,2,3,-1,2,3,-1,2,3,\dots) \)

            Then we have
            \begin{align*}
                \lim \sup  {a}_{n} &= \lim_{ n \to \infty  }  \sup \{ {a}_{k } : k  \geq n \} = \lim_{ n \to \infty  }  \sup \{ {a}_{n}, {a}_{n+1}, \dots  \}  = \lim_{ n \to \infty  }  \{  - 1 , 2 , 3 \}  = \lim_{ n \to \infty  } 3 =  3 \\
            \lim \inf {a}_{n} &= \lim_{ n \to \infty  }  \inf \{ {a}_{k } : k \geq n  \}  = \lim_{ n \to \infty  }  \inf \{ {a}_{n} , {a}_{n+1}, \dots  \} = \lim_{ n \to \infty  }  \inf \{ -1 , 2,3 \} = \lim_{ n \to \infty  }  - 1 = -1.
            \end{align*}
        \item[(iii)] Consider \( {a}_{n} = n  \)
            \begin{align*}
                \lim \sup  {a}_{n} = \lim_{ n \to \infty  }  \sup \{ {a}_{k } : k \geq n  \} = \lim_{ n \to \infty  }  \sup \{ {a}_{n}, {a}_{n+1}, \dots  \}  &= \lim_{ n \to \infty  }  \sup \{ n , n + 1 , n + 2, \dots  \}  \\
                                                                                                                                                              &= \lim_{ n \to \infty  } n = \infty. 
            \end{align*}
            and similarly, we have
            \begin{align*}
                \lim \inf {a}_{n} = \lim_{ n \to \infty  }  \inf \{ {a}_{k } : k \geq n  \} = \lim_{ n \to \infty  }  \inf \{ {a}_{n}, {a}_{n+1}, \dots  \} &= \lim_{ n \to \infty  }  \inf \{ n, n + 1 , n + 2 , \dots \}  \\
                &= \lim_{ n \to \infty  } n = \infty. 
            \end{align*}
    \end{enumerate}
\end{eg}

\begin{remark}
    \begin{enumerate}
        \item[(i)] \( \lim \inf {x}_{n} = \sup_{n} \inf \{ {x}_{k } : k \geq n  \} \)
        \item[(ii)] \( \lim \sup {x}_{n} = \inf_{n} \sup \{ {x}_{k } : k \geq n  \}  \)
    \end{enumerate}
\end{remark}

\begin{theorem}[ ]
    Let \( ({a}_{n}) \) be a sequence of real numbers. Then
    \[  \lim \inf {a}_{n} \leq \lim \sup {a}_{n}. \]
\end{theorem}
\begin{proof}
Notice that for all \( n \in \N  \)
\[  \inf \{ {a}_{k } : k \geq n  \}  \leq \sup \{ {a}_{k } : k \geq n  \}.  \]
Since we already proved that the limits of both sides exists (in \( \overline{\R} \)), it follows from the order limit theorem (in \( \overline{\R} \)) that 
\[  \lim_{ n \to \infty  }  \inf \{ {a}_{k } : k \geq n  \}  \leq \lim_{ n \to \infty   }  \sup \{ {a}_{k } : k \geq n  \}.  \]
That is, we have 
\[ \lim \inf {a}_{n} \leq \lim \sup {a}_{n}. \]
\end{proof}

\begin{theorem}[ ]
    Let \( ({a}_{n}) \) be a sequence of real numbers. Then
    \begin{center}
        \( \lim_{ n \to \infty  }  {a}_{n} \) exists in \( \overline{\R} \) if and only if \( \lim \sup  {a}_{n} = \lim \inf {a}_{n} \).
    \end{center}
    Moreover, in this case, \( \lim {a}_{n} = \lim \sup {a}_{n} = \lim \inf {a}_{n}\).
\end{theorem}

\begin{proof}
    (\( \Longleftarrow \)) Let \( A = \lim \sup {a}_{n} = \lim \inf {a}_{n} \) with \( A \in \overline{\R} \). In what follows, we will show that \( \lim {a}_{n} = A  \). We may consider three cases; that is,
    \begin{enumerate}
        \item[(1)] \( A \in \R  \)  
        \item[(2)] \( A = \infty  \) 
        \item[(3)] \( A = - \infty  \)
    \end{enumerate}
    For (1), note that for all \( n \in \N \)
    \[  \inf \{ {a}_{k } : k \geq n  \}  \leq {a}_{n} \leq \sup \{ {a}_{k } : k \geq n  \}.  \]
    Since \( \lim_{ n \to \infty  }  \sup \{ {a}_{k } : k \geq n  \}  = \lim_{ n \to \infty  }  \inf \{ {a}_{k } : k \geq n  \}  = A  \), it follows from the squeeze theorem that \( \lim_{ n \to \infty  }  {a}_{n} = A  \).

    For (2) (\( A = \infty  \)), we have for all \( n \in \N \) that \( \inf \{ {a}_{k } : k \geq n  \}  \leq {a}_{n} \) and \( \lim_{ n \to \infty  }  \inf \{ {a}_{k } : k \geq n  \}  = \infty  \) implies \( \lim_{ n \to \infty  } {a}_{n} = \infty  \) by the Order Limit Theorem in \( \overline{\R} \). 

    For (3) (\( A = - \infty  \)), we know that for all \( n \in \N \) that \( {a}_{n} \leq \sup \{ {a}_{k } : k \geq n  \}  \) and \( \lim_{ n \to \infty  }  \sup \{ {a}_{k } : k \geq n  \} = - \infty  \) implies that \( \lim_{ n \to \infty  }  {a}_{n} = - \infty   \) by the Order Limit Theorem in \( \overline{\R} \).
 
    (\( \Longrightarrow \)) Let \( A = \lim_{ n \to \infty  }  {a}_{n}  \) with \( A \in \overline{\R} \). In what follows, we will show that \( \lim \sup  {a}_{n} = A \) and \( \lim \inf {a}_{n} = A  \). We may consider three cases:
    \begin{enumerate}
        \item[(1)] \( A \in \R  \)
        \item[(2)] \( A = \infty  \)
        \item[(3)] \( A = - \infty  \)
    \end{enumerate}

    For (1), suppose that \( A \in \R  \). Our goal is to show that  
    \begin{center}
        \( A \leq \lim \inf {a}_{n}  \) and \( \lim \sup {a}_{n} \leq A  \),
    \end{center}
    and so
    \[  A \leq \lim \inf {a}_{n} \leq \lim \sup {a}_{n} \leq A.  \]
    Thus, it suffices to show that for all \( \epsilon > 0  \)
    \begin{center}
        \( A - \epsilon \leq \lim \inf {a}_{n} \) and \( \lim \sup {a}_{n} \leq A + \epsilon \).
    \end{center}
    To this end, let \( \epsilon > 0  \) be given. Since \( {a}_{n} \to A  \), there exists an \( N \in \N  \) such that 
    \[  \forall n > N \ \ | {a}_{n} - A  |  < \epsilon; \]
    that is, 
    \[  \forall n > N \ \ A - \epsilon < {a}_{n} < A + \epsilon. \]
    Now, observe that
    \begin{align*}
        \forall n > N \ \ {a}_{n} < A + \epsilon &\implies A + \epsilon \ \ \text{is an upper bound of} \ \{ {a}_{k } : k \geq n  \}  \\
                                                 &\implies \forall n > N \ \ \sup \{ {a}_{k } : k \geq n \}  \leq A + \epsilon \\
                                                 &\implies \lim_{ n \to \infty  }  \sup \{ {a}_{ k } : k \geq n  \}  \leq \lim_{ n \to \infty  }  (A + \epsilon) \tag{Order Limit Theorem} \\
                                                 &\implies  \lim \sup  {a}_{n} \leq A + \epsilon
    \end{align*}
    and similarly, we have 
    \begin{align*}
        \forall n > N \ \  A  - \epsilon <  {a}_{n}  &\implies A - \epsilon \ \text{is a lower bound of} \{ {a}_{k } : k \geq n  \}  \\
                                                     &\implies \forall n > N  \ \inf \{ {a}_{ k } : k \geq n  \}  \geq A - \epsilon \\
                                                     &\implies \lim_{ n \to \infty  }  \inf \{ {a}_{k } : k \geq n  \}  \geq \lim_{ n \to \infty  }  A - \epsilon \\
                                                     &\implies \lim \inf {a}_{n} \geq A - \epsilon.
    \end{align*}

    Now, suppose (2). Our goal is to show that \( \lim \inf  {a}_{n} = \infty  \) so that \( \lim \inf {a}_{n} \leq \lim \sup {a}_{n} \) will imply that \( \lim \sup  {a}_{n} = \infty  \). In order to show that \( \lim \inf {a}_{n} = \infty  \), it suffices to show that 
    \[  \forall M > 0 \ \ M \leq \lim \inf {a}_{n}. \]
    To this end, let \( M > 0  \) be given. Since \( {a}_{n} \to \infty   \), there exists \( N \in \N \) such that 
    \begin{align*}
        \forall  n  > N \ \ {a}_{n} > M &\implies \forall n > N  \  \inf \{ {a}_{k } : k \geq n  \}  \geq  M \\
                                        &\implies \lim_{ n \to \infty  }  \inf \{ {a}_{k } : k \geq n  \}  \geq \lim_{ n \to \infty   }  M \\
                                        &\implies \lim \inf {a}_{n} \geq M.
\end{align*}

Note that an analogous process to the above is used to prove (3). 
\end{proof}

\begin{theorem}[ ]
    Let \( ({a}_{n}) \) and \( ({b}_{n}) \) be the two sequences of real numbers. Then
    \[  \lim \sup ({a}_{n} + {b}_{n}) \leq \lim \sup {a}_{n} + \lim \sup {b}_{n}  \]
    provided that the right-hand side is not of the form \( \infty  - \infty   \) or \( - \infty  + \infty   \).
\end{theorem}

\begin{proof}
First note that, by our assumption, \( \lim \sup {a}_{n} + \lim \inf {a}_{n} \) is not of the form \( \infty  - \infty   \) or (\( - \infty  + \infty   \)), there exists \( {n}_{0} \) such that 
\[  \forall n \geq {n}_{0} \ \ \sup \{ {a}_{k } : k \geq n   \}  + \sup \{ {b}_{k } : k \geq n   \}  \ \text{is not of the form} \ \infty  - \infty \  \text{or }  - \infty  + \infty   \]
For each \( n \geq {n}_{0} \), we have 
\begin{align*}
    \forall k \geq n \ \  {a}_{k } &\leq \sup \{ {a}_{\ell} : \ell \geq n  \}  \\
    \forall k \geq n \ \ {b}_{k } &\leq \sup \{ {b}_{m} : m \geq n  \}.
\end{align*}
Thus, we have 
\[  \forall k \geq n \ \ {a}_{k } + {b}_{k } \leq \sup \{ {a}_{\ell} : \ell \geq n  \}  + \sup \{ {b}_{m} : m \geq n \}.  \]
Therefore, 
\[  \forall n \geq {n}_{0} \ \ \sup \{ {a}_{k } + {b}_{k } : k \geq n  \}  \leq \sup \{ {a}_{\ell} : \ell \geq n  \}  + \sup \{ {b}_{m} : m \geq n  \}.  \]

Now, label \( {R}_{n} = \sup \{ {a}_{k } + {b}_{k } : k \geq n  \}  \), \( {L}_{n} = \sup \{ {a}_{\ell } : \ell \geq n  \}  \) and \( {S}_{n} = \sup \{ {b}_{m} : m \geq n  \}  \). From the above, we can see that \( \lim_{ n \to \infty  }  {R}_{n} \), \( \lim_{ n \to \infty  } {L}_{n} \), and \( \lim_{ n \to \infty  }  {S}_{n} \) all exists in \( \overline{\R} \). Since \( \lim_{ n \to \infty  } {L}_{n} + \lim_{ n \to \infty  } {S}_{n} \) is not of the form \( \infty  - \infty   \), it follows form the Algebraic Limit Theorem that \( \lim_{ n \to \infty  }  ({L}_{n} + {S}_{n}) \) exists and is equal to that of \(\lim {L}_{n} + \lim {S}_{n} \). By the Order Limit Theorem, we see that 
\[  \lim \sup ({a}_{n} + {b}_{n}) \leq \lim \sup {a}_{n} + \lim \sup {b}_{n}. \]

\end{proof}

\begin{theorem}[(e)]
    If \( | x |  < 1  \), then \( \lim_{ n \to \infty  } x^{n} = 0  \).
\end{theorem}
\begin{proof}
Clearly, if x = 0, then the claim holds. So, let's assume \( x \in (-1,1) \) and \( x \neq 0  \). Our goal is to show that 
\[  \forall \epsilon > 0 \ \exists N \in \N \ \text{such that} \ \forall n > N \ | x^{n} - 0  |  < \epsilon. \]
That is, we need to show, given the setup above, that \( | x^{n} |  < \epsilon \). Since \( 0 < | x  |  < 1  \), there exists \( y > 0  \) such that \( | x  |  = \frac{ 1 }{  1 + y }  \). Note that 
\[  | x |^{n} < \epsilon \iff \frac{ 1 }{  (1+y)^{n}  }  < \epsilon. \]
Using the Binomial Theorem (\( (1+y)^{n} \geq 1  + ny \)), we can see that  
\[  \frac{ 1  }{ (1+y)^{n} } \leq \frac{ 1  }{  1 + ny }  < \frac{ 1 }{ n y }.  \]
Therefore, in order to ensure that \( | x |^{n} < \epsilon \), we just need to choose \( n  \) large enough so that \( \frac{ 1 }{ ny }  < \epsilon \). To this end, it suffices to choose \( n  \) larger than \( \frac{ 1  }{ \epsilon y }  \); that is, we can take \( N = \frac{ 1  }{  \epsilon y  }\) and the result follows.
\end{proof}

\begin{theorem}[(b)]
    If \( p > 0  \), then \( \lim_{ n \to \infty  } \sqrt[n]{ p } = 1    \).
\end{theorem}
\begin{proof}
If \( p = 1  \), the claim obviously holds. If \( p \neq 1  \), we may consider two cases.

For the first case, assume that \( p > 1 \). Then let \( {x}_{n} = \sqrt[n]{ p }  - 1  \). It suffices to show that \( \lim_{ n \to \infty  }  {x}_{n}  = 0 \). Note that since \( p > 1  \), \( {x}_{n} \geq 0  \). Also, we have
\begin{align*}
    \sqrt[n]{ p }  = 1 + {x}_{n} &\implies p = (1 + {x}_{n})^{n} \geq 1 + n {x}_{n}  \\
                                 &\implies {x}_{n} \leq \frac{ p - 1  }{ n }.
\end{align*}
Thus, we have 
\[  0 \leq {x}_{n} \leq \frac{ p - 1  }{ n }. \]
It follows from the squeeze theorem that \( \lim_{ n \to \infty  }  {x}_{n} = 0  \).

Now, suppose that \( 0  < p < 1  \). Since \( 0 < p < 1  \), we have \( 1 < \frac{ 1 }{ p }  \). So, by the previous case, we have 
\[  \lim_{ n \to \infty  }  \sqrt[n]{ \frac{ 1 }{p}  } = 1 \iff \lim_{ n \to \infty  }  \frac{ 1  }{  \sqrt[n]{ p }  }  = 1.    \]
\end{proof}

\begin{theorem}[(c)]
    \( \lim_{ n \to \infty  } \sqrt[n]{n} = 1  \)
\end{theorem}

\begin{proof}
    Let \( {x}_{n} = \sqrt[n]{n} - 1  \). Observe that, by the binomial formula, we have for all \( n \geq 2  \), 
    \begin{align*}
        \sqrt[n]{n} =  1 + {x}_{n} &\implies n = (1 + {x}_{n})^{n} \geq \begin{pmatrix} n \\ 2  \end{pmatrix}  {x}_{n}^{2} = \frac{ n(n-1) }{ 2  }  {x}_{n}^{2} \\
                                   &\implies \frac{ 2n  }{ n (n-1) } \geq {x}_{n}^{2} \\
                                   &\implies {x}_{n} \leq \sqrt{ \frac{ 2  }{ n - 1 }  }. 
    \end{align*}
    Thus, we have 
    \[  0 \leq {x}_{n} \leq \sqrt{ \frac{ 2  }{ n - 1  }  }. \]
    It follows from the squeeze theorem that \( {x}_{n} \to 0  \) and so \( \sqrt[n]{n} \to 1  \).
\end{proof}





\end{document}
