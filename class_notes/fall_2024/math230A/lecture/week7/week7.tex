\documentclass[a4paper]{report}
\usepackage{standalone}
\usepackage{import}
\usepackage[utf8]{inputenc}
\usepackage[T1]{fontenc}
% \usepackage{fourier}
\usepackage{textcomp}
\usepackage{hyperref}
\usepackage[english]{babel}
\usepackage{url}
% \usepackage{hyperref}
% \hypersetup{
%     colorlinks,
%     linkcolor={black},
%     citecolor={black},
%     urlcolor={blue!80!black}
% }
\usepackage{graphicx} \usepackage{float}
\usepackage{booktabs}
\usepackage{enumitem}
% \usepackage{parskip}
% \usepackage{parskip}
\usepackage{emptypage}
\usepackage{subcaption}
\usepackage{multicol}
\usepackage[usenames,dvipsnames]{xcolor}
\usepackage{ocgx}
% \usepackage{cmbright}


\usepackage[margin=1in]{geometry}
\usepackage{amsmath, amsfonts, mathtools, amsthm, amssymb}
\usepackage{thmtools}
\usepackage{mathrsfs}
\usepackage{cancel}
\usepackage{bm}
\newcommand\N{\ensuremath{\mathbb{N}}}
\newcommand\R{\ensuremath{\mathbb{R}}}
\newcommand\Z{\ensuremath{\mathbb{Z}}}
\renewcommand\O{\ensuremath{\emptyset}}
\newcommand\Q{\ensuremath{\mathbb{Q}}}
\newcommand\C{\ensuremath{\mathbb{C}}}
\newcommand\F{\ensuremath{\mathbb{F}}}
% \newcommand\P{\ensuremath{\mathbb{P}}}
\DeclareMathOperator{\sgn}{sgn}
\DeclareMathOperator{\diam}{diam}
\DeclareMathOperator{\LO}{LO}
\DeclareMathOperator{\UP}{UP}
\DeclareMathOperator{\card}{card}
\DeclareMathOperator{\Arg}{Arg}
\DeclareMathOperator{\Dom}{Dom}
\DeclareMathOperator{\Log}{Log}
\DeclareMathOperator{\dist}{dist}
% \DeclareMathOperator{\span}{span}
\usepackage{systeme}
\let\svlim\lim\def\lim{\svlim\limits}
\renewcommand\implies\Longrightarrow
\let\impliedby\Longleftarrow
\let\iff\Longleftrightarrow
\let\epsilon\varepsilon
\usepackage{stmaryrd} % for \lightning
\newcommand\contra{\scalebox{1.1}{$\lightning$}}
% \let\phi\varphi
\renewcommand\qedsymbol{$\blacksquare$}

% correct
\definecolor{correct}{HTML}{009900}
\newcommand\correct[2]{\ensuremath{\:}{\color{red}{#1}}\ensuremath{\to }{\color{correct}{#2}}\ensuremath{\:}}
\newcommand\green[1]{{\color{correct}{#1}}}

% horizontal rule
\newcommand\hr{
    \noindent\rule[0.5ex]{\linewidth}{0.5pt}
}

% hide parts
\newcommand\hide[1]{}

% si unitx
\usepackage{siunitx}
\sisetup{locale = FR}
% \renewcommand\vec[1]{\mathbf{#1}}
\newcommand\mat[1]{\mathbf{#1}}

% tikz
\usepackage{tikz}
\usepackage{tikz-cd}
\usetikzlibrary{intersections, angles, quotes, calc, positioning}
\usetikzlibrary{arrows.meta}
\usepackage{pgfplots}
\pgfplotsset{compat=1.13}

\tikzset{
    force/.style={thick, {Circle[length=2pt]}-stealth, shorten <=-1pt}
}

% theorems
\makeatother
\usepackage{thmtools}
\usepackage[framemethod=TikZ]{mdframed}
\mdfsetup{skipabove=1em,skipbelow=1em}

\theoremstyle{definition}

\declaretheoremstyle[
    headfont=\bfseries\sffamily\color{ForestGreen!70!black}, bodyfont=\normalfont,
    mdframed={
        linewidth=1pt,
        rightline=false, topline=false, bottomline=false,
        linecolor=ForestGreen, backgroundcolor=ForestGreen!5,
    }
]{thmgreenbox}

\declaretheoremstyle[
    headfont=\bfseries\sffamily\color{NavyBlue!70!black}, bodyfont=\normalfont,
    mdframed={
        linewidth=1pt,
        rightline=false, topline=false, bottomline=false,
        linecolor=NavyBlue, backgroundcolor=NavyBlue!5,
    }
]{thmbluebox}

\declaretheoremstyle[
    headfont=\bfseries\sffamily\color{NavyBlue!70!black}, bodyfont=\normalfont,
    mdframed={
        linewidth=1pt,
        rightline=false, topline=false, bottomline=false,
        linecolor=NavyBlue
    }
]{thmblueline}

\declaretheoremstyle[
    headfont=\bfseries\sffamily, bodyfont=\normalfont,
    numbered = no,
    mdframed={
        rightline=true, topline=true, bottomline=true,
    }
]{thmbox}

\declaretheoremstyle[
    headfont=\bfseries\sffamily, bodyfont=\normalfont,
    numbered=no,
    % mdframed={
    %     rightline=true, topline=false, bottomline=true,
    % },
    qed=\qedsymbol
]{thmproofbox}

\declaretheoremstyle[
    headfont=\bfseries\sffamily\color{NavyBlue!70!black}, bodyfont=\normalfont,
    numbered=no,
    mdframed={
        rightline=false, topline=false, bottomline=false,
        linecolor=NavyBlue, backgroundcolor=NavyBlue!1,
    },
]{thmexplanationbox}

\declaretheorem[
    style=thmbox, 
    % numberwithin = section,
    numbered = no,
    name=Definition
    ]{definition}

\declaretheorem[
    style=thmbox, 
    name=Example,
    ]{eg}

\declaretheorem[
    style=thmbox, 
    % numberwithin = section,
    name=Proposition]{prop}

\declaretheorem[
    style = thmbox,
    numbered=yes,
    name =Problem
    ]{problem}

\declaretheorem[style=thmbox, name=Theorem]{theorem}
\declaretheorem[style=thmbox, name=Lemma]{lemma}
\declaretheorem[style=thmbox, name=Corollary]{corollary}

\declaretheorem[style=thmproofbox, name=Proof]{replacementproof}

\declaretheorem[style=thmproofbox, 
                name = Solution
                ]{replacementsolution}

\renewenvironment{proof}[1][\proofname]{\vspace{-1pt}\begin{replacementproof}}{\end{replacementproof}}

\newenvironment{solution}
    {
        \vspace{-1pt}\begin{replacementsolution}
    }
    { 
            \end{replacementsolution}
    }

\declaretheorem[style=thmexplanationbox, name=Proof]{tmpexplanation}
\newenvironment{explanation}[1][]{\vspace{-10pt}\begin{tmpexplanation}}{\end{tmpexplanation}}

\declaretheorem[style=thmbox, numbered=no, name=Remark]{remark}
\declaretheorem[style=thmbox, numbered=no, name=Note]{note}

\newtheorem*{uovt}{UOVT}
\newtheorem*{notation}{Notation}
\newtheorem*{previouslyseen}{As previously seen}
% \newtheorem*{problem}{Problem}
\newtheorem*{observe}{Observe}
\newtheorem*{property}{Property}
\newtheorem*{intuition}{Intuition}

\usepackage{etoolbox}
\AtEndEnvironment{vb}{\null\hfill$\diamond$}%
\AtEndEnvironment{intermezzo}{\null\hfill$\diamond$}%
% \AtEndEnvironment{opmerking}{\null\hfill$\diamond$}%

% http://tex.stackexchange.com/questions/22119/how-can-i-change-the-spacing-before-theorems-with-amsthm
\makeatletter
% \def\thm@space@setup{%
%   \thm@preskip=\parskip \thm@postskip=0pt
% }
\newcommand{\oefening}[1]{%
    \def\@oefening{#1}%
    \subsection*{Oefening #1}
}

\newcommand{\suboefening}[1]{%
    \subsubsection*{Oefening \@oefening.#1}
}

\newcommand{\exercise}[1]{%
    \def\@exercise{#1}%
    \subsection*{Exercise #1}
}

\newcommand{\subexercise}[1]{%
    \subsubsection*{Exercise \@exercise.#1}
}


\usepackage{xifthen}

\def\testdateparts#1{\dateparts#1\relax}
\def\dateparts#1 #2 #3 #4 #5\relax{
    \marginpar{\small\textsf{\mbox{#1 #2 #3 #5}}}
}

\def\@lesson{}%
\newcommand{\lesson}[3]{
    \ifthenelse{\isempty{#3}}{%
        \def\@lesson{Lecture #1}%
    }{%
        \def\@lesson{Lecture #1: #3}%
    }%
    \subsection*{\@lesson}
    \testdateparts{#2}
}

% \renewcommand\date[1]{\marginpar{#1}}


% fancy headers
\usepackage{fancyhdr}
\pagestyle{fancy}

\makeatother

% notes
\usepackage{todonotes}
\usepackage{tcolorbox}

\tcbuselibrary{breakable}
\newenvironment{verbetering}{\begin{tcolorbox}[
    arc=0mm,
    colback=white,
    colframe=green!60!black,
    title=Opmerking,
    fonttitle=\sffamily,
    breakable
]}{\end{tcolorbox}}

\newenvironment{noot}[1]{\begin{tcolorbox}[
    arc=0mm,
    colback=white,
    colframe=white!60!black,
    title=#1,
    fonttitle=\sffamily,
    breakable
]}{\end{tcolorbox}}

% figure support
\usepackage{import}
\usepackage{xifthen}
\pdfminorversion=7
\usepackage{pdfpages}
\usepackage{transparent}
\newcommand{\incfig}[1]{%
    \def\svgwidth{\columnwidth}
    \import{./figures/}{#1.pdf_tex}
}

% %http://tex.stackexchange.com/questions/76273/multiple-pdfs-with-page-group-included-in-a-single-page-warning
\pdfsuppresswarningpagegroup=1




\begin{document}

\section{Lecture 12}

\subsection{Topics}

\begin{itemize}
    \item Definition of \( K- \)cell. 
    \item Theorem: If \( {I}_{1} \supseteq {I}_{2} \supseteq {I}_{3} \supseteq \dots  \) is a sequence of \( k- \)cells, then \( \bigcap_{ n = 1  }^{ \infty    }  {I}_{n} \) is nonempty.
    \item Theorem: Every \( k - \)cell is compact.
    \item Theorem: Suppose \( E \subseteq \R^{k} \). Then
    \item Connected sets
        \begin{center}
            \( E  \) is closed and bounded \( \iff  \) \( E  \) is compact \( \iff  \) Every infinite subset of \( E  \) has a limit point in \( E  \).
        \end{center}
    \item Theorem: Every bounded infinite subset of \( \R^{k} \) has a limit point in \( \R^{k } \). 
\end{itemize}

\begin{corollary}
If \( {I}_{1} \supseteq {I}_{2} \supseteq {I}_{3} \supseteq \dots  \) is a sequence of compact sets, then \( \bigcap_{ n = 1  }^{ \infty    }  {I}_{n} \) is nonempty.
\end{corollary}


\begin{theorem}[Nested Interval Property]
    If \( {I}_{n} = [{a}_{n}, {b}_{n}] \) is a sequence of closed intervals in \( \R  \) such that \( {I}_{1} \supseteq {I}_{2} \supseteq {I}_{3} \supseteq \dots  \), then \( \bigcap_{  n = 1 }^{ \infty   }  {I}_{n} \) is nonempty.
\end{theorem}

\begin{definition}[K-cell]
    The set \( I = [{a}_{1}, {b}_{1}] \times \cdots \times [{a}_{k}, {b}_{k}] \) is called a \( k- \)cell in \( \R^{k} \).  
\end{definition}
\begin{eg}
 Let \( I = [{a}_{1}, {a}_{2}]  \times [{a}_{2}, {b}_{2}] \) is a \( 2- \)cell in \( \R^{2} \).
\end{eg}

\begin{theorem}[Nested Cell Property]
   If \( {I}_{1} \supseteq {I}_{2} \supseteq {I}_{3} \cdots  \) is a nested sequence of \( k - \)cells, then \( \bigcap_{ n=1 }^{ \infty  }  {I}_{n} \neq \emptyset \). 
\end{theorem}
\begin{proof}
    For each \( n \in \N  \), let 
    \begin{align*}
        {I}_{n} = [{a}_{1}^{(n)}, {b}_{1}^{(n)}] \times \cdots \times [{a}_{k}^{(n)}, {b}_{k}^{(n)}]  \\
    \end{align*}
    Also, let 
    \begin{center}
        \( \forall n \in \N  \) and \( \forall 1 \leq i \leq k  \), we have \( {A}_{i}^{(n)} = [{a}_{i}^{(n)}, {b}_{i}^{(n)}] \)
    \end{center}
    Since for each \( n \in \N  \), \( {I}_{n} \supseteq {I}_{n+1} \), we have
    \[  {A}_{i}^{(n) }  \supseteq {A}_{i}^{(n+1)}\ \forall 1 \leq i \leq k.  \]
    That is,
    \begin{align*}
       {I}_{1} &= {A}_{1}^{(1)} \times \cdots \times {A}_{k }^{(1)} \\
       {I}_{2} &= {A}_{1}^{(2)} \times \cdots \times {A}_{k}^{(2)} \\
               &\vdots  \\
        {I}_{n} &= {A}_{1}^{(n)} \times \cdots {A}_{k}^{(n)}.
    \end{align*}
    Hence, it follows from the nested interval property that there exists 
    \begin{align*}
        \exists {x}_{1} &\in \bigcap_{ n=1  }^{ \infty  } {A}_{1}^{(n)} \\
        \exists {x}_{2} &\in \bigcap_{ i=1  }^{ \infty   }  {A}_{2}^{(n)} \\
                        &\vdots \\
        \exists {x}_{k} &\in \bigcap_{ n=1  }^{ \infty  }  {A}_{k}^{(n)}. 
    \end{align*}
    Thus, by a fact in set theory; that is,  
    \[  (A \cap B) \times (C \cap D ) \subseteq  (A \times C ) \cap (B \times D). \]
    \begin{align*}
        ({x}_{1}, \dots, {x}_{k}) &\in \Big[\bigcup_{ n=1  }^{ \infty   }  {A}_{1}^{(n)} \Big] \times \Big[\bigcap_{ n=1  }^{ \infty  } {A}_{2}^{(n)} \Big] \times \cdots \times \Big[ \bigcap_{  n = 1 }^{ \infty  }  {A}_{k}^{(n)}\Big] \\
                                  &\subseteq \bigcap_{ n=1  }^{ \infty  }  [{A}_{1}^{(n)} \times \cdots \times {A}_{k}^{(n)}] \\
                                  &= \bigcap_{ n=1  }^{ \infty  }  {I}_{n}.
    \end{align*}
    Hence, we see that 
    \[  \bigcap_{ n=1  }^{ \infty  }  {I}_{n} \neq \emptyset. \]
\end{proof}

\begin{theorem}[ ]
Every \( k- \)cell in \( \R^{k} \) is compact.    
\end{theorem}
\begin{proof}
    Here we will prove the claim for \( 2- \)cells. The proof for a general \( k- \)cell is completely analogous. Let \( I = [{a}_{1}, {b}_{1}] \times [{a}_{2}, {b}_{2}] \) be a \( 2- \)cell. Let \( a = ({a}_{1}, {a}_{2})  \) and \( b = ({b}_{1}, {b}_{2}) \). Let \[ \delta = d(a,b) = \| a - b\| = \sqrt{ | {a}_{1} - {b}_{1} |^{2} + | {a}_{2} - {b}_{2} |^{2} }  \]. Note that if \( x = ({x}_{1}, {x}_{2}) \) and \( y = ({y}_{1}, {y}_{2}) \) are any two points in \( I  \), then
    \begin{align*}
    {x}_{1}, {y}_{1} \in [{a}_{1}, {b}_{2}] &\implies | {x}_{1} - {y}_{1}  |  \leq | {b}_{1} - {a}_{1} |  \\
    {x}_{2}, {y}_{2} \in [{a}_{2}, {b}_{2}]                     &\implies  | {x}_{2} - {y}_{2} |  \leq | {b}_{2} - {a}_{2} |
\end{align*}
which implies that 
\[ \sqrt{ | {x}_{1} - {y}_{1} |^{2} + | {x}_{2} - {y}_{2} |^{2} } \leq \sqrt{ | {a}_{1} - {b}_{1} |^{2} + | {a}_{2} - {b}_{2}  |^{2} } = \delta.  \]
So, \( d(x,y) \leq \delta \).
Let us assume for contradiction that \( I  \) is NOT compact. So, there exists an open cover \( \{ {G}_{\alpha} \}_{\alpha \in \Lambda} \) of \( I  \) that does NOT have a finite subcover; that is, \( I \subseteq \bigcup_{ \alpha \in \Lambda }^{  } {G}_{\alpha} \). For each \( 1 \leq i \leq 2  \), divide \( [{a}_{i}, {b}_{i}] \) into two subintervals of equal length:
\[  {c}_{i} = \frac{ {a}_{i} + {b}_{i} }{ 2 }  \ \ [{a}_{i}, {b}_{i}] = [{a}_{i}, {c}_{i}] \cup [{c}_{i}, {b}_{i}]. \]
These subintervals determine \( 4  \) \( 2 - \)cells. There is at least one of these \( 4  \) \( 2- \)cells that is not covered by any finite subcollection of \( \{ {G}_{\alpha} \}_{\alpha \in \Lambda} \). Let us call this \( 2- \)cell as \( {I}_{1} \). Notice that 
\begin{align*}
    \forall x,y \in {I}_{1}  \ \  \|x - y\|_2 \leq \frac{ \delta }{ 2 }.
\end{align*}
Now, subdivide \( {I}_{1} \) into \( 4  \) \( 2- \)cells and continue this process inductively. In this manner, we will obtain a sequence of \( 2- \)cells 
\[ I, {I}_{1}, {I}_{2}, {I}_{3}, \dots  \]
such that 
\begin{align*}
    &I \supseteq {I}_{1} \supseteq {I}_{2} \supseteq {I}_{3} \supseteq \dots  \tag{1} \\
    &\forall x,y \in {I}_{n},  \ \ \| x - y \| \leq \frac{ \delta }{ 2 }  \tag{2} \\
    &\forall n \in \N  \ \   {I}_{n} \  \text{cannot be covered by a finite subcollection of} \ \  \{ {G}_{\alpha} \}_{\alpha \in I}
\end{align*}

\end{proof}


\begin{theorem}[Heine-Borel Theorem]
   Let \( E \subseteq \R^{k} \). The following statements are equivalent:
   \begin{enumerate}
       \item[(a)] \( E  \) is closed and bounded. 
        \item[(b)] \( E  \) is compact.
        \item[(c)] Every infinite subset of \( E  \) has a limit point of \( E  \). 
   \end{enumerate}
\end{theorem}
\begin{proof}
We will show that \( (a) \implies (b) \implies (c) \implies (a) \).

\( ((a) \implies (b) ) \) Assume that \( E  \) is closed and bounded. Our goal is to show that \( E  \) is compact. Since \( E  \) is bounded, \label{See Explanation} there exists a \( K- \)cell, \( I  \), such that \( E \subseteq I \). Note that by Theorem 2.40, we see that \( I  \) is compact. By Theorem 2.3.5, \( E  \) is compact.  

\( ((b) \implies (c)) \) Assume that \( E  \) is compact. Our goal is to show that \( E  \) is limit point compact; that is, every infinite subset of \( E  \) has a limit point in \( E  \). See proof from last week.

\( ((c) \implies (a)) \) Assume that every infinite subset of \( E  \) has a limit point in \( E  \). Our goal is to show that \( E  \) is closed and bounded. Suppose for sake of contradiction that \( E  \) is NOT bounded and NOT closed. 

Suppose that \( E  \) is NOT bounded. In what follows, we will construct a sequence of points \( {x}_{1}, {x}_{2}, \dots  \) in \( E  \). Since \( E  \) is not bounded, we know that 
\begin{align*}  E \not\subseteq {N}_{1}(0) &\implies \exists {x}_{1} \in E \ \text{such that}  \ d({x}_{1}, 0) = \|{x}_{1}\|_2 \geq 1. \\  
    E \not\subseteq {N}_{2}(0) &\implies \exists {x}_{2} \in E \ \text{such that} \ d({x}_{2}, 0) = \|{x}_{2}\|_{2} \geq 2 \\
    E \not\subseteq {N}_{3}(0) &\implies \exists {x}_{3} \in E \ \text{such that} \ d({x}_{3}, 0) = \|{x}_{3}\|_2 \geq 3 \\
                               &\vdots \\
    E \not\subseteq {N}_{n}(0) &\implies \exists {x}_{n} \in E \ \text{such that} \ d({x}_{n}, 0) = \|{x}_{n}\|_2 \geq n \\
                               &\vdots \\
\end{align*}
That is, we have a sequence of points \( S = \{ {x}_{n} : n \in \N  \}    \) in \( E  \) with the property that 
\[  d({x}_{n}, 0) = \|{x}_{n}\|_2 \geq n. \]
Note that \( S  \) is an infinite set; indeed, if \( S  \) were finite, then
\[  S = \{ {a}_{1}, \dots, {a}_{m} \}. \]
Now, let 
\[  r = \max \{ d({a}_{n},0) : 1 \leq n \leq m \} \]
and let \( n \in \N  \) be such that \( n > r + 1  \) (By the Archimedean Property). Since \( \|{x}_{n}\|_2 \geq n  > r + 1  \), we can conclude that none of the \( {a}_{1}, \dots, {a}_{m} \) is \( {x}_{n} \) and this contradicts how \( S  \) was constructed.

Now, we will show that \( E  \) is closed. Assume for contradiction that \( E  \) is NOT closed; that is, \( E' \not\subseteq E  \). That is, there exists \( {y}_{0} \in \R^{k} \) such that \( {y}_{0} \in E' \) but \( {y}_{0} \in E \). We will construct a sequence of points \( {y}_{1}, {y}_{2}, \dots  \) in \( E  \) as follows:
\begin{align*}
    {y}_{0} \in E' &\implies {N}_{1}({y}_{0}) \cap (E \setminus  \{ {y}_{0} \} ) \ \text{is infinite} \ \text{such that} \ \|{y}_{1} - {y}_{0}\| < 1  \\
    {y}_{0} \in E' &\implies {N}_{\frac{ 1 }{ 2 } }({y}_{0}) \cap (E \setminus  \{ {y}_{0} \} ) \ \text{is infinite such that} \ \| {y}_{2} - {y}_{0} \| < \frac{ 1 }{ 2 }  \\
                   &\vdots \\
    {y}_{0} \in E' &\implies {N}_{\frac{ 1 }{ m } }({y}_{0}) \cap (E \setminus  \{ {y}_{0} \} ) \ \text{is infinite such that} \ \| {y}_{m} - {y}_{0}\| < \frac{ 1 }{ m } \\
                   &\vdots
\end{align*}
Let \( T = \{ {y}_{1}, {y}_{2}, {y}_{3}, \dots  \}  \) and note that \( T \subseteq E  \). Note \( T  \) is infinite (by construction \( {y}_{1}, {y}_{2}, {y}_{3}, \dots  \) are distinct elements. We claim that if \( z \neq {y}_{0} \), then \( z \in T' \). To this end, we have for all \( n \in \N  \)
        \[ d({y}_{0}, z) = \|{y}_{0} - z \|_2 \leq \|{y}_{0} - {y}_{n}\|_2 + \|{y}_{n} - z \|_2 \]
        which implies further that 
        \[  \|{y}_{n} - z \|_2 \geq \|{y}_{0} - z \|_2 - \|{y}_{n} - {y}_{0}\|_2 > \|{y}_{0} - z\|_2 - \frac{ 1 }{ n }. \]
        Hence, for all \( n \in \N  \) with \( \frac{ 1 }{ n }  < \frac{ 1 }{ 2 }  \|{y}_{0} - z \|_2 \), we have 
        \begin{align*}
            d({y}_{n},z)&> \|{y}_{0} - z \|_2 - \frac{ 1 }{ n }  \\
                        &> \|{y}_{0} - z\|_2 - \frac{ 1 }{ 2 }  \|{y}_{0} - z \|_2 \\
                        &= \frac{ 1 }{ 2 }  \|{y}_{0} - z \|_2. 
        \end{align*}
        So, for all but finitely many \( n  \), we see that 
        \[  d({y}_{n},z) > \frac{ 1 }{ 2 }  \|{y}_{0} - z \|_2. \]
        Hence, if we let \( \epsilon = \frac{ 1 }{ 4 }  \|{y}_{0} - z\|_2 \), then \( {N}_{\epsilon}(z) \cap T  \) is a finite set which proves \( z \notin T' \). 
        But this implies that the only possible limit point of \( T  \) is \( {y}_{0} \), but \( {y}_{0} \notin E  \). Hence, \( T  \) is an infinite subset of \( E  \) with not limit point in \( E  \). This contradicts our hypothesis that every infinite subset of \( E  \) has a limit point of \( E  \).
\end{proof}

\begin{remark}
    Note that in any general metric space, we have \( (a) \implies (b) \) is not necessarily true.
\end{remark}

\begin{theorem}[Bolzano-Weierstrass Theorem]
 Let \( E \subseteq \R^{k }  \) and \( E  \) is an infinite set and bounded. Then \( E' \neq \emptyset \).   
\end{theorem}
\begin{proof}
Suppose that \( E  \) is bounded. Then there exists a \( k- \)cell \( I  \) such that \( E \subseteq  I  \). By Theorem 2.40, we know that \( I  \) is a compact set. Furthermore, we know that \( I  \) is limit point compact by Theorem 2.41. So, every infinite set in \( I  \) has a limit point in \( I  \). In particular, \( E  \) has a limit point in \( I  \). So, \( E \neq \emptyset \).
\end{proof}

\section{Lecture 13}

\subsection{Topics}

\begin{itemize}
    \item Separated sets, disconnected sets, connected set.
    \item Theorem: \( E \subseteq \R   \) is connected if and only if \( x,y \in E  \) and \( z \in (x,y) \) implies \( z \in E  \).
    \item Perfect Sets
    \item Theorem: \( P \subseteq \R^{k} \) is nonempty perfect implies \( P  \) is uncountable.
    \item The Cantor Set
\end{itemize}


\begin{definition}[Connected Sets, Disconnected, connected]
    Let \( (X,d) \) be a metric space. 
    \begin{enumerate}
        \item[(i)] Two sets \( A,B \subseteq X   \) are aid to be disjoint if \( A \cap B = \emptyset \).
        \item[(ii)] Two sets \( A,B \subseteq X  \) are said to be \textbf{separated} if \( \overline{A} \cap B  \) and \( A \cap \overline{B}  \) are both empty.
        \item [(iii)] A set \( E \subseteq X   \) is said to \textbf{disconnected} if it can be written as a union of tow nonempty separated sets \( A  \) and \( B  \); that is, \( E = A \cup B  \).
        \item[(iv)] A set \( E \subseteq X  \) is said to be connected if it is NOT disconnected.
    \end{enumerate}
\end{definition}

\begin{eg}[\( \R  \) with the standard metric]
\begin{enumerate}
    \item[(*)] If we have \( A = (1,2)  \) and \( B = (2,5) \) are separated, then
        \begin{align*}
            \overline{A} \cap B &= [1,2] \cap (2,5) = \emptyset \\
            A \cap \overline{B} &= (1,2) \cap [2,5] = \emptyset.
        \end{align*}
        Hence, \( E = A \cup B  \) is disconnected.
    \item[(*)] We have \( C = (1,2] \) and \( D = (2,5) \) are disjoint but not separated; that is, we have
        \begin{align*}
            C \cap \overline{D} &= (1,2] \cap [2,5] = \{ 2 \}  \\
            C \cup D &= (1,5) \ \text{is indeed connected.}
        \end{align*}
\end{enumerate}    
\end{eg}

\begin{theorem}[ ]
    Let \( E \subseteq \R \). \( E  \) is connected if and only if \( E  \) contains the following property
    \begin{center}
        If \( x,y \in E  \) and \( x < z < y  \), then \( z \in E  \).
    \end{center}
\end{theorem}
\begin{proof}

\end{proof}

\begin{remark}[Proposition 3.3.5, "Differential Calculus on Normed Space", Cartan]
    Let \( U  \) be an open set in a normed (real) vector space. The following conditions are equivalent:
    \begin{enumerate}
        \item[(a)] \( U \) is connected
        \item[(b)] \( U  \) is path connected (any two points in \( U  \) can be connected by a path in \( U  \))
    \end{enumerate}
\end{remark}

\begin{remark}[Chapter 4, "Introduction" to Topological Manifolds", John Lee]
   In any metric space,  
   \begin{center}
       path connected \( \implies \) connected.
   \end{center}
   However, the converse is not always true!
\end{remark}

\begin{definition}[Perfect Set]
    Let \( (X,d) \) be a metric space. Let \( E \subseteq  X  \). The following are equivalent definitions:
    \begin{enumerate}
        \item[(i)] \( E  \) is said to be \textbf{perfect} if \( E' = E \) 
        \item[(ii)] \( E  \) is said to be \textbf{perfect} if \( E' \subseteq E   \) and \( E \subseteq  E' \). 
        \item[(iii)] \( E  \) is said to be perfect if \( E  \) is closed and every point of \( E  \) is a limit point of \( E  \).
        \item[(iv)] \( E  \) is said to be perfect if \( E  \) is closed and \( E  \) does not have any isolated points.
    \end{enumerate}
\end{definition}

\begin{eg}
    \begin{itemize}
        \item \( E = [0,1] \implies E' = [0,1] \). Thus, we have \( E = E' \) and so \( E  \) is perfect.
        \item \( E = [0,1] \cup \{ 2 \} \implies  \) \( 2  \) is an isolated point of \( E  \) \( \implies  \) \( E  \) is NOT perfect.
        \item \( E = \Big\{ \frac{ 1 }{ n } : n \in \N   \Big\} \implies E' = \{ 0  \} \implies  E \neq E' \), so \( E  \) is not perfect. 
        \item Is \( E' \) from the last example perfect? Indeed, \( E' = \{ 0  \}  \implies (E')' = \emptyset\). Thus, \( E' \neq (E')' \implies E' \) is NOT perfect.
        \item \( E = \emptyset \) and \( E' = \emptyset \). Thus, \( E = E' \) so \( E  \) is perfect.
    \end{itemize}  
\end{eg}

\begin{theorem}[ ]
    Let \( P  \) be a nonempty perfect set in \( \R^{k} \). Then \( P \) is uncountable.
\end{theorem}

Our proof of the theorem above will use the following two lemmas:

\begin{lemma}
   Let \( (X,d) \) be a metric space. Let \( E \subseteq  X  \) be perfect. If \( V \) is any open set in \( X  \) such that \( V \cap E \neq \emptyset \), then \( V \cap E  \) is an infinite set.  
\end{lemma}
\begin{proof}
Let \( q \in V \cap E  \). Thus, \( q \in V  \) and \( q \in E  \). Then \( q \in V  \) implies that there exists \( \delta > 0  \) such that \( {N}_{\delta}(q) \subseteq  V   \) and \( q \in E  \) implies \( q \in E' \). As a consequence of these two results, we see that \( {N}_{\delta}(q) \cap E  \) is an infinite set. Thus, \( V \cap E  \) is an infinite set (Here, we are using the fact that an open set intersected with a perfect set is infinite whenever the intersection is nonempty). 
\end{proof}
\begin{lemma}
    Let \( q \in \R^{k} \). Let \( r > 0  \). Then \(  \overline{{N}_{r}(q)} = {C}_{r}(q). \)
\end{lemma}
\begin{proof}
    Note that since \( P' = P  \) and \( P \neq \emptyset \), we have \( P' \neq \emptyset \). Thus, \( P  \) is infinite. Assume for contradiction that \( P  \) is countable. Let's denote the distinct elements of \( P  \) by \( {x}_{1}, {x}_{2}, {x}_{3}, \dots \); that is, we can denote  
    \[  P = \{ {x}_{1}, {x}_{2}, {x}_{3}, \dots \}. \]
    In what follows, we will construct a sequence of neighborhoods \( {V}_{1}, {B}_{2}, {V}_{3}, \dots  \) such that
    \begin{enumerate}
        \item[(i)] For all \( n \in \N  \), we have \( \overline{{V}_{n+1}} \subseteq  {V}_{n} \).
        \item[(ii)] For all \( n \in \N  \), \( {x}_{n} \notin \overline{{V}_{n+1}} \).
        \item[(iii)] For all \( n \in \N  \), \( {V}_{n} \cap P \neq \emptyset \).
    \end{enumerate}
    Let's assume that we have constructed these neighborhoods ({\hyperref[Construction of neighborhoods]{Construction of neighborhoods}} ). Then for each \( n \in \N  \), let 
    \[  {K}_{n} = \overline{{V}_{n}} \cap P \neq \emptyset. \]
   Note that  
   \begin{enumerate}
       \item[(I)] \( \overline{{V}_{n+1}} \subseteq  {V}_{n} \subseteq \overline{{V}_{n}} \) so \( \overline{{V}_{n+1}} \cap P \subseteq  \overline{{V}_{n}} \cap P  \) and so \( {K}_{n+1} \subseteq  {K}_{n} \) for each \( n  \). 
        \item[(II)] Since \( \overline{{V}_{n}} \) is a closed and bounded set in \( \R^{k} \), we have that \( \overline{{V}_{n}} \) is compact. Furthermore, \( P \) being a perfect set implies that \( P \) is a closed set. As a consequence of these two facts, we can conclude that \( {K}_{n} = \overline{{V}_{n}} \cap P  \) is compact. 
   \end{enumerate}
   Using facts (I) and (II), we can conclude that 
   \[  \bigcap_{ n=1  }^{ \infty    } {K}_{n} \neq \emptyset \tag{*} \]
   by Theorem 2.3.6. Recall that for all \( n  \), we have \( {K}_{n} \subseteq  P  \), and so we have 
   \[  \bigcap_{ n = 1  }^{ \infty  } {K}_{n} \subseteq P. \]
   In what follows, we will contradict (*). Let \( b \in P  \) be arbitrary. Then \( b = {x}_{m} \) for some \( m \in \N  \). By property (ii), we see \( {x}_{m} \notin \overline{{V}_{m+1}} \) and so \( {x}_{m} \notin \overline{{V}_{m+1}} \cap P = {K}_{m+1}  \). This tells us that 
   \[  \bigcap_{ n = 1  }^{ \infty  }  {K}_{n} = \emptyset.  \]
\end{proof}
\begin{remark}[On the construction of neighborhoods \( V_1, V_2, \dots \)]\label{Construction of neighborhoods}
   Fix \( {r}_{1} > 0  \). Let \( {V}_{1} = {N}_{{r}_{1}}({x}_{1})  \). Clearly, \( {V}_{1} \cap P \neq \emptyset \) (because \( {x}_{1} \in {V}_{1}  \) and \( {c}_{1} \in P \)). Our goal is to construct an open neighborhood \( {V}_{2} \) such that 
   \begin{enumerate}
       \item[(i)] \( \overline{{V}_{2}} \subseteq {V}_{1} \),
       \item[(ii)] \( {x}_{1} \notin \overline{{V}_{2}} \),
        \item[(iii)] \( {V}_{2} \cap P \neq \emptyset \).
   \end{enumerate}
   We can do this just by using the fact that \( {V}_{1} \cap P \neq \emptyset \). By the first lemma found above, there exists \( {y}_{1} \in {V}_{1} \cap P  \) such that \( {y}_{1} \neq {x}_{1} \). Since \( {V}_{1}  \) is open and \( {y}_{1} \in {V}_{1} \), there exists \( {\delta}_{1} > 0  \) such that \( {N}_{{\delta}_{1}} ({y}_{1}) \subseteq  {V}_{1} \). 
   \begin{center}
        Let \( {r}_{2} = \frac{ 1 }{ 2 }  \min \{ d({x}_{1}, {y}_{1}), {\delta}_{1} \}.  \) 
   \end{center}
   Let \( {V}_{2} = {N}_{{r}_{2}}({y}_{1}) \). We claim \( {V}_{2} \) has all the desired properties above. Indeed, we see that 
   \begin{enumerate}
       \item[(i)] Observe that \begin{align*}
           \overline{{V}_{2}} = \overline{{N}_{{r}_{2}}({y}_{1})} &= \{ z \in \R^{k} : \| z - {y}_{1}\|_2 \leq {r}_{2} \} \\
                                                                  &\subseteq \{ z \in \R^{k} : \|z - {y}_{1}\|_2 < {\delta}_{1} \} = {N}_{{\delta}_{1}}({y}_{1}) \\ 
                                                                  &\subseteq {V}_{1}.
       \end{align*}
    \item[(ii)] Notice that \( d({x}_{1}, {y}_{1}) > {r}_{2} \) implies that 
        \[  {x}_{1} \notin \overline{{N}_{{r}_{2}}({y}_{1})} = \{ z \in \R^{k} : \| z - {y}_{1} \|_2 \leq {r}_{2} \}.  \]
    \item[(iii)] Since \( {y}_{1} \in {V}_{2}  \) and \( {y}_{1} \in P  \), we clearly have that \( {V}_{2} \cap P  \neq \emptyset \).
   \end{enumerate}
   We can construct \( {V}_{3}, {V}_{4}, \dots \) in a similar manner.
\end{remark}

As a consequence of the theorem above, we have

\begin{corollary}
    The interval \( [0,1]  \) is uncountable.
\end{corollary}

\subsection{The Cantor Set}

The construction of the Cantor set is typically done in stages.

\subsubsection{Step 0:}

Let \( {E}_{0} = [0,1] \).

\subsubsection{Step 1:}
Remove the segment \( \Big(  \frac{ 1 }{ 3 } , \frac{ 2 }{ 3 }  \Big)  \), that is, remove the middle third, and define
\[  {E}_{1} = \Big[ 0 , \frac{ 1 }{ 3 } \Big] \cup \Big[ \frac{ 2 }{ 3 } , 1 \Big]. \] The middle third, in this case, will be calculated by the following
\[  \Big(  \frac{ 1 }{ 3 } , \frac{ 2 }{ 3 }  \Big) = \Big(  \frac{ 3(0) + 1  }{ 3^{2} } , \frac{ 3 (0) + 2  }{ 3^{2} }  \Big). \]

\subsubsection{Step 2:}

Take each of the intervals \( \Big[0 , \frac{ 1 }{ 3 } \Big]  \) and \( \Big[ \frac{ 2 }{ 3 } , 1 \Big] \) and remove the middle third of each of those, and define 
\[  {E}_{2} = \Big[ 0 , \frac{ 1 }{ 9 } \Big] \cup \Big[ \frac{ 2 }{ 9 }, \frac{ 3 }{ 9 } \Big] \cup \Big[ \frac{ 6 }{ 9 } , \frac{ 7 }{ 9 } \Big] \cup \Big[ \frac{ 8 }{ 9 }  , 1 \Big].  \]
Likewise, we can remove the middle third move explicitly by calculating the following
\[  \Big(  \frac{ 3(0) + 1  }{ 3^{2} } , \frac{ 3(0) + 2  }{ 3^{2} }  \Big) \ \ \text{and} \ \ \Big(  \frac{ 3(2) + 1  }{ 3^{2} } , \frac{ 3(2) + 2  }{ 3^{2} }  \Big). \]

Continue in this manner until we obtain a sequence of compact sets:
\[ {E}_{1}, {E}_{2}, {E}_{3}, {E}_{4}, \dots \]
with the following properties
\begin{enumerate}
    \item[(1)] \( {E}_{1} \supseteq {E}_{2} \supseteq {E}_{3} \supseteq {E}_{4} \ \cdots  \)
    \item[(2)] For each \( n \in \N  \), \( {E}_{n} \) is the union of \( 2^{n} \) intervals of length \( \frac{ 1 }{ 3^{n} }  \).
\end{enumerate}

The set \( P = \bigcap_{ n=1  }^{ \infty    }  {E}_{n} \) is called the \textbf{Cantor Set}.

\begin{remark}
    Notice that in order to obtain \( {E}_{n} \), we remove intervals of the form \( \Big(  \frac{ 3 k + 1  }{ 3^{n} } , \frac{ 3 k + 2  }{ 3^{n} }  \Big) \) from \( {E}_{n-1} \); that is, \( k  \) is such that \( 0 \leq k  \) and \( 3 k + 2 < 3^{n} \).
\end{remark}

\begin{theorem}[Properties of the Cantor Set]
    Let \( P  \) denote the Cantor set. Then
    \begin{enumerate}
        \item[(1)] \( P \) is compact
        \item[(2)] \( P  \) is nonempty
        \item[(3)] \( P  \) contains no segment
        \item[(4)] \( P  \) is perfect (and so it is uncountable)
        \item[(5)] \( P  \) has measure zero.
    \end{enumerate}
\end{theorem}

\begin{proof}
\begin{enumerate}
    \item[(1)] Note that \( P  \) is an intersection of compact sets. Hence, \( P  \) is compact (see hw5).
    \item[(2)] It follows from Theorem 2.3.6 that the intersection of a sequence of nested nonempty compact sets is nonempty. (In fact, the endpoint of each interval that appears at any state belong to \( P \))
    \item[(3)] Our goal is to show that \( P  \) does NOT contain any set of the form \( (\alpha, \beta) \) (where \( 0 \leq \alpha, \beta \leq 1 \)). Note that, by the construction of \( P  \), the intervals of the form:
        \[  {I}_{k,n} = \Big(  \frac{ 3k+1 }{  3^{n} } , \frac{ 3k + 2  }{ 3^{n} }  \Big) \  n \in \N, \\ 3k + 2 < 3^{n}  \]
        have no intersection with \( P  \). However, \( (\alpha, \beta) \) contains at least one of \( {I}_{k,n} \)'s. Indeed, \( (\alpha, \beta) \) contains \( \Big(  \frac{ 3k+1  }{ 3^{n} } , \frac{ 3k+2  }{ 3^{n} }  \Big) \). That is,  
\end{enumerate}
\end{proof}





\end{document}

