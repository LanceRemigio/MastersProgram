\documentclass[a4paper]{report}
\usepackage{standalone}
\usepackage{import}
\usepackage[utf8]{inputenc}
\usepackage[T1]{fontenc}
% \usepackage{fourier}
\usepackage{textcomp}
\usepackage{hyperref}
\usepackage[english]{babel}
\usepackage{url}
% \usepackage{hyperref}
% \hypersetup{
%     colorlinks,
%     linkcolor={black},
%     citecolor={black},
%     urlcolor={blue!80!black}
% }
\usepackage{graphicx} \usepackage{float}
\usepackage{booktabs}
\usepackage{enumitem}
% \usepackage{parskip}
% \usepackage{parskip}
\usepackage{emptypage}
\usepackage{subcaption}
\usepackage{multicol}
\usepackage[usenames,dvipsnames]{xcolor}
\usepackage{ocgx}
% \usepackage{cmbright}


\usepackage[margin=1in]{geometry}
\usepackage{amsmath, amsfonts, mathtools, amsthm, amssymb}
\usepackage{thmtools}
\usepackage{mathrsfs}
\usepackage{cancel}
\usepackage{bm}
\newcommand\N{\ensuremath{\mathbb{N}}}
\newcommand\R{\ensuremath{\mathbb{R}}}
\newcommand\Z{\ensuremath{\mathbb{Z}}}
\renewcommand\O{\ensuremath{\emptyset}}
\newcommand\Q{\ensuremath{\mathbb{Q}}}
\newcommand\C{\ensuremath{\mathbb{C}}}
\newcommand\F{\ensuremath{\mathbb{F}}}
% \newcommand\P{\ensuremath{\mathbb{P}}}
\DeclareMathOperator{\sgn}{sgn}
\DeclareMathOperator{\diam}{diam}
\DeclareMathOperator{\LO}{LO}
\DeclareMathOperator{\UP}{UP}
\DeclareMathOperator{\card}{card}
\DeclareMathOperator{\Arg}{Arg}
\DeclareMathOperator{\Dom}{Dom}
\DeclareMathOperator{\Log}{Log}
\DeclareMathOperator{\dist}{dist}
% \DeclareMathOperator{\span}{span}
\usepackage{systeme}
\let\svlim\lim\def\lim{\svlim\limits}
\renewcommand\implies\Longrightarrow
\let\impliedby\Longleftarrow
\let\iff\Longleftrightarrow
\let\epsilon\varepsilon
\usepackage{stmaryrd} % for \lightning
\newcommand\contra{\scalebox{1.1}{$\lightning$}}
% \let\phi\varphi
\renewcommand\qedsymbol{$\blacksquare$}

% correct
\definecolor{correct}{HTML}{009900}
\newcommand\correct[2]{\ensuremath{\:}{\color{red}{#1}}\ensuremath{\to }{\color{correct}{#2}}\ensuremath{\:}}
\newcommand\green[1]{{\color{correct}{#1}}}

% horizontal rule
\newcommand\hr{
    \noindent\rule[0.5ex]{\linewidth}{0.5pt}
}

% hide parts
\newcommand\hide[1]{}

% si unitx
\usepackage{siunitx}
\sisetup{locale = FR}
% \renewcommand\vec[1]{\mathbf{#1}}
\newcommand\mat[1]{\mathbf{#1}}

% tikz
\usepackage{tikz}
\usepackage{tikz-cd}
\usetikzlibrary{intersections, angles, quotes, calc, positioning}
\usetikzlibrary{arrows.meta}
\usepackage{pgfplots}
\pgfplotsset{compat=1.13}

\tikzset{
    force/.style={thick, {Circle[length=2pt]}-stealth, shorten <=-1pt}
}

% theorems
\makeatother
\usepackage{thmtools}
\usepackage[framemethod=TikZ]{mdframed}
\mdfsetup{skipabove=1em,skipbelow=1em}

\theoremstyle{definition}

\declaretheoremstyle[
    headfont=\bfseries\sffamily\color{ForestGreen!70!black}, bodyfont=\normalfont,
    mdframed={
        linewidth=1pt,
        rightline=false, topline=false, bottomline=false,
        linecolor=ForestGreen, backgroundcolor=ForestGreen!5,
    }
]{thmgreenbox}

\declaretheoremstyle[
    headfont=\bfseries\sffamily\color{NavyBlue!70!black}, bodyfont=\normalfont,
    mdframed={
        linewidth=1pt,
        rightline=false, topline=false, bottomline=false,
        linecolor=NavyBlue, backgroundcolor=NavyBlue!5,
    }
]{thmbluebox}

\declaretheoremstyle[
    headfont=\bfseries\sffamily\color{NavyBlue!70!black}, bodyfont=\normalfont,
    mdframed={
        linewidth=1pt,
        rightline=false, topline=false, bottomline=false,
        linecolor=NavyBlue
    }
]{thmblueline}

\declaretheoremstyle[
    headfont=\bfseries\sffamily, bodyfont=\normalfont,
    numbered = no,
    mdframed={
        rightline=true, topline=true, bottomline=true,
    }
]{thmbox}

\declaretheoremstyle[
    headfont=\bfseries\sffamily, bodyfont=\normalfont,
    numbered=no,
    % mdframed={
    %     rightline=true, topline=false, bottomline=true,
    % },
    qed=\qedsymbol
]{thmproofbox}

\declaretheoremstyle[
    headfont=\bfseries\sffamily\color{NavyBlue!70!black}, bodyfont=\normalfont,
    numbered=no,
    mdframed={
        rightline=false, topline=false, bottomline=false,
        linecolor=NavyBlue, backgroundcolor=NavyBlue!1,
    },
]{thmexplanationbox}

\declaretheorem[
    style=thmbox, 
    % numberwithin = section,
    numbered = no,
    name=Definition
    ]{definition}

\declaretheorem[
    style=thmbox, 
    name=Example,
    ]{eg}

\declaretheorem[
    style=thmbox, 
    % numberwithin = section,
    name=Proposition]{prop}

\declaretheorem[
    style = thmbox,
    numbered=yes,
    name =Problem
    ]{problem}

\declaretheorem[style=thmbox, name=Theorem]{theorem}
\declaretheorem[style=thmbox, name=Lemma]{lemma}
\declaretheorem[style=thmbox, name=Corollary]{corollary}

\declaretheorem[style=thmproofbox, name=Proof]{replacementproof}

\declaretheorem[style=thmproofbox, 
                name = Solution
                ]{replacementsolution}

\renewenvironment{proof}[1][\proofname]{\vspace{-1pt}\begin{replacementproof}}{\end{replacementproof}}

\newenvironment{solution}
    {
        \vspace{-1pt}\begin{replacementsolution}
    }
    { 
            \end{replacementsolution}
    }

\declaretheorem[style=thmexplanationbox, name=Proof]{tmpexplanation}
\newenvironment{explanation}[1][]{\vspace{-10pt}\begin{tmpexplanation}}{\end{tmpexplanation}}

\declaretheorem[style=thmbox, numbered=no, name=Remark]{remark}
\declaretheorem[style=thmbox, numbered=no, name=Note]{note}

\newtheorem*{uovt}{UOVT}
\newtheorem*{notation}{Notation}
\newtheorem*{previouslyseen}{As previously seen}
% \newtheorem*{problem}{Problem}
\newtheorem*{observe}{Observe}
\newtheorem*{property}{Property}
\newtheorem*{intuition}{Intuition}

\usepackage{etoolbox}
\AtEndEnvironment{vb}{\null\hfill$\diamond$}%
\AtEndEnvironment{intermezzo}{\null\hfill$\diamond$}%
% \AtEndEnvironment{opmerking}{\null\hfill$\diamond$}%

% http://tex.stackexchange.com/questions/22119/how-can-i-change-the-spacing-before-theorems-with-amsthm
\makeatletter
% \def\thm@space@setup{%
%   \thm@preskip=\parskip \thm@postskip=0pt
% }
\newcommand{\oefening}[1]{%
    \def\@oefening{#1}%
    \subsection*{Oefening #1}
}

\newcommand{\suboefening}[1]{%
    \subsubsection*{Oefening \@oefening.#1}
}

\newcommand{\exercise}[1]{%
    \def\@exercise{#1}%
    \subsection*{Exercise #1}
}

\newcommand{\subexercise}[1]{%
    \subsubsection*{Exercise \@exercise.#1}
}


\usepackage{xifthen}

\def\testdateparts#1{\dateparts#1\relax}
\def\dateparts#1 #2 #3 #4 #5\relax{
    \marginpar{\small\textsf{\mbox{#1 #2 #3 #5}}}
}

\def\@lesson{}%
\newcommand{\lesson}[3]{
    \ifthenelse{\isempty{#3}}{%
        \def\@lesson{Lecture #1}%
    }{%
        \def\@lesson{Lecture #1: #3}%
    }%
    \subsection*{\@lesson}
    \testdateparts{#2}
}

% \renewcommand\date[1]{\marginpar{#1}}


% fancy headers
\usepackage{fancyhdr}
\pagestyle{fancy}

\makeatother

% notes
\usepackage{todonotes}
\usepackage{tcolorbox}

\tcbuselibrary{breakable}
\newenvironment{verbetering}{\begin{tcolorbox}[
    arc=0mm,
    colback=white,
    colframe=green!60!black,
    title=Opmerking,
    fonttitle=\sffamily,
    breakable
]}{\end{tcolorbox}}

\newenvironment{noot}[1]{\begin{tcolorbox}[
    arc=0mm,
    colback=white,
    colframe=white!60!black,
    title=#1,
    fonttitle=\sffamily,
    breakable
]}{\end{tcolorbox}}

% figure support
\usepackage{import}
\usepackage{xifthen}
\pdfminorversion=7
\usepackage{pdfpages}
\usepackage{transparent}
\newcommand{\incfig}[1]{%
    \def\svgwidth{\columnwidth}
    \import{./figures/}{#1.pdf_tex}
}

% %http://tex.stackexchange.com/questions/76273/multiple-pdfs-with-page-group-included-in-a-single-page-warning
\pdfsuppresswarningpagegroup=1



\usepackage{fancyhdr}
\pagestyle{fancy}

\begin{document}

\section{Lecture 6}

\subsection{A few examples of Metrics}

\begin{eg}
    Consider \( (\R, d) \) where \( d: \R \times \R \to [0,\infty )  \) is defined by   
    \[  d(x,y) = \frac{ | x - y |  }{ 1 + | x - y |  }.  \]
    This is a metric on \( (\R, d) \). \textbf{Prove this on homework!}
\end{eg}

\begin{remark}
    If \( (X,D) \) is a metric space, then \( (X,D)  \) is also a metric space where
    \[  D(x,y) = \frac{ d(x,y) }{ 1 + d(x,y) }. \]
\end{remark}

We can define a metric that will always be less than or equal to \( 1 \).

\begin{eg}[Taxi Cab Metric]
    Consider \( (\R^{2}, d) \) where \( d: \R^{2} \times \R^{2} \to [0,\infty)  \) is defined by
    \[  d((a,b), (x,y)) = | a - x  |  + | b - y |   \]
    We want to show that this is a metric. 
    \begin{enumerate}
        \item[(i)] For all \( (a,b), (x,y) \in \R^{2} \), we have \( d((a,b),(x,y)) = | a - x  |  + |  b - y |  \geq 0  \) by property of the absolute value \( | \cdot |  \).
        \item[(ii)] For all \( (a,b), (x,y) \in \R^{2} \), we have 
            \begin{align*}
            d((a,b), (x,y)) = 0 &\iff | a - x  | + | b - y | = 0   \\
                                &\iff |  a - x  |  = 0 \  \text{and} \ | b - y |  = 0 \\
                                &\iff a - x = 0 \   \text{and} \  b - y = 0  \\
                                &\iff (a,b) = (x,y).
        \end{align*}
    \item[(iii)] For all \( (a,b), (x,y) \in \R^{2} \). 
        \[ d((a,b), (x,y)) = | a - x   | + |  b - y |  = |  x - a  |  + |  y - b  | = d((x,y), (a,b)).  \]
    \item[(iv)] For all \( (a,b), (x,y), (t,s) \in \R^{2}  \), we want to show that 
        \[  d((a,b), (x,y)) \leq d((a,b), (t,s)) + d((t,s), (x,y)). \]
        We have 
        \begin{align*}
            d((a,b),(t,s)) + d((t,s), (x,y)) &= | a - t |  + | b - s |  + | t - x  |  + | s -y  |  \\
                                             &= (| a - t |  + | t - x  | ) + (| b - s |  + |  s  -y  | ) \\
                                             &\geq | a - x  |  + | b - y | \\
                                             &= d((a,b), (x,y))
        \end{align*}
    \end{enumerate}
\end{eg}

\begin{eg}[Discrete Metric]
    Consider \( X \to \text{any nonempty set}  \). Consider \( d: X \times X \to [0,\infty ) \),   
    \[  d(x,y) = 
    \begin{cases}
        1 &\text{if} \  x \neq y  \\
        0 &\text{if} \ x = y
    \end{cases}. \]
    We want to show that this is a metric on \( (X, \text{any non-empty set}) \).
    \begin{enumerate}
        \item[(i)] For all \( x,y \in X  \), we have \( d(x,y) \geq 0  \) because either the \( d(x,y)  = 1  \) or \( d(x,y) = 0  \).
        \item[(ii)] Let \( x,y \in X  \). Clearly, we have \( x = y  \) if and only if \(  d(x,y) = 0 \) by definition of \( d(x,y) \).
        \item[(iii)] Let \( x,y \in X  \). Clearly, we have \( d(x,y) = d(y,x) \) by definition.  
        \item[(iv)] Let \( x,y,z \in X  \). We want to show that 
            \[  d(x,y) \leq d(x,z) + d(z,y) \]
            Let us consider two cases:
            \begin{enumerate}
                \item[(1)] \( x = y \). Thus, both sides of the triangle inequality clearly hold since \( d(x,y) = 0  \).
                \item[(2)] \( x \neq y  \). In this case, we have \( d(x,y) = 1  \). So, we need to show that 
                    \[  d(x,z) + d(z,y) \geq 1. \]
                    Since \( x \neq y  \), at least one of the statements \( z \neq y  \) or \( z \neq x  \) is true. If \( z \neq x  \), then \( d(z,x ) = 1  \) and so 
                    \[  d(x,z) + d(z,y) =  1 + d(z,y) \geq  1.  \]
                    If \( z \neq y  \), then \( d(z,y) = 1  \), and so
                    \[  d(x,z) + d(z,y) = d(x,z) + 1 \geq 1.  \]
            \end{enumerate}
    \end{enumerate}
\end{eg}


\begin{eg}
    Consider \( (V, \| \cdot \|) \to \text{any normed vector space} \) with the metric \( d: V \times V \to [0,\infty ) \) and \( d(x,y) = \| x - y \| \).
    We wan to show that this is a metric on \( V  \). 
    \begin{enumerate}
        \item[(i)] Let \( x,y \in V  \). By the property of the norm, we have \( d(x,y) = \| x - y \| \geq 0  \).
        \item[(ii)] Let \( x,y \in V  \). By the 2nd property of the norm, we have 
            \begin{align*}
                d(x,y ) = 0 &\iff \| x - y \| = 0  \\
                            &\iff x - y = 0 \\
                            &\iff x = y. 
            \end{align*}
        \item[(iii)] Let \( x,y \in V  \). We have
            \[  d(x,y) = \| x - y \| = \| - (y - x)\| = |  - 1 | \| y - x \| = \| y - x \| = d(y,x).  \]
        \item[(iv)] Let \( x,y,z \in V  \). We want to show that 
            \[  d(x,y) \leq d(x,z) + d(z,y). \]
            We have,
            \begin{align*}
                d(x,z) + d(z,y) &= \|x - z \| + \|z - y\| \\
                                &\geq \| (x-z) + (z-y) \| \\
                                &= \|x  - y\| \\
                                &= d(x,y).
            \end{align*}
    This example illustrates how every time we have a norm, we immediately have a metric space. But the other around is not necessarily true!
    \end{enumerate}
\end{eg}

\begin{eg}[Standard Distance in \( \R^n \)]
    Consider \( (\R^{n}, d) \) where \( d: \R^{n} \times \R^{n} \to \R \) is defined by 
    for all \( x,y \in \R^{n} \) with \( x = ({x}_{1}, {x}_{2}, \dots, {x}_{n}) \) and \( y = ({y}_{1}, {y}_{2}, \dots, {y}_{n}) \), we have 
    \[  d(x,y) = \sqrt{ | {x}_{1} - {y}_{1} |^{2} + \cdots + | {x}_{n} - {y}_{n} |^{2} }  \]
Note that if we define for all \( x \in \R^{n} \), we have
\[  \|x \|_2 = \sqrt{ | {x}_{1} |^{2} + \cdots + | {x}_{n} |^{2} }.   \]
Then 
\[  d(x,y) = \| x - y\|_2. \]
Proving that the above is a norm is enough to show that the original in question is a metric.  

Let \( x,y \in \R^{n} \).

\begin{enumerate}
    \item[(i)] We have 
        \[  \|x \|_2 = \sqrt{ | {x}_{1} |^{2} + \cdots + | {x}_{n} |^{2}  }  \geq 0. \]
    \item[(ii)] We have 
        \begin{align*}
\|{x}_{2} \|_2 = 0 &\iff \sqrt{ | {x}_{1} |^{2} + \cdots + | {x}_{n} |^{2}  }  = 0  \\
                   &\iff | {x}_{1} |^{2} + \cdots + | {x}_{n} |^{2} = 0 \\ 
                   &\iff |  {x}_{1} |  = 0, | {x}_{2} | = 0, \dots,  | {x}_{n} |  = 0 \\
                   &\iff {x}_{1} = 0, {x}_{2} = 0, \dots, {x}_{n} = 0. 
        \end{align*}
        Thus, we have \( x = 0  \).
    \item[(iii)] For all \( \alpha \in \R  \), we have
        \begin{align*}
            \| \alpha x  \|_2 &= \sqrt{ (\alpha {x}_{1})^{2} + \cdots + (\alpha {x}_{n})^{2}  }  \\
                            &= \sqrt{ \alpha^{2} ({x}_{1}^{2} + \cdots + {x}_{n}^{2}) } \\
                            &= | \alpha |  \sqrt{ {x}_{1}^{2} + \cdots + {x}_{n}^{2} }  \\
                            &= | \alpha | \| x\|_2.
        \end{align*}
    \item[(iv)] Now, we want to show that 
        \[  \|x + y\|_2 \leq \| x\|_2 + \|y\|_2.   \]
        That is, we want to show that
        \begin{align*}
            \sqrt{ ({x}_{1} + {y}_{1})^{2} + \cdots + ({x}_{n} + {y}_{n})^{2} } &\leq \sqrt{ {x}_{1}^{2} + \cdots + {x}_{n}^{2} }  + \sqrt{ {y}_{1}^{2} + \cdots + {y}_{n}^{2} }.
        \end{align*}
        \textbf{We will show this later!}
\end{enumerate}
\end{eg}

\begin{eg}
   Consider \( (\R^{n} ,d) \) where \( d: \R^{n} \times \R^{n} \to \R  \) is defined by 
   for all \( x,y \in \R^{n} \), we have
   \[  {d}_{p}(x,y) = \Big[|  {x}_{1} - {y}_{1} |^{p } + \cdots + | {x}_{n} - {y}_{n} |^{p} \Big]^{\frac{ 1 }{ p } }. \]
\end{eg}

\subsection{Inequalities}

In what follows we will list several key inequalities that can be used in proving that, for example, a given expression is a metric or a norm.
\begin{enumerate}
    \item[(1)] The triangle inequality for the standard norm in \( \R  \). We have for all \( x,y \in \R  \), we have 
        \[  |  x + y  |  \leq | x  |  + | y |. \]
        More generally, we have for all \( {x}_{1}, \dots, {x}_{n} \in \R  \), we have
        \[  \Big| \sum_{ i=1  }^{ n } {x}_{i} \Big|  \leq \sum_{ i=1  }^{ n } | {x}_{i} |. \]
    \item[(2)] Reverse triangle inequality for the standard norm in \( \R \); that is, we have for all \( x,y \in \R  \), 
        \[  | |  x  |  - | y |  |  \leq |  x  - y  |. \]
    \item[(3)] For all \( a,b \geq 0  \) and for all \( \rho > 0  \), we have 
        \[  ab \leq \frac{ 1 }{ 2 }  \Big(\rho a^{2} + \frac{ 1 }{ \rho }  b^{2} \Big). \]
        Note that we are dealing with real numbers here!
    \item[(4)] Cauchy-Schwarz Inequality. For al \( x,y \in \R^{n} \), we have
        \[  |  {x}_{1} {y}_{1} + \cdots + {x}_{n}{y}_{n}  |  \leq \Big(  \sqrt{  {x}_{1}^{2} + \cdots + {x}_{n}^{2} }  \Big) \Big(  \sqrt{ {y}_{1}^{2} + \cdots + {y}_{n}^{2} }  \Big) \tag{*}.  \]
\end{enumerate}

\section{Lecture 7}

\subsection{Topics}

\begin{itemize}
    \item Review of Inequalities
    \item Neighborhood of a point
    \item Limit point of a set, isolated point of a set
    \item Closed set
    \item Interior point of a set
    \item Open set
    \item Bounded set
    \item Closure
    \item Dense
\end{itemize}

\subsection{Inequalities}
    
\subsection{Minkowski}
    We wan to show the triangle inequality for \( \|\cdot\|_2  \) in \( \R^{n} \); that is, we want to show that
    \begin{prop}
         Let \( \|\cdot\|_2 \) be a norm in \( \R^{n} \). Then for all \( x,y \in \R^{n} \), \( \| x + y \|_2 \leq \|x\|_2 + \|y\|_2   \).
    \end{prop}
    \begin{proof}
        We see that \( x \cdot y  \leq | x \cdot y  |  \leq \|x\|_2 \|y\|_2 \) by the Cauchy Cauchy-Schwarz Inequality. Thus, we have
        \begin{align*}
            \|x + y \|_2^{2} &= (x +y) \cdot (x +y) \\
                             &= x \cdot x + x \cdot y + y \cdot x + y \cdot y \\
                             &= \|x\|_2^{2} + 2 x \cdot y + \|y\|_2^{2} \\
                             &\leq \|x\|_2^{2} + 2 \|x\|_2 \|y\|_2 + \|y\|_2^{2} \\
                             &= (\|x\|_2 + \|y\|_2 )^{2}. 
    \end{align*}
    Hence, we have 
    \[  \|x + y\|_2^{2} \leq (\|x\|_2 + \|y\|_2)^{2}.  \]
    Therefore, we have
    \[  \|x  + y\|_2 \leq \|x\|_2 + \|y\|_2. \]
    \end{proof}

\subsection{Minkowski for General p}
For general \( p  \), we have Holder's Inequality which is 
\[  \|x +y\|_p \leq \|x\|_p + \|y\|_p \]
for any fixed real number \( p \geq 1  \) and for any \( x,y \in \R^{n} \).

\subsection{\( (x+1)^n \)}
Recall that for all \( a,b \in \R  \), for all \( n \in \N  \), we have
\[  (a+b)^{n} = \sum_{ k = 0  }^{  n  } \begin{pmatrix} n \\ k  \end{pmatrix} a^{k } b^{n - k }. \]
So, in particular, if \( x \geq 0  \) and \( n \in \N  \), then
\begin{align*}  (x+1)^{n} &= \sum_{ k = 0  }^{ n } \begin{pmatrix} n \\ k  \end{pmatrix}  x^{k } 1^{n-k} \\ &= \sum_{ k = 0  }^{ n  } \begin{pmatrix} n \\ k  \end{pmatrix}  x^{k}  \\
                          &= \begin{pmatrix} n \\ 0  \end{pmatrix}  x^{0} + \begin{pmatrix} n \\ 1  \end{pmatrix}  x^{1} + \cdots + \begin{pmatrix} n \\ n  \end{pmatrix} x^{n} \\
                          &\geq 1 + nx.
\end{align*}
Hence, we have for all \( x \geq 0  \) and for all \( n \in \N  \), we have
\[  (x+1)^{n} \geq 1 + nx.  \]

\subsection{\( p- \)means}
Let \( {x}_{1}, \dots, {x}_{n}  \) be positive real numbers. Let \( p \in \N \cup \{ 0  \}  \). By the \( p- \)mean of \( {x}_{1}, \dots, {x}_{n} \) denoted by \( {A}_{p}({x}_{1}, \dots, {x}_{n}) \), we mean
\begin{align*}
    {A}_{p}({x}_{1}, \dots, {x}_{n}) &= 
    \begin{cases}
        \frac{ \sqrt[p]{  x^{p}_1 + \cdots + {x}_{n}s^{p} }{ n }  } &\text{if} \ p \neq 0 \\
        \sqrt[n]{{x}_{1} \dots {x}_{n} } &\text{if} \  p = 0. 
    \end{cases}.
\end{align*}
For example, if \( p = 1  \), we have
\[  {A}_{1}({x}_{1}, {x}_{2}, \dots, {x}_{n}) = \frac{ {x}_{1} + {x}_{2} + \cdots + {x}_{n}  }{ n }   \]
which is the Arithmetic Mean. If \( p = 2  \), we have
\[  {A}_{2}({x}_{1}, \dots, {x}_{n}) = \sqrt{ \frac{ {x}_{1}^{2} + \cdots + {x}_{n}^{2} }{ n }  }.  \]
If \( p = 0  \), we have
\[  {A}_{0}({x}_{1}, \dots, {x}_{n}) = \sqrt[n]{ {x}_{1} \dots {x}_{n}  } \]
which is the geometric mean. It can be shown that
\[  {A}_{0}({x}_{1}, \dots, {x}_{n}) \leq {A}_{1}({x}_{1}, \dots, {x}_{n}) \leq {A}_{2}({x}_{1}, \dots, {x}_{n}) \leq \dots \ .  \]
In particular, we have \( {A}_{0} \leq {A}_{1} \); that is, 
\[  \sqrt[n]{{x}_{1} \dots {x}_{n}  } \leq \frac{ {x}_{1} + \cdots + {x}_{n} }{ n } \tag{AM-GM Inequality}. \]
We can prove this inequality using differentiation. Without differentiation, we can prove the same inequality using \textbf{Cauchy Induction}.

\subsection{Jensen's Inequality}

Suppose \( f: (a,b) \to \R  \) is a convex function (\( f"(x) \geq 0  \) for all \( x \in (a,b) \)). Let \( {x}_{1}, \dots, {x}_{n}  \) be points in \( (a,b) \). Let \( {\lambda}_{1}, \dots, {\lambda}_{n} \geq 0  \) such that \( {\lambda}_{1} + \cdots + {\lambda}_{n} = 1  \). Then   
\begin{align*}
    f({\lambda}_{1} {x}_{1} + \cdots + {\lambda}_{n} {x}_{n}) &= {\lambda}_{1} f({x}_{1}) + \cdots + {\lambda}_{n} f({x}_{n}).
\end{align*}
We want to show this inequality holds for \( n = 2  \); that is, 
\[  f({\lambda}_{1} {x}_{1} + {\lambda}_{2} {x}_{2}) \leq {\lambda}_{1} f({x}_{1}) + {\lambda}_{2} f({x}_{2}) \]
where \( {\lambda}_{1} + {\lambda}_{2} = 1  \). Note that 
\[  f((1 - {\lambda}_{2}){x}_{1} + {\lambda}_{2} {x}_{2}) \leq (1 - {\lambda}_{2})f({x}_{2}) + {\lambda}_{2} f({x}_{2}). \]

\begin{remark}
    If we want to have an expression that defines a function that gives us any number in between two points \( e<h  \), we can have
    \[ f(\lambda) =  (1-\lambda)e + \lambda h   \]
    for any \( 0 \leq \lambda \leq 1  \).
\end{remark}

\subsection{Neighborhood of a point}

\begin{definition}[Neighborhood]
   Let \( (X,d) \) be a metric space. Let \( p \in X  \). For any \( \epsilon > 0  \), we call  
   \begin{align*}
       {N}_{\epsilon}(p) &= \{ x \in X : d(p,x) < \epsilon \}.  
   \end{align*}
   the \textbf{neighborhood of \( p \) of radius \( \epsilon \)}. 
\end{definition}

\begin{eg}
    Let \( (\R, d) \) and \( d(x,y) = | x - y  |  \). The neighborhood of any \( p \in \R  \) with radius \( \epsilon > 0  \) is
    \[  {N}_{\epsilon}(p) = \{ x \in \R : d(x,p) < \epsilon \}  = \{ x \in \R : |  x - p  |  < \epsilon \}.  \]
    Note that \( | x - p  |  < \epsilon  \) is the same thing as \( p - \epsilon < x < p + \epsilon  \) or that \( x \in (p - \epsilon, p + \epsilon) \).
\end{eg}

\begin{eg}
    Let \( (\R^{2}, d) \) with \( d((a,b), (x,y)) = \sqrt{ (a-x)^{2} + (b-y)^{2} }  \).
    Let \( (a,b) \in \R^{2} \) with \( \epsilon > 0  \). Then
    \begin{align*} 
    {N}_{\epsilon}((a,b)) &= \{ (x,y) \in \R^{2} : d((x,y)(a,b)) < \epsilon \}  \\
                          &= \{ (x,y) \in \R^{2} : \sqrt{ (x-a)^{2} + (y-b)^{2} } < \epsilon \} \\
                          &= \{ (x,y) \in \R^{2} : (x-a)^{2} + (y-b)^{2} < \epsilon^{2}. \} 
\end{align*}
Thus, \( {N}_{\epsilon}((a,b))  \) consists of the points inside the circle of radius \( \epsilon  \) centered in \( (a,b) \).
\end{eg}

\begin{eg}
    \( (\R^{2}, d) \) with \( d((a,b), (x,y)) = | a - x  |  + |  b - y  |  \) and let \( \epsilon = 1  \). We have
    \begin{align*}
    {N}_{1}((0,0)) &= \{ (x,y) \in \R^{2} : d((x,y),(0,0)) < 1  \}  \\
                &= \{ (x,y) \in \R^{2} : | x - 0  |  + | y - 0  |  < 1  \} \\
                   &= \{ (x,y) \in \R^{2} : | x  |  + | y  |  < 1  \}.
\end{align*}
If we graph this out in \( \R^{2} \), then the shape of the neighborhood will take on a rhombus.
\end{eg}

\begin{eg}
   Let \( (\R, d) \) with the discrete metric
   \[  d(x,y) = 
   \begin{cases}
       1 &\text{if} \ x \neq y \\
       0 &\text{if} \ x =  y 
   \end{cases}.  \]
   Let \( p \in \R  \). Let \( \epsilon > 0  \). Let us consider two cases:
   \begin{enumerate}
       \item[(1)] Let \( \epsilon \leq 1  \). Note that if
           \[  d(x,p) < \epsilon \leq 1, \]
           then \( d(x,p) < 1  \), and so \( d(x,p) = 0  \). Hence, \( x = p  \). Then the neighborhood is 
           \[  {N}_{\epsilon}(p) = \{ x \in \R : d(x,p) < \epsilon \}  = \{ p \}. \]
        \item[(2)] Let \( \epsilon > 1  \). Clearly, for all \( x \in \R  \), we have \( d(x,p) \leq 1 < \epsilon. \)
            So,
            \[  {N}_{\epsilon}(p) = \{ x \in \R : d(x,p) < \epsilon \}  = \R.  \]
   \end{enumerate}
\end{eg}

\subsection{Limit Points}

\begin{definition}[Limit Points, Isolated Points]
    Let \( (X,d) \) is a metric space with \( E \subseteq X  \). Then we call
   \begin{enumerate}
       \item[(1)] A point \( p \in X  \) is said to be a \textbf{limit point of \( E  \)} if for all \( \epsilon > 0  \), 
           \[  {N}_{\epsilon}(p) \cap (E \setminus  \{ p \} ) \neq \emptyset. \]
       \item[(2)] The \textbf{collection of all the limit points of \( E  \)} is denoted by \( E' \); that is,
           \[  E' = \{ p \in X : \text{for all} \ \epsilon > 0, {N}_{\epsilon}(p) \cap (E \setminus  \{ p \} ) \neq \emptyset \}.  \]
        \item[(3)] A point \( p \in E  \) is said to be an \textbf{isolated point of \( E  \)} if \( p  \) is NOT a limit point; that is,  \( p \in E  \) but \( p \notin E' \). Another way of saying this is \( E \setminus  E' \). 
        \item[(4)] If \( p  \) is NOT a limit point, we have \( p \notin E' \) if and only if there exists \( \epsilon > 0  \) such that 
    \[  {N}_{\epsilon}(p) \cap (E \setminus  \{ p \} ) = \emptyset. \]
\end{enumerate} 
\end{definition}

\begin{remark}
    The statement for all \( \epsilon > 0 \), \( {N}_{\epsilon}(p) \) is equivalent to for all \( {N}_{\epsilon}(p) \).
\end{remark}

\begin{eg}
    Let \( (\R, d) \) with \( d(x,y) = | x - y |  \). Note that \(  0 \notin E  \). Also, recall that \( 0 \in E' \) if and only if for all \( \epsilon > 0  \), \( {N}_{\epsilon}(0) \cap (E \setminus  \{ 0  \} ) \neq \emptyset \). If \( 0 \in E' \), we just need to show that for all \( \epsilon > 0  \), \( {N}_{\epsilon}(0) \cap E \neq \emptyset \); that is, we need to show that  
    \begin{center}
        for all \( \epsilon > 0  \), \( (-\epsilon, \epsilon) \cap E \neq \emptyset \).
    \end{center}
    Let \( \epsilon > 0  \). By the Archimedean Property of \( \R  \), there exists \( m \in \N  \) such that \( \frac{ 1 }{ m }  < \epsilon  \). Clearly, \( 1/m \in (-\epsilon, \epsilon ) \cap E  \).
\end{eg}

\begin{eg}
   Let \( (\R, d) \) with \( d(x,y) = | x - y  |  \) and 
   \[  E = (1,2) \cup \{ 5  \}.  \]
   Prove that \( 5  \) is an isolated point.
   Since \( 5 \in E  \), it is enough to show that \( 5  \) is not in \( E' \). Recall that \( 5 \in E^{'}  \) if and only if there exists an \( \epsilon > 0  \) such that   
   \[  {N}_{\epsilon}(5) \cap (E \setminus  \{ 5 \} ) = \emptyset. \]
   Noticing that \( E \setminus  \{ 5  \} = (1,2) \), we can write \( 5 \notin E' \) if and only if there exists an \( \epsilon > 0  \) such that \( (5 - \epsilon , 5 + \epsilon) \cap (1,2) = \emptyset \). Clearly, \( \epsilon = 1  \) does the job and we are done; that is,
   \[  (5 - \epsilon, 5 + \epsilon) \cap (1,2) = (4, 6) \cap (1,2) = \emptyset. \]
\end{eg}

\begin{eg}
   Let \( (\R^{2}, d) \) and \( d((a,b),(x,y)) = \sqrt{ (a-x)^{2} + (b-y)^{2} }   \) 
   and 
   \[  E = \{ (x,y) \in \R^{2} : x^{2} + y^{2} < 4   \}. \]
   What is \( E' \)?
   \[  E' = \{ (x,y) \in \R^{2} : x^{2} + y^{2} \leq 4.  \}  \]
   For example, if \( (a,b) \) is such that \( a^{2} + b^{2} > 4  \), then \( (a,b) \notin E' \). Let \( \delta = \frac{ 1 }{ 2 }  (\sqrt{ a^{2} + b^{2} } - 2 ) \). Clearly, \( {N}_{\delta} \cap (E \setminus  \{ p \}  ) = \emptyset \).
\end{eg}

\begin{definition}[Closed Set]
    Let \( (X,d)  \) be a metric space, \( E \subseteq  X  \). We say that \( E  \) is \textbf{closed} if every limit point of \( E  \) is contained within \( E  \); that is,
    \begin{center}
        \( E  \) is closed \( \Longleftrightarrow  \)  \( E' \subseteq  E  \).
    \end{center}
\end{definition}

\begin{eg}
    Let \( (\R,d) \), \( d(x,y) = | x - y  |  \) and \( E = \{ 1,2,3 \}  \).
    \begin{enumerate}
        \item[(i)] What is \( E' \)?
          
            \textbf{Claim:} \( E' = \emptyset \). Let \( p \in \R  \). Our goal is to show that \( p  \) is not in \( E' \). That is, we want to show that there exists an \( \epsilon >  0  \) such that \( {N}_{\epsilon}(p) \cap (E \setminus  \{ p \} )  = \emptyset \)
          where \( {N}_{\epsilon}(p) = (p - \epsilon, p + \epsilon) \).

          We may consider the following cases:
          \begin{enumerate}
              \item[(1)] If \( p < 1  \). Let \( \epsilon = \frac{ 1 - p }{ 2 }  \) works. 
               \item[(2)] If \( p > 3  \), we have \( \epsilon = \frac{ p - 3  }{ 2  }   \) works.
                \item[(3)] If \( p \in \{ 1,2,3 \}  \), then \( \epsilon = \frac{ 1 }{ 4 }  \) works.
                \item[(4)] If \( 1 < p < 2  \), then let \( \epsilon = \frac{ 1 }{ 2 }  \min \{ p - 1 , 2 - p \}   \) works. 
            \item[(5)] If \( 2 < p < 3   \), then \( \epsilon = \frac{ 1 }{ 2 } \{  \min p - 2 , 3 -p \}   \) works.
          \end{enumerate}
        \item[(ii)] Is \( E  \) closed?
            Since \( E' = \emptyset \), we have \( E' \subseteq E  \) and so \( E  \) is closed.
    \end{enumerate}
\end{eg}

\begin{remark}
    Any finite set is closed!
\end{remark}

\begin{definition}[Interior Point]
   Let \( (X,d) \) is a metric space and let \( E \subseteq  X  \). We say that a point \( p \in E  \) is said to be an \textbf{interior point of \( E  \)} if there exists a neighborhood \( {N}_{\epsilon}(x)  \) such that \( {N}_{\epsilon}(x) \subseteq E  \). The collection of all interior points of \( E  \) is called the \textbf{interior of \( E  \)} and is denoted by the set:
   \[  E^{\circ} = \{ x \in E : \exists \ {N}_{\epsilon}(x) \subseteq E  \}. \]
\end{definition}

\begin{remark}
    Note that by definition we know that the interior of \( E  \) is always contained within \( E  \); that is, \( E^{\circ} \subseteq  E  \). Also, 
    \begin{center}
        \( p \in E^{\circ}  \) if and only if there exists a neighborhood \( {N}_{\delta}(p)  \) such that \( {N}_{\delta}(p) \subseteq E  \).
    \end{center}
\end{remark}

\begin{eg}
    Let \( (\R, d) \) with \( d(x,y) = | x - y  |  \) with \( E = (1,3] \). What is \( E^{\circ} = ?  \). We claim that \( E^{\circ} = (1,3) \).

    Let \( p \in  (1,3) \). We want to show that 
    \begin{enumerate}
        \item[(1)] If \( p \in (1,3) \), then \( p  \) is an interior point.
        \item[(2)] If \( p = 3  \), then \( p \notin E^{\circ} \).
    \end{enumerate}
    We proceed by showing each case above:
    \begin{enumerate}
        \item[(1)] It suffices to show that there exists \( \delta > 0  \) such that \( {N}_{\delta}(p) \subseteq  E  \). Clearly, choose \( \delta = \frac{ 1 }{ 2 }  \min \{  p - 1 , 3 - p \}  \) and we are done.
        \item[(2)] Suppose \( p = 3  \). It suffices to show that 
            \begin{center}
                \( \forall \ \epsilon > 0  \), \( {N}_{\epsilon}(3) \not\subseteq E  \).
            \end{center}
            That is, we want to show that
            \[  \forall \ \epsilon > 0 , (3 - \epsilon, 3 + \epsilon) \cap E^{c} \neq \emptyset. \]
            Clearly, for all \( \epsilon > 0  \), we have \( 3 + \frac{ \epsilon  }{  2  }  \in (3 - \epsilon , 3 + \epsilon) \). Thus, \( 3  + \frac{ \epsilon  }{  2  }  \in E^{c} \). Hence, we have 
            \[  (3 - \epsilon , 3 + \epsilon) \cap E^{c} \neq \emptyset. \]
    \end{enumerate}
\end{eg}

To show that boundary points are not interior points, it suffices to show that intersection with each neighborhood and the complement of the set is question is nonmepty.

\begin{eg}
    Let \( (\R, d)  \) with \( d(x,y) = |  x - y  |  \) and \( E = \{ 1,2,3 \}  \). What is \( E^{\circ} = ?  \). We claim that \( E^{\circ} = \emptyset  \). The reason is as follows: Let \( p \in \{ 1,2,3 \}  \); that is, for all \( \epsilon > 0  \), we have \( {N}_{\epsilon}(p) \not\subseteq E  \) but \( {N}_{\epsilon}(p) = (p - \epsilon, p + \epsilon) \) has infinitely many points. We proved that if \( p \in E  \), then \( p \notin E^{\circ} \). So, \( E^{\circ} = \emptyset \).
\end{eg}

\begin{definition}[Open Sets]
   Let \( (X,d) \) be a metric space and \( E \subseteq  X \). We say that \( E  \) is \textbf{open} if every point of \( E  \) is an interior point of \( E  \); that is, 
   \begin{center}
       \( E  \) is open \( \Longleftrightarrow  \) \( E \subseteq  E^{\circ} \).
   \end{center}
\end{definition}

\begin{remark}
    We know that, for any set, \( E^{\circ} \subseteq  E  \). So, we can rewrite our definition as follows:
    \begin{center}
        \( E  \) is open \( \Longleftrightarrow  \) \( E = E^{\circ} \).
    \end{center}
    That is, the other inclusion holds!
\end{remark}

\begin{eg}
    Let \( (\R ,d ) \) with \( d(x,y) = | x -y  |  \) and \( E = \{ 1,2,3 \}  \). Is \( E  \) open? Note that \( E^{\circ} = \emptyset  \). So, \( E^{\circ} \neq E  \) and so \( E  \) is NOT open.
\end{eg}

\begin{eg}
    Let \( (\R,d) \) with \( d(x,y) = |  x - y  |  \) and \( E = (1,4) \). Prove that \( E  \) is open. It suffices to show that every point \( p \in E   \) is an interior point. Let \(  p \in E  \). That is, we want to show that there exists \( \delta > 0  \) such that \( {N}_{\delta}(p) \subseteq  E  \). If we choose \( \delta = \frac{ 1 }{ 2 }  \{ p - 1 , 4 - p \}  \) does the job and we are done.
\end{eg}

\begin{definition}[Bounded Sets]
    Let \( (X,d) \) be a metric space and \( E \subseteq  X \). We say that \( E  \) is \textbf{bounded} if there exists \( \epsilon >0    \) and \( q \in X  \) such that \( E \subseteq {N}_{\epsilon}(q) \).   
\end{definition}

\begin{eg}
    Let \( (\R , d ) \) with \( d(x,y) = | x - y  |  \) and \( E = [0,\infty ) \). Is \( E  \) bounded? \textbf{NO!} This is because for all \( q \in \R  \) and \( \epsilon > 0  \), we have
    \[  [0,\infty ) \not\subseteq (q - \epsilon, q + \epsilon).   \]
\end{eg}

\begin{eg}
    Let \( (\R, d) \) with the discrete metric and \( E = [0,\infty )  \). Is \( E  \) bounded? \textbf{YES!} For example, we have 
    \[  E \subseteq  {N}_{10}(0) = \R.  \]
\end{eg}

\begin{definition}[Closure]
    Let \( (X,d) \) be a metric space and \( E \subseteq X  \). The \textbf{closure} of \( E  \), denoted by \( \overline{E} \), is defined as follows:  
    \[ \overline{E} = E \cup E' \]
    that is, the closure of \( E  \) is the union of the isolated points and limit points.
\end{definition}

\begin{eg}
   Let \( (\R,d) \) and \( d(x,y) = | x - y  |  \). What is \( \overline{Q} \)? Show within the homework that \( \Q' = \R  \). So,  
   \[   \]
\end{eg}

\begin{definition}[Dense]
    Let \( (X,d) \) metric space and \( E \subseteq X  \). We say that \( E  \) is dense in \( X  \) if \( \overline{E} = X  \). (that is, every point of \( X  \) is either in \( E  \) or is a limit point of \( E  \)).
\end{definition} 

\begin{eg}
    \( \overline{\Q} = \R    \), so \( \Q \) is dense in \( \R  \). 
\end{eg}






\end{document}
