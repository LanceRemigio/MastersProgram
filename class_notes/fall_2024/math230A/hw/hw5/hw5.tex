\documentclass[a4paper]{article}
\usepackage{standalone}
\usepackage{import}
\usepackage[utf8]{inputenc}
\usepackage[T1]{fontenc}
% \usepackage{fourier}
\usepackage{textcomp}
\usepackage{hyperref}
\usepackage[english]{babel}
\usepackage{url}
% \usepackage{hyperref}
% \hypersetup{
%     colorlinks,
%     linkcolor={black},
%     citecolor={black},
%     urlcolor={blue!80!black}
% }
\usepackage{graphicx} \usepackage{float}
\usepackage{booktabs}
\usepackage{enumitem}
% \usepackage{parskip}
% \usepackage{parskip}
\usepackage{emptypage}
\usepackage{subcaption}
\usepackage{multicol}
\usepackage[usenames,dvipsnames]{xcolor}
\usepackage{ocgx}
% \usepackage{cmbright}


\usepackage[margin=1in]{geometry}
\usepackage{amsmath, amsfonts, mathtools, amsthm, amssymb}
\usepackage{thmtools}
\usepackage{mathrsfs}
\usepackage{cancel}
\usepackage{bm}
\newcommand\N{\ensuremath{\mathbb{N}}}
\newcommand\R{\ensuremath{\mathbb{R}}}
\newcommand\Z{\ensuremath{\mathbb{Z}}}
\renewcommand\O{\ensuremath{\emptyset}}
\newcommand\Q{\ensuremath{\mathbb{Q}}}
\newcommand\C{\ensuremath{\mathbb{C}}}
\newcommand\F{\ensuremath{\mathbb{F}}}
% \newcommand\P{\ensuremath{\mathbb{P}}}
\DeclareMathOperator{\sgn}{sgn}
\DeclareMathOperator{\diam}{diam}
\DeclareMathOperator{\LO}{LO}
\DeclareMathOperator{\UP}{UP}
\DeclareMathOperator{\card}{card}
\DeclareMathOperator{\Arg}{Arg}
\DeclareMathOperator{\Dom}{Dom}
\DeclareMathOperator{\Log}{Log}
\DeclareMathOperator{\dist}{dist}
% \DeclareMathOperator{\span}{span}
\usepackage{systeme}
\let\svlim\lim\def\lim{\svlim\limits}
\renewcommand\implies\Longrightarrow
\let\impliedby\Longleftarrow
\let\iff\Longleftrightarrow
\let\epsilon\varepsilon
\usepackage{stmaryrd} % for \lightning
\newcommand\contra{\scalebox{1.1}{$\lightning$}}
% \let\phi\varphi
\renewcommand\qedsymbol{$\blacksquare$}

% correct
\definecolor{correct}{HTML}{009900}
\newcommand\correct[2]{\ensuremath{\:}{\color{red}{#1}}\ensuremath{\to }{\color{correct}{#2}}\ensuremath{\:}}
\newcommand\green[1]{{\color{correct}{#1}}}

% horizontal rule
\newcommand\hr{
    \noindent\rule[0.5ex]{\linewidth}{0.5pt}
}

% hide parts
\newcommand\hide[1]{}

% si unitx
\usepackage{siunitx}
\sisetup{locale = FR}
% \renewcommand\vec[1]{\mathbf{#1}}
\newcommand\mat[1]{\mathbf{#1}}

% tikz
\usepackage{tikz}
\usepackage{tikz-cd}
\usetikzlibrary{intersections, angles, quotes, calc, positioning}
\usetikzlibrary{arrows.meta}
\usepackage{pgfplots}
\pgfplotsset{compat=1.13}

\tikzset{
    force/.style={thick, {Circle[length=2pt]}-stealth, shorten <=-1pt}
}

% theorems
\makeatother
\usepackage{thmtools}
\usepackage[framemethod=TikZ]{mdframed}
\mdfsetup{skipabove=1em,skipbelow=1em}

\theoremstyle{definition}

\declaretheoremstyle[
    headfont=\bfseries\sffamily\color{ForestGreen!70!black}, bodyfont=\normalfont,
    mdframed={
        linewidth=1pt,
        rightline=false, topline=false, bottomline=false,
        linecolor=ForestGreen, backgroundcolor=ForestGreen!5,
    }
]{thmgreenbox}

\declaretheoremstyle[
    headfont=\bfseries\sffamily\color{NavyBlue!70!black}, bodyfont=\normalfont,
    mdframed={
        linewidth=1pt,
        rightline=false, topline=false, bottomline=false,
        linecolor=NavyBlue, backgroundcolor=NavyBlue!5,
    }
]{thmbluebox}

\declaretheoremstyle[
    headfont=\bfseries\sffamily\color{NavyBlue!70!black}, bodyfont=\normalfont,
    mdframed={
        linewidth=1pt,
        rightline=false, topline=false, bottomline=false,
        linecolor=NavyBlue
    }
]{thmblueline}

\declaretheoremstyle[
    headfont=\bfseries\sffamily, bodyfont=\normalfont,
    numbered = no,
    mdframed={
        rightline=true, topline=true, bottomline=true,
    }
]{thmbox}

\declaretheoremstyle[
    headfont=\bfseries\sffamily, bodyfont=\normalfont,
    numbered=no,
    % mdframed={
    %     rightline=true, topline=false, bottomline=true,
    % },
    qed=\qedsymbol
]{thmproofbox}

\declaretheoremstyle[
    headfont=\bfseries\sffamily\color{NavyBlue!70!black}, bodyfont=\normalfont,
    numbered=no,
    mdframed={
        rightline=false, topline=false, bottomline=false,
        linecolor=NavyBlue, backgroundcolor=NavyBlue!1,
    },
]{thmexplanationbox}

\declaretheorem[
    style=thmbox, 
    % numberwithin = section,
    numbered = no,
    name=Definition
    ]{definition}

\declaretheorem[
    style=thmbox, 
    name=Example,
    ]{eg}

\declaretheorem[
    style=thmbox, 
    % numberwithin = section,
    name=Proposition]{prop}

\declaretheorem[
    style = thmbox,
    numbered=yes,
    name =Problem
    ]{problem}

\declaretheorem[style=thmbox, name=Theorem]{theorem}
\declaretheorem[style=thmbox, name=Lemma]{lemma}
\declaretheorem[style=thmbox, name=Corollary]{corollary}

\declaretheorem[style=thmproofbox, name=Proof]{replacementproof}

\declaretheorem[style=thmproofbox, 
                name = Solution
                ]{replacementsolution}

\renewenvironment{proof}[1][\proofname]{\vspace{-1pt}\begin{replacementproof}}{\end{replacementproof}}

\newenvironment{solution}
    {
        \vspace{-1pt}\begin{replacementsolution}
    }
    { 
            \end{replacementsolution}
    }

\declaretheorem[style=thmexplanationbox, name=Proof]{tmpexplanation}
\newenvironment{explanation}[1][]{\vspace{-10pt}\begin{tmpexplanation}}{\end{tmpexplanation}}

\declaretheorem[style=thmbox, numbered=no, name=Remark]{remark}
\declaretheorem[style=thmbox, numbered=no, name=Note]{note}

\newtheorem*{uovt}{UOVT}
\newtheorem*{notation}{Notation}
\newtheorem*{previouslyseen}{As previously seen}
% \newtheorem*{problem}{Problem}
\newtheorem*{observe}{Observe}
\newtheorem*{property}{Property}
\newtheorem*{intuition}{Intuition}

\usepackage{etoolbox}
\AtEndEnvironment{vb}{\null\hfill$\diamond$}%
\AtEndEnvironment{intermezzo}{\null\hfill$\diamond$}%
% \AtEndEnvironment{opmerking}{\null\hfill$\diamond$}%

% http://tex.stackexchange.com/questions/22119/how-can-i-change-the-spacing-before-theorems-with-amsthm
\makeatletter
% \def\thm@space@setup{%
%   \thm@preskip=\parskip \thm@postskip=0pt
% }
\newcommand{\oefening}[1]{%
    \def\@oefening{#1}%
    \subsection*{Oefening #1}
}

\newcommand{\suboefening}[1]{%
    \subsubsection*{Oefening \@oefening.#1}
}

\newcommand{\exercise}[1]{%
    \def\@exercise{#1}%
    \subsection*{Exercise #1}
}

\newcommand{\subexercise}[1]{%
    \subsubsection*{Exercise \@exercise.#1}
}


\usepackage{xifthen}

\def\testdateparts#1{\dateparts#1\relax}
\def\dateparts#1 #2 #3 #4 #5\relax{
    \marginpar{\small\textsf{\mbox{#1 #2 #3 #5}}}
}

\def\@lesson{}%
\newcommand{\lesson}[3]{
    \ifthenelse{\isempty{#3}}{%
        \def\@lesson{Lecture #1}%
    }{%
        \def\@lesson{Lecture #1: #3}%
    }%
    \subsection*{\@lesson}
    \testdateparts{#2}
}

% \renewcommand\date[1]{\marginpar{#1}}


% fancy headers
\usepackage{fancyhdr}
\pagestyle{fancy}

\makeatother

% notes
\usepackage{todonotes}
\usepackage{tcolorbox}

\tcbuselibrary{breakable}
\newenvironment{verbetering}{\begin{tcolorbox}[
    arc=0mm,
    colback=white,
    colframe=green!60!black,
    title=Opmerking,
    fonttitle=\sffamily,
    breakable
]}{\end{tcolorbox}}

\newenvironment{noot}[1]{\begin{tcolorbox}[
    arc=0mm,
    colback=white,
    colframe=white!60!black,
    title=#1,
    fonttitle=\sffamily,
    breakable
]}{\end{tcolorbox}}

% figure support
\usepackage{import}
\usepackage{xifthen}
\pdfminorversion=7
\usepackage{pdfpages}
\usepackage{transparent}
\newcommand{\incfig}[1]{%
    \def\svgwidth{\columnwidth}
    \import{./figures/}{#1.pdf_tex}
}

% %http://tex.stackexchange.com/questions/76273/multiple-pdfs-with-page-group-included-in-a-single-page-warning
\pdfsuppresswarningpagegroup=1


% \usepackage{fourier}

\pagestyle{fancy}
\fancyhf{}

\title{Math 230A: Homework 5}
\author{Lance Remigio}

\begin{document}
\maketitle    
\lhead{Math 230A: Homework 5}
\chead{Lance Remigio}
\rhead{\thepage}

\begin{problem}
   Mark each statement True or False. Let \( (X,d) \) be a metric space and \( K \subseteq X  \). 
   \begin{enumerate}
       \item If \( Y \subseteq X   \) and \( \{ {G}_{\alpha} \}  \) is a collection of subsets of \( Y  \) that are open relative to \( Y  \), then \( \bigcup_{ \alpha }^{  } {G}_{\alpha} \) is open relative to \( Y  \). \textbf{True.}
        \item If there exists some open cover of \( K   \) which has a finite subcover, then \( K  \) is compact. \textbf{False.}
        \item If \( K  \) is compact, then \( K' \subseteq K  \). \textbf{True.}
        \item If \( K  \) is closed, then \( K  \) is compact. \textbf{False.}
        \item If \( K  \) is compact and \( E \subseteq K \), then \( E  \) is compact. \textbf{False.}
        \item Consider \( E = [-10,10] \subseteq \R  \) and the open cover of \( E  \) by \( \Phi = \{ (x-1,x + 1) : x \in E  \}  \). Then the collection \( \{ (x - \frac{ 1 }{ 2 }  , x + \frac{ 1 }{ 2 } ) : x \in E  \}  \) is a subcover of \( \Phi \). \textbf{False.}
        \item Let \( E  \) and \( \Phi \) be as above. Then the collection \( \{ (x-1, x+1): x = -10,-9,-8, \dots, 8, 9, 10 \}  \) is a subcover of \( \Phi \). \textbf{True.}
        \item Let \( E  \) and \( \Phi \) be as above. Then the collection \( \{ (x-1, x+1): x = -10,-8, -6, \dots, 6, 8, 10 \}  \) is a subcover of \( \Phi \). \textbf{False.} (this is a subcollection of \( \Phi  \) but it is NOT a subcover)
   \end{enumerate}
\end{problem}

   \begin{problem}
       Show that compact implies bounded.
   \end{problem}
   \begin{proof}
       Assume that \( K  \) is compact. We want to show that \( K  \) is bounded. Fix \( \delta > 0  \) and consider \( K \subseteq \bigcup_{ x \in k  }^{  } {N}_{\delta}(x) \) where \( \{ {N}_{\delta}(x) \}_{x \in K} \) is an open cover of \( K  \). Since \( K  \) is compact, we can find a finite subcover; that is, there exists \( {x}_{1}, \dots, {x}_{n} \in K     \) such that 
       \[ K \subseteq \bigcup_{ i=1  }^{ n } {N}_{\delta}({x}_{i}). \tag{*}  \]
       Note that each \( {N}_{\delta}({x}_{i}) \) is bounded. Using the results from lemmas {\hyperref[lemma 1]{1}} and {\hyperref[lemma 2]{2}}, we can conclude that the finite union in (*) must be bounded as well as its subset \( K  \). Thus, we can see that \( K  \) must be bounded and we are done.   
   \end{proof}



   \begin{problem}
      Show the union of finitely many compact sets is compact. 
   \end{problem}
\begin{proof}
    Let \( {E}_{1}, \dots, {E}_{n} \) be compact sets in the metric space \( (X,d) \). Our goal is to show that  
    \[  K = \bigcup_{ i=1  }^{ \infty   }  {K}_{i}  \]
    is a compact set in \( X \). We proceed to show this result via induction on \( n  \).
    Let \( n = 2  \) be our base case. Our goal is to show that \( {E}_{1} \cup {E}_{2} \) is compact. Let \( \{ {O}_{\alpha} \}_{\alpha \in \Lambda} \) be an open cover of \( {E}_{1} \cup {E}_{2} \); that is,
    \[  {E}_{1} \subseteq  \bigcup_{ \alpha \in \Lambda }^{  }  {O}_{\alpha} \ \text{and} \ {E}_{2} \subseteq  \bigcup_{ \alpha \in \Lambda }^{  }  {O}_{\alpha}. \]
    Since \( {E}_{1} \) is compact, there exists \( {\alpha}_{1}, \dots, {\alpha}_{n} \in \Lambda  \) such that 
    \[  {E}_{1} \subseteq  \bigcup_{ i=1  }^{ n }  {O}_{{\alpha}_{i}}. \]
    Likewise, \( {E}_{2} \) being compact implies that there exists \( {\beta}_{1}, \dots, {\beta}_{m} \in \Lambda \) such that  
    \[  {E}_{2} \subseteq  \bigcup_{ j=1  }^{ m  }  {O}_{{\beta}_{j}}. \]
    Thus, we see that 
    \[  {E}_{1} \cup {E}_{2} \subseteq \Big( \bigcup_{ i=1  }^{ n  } {O}_{{\alpha}_{i}} \Big) \cup \bigcup_{ j=1  }^{ m }  {O}_{{\beta}_{j}}.  \]
    Suppose the claim is true for \( n = k \geq 1  \). If \( {E}_{1}, \dots, {E}_{k} \) are compact sets, then \( \bigcup_{ i=1  }^{ k  }  {E}_{i} \) is compact. Observe that 
    \begin{align*}
        \bigcup_{ i=1  }^{ k + 1  } {E}_{i} &= \Big(  \bigcup_{ i=1  }^{ k  }  {E}_{i} \Big) \cup {E}_{k+1} \tag{\( \bigcup_{ i=1  }^{ k  } {E}_{i}  \) is compact}.
    \end{align*}
    But by our base case \( n = 2  \), we know that the right-hand side of the equation above is compact and we are done.
\end{proof}

   \begin{problem}
       Show that an arbitrary intersection of compact sets is compact. (Hint. A closed subset of a compact set is compact)
   \end{problem}
   \begin{proof}
       Let \( \{ {O}_{\alpha} \}_{\alpha \in \Lambda} \) be a collection of open sets in \( X  \) and let \( \{ {K}_{\alpha} \}_{\alpha \in \Lambda} \) be a collection of compact sets in \( X  \). Define
       \[  K = \bigcap_{  \alpha  }^{  } {K}_{\alpha}. \tag{1} \]
       Since each \( {K}_{\alpha} \) is compact, we know that each \( {K}_{\alpha} \) must be closed. Therefore, the arbitrary intersection above must be closed and so \( K  \) is closed. But observe that 
       \[  \bigcap_{ \alpha }^{  } {K}_{\alpha} \subseteq  {K}_{\alpha} \]
       and that \( K_{\alpha}  \) is a closed set. Thus, \( K  \) must be compact as well! 
   \end{proof}

    \begin{problem}
        A metric space \( (X,d) \) is called \textbf{separable} if it contains a countable subset \( E  \) which is dense in \( X  \). For example, \( \R  \) is separable because \( \Q  \) is a countable set which is dense in \( \R  \). Show that \( \R^{2}  \) is separable. (Hint: Consider the set of points which have only rational coordinates.) 
    \end{problem}
    \begin{proof}
        Let 
        \[  E  = \Q \times \Q = \{ (p,q) \in \R^{2} : p,q \in \Q  \}. \]
        Since \( \Q  \) is a countable set and that a finite product of countable sets is countable, we must have that \( E  \) is countable. All that is left to show is that \( E  \) is dense in \( \R^{2} \). Our goal is to show that \( \overline{E} = \R^{2} \); that is, 
        \[  \overline{E} = \{ (x,y) \in \R^{2} : \forall \epsilon > 0, \ {N}_{\epsilon}((x,y)) \cap E \neq \emptyset \}. \]
        Hence, our goal is to show that for all \( (x,y) \in \R^{2} \) and for all \( \epsilon > 0  \) that 
        \[  {N}_{\epsilon}((x,y)) \cap E \neq \emptyset. \]
        To this end, let \( (x,y) \in \R^{2} \) and let \( \epsilon > 0  \) be given. Since \( x \in \R  \) and \( \Q  \) is dense in \( \R  \), we know that there exists \( p \in \Q  \) such that  
        \[ p \in {N}_{\frac{ \epsilon }{ \sqrt{ 2 }  } }(x) \iff  | x - p  | < \frac{ \epsilon }{ \sqrt{ 2 }  }. \tag{1} \]
            Likewise, \( y \in \R  \) implies that there exists \( q \in \Q  \) such that 
            \[  q \in {N}_{\frac{ \epsilon }{ \sqrt{ 2 }  } } \iff | y - q  |  < \frac{ \epsilon }{ \sqrt{ 2 }  }. \tag{2} \]
            We claim that \( (p,q) \in {N}_{\epsilon}((x,y)) \cap E  \) so that 
            \[  {N}_{\epsilon}((x,y)) \cap E \neq \emptyset. \]
            By using (1) and (2) as well as the triangle inequality, we see that 
            \[  d((p,q), (x,y)) = \sqrt{ | x - p |^{2} + | y - q  |^{2} } < \sqrt{ \frac{ \epsilon^{2} }{ 2 }  + \frac{ \epsilon^{2} }{ 2 }  }  = \sqrt{ \epsilon^{2} }  = \epsilon.  \]
            Hence, we see that \( {N}_{\epsilon}((x,y)) \). Clearly, we see that \( (p,q) \in E  \) and so we conclude that 
            \[  {N}_{\epsilon}((x,y)) \cap E \neq \emptyset. \]
            Thus, \( E = \Q^{2} \) is dense in \( \R^{2} \).
    \end{proof}
   \begin{problem}
       Let \( (X,d) \) be a separable metric space and \( \O \neq A \subseteq X  \). Prove that the collection of the isolated points of \( A  \) is at most countable.
   \end{problem}
   \begin{proof}
   Let \( (X,d) \) be a separable metric space and \( \emptyset \neq A \subseteq  X  \). Denote the set \( {A}_{I} \) as the set of all isolated points of \( A  \). Since \( (X,d) \) is separable, we know that there must exists a subset of \( X  \) denoted by \( E  \) such that \( E  \) is both countable and dense in \( X  \). Our goal is to construct an injective map \( f: {A}_{I} \to E  \) which proves that \( {A}_{I} \) is at most countable. 
   
   Observe that for every \( p \in {A}_{I} \), we know that \( p \in A \setminus  A'  \) if and only if there exists \( {\epsilon}_{p} > 0  \) such that 
   \[  {N}_{{\epsilon}_{p}}(p) \cap A = \{ p \}. \tag{1} \]
   Furthermore, since \( E  \) is dense in \( X  \), we know that there exists \( {\epsilon}_{p} > 0  \) such that  
   \[  {N}_{{\epsilon}_{p}}(p) \cap E \neq \emptyset. \tag{2} \]
   Hence, (1) and (2) imply that there exists \( {y}_{p} \in X  \) such that \( {y}_{p} \in {N}_{\frac{ {\epsilon}_{p} }{ 2 }} \cap E  \). We define \( f: {A}_{I} \to E  \) by
   \[  f(p) = {y}_{p}. \]
   Our goal is to show that \( f  \) is injective. Let \( p,q \in X  \). Suppose \( f(p) = f(q) \). From this, we will show that \( p = q  \). Since \( f(p) = f(q) \), we know that    \[  {y}_{p} = {y}_{q}. \] 
   Observe that  
   \begin{center} 
       \( {y}_{p} \in {N}_{\frac{ {\epsilon}_{p} }{ 2 } } \cap E  \) and \( {y}_{q} \in {N}_{\frac{ {\epsilon}_{q} }{ 2 } }(q) \cap E  \),
   \end{center}
   Since \( {y}_{p} = {y}_{q} \), we have 
   \[  {N}_{\frac{ {\epsilon}_{p} }{ 2 }  }(p) \cap {N}_{\frac{ {\epsilon}_{q} }{ 2 } }(q) \neq \emptyset. \]
   Let \( z \in {N}_{\frac{ {\epsilon}_{p} }{ 2 } } \cap {N}_{ \frac{ {\epsilon}_{q} }{ 2 } }(q) \). Then we have 
   \[  d(p,z) < \frac{ {\epsilon}_{p} }{ 2 }  \ \text{and} \ d(z,q) < \frac{ {\epsilon}_{q} }{ 2 } \]
   which implies that 
   \[  d(p,z) + d(z,q) < \frac{ {\epsilon}_{p} }{ 2 }  + \frac{ {\epsilon}_{q} }{ 2 }.  \]
   Without loss of generality, assume that \( {\epsilon}_{q} < {\epsilon}_{p}    \). By the triangle inequality, we see that 
   \[  d(p,q) \leq d(p,z) + d(z,q) < \frac{ {\epsilon}_{p} }{ 2 }  + \frac{ {\epsilon}_{q} }{ 2 } < {\epsilon}_{p}.  \]
 This implies that \( q  \) must be contained in \( {N}_{{\epsilon}_{p}}(p) \cap A  \). However, the only point contained in \( {N}_{{\epsilon}_{p}} \cap A  \) is \( p  \). Hence, we must have \( p = q  \) and so \( f  \) is injective. Thus, \( {A}_{I} \) is at most countable.
   \end{proof}

   \begin{problem}
       Let \( (X,d) \) be a metric space. A collection \( \{ {V}_{\alpha} \} \) of open subsets of \( X  \) is said to be a \textbf{base} for \( X  \) if the following is true: For every \( x \in X  \) and every open set \( G \subseteq X   \) such that \( x \in G  \), we have \( {V}_{\alpha} \subseteq G   \) for some \( \alpha \). In other words, every open set in \( X  \) is the union of a subcollection of \( \{ {V}_{\alpha} \}  \). 

       Prove that the every separable metric space has a countable base. (Hint: Take all neighborhoods with rational radius and center in some countable dense subset of \( X  \).)
   \end{problem}
   \begin{proof}
   Since \( (X,d) \) is a separable metric space, we know that \( X  \) contains a countable dense subset \( E  \). Let \( \{ {V}_{\alpha} \}_{\alpha \in \Lambda} \) be a collection of open subsets in \( X  \). Let \( x \in X  \) and let \( G \subseteq  X   \) be an open set such that \( x \in G  \). Our goal is to show that \( {V}_{\alpha} \subseteq  G  \) for some \( \alpha \). Since \( X = \overline{E } \), we must have \( x \in \overline{E} \); that is, for all \( \epsilon > 0  \),
   \[  {N}_{\epsilon}(x) \cap E \neq \emptyset. \]
   To this end, let \( \epsilon > 0 \) be rational and pick a point \( y \in {N}_{\epsilon}(x) \cap E  \). Then
   \begin{center}
       \( y \in {N}_{\epsilon}(x) \) and \( y \in E  \).
   \end{center}
    Note that \( {N}_{\epsilon}(x) \) is an open set in \( X  \). Thus, we can write \( {N}_{\epsilon}(x) \) as a union of open sets \( {V}_{\alpha} \) where \( \alpha \in \Lambda \); that is, 
    \[  {N}_{\epsilon}(x) = \bigcup_{ \alpha }^{  } {V}_{\alpha}. \]
    Thus, \( y  \) must be contained in the union above and so \( y \in {V}_{\alpha_0} \) for some \( {\alpha}_{0} \in \Lambda \). Since \( y \in G  \), we must also have \( {V}_{\alpha} \subseteq G  \) and we are done.
   \end{proof}

   \begin{problem}
      Let \( (X,d)  \) be a metric space in which every infinite subset has a limit point. Prove that \( X  \) is separable. (Hint: Fix \( \delta > 0  \), and pick \( {x}_{1} \in X  \). Having chosen \( {x}_{1}, \dots, {x}_{j} \in X  \), choose \( {x}_{j+1} \in X  \), if possible, so that \( d({x}_{i}, {x}_{j+1}) \geq \delta \) for \( i = 1,\dots, j \). Show that this process must stop after a finite number of steps, and that \( X  \) can therefore be covered by finitely many neighborhoods of radius \( \delta  \). Take \( \delta = 1/n \ (n = 1,2,3,\dots) \), and consider the centers of the corresponding neighborhoods.)
   \end{problem}
   \begin{proof}
   Let \( (X,d) \) be a metric space in which every infinite subset has a limit point. Our goal is to show that \( X  \) is separable. Fix \( \delta > 0  \), and pick \( {x}_{1} \in X  \). Now, choose \( {x}_{1}, \dots, {x}_{j } \in X  \) and then choose \( {x}_{j+1} \in X  \) such that \( d({x}_{i}, {x}_{j+1}) \geq \delta \) for \( i = 1, \dots, j  \). 

     By the lemma {\hyperref[lemma 3]{3}}, the process outlined in the first paragraph must terminate; that is, we can consider finitely many neighborhoods of radius \( \delta > 0   \) that cover \( X  \).

   Define
   \[  S = \{ {x}_{{n}_{j}} : 1 \leq j \leq {N}_{n} : n = 1,2,3,\dots \} \]
    and fix \( \delta = \frac{ 1 }{ n }   \) for \( n = 1,2,3,\dots \ \) such that  
   \[  X = \bigcup_{ j=1  }^{ {N}_{n} }  {N}_{\delta}({x}_{{n}_{j}}) \ \ n = 1,2,3,\dots \ . \]
   Clearly, \( S \) is countable. Hence, all that is left to show is that \( S  \) is dense in \( X  \). That is, we would like to show that for every \( x \in X  \) and \( \epsilon > 0  \) that
   \[  {N}_{\epsilon}(x) \cap S \neq \emptyset.  \]

   To this end, pick \( n \in \N  \) such that \( \delta =  \frac{ 1 }{ n }  < \epsilon  \) by the Archimeadean Property. Consider the neighborhood \( {N}_{\epsilon}(x)  \). If we let \( y \in S \) with \( x \in {N}_{\delta}(y) \), then we see that  
   \[  d(x,y) < \frac{ 1 }{ n }  < \epsilon. \]
   This implies that \( y \in {N}_{\epsilon}(x) \). Since \( y  \) is also contained in \( S  \), we can conclude that
   \[  y \in {N}_{\epsilon}(x) \cap S \iff {N}_{\epsilon}(x) \cap S \neq \emptyset, \]
   showing that \( S  \) is dense in \( X  \). Hence, \( X  \) is a separable metric space.
   \end{proof}

   \begin{problem}[Extra Credit]
      Let \( (X,d) \) be a metric space and \( Y  \) be a nonempty subset of \( X  \). Let \( E \subseteq Y \). Prove that   
      \begin{center}
          \( E  \) is closed relative to \( Y  \) \( \iff \) \( E = A \cap Y  \) for some closed set \( A \subseteq  X \).
      \end{center}
   \end{problem} 
   \begin{proof}
    \( (\Longrightarrow) \) Assume that \( E  \) is closed relative to \( Y  \). Our goal is to show that \( E = A \cap Y  \). Since \( E  \) is closed relative in \( Y \), we know that \( E^{c} \) is open relative in \( Y  \); that is, 
    \[ (Y \setminus  E ) = G \cap Y \ \text{for some} \  G \subseteq Y. \]
    Our goal is to show that \( E = G^{c} \cap Y  \) for some \( G^{c} \subseteq Y \) (\( G \) is open \( \iff \) \( G^{c} \) is closed). Observe that 
    \begin{align*}
        E = Y \setminus  (Y \setminus  E ) &= Y \setminus  ( G \cap Y) \\
                                           &= (Y \cap Y) \setminus  (G \cap Y) \\
                                           &=  (Y \setminus  G) \cap Y \\
                                           &=  G^{c} \cap Y \tag{for some \( G^{c} \subseteq Y \)}
    \end{align*}
    and we are done.
    
    
        
    \( (\impliedby) \) Suppose \( E = A \cap Y  \) for some closed set \( A \subseteq X  \). Observe that  
    \[  Y \setminus  E = (X \setminus  A ) \cap Y  \tag{1} \]
    where \( X \setminus  A  \) is open in \( X  \). By (1), we must have \( Y \setminus  E  \) is open in \( Y  \). Hence, \( E  \) is closed relative to \( Y  \). 
\end{proof}
   
   \begin{problem}[Extra Credit]
   Let \( (X,d) \) be a metric space. Let \( E \subseteq  X \). Prove that the following definitions of boundedness are equivalent:
    \begin{itemize}
        \item Rudin's Definition: There exists \( q \in X  \) and \( \epsilon > 0  \) such that \( E \subseteq {N}_{\epsilon}(q) \).
        \item Anthony's Definition: There exists \( R > 0  \) such that for all \( x  \) and \( y  \) in \( E  \), we have \( d(x,y) < R  \).
    \end{itemize}
   \end{problem}
   \begin{proof}
    \( (\implies) \) Suppose there exists \( q \in X  \) and \(  \epsilon > 0 \) such that \( E \subseteq  {N}_{\epsilon}(q) \). Let \( x,y \in E  \). Our goal is to show that there exists \( R > 0  \) such that for all \( x,y \in E  \), \( d(x,y) < R  \). To this end, let \( x,y \in E  \) be given. Choose \( R = 2 \epsilon > 0   \). Since \( E \subseteq  {N}_{\epsilon}(q) \), we have that         
    \[  d(x,y) \leq d(x,q) + d(q,y) < \epsilon + \epsilon = 2 \epsilon = R.  \]
    Thus, we see that \( d(x,y) < R  \) for any \( x,y \in E  \).

    \( (\impliedby) \) Suppose there exists \( R > 0  \) such that for all \( x,y \in E  \), \( d(x,y) < R  \). We will show that there exists \( q \in X  \) and \( \epsilon > 0  \) such that \( E \subseteq {N}_{\epsilon}(q) \). Let \( x \in E  \). Fix \( y \in E  \) such that \( q = y  \). Choose \( \epsilon = R + 1 > 0  \). By assumption,  
    \[  d(x,q) < R < R + 1 = \epsilon.  \]
    Hence, \( x \in {N}_{\epsilon}(q) \).
   \end{proof}

   \begin{lemma}[1]\label{lemma 1}
   Let \( (X,d) \) be a metric space. A subset of a bounded set is bounded. 
\end{lemma}
\begin{proof}
Since \( B  \) is bounded, there exists \( q \in X  \) and \( \epsilon > 0  \) such that \(  B \subseteq  {N}_{\epsilon}(q) \). Since \(  A \subseteq  B  \), we conclude that \( A \subseteq  {N}_{\epsilon}(q) \). So, \( A  \) is bounded.
\end{proof}

\begin{lemma}[2]\label{lemma 2}
    Let \( (X,d) \) be a metric space. A finite union of bounded sets is bounded.    
\end{lemma}
\begin{proof}
    Let \( {E}_{1}, \dots, {E}_{n} \) be a bounded sets in a metric space \( (X,d) \). We will show via induction on \( n  \) that 
    \[  \bigcup_{ i=1  }^{ n } {E}_{i}  \]
    is bounded. Note that if \( n = 1  \), then \( {E}_{1} \) is bounded and so the result is immediate. Thus, let our base case be \( n = 2  \). We will show that the union \( {E}_{1} \cup {E}_{2} \) is bounded. If \( {E}_{1} \) is bounded, then we can find an \( {\epsilon}_{1} > 0  \) and \( {q}_{1} \in X  \) such that 
    \[  {E}_{1} \subseteq {N}_{{\epsilon}_{1}}({q}_{1}). \tag{1} \]
    Likewise, if \( {E}_{2} \) is bounded, then we can find an \( {\epsilon}_{2} > 0  \) and \( {q}_{2} \in X  \) such that 
    \[  {E}_{2} \subseteq {N}_{{\epsilon}_{2}}({q}_{2}). \tag{2} \]
    Choose \( \delta = 2 [ {\epsilon}_{1} + {\epsilon}_{2} + d({q}_{1}, {q}_{2})]  \). From this, we will show that \( {E}_{1} \cup {E}_{2} \subseteq  {N}_{\delta}({q}_{1}) \) so that \( {E}_{1} \cup {E}_{2} \) will be bounded. Let \( x \in {E}_{1} \cup {E}_{2} \). Then we either have \( x \in {E}_{1} \) or \( x \in {E}_{2} \). If \( x \in {E}_{1} \), then   
    \[  x \in {N}_{{\epsilon}_{1}}(q) \subseteq {N}_{\delta}({q}_{1}) \tag{\( {\epsilon}_{1} < \delta \)}. \]
    If \( x \in {E}_{2} \), then 
    \begin{align*}
        d(x,{q}_{1}) &\leq d(x,{q}_{2}) + d({q}_{2}, {q}_{1}) \\
                     &\leq {\epsilon}_{2} + d({q}_{1}, {q}_{2}) \\
                     &< \delta.
    \end{align*} 
    This implies that \( x \in {N}_{\delta}({q}_{1}) \). Thus, (i) and (ii) imply that \( {E}_{1} \cup {E}_{2} \subseteq {N}_{\delta}({q}_{1}) \). Now, assume that the result holds for the \( n \)th case. We will now show that the result holds for \( n = k + 1  \). Thus, observe that 
    \begin{align*}
        \bigcup_{ n=1 }^{ k + 1  } {E}_{n} &= \Big( \bigcup_{ n=1  }^{ k  }  {E}_{n}   \Big) \cup {E}_{k+1}. \tag{3} 
    \end{align*}
    By our inductive hypothesis, we know that 
    \[  \bigcup_{ n=1  }^{ k  }  {E}_{n} \]
    is bounded. Clearly, with \( {E}_{k+1} \) being also bounded along with our base case \( n = 2  \), we can conclude that the union in (3) is also bounded which concludes our induction proof.
\end{proof}

\begin{lemma}[3]\label{lemma 3}
  Let \( (X,d) \) be a metric space and \( E \subseteq  X \) be infinite. Suppose \( d({x}_{n},{x}_{m}) > \epsilon  \) for all \( n \neq m  \in \N  \) such that \( {x}_{n} \neq {x}_{m} \). Then \( E  \) has no limit points.  
\end{lemma}
\begin{proof}
Suppose for sake of contradiction that \( E  \) does have a limit point. Denote this limit point of \( E  \) as \( x  \). Thus, for any \( \delta > 0  \), we have  
   \[  {N}_{\delta}(x) \cap (E \setminus  \{ x \} ) \neq \emptyset.  \tag{1} \]
   Since \( E  \) is an infinite subset of \( X  \), pick \( n \in \N  \) such that 
   \begin{center}
       \( {x}_{n} \in {N}_{\delta}(x)  \) and \( {x}_{n} \in E  \).
   \end{center}
   Hence, we see that \( d(x,{x}_{n}) < \delta \). Similarly, pick \( m \in \N   \) such that \( n \neq m  \) in the intersection in (1) such that \( {x}_{m} \in {N}_{\delta}(w) \). Then we have \( d({x}_{m},x) < \delta \). Our goal is to find some \( \epsilon > 0  \) such that \( d({x}_{n},{x}_{m}) < \epsilon  \), contradicting our assumption that \( d({x}_{n},{x}_{m}) \geq \delta \) for all \( \delta > 0 \) and \( n \neq m \in \N \). Choose \( \epsilon = 2 \delta \). Using the triangle inequality, we see that 
   \[  d({x}_{n},{x}_{m}) \leq d({x}_{n},x) + d(x,{x}_{m}) < \delta + \delta = 2 \delta = \epsilon \]
   which produces desired contradiction.
\end{proof}

\end{document}
