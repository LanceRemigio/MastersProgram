\documentclass[a4paper]{article}

\usepackage[utf8]{inputenc}
\usepackage[T1]{fontenc}
% \usepackage{fourier}
\usepackage{textcomp}
\usepackage{hyperref}
\usepackage[english]{babel}
\usepackage{url}
% \usepackage{hyperref}
% \hypersetup{
%     colorlinks,
%     linkcolor={black},
%     citecolor={black},
%     urlcolor={blue!80!black}
% }
\usepackage{graphicx} \usepackage{float}
\usepackage{booktabs}
\usepackage{enumitem}
% \usepackage{parskip}
% \usepackage{parskip}
\usepackage{emptypage}
\usepackage{subcaption}
\usepackage{multicol}
\usepackage[usenames,dvipsnames]{xcolor}
\usepackage{ocgx}
% \usepackage{cmbright}


\usepackage[margin=1in]{geometry}
\usepackage{amsmath, amsfonts, mathtools, amsthm, amssymb}
\usepackage{thmtools}
\usepackage{mathrsfs}
\usepackage{cancel}
\usepackage{bm}
\newcommand\N{\ensuremath{\mathbb{N}}}
\newcommand\R{\ensuremath{\mathbb{R}}}
\newcommand\Z{\ensuremath{\mathbb{Z}}}
\renewcommand\O{\ensuremath{\emptyset}}
\newcommand\Q{\ensuremath{\mathbb{Q}}}
\newcommand\C{\ensuremath{\mathbb{C}}}
\newcommand\F{\ensuremath{\mathbb{F}}}
% \newcommand\P{\ensuremath{\mathbb{P}}}
\DeclareMathOperator{\sgn}{sgn}
\DeclareMathOperator{\diam}{diam}
\DeclareMathOperator{\LO}{LO}
\DeclareMathOperator{\UP}{UP}
\DeclareMathOperator{\card}{card}
\DeclareMathOperator{\Arg}{Arg}
\DeclareMathOperator{\Dom}{Dom}
\DeclareMathOperator{\Log}{Log}
\DeclareMathOperator{\dist}{dist}
% \DeclareMathOperator{\span}{span}
\usepackage{systeme}
\let\svlim\lim\def\lim{\svlim\limits}
\renewcommand\implies\Longrightarrow
\let\impliedby\Longleftarrow
\let\iff\Longleftrightarrow
\let\epsilon\varepsilon
\usepackage{stmaryrd} % for \lightning
\newcommand\contra{\scalebox{1.1}{$\lightning$}}
% \let\phi\varphi
\renewcommand\qedsymbol{$\blacksquare$}

% correct
\definecolor{correct}{HTML}{009900}
\newcommand\correct[2]{\ensuremath{\:}{\color{red}{#1}}\ensuremath{\to }{\color{correct}{#2}}\ensuremath{\:}}
\newcommand\green[1]{{\color{correct}{#1}}}

% horizontal rule
\newcommand\hr{
    \noindent\rule[0.5ex]{\linewidth}{0.5pt}
}

% hide parts
\newcommand\hide[1]{}

% si unitx
\usepackage{siunitx}
\sisetup{locale = FR}
% \renewcommand\vec[1]{\mathbf{#1}}
\newcommand\mat[1]{\mathbf{#1}}

% tikz
\usepackage{tikz}
\usepackage{tikz-cd}
\usetikzlibrary{intersections, angles, quotes, calc, positioning}
\usetikzlibrary{arrows.meta}
\usepackage{pgfplots}
\pgfplotsset{compat=1.13}

\tikzset{
    force/.style={thick, {Circle[length=2pt]}-stealth, shorten <=-1pt}
}

% theorems
\makeatother
\usepackage{thmtools}
\usepackage[framemethod=TikZ]{mdframed}
\mdfsetup{skipabove=1em,skipbelow=1em}

\theoremstyle{definition}

\declaretheoremstyle[
    headfont=\bfseries\sffamily\color{ForestGreen!70!black}, bodyfont=\normalfont,
    mdframed={
        linewidth=1pt,
        rightline=false, topline=false, bottomline=false,
        linecolor=ForestGreen, backgroundcolor=ForestGreen!5,
    }
]{thmgreenbox}

\declaretheoremstyle[
    headfont=\bfseries\sffamily\color{NavyBlue!70!black}, bodyfont=\normalfont,
    mdframed={
        linewidth=1pt,
        rightline=false, topline=false, bottomline=false,
        linecolor=NavyBlue, backgroundcolor=NavyBlue!5,
    }
]{thmbluebox}

\declaretheoremstyle[
    headfont=\bfseries\sffamily\color{NavyBlue!70!black}, bodyfont=\normalfont,
    mdframed={
        linewidth=1pt,
        rightline=false, topline=false, bottomline=false,
        linecolor=NavyBlue
    }
]{thmblueline}

\declaretheoremstyle[
    headfont=\bfseries\sffamily, bodyfont=\normalfont,
    numbered = no,
    mdframed={
        rightline=true, topline=true, bottomline=true,
    }
]{thmbox}

\declaretheoremstyle[
    headfont=\bfseries\sffamily, bodyfont=\normalfont,
    numbered=no,
    % mdframed={
    %     rightline=true, topline=false, bottomline=true,
    % },
    qed=\qedsymbol
]{thmproofbox}

\declaretheoremstyle[
    headfont=\bfseries\sffamily\color{NavyBlue!70!black}, bodyfont=\normalfont,
    numbered=no,
    mdframed={
        rightline=false, topline=false, bottomline=false,
        linecolor=NavyBlue, backgroundcolor=NavyBlue!1,
    },
]{thmexplanationbox}

\declaretheorem[
    style=thmbox, 
    % numberwithin = section,
    numbered = no,
    name=Definition
    ]{definition}

\declaretheorem[
    style=thmbox, 
    name=Example,
    ]{eg}

\declaretheorem[
    style=thmbox, 
    % numberwithin = section,
    name=Proposition]{prop}

\declaretheorem[
    style = thmbox,
    numbered=yes,
    name =Problem
    ]{problem}

\declaretheorem[style=thmbox, name=Theorem]{theorem}
\declaretheorem[style=thmbox, name=Lemma]{lemma}
\declaretheorem[style=thmbox, name=Corollary]{corollary}

\declaretheorem[style=thmproofbox, name=Proof]{replacementproof}

\declaretheorem[style=thmproofbox, 
                name = Solution
                ]{replacementsolution}

\renewenvironment{proof}[1][\proofname]{\vspace{-1pt}\begin{replacementproof}}{\end{replacementproof}}

\newenvironment{solution}
    {
        \vspace{-1pt}\begin{replacementsolution}
    }
    { 
            \end{replacementsolution}
    }

\declaretheorem[style=thmexplanationbox, name=Proof]{tmpexplanation}
\newenvironment{explanation}[1][]{\vspace{-10pt}\begin{tmpexplanation}}{\end{tmpexplanation}}

\declaretheorem[style=thmbox, numbered=no, name=Remark]{remark}
\declaretheorem[style=thmbox, numbered=no, name=Note]{note}

\newtheorem*{uovt}{UOVT}
\newtheorem*{notation}{Notation}
\newtheorem*{previouslyseen}{As previously seen}
% \newtheorem*{problem}{Problem}
\newtheorem*{observe}{Observe}
\newtheorem*{property}{Property}
\newtheorem*{intuition}{Intuition}

\usepackage{etoolbox}
\AtEndEnvironment{vb}{\null\hfill$\diamond$}%
\AtEndEnvironment{intermezzo}{\null\hfill$\diamond$}%
% \AtEndEnvironment{opmerking}{\null\hfill$\diamond$}%

% http://tex.stackexchange.com/questions/22119/how-can-i-change-the-spacing-before-theorems-with-amsthm
\makeatletter
% \def\thm@space@setup{%
%   \thm@preskip=\parskip \thm@postskip=0pt
% }
\newcommand{\oefening}[1]{%
    \def\@oefening{#1}%
    \subsection*{Oefening #1}
}

\newcommand{\suboefening}[1]{%
    \subsubsection*{Oefening \@oefening.#1}
}

\newcommand{\exercise}[1]{%
    \def\@exercise{#1}%
    \subsection*{Exercise #1}
}

\newcommand{\subexercise}[1]{%
    \subsubsection*{Exercise \@exercise.#1}
}


\usepackage{xifthen}

\def\testdateparts#1{\dateparts#1\relax}
\def\dateparts#1 #2 #3 #4 #5\relax{
    \marginpar{\small\textsf{\mbox{#1 #2 #3 #5}}}
}

\def\@lesson{}%
\newcommand{\lesson}[3]{
    \ifthenelse{\isempty{#3}}{%
        \def\@lesson{Lecture #1}%
    }{%
        \def\@lesson{Lecture #1: #3}%
    }%
    \subsection*{\@lesson}
    \testdateparts{#2}
}

% \renewcommand\date[1]{\marginpar{#1}}


% fancy headers
\usepackage{fancyhdr}
\pagestyle{fancy}

\makeatother

% notes
\usepackage{todonotes}
\usepackage{tcolorbox}

\tcbuselibrary{breakable}
\newenvironment{verbetering}{\begin{tcolorbox}[
    arc=0mm,
    colback=white,
    colframe=green!60!black,
    title=Opmerking,
    fonttitle=\sffamily,
    breakable
]}{\end{tcolorbox}}

\newenvironment{noot}[1]{\begin{tcolorbox}[
    arc=0mm,
    colback=white,
    colframe=white!60!black,
    title=#1,
    fonttitle=\sffamily,
    breakable
]}{\end{tcolorbox}}

% figure support
\usepackage{import}
\usepackage{xifthen}
\pdfminorversion=7
\usepackage{pdfpages}
\usepackage{transparent}
\newcommand{\incfig}[1]{%
    \def\svgwidth{\columnwidth}
    \import{./figures/}{#1.pdf_tex}
}

% %http://tex.stackexchange.com/questions/76273/multiple-pdfs-with-page-group-included-in-a-single-page-warning
\pdfsuppresswarningpagegroup=1



\title{Homework 9 Extra Credit}
\author{Lance Remigio}

\begin{document}
   \maketitle 

   \begin{lemma}\label{lemma}
    Let \( (X,\|\cdot\|) \) be a normed space and let \( ({x}_{n}) \) and \( ({y}_{n}) \) be two sequences in \( X  \) where \( {y}_{n} = {x}_{n+1} - {x}_{n} \).  If \( \sum_{ n=1  }^{ \infty  } {y}_{n} \) converges, then \( ({x}_{n}) \) converges.
\end{lemma}
\begin{proof}
Let \( {y}_{n} = {x}_{n+1} - {x}_{n} \) and suppose \( \sum_{ n=1  }^{ \infty  } {y}_{n} \) converges. Define
\[  {s}_{n} = \sum_{ k=1  }^{ n } {y}_{k} = \sum_{ k=1 }^{ n } ({x}_{k+1} - {x}_{k}). \]
Since \( \sum_{ n=1  }^{ \infty  } {y}_{n} \) converges, the sequence of partial sums \( ({s}_{n}) \) of \( \sum_{ n=1  }^{ \infty  } {y}_{n} \) must converge to some \( s \in X  \). Our goal is to show that \( ({x}_{n}) \) converges. Observe that 
\begin{align*}
    {x}_{1} + \sum_{ k=1  }^{ n - 1  } {y}_{k} &= {x}_{1} + \sum_{ k=1  }^{ n-1  } ({x}_{k+1} - {x}_{k}) \\
                                               &= {x}_{1} + ({x}_{2} - {x}_{1}) + ({x}_{3} - {x}_{2}) + \cdots + ({x}_{n} - {x}_{n-1}) \\
                                               &= {x}_{n}.
\end{align*}
Hence, we have for all \( n \in \N  \) that 
\[  {x}_{n} = {x}_{1} + \sum_{ k=1  }^{ n - 1  } {y}_{k }. \]
Since \( \sum_{ n=1  }^{ \infty  } {y}_{n} \) converges, we can use the algebraic limit theorem on the equation above to conclude that 
\[  \lim_{ n \to \infty  }  {x}_{n} = {x}_{1} + s.  \]

\end{proof}

\begin{problem}
    Prove that a normed space \( (X, \| \cdot \|) \) is a Banach space if and only if every absolutely convergent series is convergent. 
\end{problem}

\begin{proof}
\( (\Longrightarrow) \) Note that this direction was proven during class.

\( (\Longleftarrow) \) Suppose every absolutely convergent series converges. Our goal is to show that \( X  \) is a Banach Space; that is, we need to show that \( X  \) is a complete normed space. Hence, we need to show that every Cauchy sequence in \( X  \) is convergent. To this end, let \( ({x}_{n}) \) be a Cauchy sequence in \( X  \). Our strategy is to find a subsequence \( ({x}_{{n}_{k }}) \) of \( ({x}_{n}) \) that converges to some \( x \in X  \).

Since \( ({x}_{n}) \) is a Cauchy Sequence, there exists an \( N \in \N  \) such that for any \( n > m > N  \), we have
\[  \|{x}_{n} - {x}_{m} \| < \epsilon.  \]
We claim that for all \( k \in \N  \), there exists \( {m}_{k} \in \N  \) such that for all \( n > m  > {m}_{k} \),  
\[  \|{x}_{n} - {x}_{m}\| < \frac{ 1 }{ 2^{k-1} }. \]
Indeed, using the fact that \( ({x}_{n}) \) is Cauchy, we have
\begin{align*}
    &\text{For} \ \epsilon = 1  \ \exists {m}_{1} \in \N \ \text{such that} \ \forall n,m > {m}_{1}, \ \|{x}_{n} - {x}_{m} \| < 1    \\
    &\text{For} \ \epsilon = \frac{ 1 }{ 2^{1} }  \ \exists {m}_{2} \in \N \ \text{such that} \ \forall n,m > {m}_{2}, \ \|{x}_{n} - {x}_{m} \| < \frac{ 1 }{ 2^{1} }    \\
    &\text{For} \ \epsilon = \frac{ 1 }{ 2^{2} }  \ \exists {m}_{3} \in \N \ \text{such that} \ \forall n,m > {m}_{3}, \ \|{x}_{n} - {x}_{m} \| < \frac{ 1 }{ 2^{2} }    \\
                                               &\vdots \\
                                               &\text{For} \ \epsilon = \frac{ 1 }{ 2^{k-1} }  \ \exists {m}_{k} \in \N \ \text{such that} \ \forall n,m > {m}_{k}, \ \|{x}_{n} - {x}_{m} \| < \frac{ 1 }{ 2^{k-1} } \\
                                                     &\vdots
\end{align*}
More generally, we see that for any \( k \in \N  \), there exists \( {m}_{k} \) such that for any \( n > m > {m}_{k} \)
\[  \|{x}_{n} - {x}_{m} \| < \frac{ 1 }{ 2^{k-1} }. \tag{*}  \]
In what follows, we will construct \( ({x}_{{n}_{k}}) \) of \( ({x}_{n}) \). For every \( k \in \N  \), choose \( {n}_{k} > {m}_{k} \) defined by \( {n}_{k} = {m}_{k} + 1     \) such that \( {n}_{k} > {m}_{k} \). Similarly, for all \( k \in \N \), we can choose \( {n}_{k+1} > {n}_{k}   \) where \( {n}_{k+1} = {n}_{k} + 1  \).  Clearly, we see that for any \( k \in \N  \), we have \( {n}_{k+1} > {n}_{k} > {m}_{k} \). By (*), we can see that        
\[  \|{x}_{{n}_{k+1}} - {x}_{{n}_{k}} \| < \frac{ 1 }{ 2^{k-1} }. \]
Now, observe that 
\begin{align*}
    0 < \underbrace{\sum_{ k=1  }^{ \infty   } \| {x}_{{n}_{k+1}} - {x}_{{n}_{k}} \|}_{\text{This is a series in \( \R  \)}} &< \sum_{ k=1  }^{ \infty   } \frac{ 1 }{ 2^{k-1} } = 1.
\end{align*}
Note that the above holds because of (*) and the fact that \( \displaystyle \sum_{ k=1  }^{ \infty    } \frac{ 1 }{ 2^{k-1} } = 1  \) is a geometric series. Hence, we see that the series above converges absolutely. By assumption, we must have 
\[  \sum_{ k=1  }^{ \infty  } ({x}_{{n}_{k+1}} - {x}_{{n}_{k}} ) \ \text{converges}. \]
From the {\hyperref[lemma]{lemma}} above, we can see that \( \lim_{ k  \to  \infty   }  {x}_{{n}_{k}} = x \) for some \( x \in X \). Using this result along with the result found in exercise 18 from homework 8, we can say that \( ({x}_{n}) \) must converge to \( x  \) as well. Thus, we conclude that \( X  \) is a Banach Space.
\end{proof} 

If \( ({x}_{n}) \) is a Cauchy sequence, then \( ({x}_{n}) \) contains a subsequence \( ({x}_{{n}_{k}}) \) such that for all \( k \in \N  \), we have
\[  \| {x}_{{n}_{k+1}} - {x}_{{n}_{k}} \| < \frac{ 1 }{ 2^{k} }.  \]


\end{document}
