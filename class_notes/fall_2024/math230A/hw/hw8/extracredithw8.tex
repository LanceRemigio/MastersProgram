\documentclass[a4paper]{article}
\usepackage[utf8]{inputenc}
\usepackage[T1]{fontenc}
% \usepackage{fourier}
\usepackage{textcomp}
\usepackage{hyperref}
\usepackage[english]{babel}
\usepackage{url}
% \usepackage{hyperref}
% \hypersetup{
%     colorlinks,
%     linkcolor={black},
%     citecolor={black},
%     urlcolor={blue!80!black}
% }
\usepackage{graphicx} \usepackage{float}
\usepackage{booktabs}
\usepackage{enumitem}
% \usepackage{parskip}
% \usepackage{parskip}
\usepackage{emptypage}
\usepackage{subcaption}
\usepackage{multicol}
\usepackage[usenames,dvipsnames]{xcolor}
\usepackage{ocgx}
% \usepackage{cmbright}


\usepackage[margin=1in]{geometry}
\usepackage{amsmath, amsfonts, mathtools, amsthm, amssymb}
\usepackage{thmtools}
\usepackage{mathrsfs}
\usepackage{cancel}
\usepackage{bm}
\newcommand\N{\ensuremath{\mathbb{N}}}
\newcommand\R{\ensuremath{\mathbb{R}}}
\newcommand\Z{\ensuremath{\mathbb{Z}}}
\renewcommand\O{\ensuremath{\emptyset}}
\newcommand\Q{\ensuremath{\mathbb{Q}}}
\newcommand\C{\ensuremath{\mathbb{C}}}
\newcommand\F{\ensuremath{\mathbb{F}}}
% \newcommand\P{\ensuremath{\mathbb{P}}}
\DeclareMathOperator{\sgn}{sgn}
\DeclareMathOperator{\diam}{diam}
\DeclareMathOperator{\LO}{LO}
\DeclareMathOperator{\UP}{UP}
\DeclareMathOperator{\card}{card}
\DeclareMathOperator{\Arg}{Arg}
\DeclareMathOperator{\Dom}{Dom}
\DeclareMathOperator{\Log}{Log}
\DeclareMathOperator{\dist}{dist}
% \DeclareMathOperator{\span}{span}
\usepackage{systeme}
\let\svlim\lim\def\lim{\svlim\limits}
\renewcommand\implies\Longrightarrow
\let\impliedby\Longleftarrow
\let\iff\Longleftrightarrow
\let\epsilon\varepsilon
\usepackage{stmaryrd} % for \lightning
\newcommand\contra{\scalebox{1.1}{$\lightning$}}
% \let\phi\varphi
\renewcommand\qedsymbol{$\blacksquare$}

% correct
\definecolor{correct}{HTML}{009900}
\newcommand\correct[2]{\ensuremath{\:}{\color{red}{#1}}\ensuremath{\to }{\color{correct}{#2}}\ensuremath{\:}}
\newcommand\green[1]{{\color{correct}{#1}}}

% horizontal rule
\newcommand\hr{
    \noindent\rule[0.5ex]{\linewidth}{0.5pt}
}

% hide parts
\newcommand\hide[1]{}

% si unitx
\usepackage{siunitx}
\sisetup{locale = FR}
% \renewcommand\vec[1]{\mathbf{#1}}
\newcommand\mat[1]{\mathbf{#1}}

% tikz
\usepackage{tikz}
\usepackage{tikz-cd}
\usetikzlibrary{intersections, angles, quotes, calc, positioning}
\usetikzlibrary{arrows.meta}
\usepackage{pgfplots}
\pgfplotsset{compat=1.13}

\tikzset{
    force/.style={thick, {Circle[length=2pt]}-stealth, shorten <=-1pt}
}

% theorems
\makeatother
\usepackage{thmtools}
\usepackage[framemethod=TikZ]{mdframed}
\mdfsetup{skipabove=1em,skipbelow=1em}

\theoremstyle{definition}

\declaretheoremstyle[
    headfont=\bfseries\sffamily\color{ForestGreen!70!black}, bodyfont=\normalfont,
    mdframed={
        linewidth=1pt,
        rightline=false, topline=false, bottomline=false,
        linecolor=ForestGreen, backgroundcolor=ForestGreen!5,
    }
]{thmgreenbox}

\declaretheoremstyle[
    headfont=\bfseries\sffamily\color{NavyBlue!70!black}, bodyfont=\normalfont,
    mdframed={
        linewidth=1pt,
        rightline=false, topline=false, bottomline=false,
        linecolor=NavyBlue, backgroundcolor=NavyBlue!5,
    }
]{thmbluebox}

\declaretheoremstyle[
    headfont=\bfseries\sffamily\color{NavyBlue!70!black}, bodyfont=\normalfont,
    mdframed={
        linewidth=1pt,
        rightline=false, topline=false, bottomline=false,
        linecolor=NavyBlue
    }
]{thmblueline}

\declaretheoremstyle[
    headfont=\bfseries\sffamily, bodyfont=\normalfont,
    numbered = no,
    mdframed={
        rightline=true, topline=true, bottomline=true,
    }
]{thmbox}

\declaretheoremstyle[
    headfont=\bfseries\sffamily, bodyfont=\normalfont,
    numbered=no,
    % mdframed={
    %     rightline=true, topline=false, bottomline=true,
    % },
    qed=\qedsymbol
]{thmproofbox}

\declaretheoremstyle[
    headfont=\bfseries\sffamily\color{NavyBlue!70!black}, bodyfont=\normalfont,
    numbered=no,
    mdframed={
        rightline=false, topline=false, bottomline=false,
        linecolor=NavyBlue, backgroundcolor=NavyBlue!1,
    },
]{thmexplanationbox}

\declaretheorem[
    style=thmbox, 
    % numberwithin = section,
    numbered = no,
    name=Definition
    ]{definition}

\declaretheorem[
    style=thmbox, 
    name=Example,
    ]{eg}

\declaretheorem[
    style=thmbox, 
    % numberwithin = section,
    name=Proposition]{prop}

\declaretheorem[
    style = thmbox,
    numbered=yes,
    name =Problem
    ]{problem}

\declaretheorem[style=thmbox, name=Theorem]{theorem}
\declaretheorem[style=thmbox, name=Lemma]{lemma}
\declaretheorem[style=thmbox, name=Corollary]{corollary}

\declaretheorem[style=thmproofbox, name=Proof]{replacementproof}

\declaretheorem[style=thmproofbox, 
                name = Solution
                ]{replacementsolution}

\renewenvironment{proof}[1][\proofname]{\vspace{-1pt}\begin{replacementproof}}{\end{replacementproof}}

\newenvironment{solution}
    {
        \vspace{-1pt}\begin{replacementsolution}
    }
    { 
            \end{replacementsolution}
    }

\declaretheorem[style=thmexplanationbox, name=Proof]{tmpexplanation}
\newenvironment{explanation}[1][]{\vspace{-10pt}\begin{tmpexplanation}}{\end{tmpexplanation}}

\declaretheorem[style=thmbox, numbered=no, name=Remark]{remark}
\declaretheorem[style=thmbox, numbered=no, name=Note]{note}

\newtheorem*{uovt}{UOVT}
\newtheorem*{notation}{Notation}
\newtheorem*{previouslyseen}{As previously seen}
% \newtheorem*{problem}{Problem}
\newtheorem*{observe}{Observe}
\newtheorem*{property}{Property}
\newtheorem*{intuition}{Intuition}

\usepackage{etoolbox}
\AtEndEnvironment{vb}{\null\hfill$\diamond$}%
\AtEndEnvironment{intermezzo}{\null\hfill$\diamond$}%
% \AtEndEnvironment{opmerking}{\null\hfill$\diamond$}%

% http://tex.stackexchange.com/questions/22119/how-can-i-change-the-spacing-before-theorems-with-amsthm
\makeatletter
% \def\thm@space@setup{%
%   \thm@preskip=\parskip \thm@postskip=0pt
% }
\newcommand{\oefening}[1]{%
    \def\@oefening{#1}%
    \subsection*{Oefening #1}
}

\newcommand{\suboefening}[1]{%
    \subsubsection*{Oefening \@oefening.#1}
}

\newcommand{\exercise}[1]{%
    \def\@exercise{#1}%
    \subsection*{Exercise #1}
}

\newcommand{\subexercise}[1]{%
    \subsubsection*{Exercise \@exercise.#1}
}


\usepackage{xifthen}

\def\testdateparts#1{\dateparts#1\relax}
\def\dateparts#1 #2 #3 #4 #5\relax{
    \marginpar{\small\textsf{\mbox{#1 #2 #3 #5}}}
}

\def\@lesson{}%
\newcommand{\lesson}[3]{
    \ifthenelse{\isempty{#3}}{%
        \def\@lesson{Lecture #1}%
    }{%
        \def\@lesson{Lecture #1: #3}%
    }%
    \subsection*{\@lesson}
    \testdateparts{#2}
}

% \renewcommand\date[1]{\marginpar{#1}}


% fancy headers
\usepackage{fancyhdr}
\pagestyle{fancy}

\makeatother

% notes
\usepackage{todonotes}
\usepackage{tcolorbox}

\tcbuselibrary{breakable}
\newenvironment{verbetering}{\begin{tcolorbox}[
    arc=0mm,
    colback=white,
    colframe=green!60!black,
    title=Opmerking,
    fonttitle=\sffamily,
    breakable
]}{\end{tcolorbox}}

\newenvironment{noot}[1]{\begin{tcolorbox}[
    arc=0mm,
    colback=white,
    colframe=white!60!black,
    title=#1,
    fonttitle=\sffamily,
    breakable
]}{\end{tcolorbox}}

% figure support
\usepackage{import}
\usepackage{xifthen}
\pdfminorversion=7
\usepackage{pdfpages}
\usepackage{transparent}
\newcommand{\incfig}[1]{%
    \def\svgwidth{\columnwidth}
    \import{./figures/}{#1.pdf_tex}
}

% %http://tex.stackexchange.com/questions/76273/multiple-pdfs-with-page-group-included-in-a-single-page-warning
\pdfsuppresswarningpagegroup=1


\title{Homework 8 Extra Credit}

\author{Lance Remigio}

\begin{document}
\maketitle

\begin{problem}
   Prove that  
   \[  \lim \inf {a}_{n} + \lim \inf {b}_{n} \leq \lim \inf({a}_{n} + {b}_{n}) \leq \lim \sup {a}_{n} + \lim \inf {b}_{n}  \]
   provided that all expressions are meaningful.
\end{problem}
\begin{proof}
Our goal is to show that 
\begin{enumerate}
    \item[(1)] \( \lim \inf {a}_{n} + \lim \inf {b}_{n} \leq \lim \inf({a}_{n} + {b}_{n})  \)
    \item[(2)]\( \inf({a}_{n} + {b}_{n}) \leq \lim \sup {a}_{n} + \lim \inf {b}_{n} \). 
\end{enumerate}
To show (1), we will assume that the left-hand side is NOT of the form \( \infty  + \infty   \). Hence, there exists \( n \geq {n}_{0}  \) and \( \ell \geq n  \) such that 
\begin{align*}
    {a}_{\ell} &\geq \inf \{ {a}_{k} : k \geq n  \},  \\
    {b}_{\ell} &\geq \inf \{ {b}_{k} : k \geq n \}. 
\end{align*}
Adding both inequalities above gives us 
\[  {a}_{\ell} + {b}_{\ell} \geq \inf \{ {a}_{k} : k \geq n  \}  + \inf \{ {b}_{k} : k \geq n  \}.  \]
Using the order limit theorem and algebraic limit theorem, we can write
\begin{align*}
    \lim_{ n \to \infty  }  \inf \{ {a}_{\ell } + {b}_{\ell} : \ell \geq n  \}   &\geq \lim_{ n \to \infty  }  [ \inf \{ {a}_{k } : k \geq n  \}  + \inf \{ {b}_{k } : k \geq n  \} ] \\
                                                                                 &= \lim_{ n \to \infty  }  \inf \{ {a}_{k } : k \geq n  \}  + \lim_{ n \to \infty  }  \inf \{ {b}_{k } : k \geq n  \}.
\end{align*}
Thus, we can conclude that 
\[  \lim \inf ({a}_{n} + {b}_{n}) \geq \lim \inf ({a}_{n}) + \lim \inf ({b}_{n}) \]
which establishes (1).

To show (2), we will consider the three cases; that is, 
\begin{itemize}
    \item \( \lim \inf ({a}_{n} + {b}_{n}) \neq -\infty \)
    \item \( \lim \sup {a}_{n} = \infty \)
    \item \( \lim \inf ({a}_{n} + {b}_{n}) \neq - \infty   \) and \( \lim \sup {a}_{n} \neq \infty \).
\end{itemize}

Suppose \( \lim \inf ({a}_{n} + {b}_{n}) \neq - \infty   \). If this is the case, the right-hand side of (2) will always hold. Next, suppose \( \lim \sup {a}_{n} = \infty \). Note that the right-hand side of (2) is NOT of the form \( \infty  + \infty   \). So, we can conclude that \( \lim \inf {b}_{n} \neq - \infty  \) and so 
\[ \lim \inf {b}_{n} + \lim \sup {a}_{n} = \infty. \]

Now, suppose that \( \lim \inf ({a}_{n} + {b}_{n}) \neq - \infty   \) and \( \lim \sup {a}_{n} \neq \infty \). Then using (1) and the algebraic limit theorem, we can write
\begin{align*}
    \lim \inf {b}_{n} &= \lim \inf [({b}_{n} + {a}_{n}) + (- {a}_{n})] \\
                      &\geq \lim \inf ({a}_{n} + {b}_{n}) + \lim \inf ({a}_{n}) \tag{ALT and (1)} \\
                      &= \lim \inf ({a}_{n} + {b}_{n}) - \lim \sup ({a}_{n})   \tag{Exercise 9}.
\end{align*}
Then we have 
\[  \lim \inf ({a}_{n} + {b}_{n}) \leq \lim \inf {b}_{n} + \lim \sup {a}_{n} \]
which establishes (2). Hence, we now conclude that 
   \[  \lim \inf {a}_{n} + \lim \inf {b}_{n} \leq \lim \inf({a}_{n} + {b}_{n}) \leq \lim \sup {a}_{n} + \lim \inf {b}_{n}  \]
\end{proof}

\begin{problem}
    Suppose \( X  \) is a nonempty complete metric space, and \( ({G}_{n}) \) is a sequence of dense open subsets of \( X  \). Prove Baire's Theorem, namely, \( \bigcap_{ n=1  }^{ \infty  }  {G}_{n}  \) is nonempty.
\end{problem}
\begin{proof}

\end{proof}

\begin{problem}
   Prove the following theorem: 
   \begin{center}
       Suppose \( ({s}_{n}) \) and \( ({b}_{n}) \) are two sequences of real numbers, \( ({b}_{n}) \) is a strictly increasing sequence that diverges to \( \infty   \), and \( \lim_{ n \to \infty  }  \frac{ {s}_{n+1} - {s}_{n}  }{ {b}_{n+1} - {b}_{n} } = L \in \R  \). Then \( \lim_{ n \to \infty  }  \frac{ {s}_{n} }{  {b}_{n} }  = L  \).
   \end{center}
\end{problem}
\begin{proof}
Suppose \( ({s}_{n}) \) and \( ({b}_{n}) \) are two sequences of real numbers and \( ({b}_{n}) \) is a strictly increasing sequence that diverges to \( \infty  \), and that 
\[  \lim_{ n \to \infty  }  \frac{ {s}_{n+1} - {s}_{n} }{  {b}_{n+1} - {b}_{n} }  =  L \in \R.  \]
Our goal is to show that \( \lim_{ n \to \infty  }  \frac{ {s}_{n} }{ {b}_{n} }  = L  \). In order to do this, we need to show that for any given \( \epsilon > 0  \), we have
\begin{align*}
    \lim \sup \frac{ {s}_{n} }{ {b}_{n} } &\leq L + \epsilon \tag{1}  \\
    \lim \inf \frac{ {s}_{n} }{ {b}_{n}  }  &\geq L + \epsilon \tag{2}
\end{align*}
To this end, let \( \epsilon > 0  \) be given. Since \( \lim_{ n \to \infty  }  \frac{ {s}_{n+1} - {s}_{n} }{  {b}_{n+1} - {b}_{n} }  = L  \), there exists an \( N \in \N  \) such that for any \( n > N \), we have 
\[  \Big|  \frac{ {s}_{n+1} - {s}_{n} }{  {b}_{n+1} - {b}_{n} } - L  \Big| < \epsilon \]
which can be written in the following way:
\[  (L-\epsilon) ({b}_{n+1} - {b}_{n}) < {s}_{n+1} - {s}_{n} < (L + \epsilon)({b}_{n+1} - {b}_{n}). \tag{*} \]
Now, fix an \( \hat{N} > N    \) denoted by \( \hat{N} = N + 2 \). Then observe that 
\[  {s}_{n} = [({s}_{n} - {s}_{n-1}) + ({s}_{n-1} - {s}_{n-2}) + \cdots + ({s}_{\hat{N}} - {s}_{\hat{N} - 1})] + {s}_{\hat{N} - 1}. \]
Applying (*) to the equation above gives us 
\begin{align*}
    {s}_{n} &< (L+\epsilon)[({b}_{n} - {b}_{n-1}) + ({b}_{n-1} - {b}_{n-2}) + \cdots + ({b}_{\hat{N}} - {b}_{\hat{N} -1})] + {s}_{\hat{N}-1} \\
            &= (L+\epsilon)[{b}_{n} - {b}_{\hat{N} - 1}] + {s}_{\hat{N} - 1}
\end{align*}
Multiplying by \( {b}_{n} \) on both sides, we get
\begin{align*} \frac{ {s}_{n} }{ {b}_{n} }  &< (L+\epsilon) \Big[ \frac{ {b}_{n} - {b}_{\hat{N} - 1} }{ {b}_{n} } \Big]  + \frac{ {s}_{\hat{N} - 1} }{ {b}_{n} } \\ 
    &= (L+\epsilon) +  \frac{ {s}_{\hat{N} -1} -  {b}_{\hat{N} -1} (L+\epsilon) }{ {b}_{n} } 
\end{align*}
Since we have \( ({b}_{n}) \) is a strictly increasing sequence that diverges to \( \infty   \), we must have \( \frac{ 1 }{ {b}_{n} }  \to 0  \). Since 
\[  \alpha = {s}_{\hat{N} - 1} - {b}_{\hat{N} - 1} (L+\epsilon)  \]
is a fixed quantity (because \( \hat{N} \) is fixed), we have that \( \frac{ \alpha }{ {b}_{n} } \to 0  \) by the algebraic limit theorem. By the order limit theorem, we can see that 
\begin{align*}
    \lim_{n \to \infty } \sup \frac{ {a}_{n} }{ {b}_{n} } &\leq  \lim_{ n \to \infty  }  \Big[ (L+\epsilon) + \frac{ \alpha  }{ {b}_{n} } \Big]  \\
                                                          &= (L+\epsilon) + \lim_{ n \to \infty  }  \frac{ \alpha }{ {b}_{n} } \tag{ALT}  \\
                                                          &= L+\epsilon + 0 \\ 
                                                          &= L + \epsilon.
\end{align*}
which establishes (1).

We can apply an analogous process to establish (2). By applying (*), we can see that 
\[ \frac{ {s}_{n} }{ {b}_{n} }  > (L - \epsilon) + \frac{ {s}_{\hat{N} - 1} - {b}_{\hat{N} - 1}(L-\epsilon) }{ {b}_{n} }.  \]
Now, set
\[  \beta = {s}_{\hat{N} - 1} - {b}_{\hat{N} - 1}(L - \epsilon). \]
By the order limit theorem, we see that 
\begin{align*}
    \lim_{ n \to \infty  }  \inf \frac{ {s}_{n} }{ {b}_{n} }  &\geq \lim_{ n \to \infty  }  \Big[ (L-\epsilon) + \frac{\beta}{ {b}_{n} } \Big]  \\
                                                              &= (L-\epsilon) + \lim_{ n \to \infty  }  \frac{ \beta }{ {b}_{n} } \tag{ALT}  \\
                                                              &= L - \epsilon + 0 \\
                                                              &= L - \epsilon
\end{align*} 
which establishes (2). Since \( \epsilon > 0  \) is arbitrary, (1) and (2) imply that 
\[  \lim \sup \frac{ {s}_{n} }{ {b}_{n} }  \leq L \ \ \text{and} \ \ \lim \inf \frac{  {s}_{n} }{ {b}_{n} } \geq L.   \]
Thus, we can see that
\[  \lim \sup \frac{ {s}_{n} }{ {b}_{n} }  = \lim \inf \frac{ {s}_{n} }{ {b}_{n} }  = L.   \]
This tells us that 
\[  \lim_{ n \to \infty  }  \frac{ {s}_{n} }{ {b}_{n} }  = L.  \]



\end{proof}



\end{document}
